% HEAD BEGIN
\documentclass[letterpaper, 12pt]{article}
\newsavebox\colbbox
\usepackage{graphicx}
\usepackage{multicol}
\usepackage{anysize}
\usepackage{fontspec}
\usepackage[fontset=none]{ctex}
\usepackage{tabularx}
\usepackage{longtable}
\PassOptionsToPackage{hyphens}{url}
\usepackage[breaklinks=true, colorlinks=true]{hyperref}
\expandafter\def\expandafter\UrlBreaks\expandafter{\UrlBreaks\do\a\do\b\do\c\do\d\do\e\do\f\do\g\do\h\do\i\do\j\do\k\do\l\do\m\do\n\do\o\do\p\do\q\do\r\do\s\do\t\do\u\do\v\do\w\do\x\do\y\do\z\do\A\do\B\do\C\do\D\do\E\do\F\do\G\do\H\do\I\do\J\do\K\do\L\do\M\do\N\do\O\do\P\do\Q\do\R\do\S\do\T\do\U\do\V\do\W\do\X\do\Y\do\Z}
% \let\oldurl\url
% \renewcommand{\url}[1]{\begin{sloppypar}\oldurl{#1}\end{sloppypar}}
\setlength\columnsep{30pt}
\marginsize{30pt}{30pt}{10pt}{20pt}
\setmainfont{TeX Gyre Bonum}
\setCJKmainfont[BoldFont=Noto Serif CJK SC Bold, ItalicFont=FandolKai]{Noto Sans CJK SC}
\setlength{\parindent}{0cm}
% \setCJKmonofont{Noto Sans CJK SC}
\begin{document}
\begin{center}
    \Huge\textbf{南哪大专醒前消息}
\end{center}
\vspace{4mm}
\hrule
\renewcommand\tabularxcolumn[1]{m{#1}}
\begin{tabularx}{\textwidth}{>{\hsize.2\hsize}X>{\hsize.6\hsize}X>{\hsize.2\hsize}X}
    \begin{flushleft}
        2025.1.26\, No.165
    \end{flushleft}
    &
    \begin{center}
        \textit{“秉中持正、求新博闻。”}
    \end{center}
    &
    \begin{flushright}
        \textbf{南京市栖霞区}
    \end{flushright}
\end{tabularx}
\vspace{-3.5mm}
\hrule
\vspace{4mm}
% HEAD END
\centerline{\huge\textbf{活动预告}}
\begin{multicols}{2}
    \section{订阅方式和加入编辑部}  
编辑部招聘人才,用爱发电,工作轻松,详情可联系QQ:1329527951 客服小祥\\想订阅本消息或获取PDF版(便于查看超链接和往期),可加QQ群:\href{https://qm.qq.com/q/VXIW7fgsEe}{849644979}.
\section{Deadline Ongoing}
\setbox\colbbox\vbox{
\makeatletter\col@number\@ne
\begin{longtable}{|c|c|c|}
    \hline
    消息(未见ddl的,不刊) & 截止日期 & 刊载日期\\
    \hline\hline
    全国大学生家史大赛 & 1.31 & 12.2\\
    西安史学论坛征稿 & 3.20 & 12.9\\
    12306学生优惠票 & 2.12 & 12.13\\
    南大博物馆展览 & 6.16 & 12.17\\
    南星小红书创作 & 2.6 & 12.27\\
    挂职干部选拔 & 2.13 & 12.31\\
    ASC25报名 & 2.21 & 1.6\\
    天文台开放日 & / & 1.6\\
    开甲书院科研作坊 & 2.17 & 1.6\\
    PL读书会 & 2.12 & 1.9\\
    春季选课 & 2.7 & 1.9\\
    原创剧本联合孵化报名 & 3.20 & 1.10\\
    阅读分享活动征稿 & 3.7 & 1.10\\
    njumun代表报名 & 3.2 & 1.16\\
    大气院寒假打卡 & 2.16 & 1.20\\
    冬季代码2025报名 & 2.8 & 1.26\\

    \hline
\end{longtable}
\unskip
\unpenalty
\unpenalty}\unvbox\colbbox
\end{multicols}
\hrule
\pagebreak
\begin{multicols}{2}

\section{讲座}
\begin{tabularx}{0.5\textwidth}{|X|X|X|}
    \hline
    讲座 & 开展时间 & 刊载时间\\
    \hline\hline
/ & / & /\\\hline
\end{tabularx}

1.《南天学堂》云讲堂第十九期|新时代的天体测量学\\
讲座专家:刘佳成,南京大学天文与空间科学学院教授讲座视频:\url{https://mp.weixin.qq.com/s/1jY1aYTCzUEzB-5eHQbFWw}\\

\section{选课指南 | “科学之光”(青年学者系列)}
报名截止时间:2025年1月26日24点\\
报 名 链 接 :\url{https://table.nju.edu.cn/dtable/forms/6afb4406-efd4-43e7-a969-5e3f4b439902/}\\
14.《药物发现》\\
15.《气候变化与碳中和》\\
16.《古老的光合与小微创新》\\
17.《环境科学科研训练营》\\
18.《稳定同位素解锁自然奥秘》\\
19.《数字医疗与人口老龄化》\\
20.《免疫攻坚战》\\
课程介绍和人数详见:\url{https://mp.weixin.qq.com/s/SIBECwIHzzzgcmiFfseZ4w}\\
注:前13个课程在上一期寒假小报中已刊登。或可观看新生学院汇总推文\url{https://mp.weixin.qq.com/s/i07h1IaZ8OjDOnpkhGQ3kA}\\
报名同学需同时向任课老师提交报名材料、填写table报名链接,选课结果待老师审核后,会自动导入课表,请大家耐心等待。\\

\section{南大电子校友卡正式开通食堂就餐权限}
在食堂窗口消费结算时,出示电子校友码,在刷卡机上刷码即可消费。\\
校友在食堂就餐消费,需收取20\%搭伙费,付款时自动计算。\\
校友卡目前已有权限:进出校园、进出图书馆、使用南京大学校友云书房、食堂就餐。\\
申请和使用的详细步骤见推文\url{https://mp.weixin.qq.com/s/hIbpdHSB6DtP6RpmZ3pdjg}\\


\section{科创竞赛 | Winter Code 2025}
同时招募参赛选手与志愿者\\
1.Winter Code 2025将面向全校本科生开展,主办方准备了基础赛道与提高赛道供参与者选择,每条赛道分别设置有算法设计和项目开发两个部分。\\
基础赛道:限定参与者为2024级本科生(不限专业),限定程序语言为C/C++,同时设置有志愿者进行辅导。\\
提高赛道:面向全校本科生,不限制年级与专业,不限制编程语言。\\
2.活动时间\\
基础赛道\\
预热阶段:2025年1月26日发布手册,不计分,不设截止时间。\\
算法阶段:2025年2月3日至2月9日。\\
项目阶段:2025年2月10日至2月23日。\\
提高赛道\\
算法阶段:将分别于2025年1月26号,2月5号,2月7号,2月9号,2月11号,2月13号,各举办一场比赛,共六场。每场会从早上九点持续到下午三点,持续六小时。参与者可以任选场次参加,并自行选择其中4场比赛的分数记入总成绩。\\
项目阶段:2025年2月3日至2月18日。\\
3.报名途径\\
加入活动QQ群(群号:1028669220),报名表将在群中发布。\\
建议在1月26日活动正式开始前报名,基础赛道最晚报名时间为2月8日(算法阶段最后一天),提高赛道不建议晚于2月6日(算法阶段第三场前一天)报名,晚于该时间报名将不可避免地影响算法阶段得分。\\
4.志愿者招募\\
职责:在基础赛道对萌新们给予一定帮助,可能参与两条赛道项目部分的打分。\\
要求:需要对面向对象编程、数据结构有一定掌握,并具有一定的C/C++的GUI编程或项目开发经验,报名参加提高赛道者亦可参与。\\
报名通道(志愿者招募的截止时间将不早于2月3日,问卷正在收集即表示招募正在进行):\url{https://docs.qq.com/form/page/DQkZtSU9ubXhoZWtO}\\
奖项设置等活动详情请参考推文:\url{https://mp.weixin.qq.com/s/p8EI7DoEUEyM7O0sNfCAKA}\\
注:Winter Code比赛提高组的算法部分将同步使用算法天梯的OJ平台与题库,可同时报名参加两个活动,但最终仅能选择在一个活动中获得奖品。\\

\end{multicols} 

\end{document}