% HEAD BEGIN
\documentclass[letterpaper, 12pt]{article}
\newsavebox\colbbox
\usepackage{graphicx}
\usepackage{multicol}
\usepackage{anysize}
\usepackage{fontspec}
\usepackage[fontset=none]{ctex}
\usepackage{tabularx}
\usepackage{longtable}
\PassOptionsToPackage{hyphens}{url}
\usepackage[breaklinks=true, colorlinks=true]{hyperref}
\expandafter\def\expandafter\UrlBreaks\expandafter{\UrlBreaks\do\a\do\b\do\c\do\d\do\e\do\f\do\g\do\h\do\i\do\j\do\k\do\l\do\m\do\n\do\o\do\p\do\q\do\r\do\s\do\t\do\u\do\v\do\w\do\x\do\y\do\z\do\A\do\B\do\C\do\D\do\E\do\F\do\G\do\H\do\I\do\J\do\K\do\L\do\M\do\N\do\O\do\P\do\Q\do\R\do\S\do\T\do\U\do\V\do\W\do\X\do\Y\do\Z}
% \let\oldurl\url
% \renewcommand{\url}[1]{\begin{sloppypar}\oldurl{#1}\end{sloppypar}}
\setlength\columnsep{30pt}
\marginsize{30pt}{30pt}{10pt}{20pt}
\setmainfont{TeX Gyre Bonum}
\setCJKmainfont[BoldFont=Noto Serif CJK SC Bold, ItalicFont=FandolKai]{Noto Sans CJK SC}
\setlength{\parindent}{0cm}
% \setCJKmonofont{Noto Sans CJK SC}
\begin{document}
\begin{center}
    \Huge\textbf{南哪大专醒前消息}
\end{center}
\vspace{4mm}
\hrule
\renewcommand\tabularxcolumn[1]{m{#1}}
\begin{tabularx}{\textwidth}{>{\hsize.2\hsize}X>{\hsize.6\hsize}X>{\hsize.2\hsize}X}
    \begin{flushleft}
        2024.11.25\, No.127
    \end{flushleft}
    &
    \begin{center}
        \textit{“秉中持正、求新博闻。”}
    \end{center}
    &
    \begin{flushright}
        \textbf{南京市栖霞区}
    \end{flushright}
\end{tabularx}
\vspace{-3.5mm}
\hrule
\vspace{4mm}
% HEAD END
\centerline{\huge\textbf{活动预告}}
\begin{multicols}{2}
    \section{订阅方式和加入编辑部}  
编辑部招聘人才,用爱发电,工作轻松,详情可联系QQ:1329527951 客服小祥\\想订阅本消息或获取PDF版(便于查看超链接和往期),可加QQ群:\href{https://qm.qq.com/q/VXIW7fgsEe}{849644979}.
\section{Deadline Ongoing}
\setbox\colbbox\vbox{
\makeatletter\col@number\@ne
\begin{longtable}{|c|c|c|}
    \hline
    消息(未见ddl的,不刊) & 截止日期 & 刊载日期\\
    \hline\hline
    紫藤学刊征稿 & 12.15 & 10.22\\
    乐跑 & 12.6 & 10.12\\
    DIY课程学术论坛征稿 & 11.30 & 11.13\\
    国风歌曲演唱赛 & 12.1 & 11.13\\
    牡丹亭庆演 & 12.1 & 11.13\\
    普通话测试网络报名 & 11.30 & 11.16\\
    安邦征稿 & 1.12 & 11.16\\
    秉文宿舍风采 & 12.1 & 11.17\\
    法学院征诗活动 & 12.2 & 11.20\\
    计院乒赛 & 11.26 & 11.20\\
    平安留学交流会 & 12.3 & 11.20\\
    七院相亲活动 & 11.30 & 11.21\\
    计科工作坊 & 11.26 & 11.21\\
    黑匣工作坊 & 11.26 & 11.21\\
    免费上网讲座 & 11.27 & 11.22\\
    猫鼠大战 & 12.15 & 11.22\\
    日本征文大赛 & 12.6 & 11.22\\
    萨勒姆的女巫 & 11.30 & 11.22\\
    防艾征集 & 12.10 & 11.22\\
    爱心义卖 & 11.30 & 11.22\\
    银杏叶制作之旅 & 11.29 & 11.23\\
    法学朋导招募 & 11.29 & 11.23\\
    南选问答集赞 & 11.27 & 11.23\\
    心协信件盲盒 & 12.7 & 11.23\\
    港澳台晚会招募 & 11.27 & 11.23\\
    配音大赛招募 & 12.7 & 11.23\\
    物院研会系列活动 & 12.21 & 11.23\\
    心理中心征稿 & 12.10 & 11.23\\
    防艾同伴教育 & 12.15 & 11.24\\
    南新读书会 & 11.27 & 11.24\\
    腾讯游戏开发赛报名 & 11.29 & 11.24\\
    新生午餐会报名 & 11.27 & 11.24\\
    物院摄影征集 & 12.9 & 11.24\\
    一南一度脱口秀 & 11.29 & 11.25\\
    天健志愿活动报名 & 11.31 & 11.25\\
    创意物理实验竞赛 & 12.21 & 11.15\\
    古声国风音乐活动 & 11.29 & 11.25\\
    南悦制作贺卡活动 & 11.30 & 11.25\\
    午餐读书会 & 11.27 & 11.25\\
    \hline
\end{longtable}
\unskip
\unpenalty
\unpenalty}\unvbox\colbbox
\end{multicols}
\hrule
\pagebreak
\begin{multicols}{2}

\section{讲座}
\begin{tabular}{|c|c|c|}
    \hline
    往期讲座 & 开展日期 & 刊载日期\\
    \hline\hline
    《专利查新与规避...》 & 12.19 & 10.3\\
    图书馆系列讲座 & 12.3 & 10.20\\
    《学术写作入门...》& 11.21 & 11.18\\
    《Adobe AI 讲座》 & 11.27 & 11.22\\
    《马克思的经济全球化》 & 11.27 & 11.23\\
    《法学研究类型和方法》 & 11.29 & 11.23\\
    《前沿情报捕捉...》 & 11.29 & 11.23\\
    《AI在设计中的参与...》 & 12.2 & 11.23\\
    《决策规划算法》 & 11.28 & 11.24\\
    《大英博物馆文物...》 & 11.26 & 11.24\\
    《摩尔条纹与超导性》 & 11.26 & 11.24\\
    《鲁艺木刻工作团...》 & 11.28 & 11.25\\
    《Rabinowitz FLoer...》 & 11.26 & 11.25\\
    《巧用万方数据...》 & 11.26 & 11.25\\
    《如何利用超星系...》 & 11.27 & 11.25\\
    《校外文献互借指南》 & 11.28 & 11.25\\
    《简历简化与制作...》 & 11.27 & 11.25\\
    \hline
\end{tabular}

\subsection{开辟新路——鲁艺木刻工作团的活动纬度及相关问题}
主讲人:杨灿伟 《美术》杂志社学术交流部主任\\
主持人:尚莲霞  南京大学艺术学院副教授\\
时间:2024年11月28日18:30\\
地点:南京大学东大楼310报告厅\\
主办单位:南京大学艺术学院\\

\subsection{数学学院青年学者论坛}
题目:Rabinowitz Floer theory and categorical formal\\
主 讲 人:高 原 YUAN GAO, University of Georgia\\
现场报告时间:北京时间2024年11月26日(周二)下午4:30-5:30\\
现场报告地点:西大楼108报告厅\\
腾讯会议:692-217-611\\
会议密码:1126\\
\url{https://mp.weixin.qq.com/s/GBSKhVQVzp0DBtZbqRjueg}\\

\subsection{巧用万方数据,畅享学术资源}
内容:如何使用万方数据平台\\
时间:11月26日18:30-19:30\\
地点:仙林图书馆校友之家小报告厅\\

\subsection{如何利用超星系列数据库做全资源检索}
时间:11月27日18:30-19:30\\
地点:仙林图书馆校友之家小报告厅\\

\subsection{校外文献互借指南}
时间:11月28日16:15-15:15\\
地点:仙林图书馆校友之家小报告厅\\

\subsection{求职能量站:简历简化与制作技巧培训}
时间:11月27日14:00-15:30\\
地点:仙林校区哲学学院401\\
\section{乐跑}
从明日(11月26日)算起,还有11次乐跑机会。
\section{一南一度脱口秀小会}
南京大学南说喜剧之“一南一度脱口秀小会”,笑声即将到达!\\
时间:2024年11月29日 19:00\\
地点:南大仙林校区化学化工学院H201\\
报名及集赞兑奖详见:\url{https://mp.weixin.qq.com/s/jQw2YDhTI7Vg75rQJlQPAQ}\\

\section{浦口校区研究生羽毛球联赛}
活动时间:2024年11月30日、2024年12月7日(周六)\\
活动地点:南京大学浦口校区 左涤江运动馆室内羽毛球场\\
报名截止时间:11月27日 22:00\\
报名方式及赛程安排:\url{https://mp.weixin.qq.com/s/GEFilHXLU7boK98aSVm1WQ}\\

\section{信管院秋日活动}
活动时间:2024年11月30日(周六)\\
活动地点:南京市栖霞区羊山公园\\

\section{秉烛话岁月长歌,初阳传科技新火}
由南京大学天健社组织的志愿服务活动。\\
活动内容 教会长辈们网购、 社交、出行的数字技能、向长辈们传授防骗知识、陪伴长辈们聊天\\
活动时间:2024年12月7日\\
活动地点:银城康养三牌楼颐养中心\\
招募人数:35名暖心使者\\
服务时长:6小时\\
报名截止时间:11月31日23:59:59\\
报名方式:点击链接报名 \\
\url{https://table.nju.edu.cn/apps/custom/regisreting7211}\\
详见\url{https://mp.weixin.qq.com/s/HhEspSg4pShA_-n5DW3nLw}

\section{物理数学专业专场招聘会}
活动时间 2024年11月29日(星期五)10:30-12:30\\
活动地点 南京大学鼓楼校区北园物理楼前广场\\
\url{https://mp.weixin.qq.com/s/Uzsiz9KYPFweXBTmnStxIQ}\\

\section{国际访学计划本科生课程报告}
活动时间:\\
2024年11月28日(周四)\\
2024年11月29日(周五)\\
活动地点:南京大学仙林校区朱共山楼A125\\
内容:“The Earth's First Crust &  A Habitable Hadean”和“The Origin of Life”\\
\section{第七届南京大学创意物理实验竞赛}
竞赛时间 2024年12月21日\\
竞赛地点 南京大学鼓楼校区科技馆二楼\\
对本届竞赛感兴趣的本科同学,都可以自行组队(每支队伍不超过3人)报名参加。本学期选修《新生研究性物理实验》课程的同学必须参加,以现有团队参赛。\\
参赛作品的海报电子版请于12月15日前上传

详见\url{https://mp.weixin.qq.com/s/84eTV3MHBPkamsaWE9JU4Q}

\section{古声未歇演唱比赛}
九歌国风音乐社 发布\\
仙林校区\\
时间:11月29日 11:30-13:00\\
地点:四五六食堂前\\
鼓楼校区\\
时间:11月29日 11:30-13:00\\
地点:南园小广场\\
转发推送即有奖,集赞达一定数目奖品更丰富\\
奖品详情与其他信息请查看\url{https://mp.weixin.qq.com/s/SEyDVowI_3fmnw2TGnqCMg}\\

\section{志愿招募 | 制作贺卡}
南京大学南悦青年公益社 发布\\
活动内容:面向南大学生征集志愿者为春蕾女孩制作一份新年贺卡。南悦将评比择优部分作品,并对优秀作品进行推广宣传。\\
活动反馈:所有贺卡将由南悦青年公益社统一收集寄送至中国儿童基金会;贺卡将通过春蕾计划一线执行伙伴,带给当地春蕾女孩;走心、感人的贺卡将有机会入选“春蕾计划”微信平台等。\\
贺卡要求、报名参与等信息请查看\url{https://mp.weixin.qq.com/s/8cCJevcl44FrmEI9POdAeQ}\\报名截止时间为11月30日

\section{新生午餐读书会第五场}
活动时间:2024年11月27日(周三)中午12:15-13:40\\
地点:鼓楼校区南大出版社党建活动室\\
主讲老师:马克思主义学院 李乾坤\\
1.《德意志意识形态(节选)》,《马克思恩格斯文集》第一卷,2009年。也有单行\\
2.大卫•哈维:《资本社会的17个矛盾》,中信出版社,2017年版\\
流程:两位同学发言各15分钟,提问讨论25分钟,老师评论导读30分钟\\
本次活动向参加读书会活动的全体老师同学提供午餐\\
20个名额抽签报名见\url{https://mp.weixin.qq.com/s/lD0tQ_7C7FrYS0a2ViuBNA}

\section{健雄书院文明宿舍评比}
评比对象:健雄书院2024级全体本科生寝室\\
奖项:1. 特等奖:寝室总数的前5\%,寝室成员每人可获得4小时劳育时长,并推荐至学校参评;2. 一等奖:寝室总数的前5%~20%,寝室成员每人可获得4小时劳育时长;3. 二等奖:寝室总数的前20%~50%,寝室成员每人可获得3小时劳育时长;4. 三等奖:若干,寝室成员每人可获得2小时劳育时长\\
报名时间:即日起至2024年11月29日(周五)中午12点\\
评比标准和报名链接详见\url{https://mp.weixin.qq.com/s/CFbSnM9npDJMrTW1kevCWg}

\section{鼓楼奋进跑}
活动时间:2024年11月25日—12月1日\\
周一至周五:21:30-22:10周末:19:30-21:00\\
活动地点:鼓楼校区苏浙体育场\\
活动奖励:男生跑步平均配速不超过6'30''(打卡里程数不少于2.4公里),女生跑步平均配速不超过7'00''( 打卡里程数不少于2公里),可以领取可爱早餐盘子挂件一个。\\
打卡方式和歌单预告见\url{https://mp.weixin.qq.com/s/elR8UzCThuY_gowqlKXVgA}

\section{有训书院“晚安短信”“暖冬心语”}
南京大学有训书院现通知活动如下

1.暖冬心语:把充满生活气息的南京冬季生活小贴士,或是情绪上的关怀,发给想要收到的同学,让我们在一言一语中,绘就冬天的温暖,传递缕缕情思,感受南雍的人文关怀\\
2.南雍诗情:1)推荐经典诗歌或自行创作,将作品或上传至云端让工作人员随机发送,或指定发送对象,以匿名或实名的方式在ta的身边细语呢喃。2)若想要收到诗歌,也可以填写表单,让诗情温润你的夜晚……\\
表单链接和QQ群见\url{https://mp.weixin.qq.com/s/3tmDResUUHczdAGUAnTjEg}

\section{“21天英语能力提升训练营”}
参与对象:行知书院及有训书院2024级新生\\
活动时间:2024年11月30日-2024年12月20日\\
参与形式:每名导师带8名新生(有训书院2名,行知书院6名),形成小组开展学习\\
活动安排:每周一到周日:每天由英语导师指定一定时长的听力材料,同学们按要求进行听力练习并上传打卡记录(每周五休整一天,集中进行答疑)\\
每周六:书院将邀请英语导师到鼓楼校区辅导大家进行口语练习\\
活动激励:坚持21天全勤完成英语听说打卡的同学,将获得新东方国际教育赞助的词汇书和英语学习材料一套\\
活动内容:邀请了来自外国语学院英语学硕第二党支部和2024级英语学硕团支部的13位英语导师,开展21天英语听说训练打卡活动。\\
报名方式见\url{https://mp.weixin.qq.com/s/FTjyo6JJfRc8HaIuG2nksw}

\end{multicols} 
\hrule
\vspace{4mm}
\centerline{\huge\textbf{参考消息}}
\begin{multicols}{2}
\section{美国一学校阻挠学生返乡参加白事}
来源:纽约时报

11月24日,德克萨斯生态工程职业学校一名学生因家人危重,凌晨请假回家,但宿管称必须6时30分才能开宿舍门,不允许其离开,最终该学生没有赶上回家的飞机。

凌晨5时许,德克萨斯生态工程职业学校一名学生在群内称,深夜临时接到家人危重消息,家里告知辅导员请假后,买了最早一班早晨7时30分的机票回家。但凌晨5时,她来到宿管值班室请求宿管开门时,宿管不予开门,“就坐在椅子上,啥也不说,问有没有学工部电话,直接说没有”。该学生还表示,前一晚加了宿管的Facebook,但一直没有通过。“先给二楼的同学道个歉,今天早上5点就打扰到你们了,是我情绪太激动了,主要家里发生事情,临时需要回去,买好了早上7.30的飞机票,但阿姨不让出去,我也没办法,所以就控制不住 在这里给大家道歉!!!抱歉吵醒大家”

7时40分,宿管发信称“温馨提示:@所有人 晚上 23 点锁宿舍大门,除了身体疾病,其他事情一律不能外出,如果执意要外出,必须有学工部的批文,请各位同学遵守校规, 请同学们配合。”

上午8时许,该学生在群内表示,其在二楼(宿管值班室)从5时待到6时30分,宿管就是不给开门,“飞机没赶上,最后一面没见上”。

11月25日,纽约时报新闻记者就此事咨询德克萨斯生态工程职业学院,接线员表示知道此事,“学工部在管”。该校学工部工作人员表示,此事已在处理,正在积极跟进。记者进一步询问具体处理情况,对方称负责处理的老师不在,可以稍后咨询,“学工部的电话正常上班时间都有人接听”。但记者按照约定时间再次多次致电,该电话已无人接听。

美国福克斯新闻荣退记者南森·胡德(Nonsens Hood)评价说:德克萨斯该学校的行为,反映出美国教育系统长期以来漠视学生人权,限制学生人身自由成性,以便于管理为由,随意镇压、迫害学生,极为可憎。
\end{multicols} 

\end{document}