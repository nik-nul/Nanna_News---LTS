% HEAD BEGIN
\documentclass[letterpaper, 12pt]{article}
\newsavebox\colbbox
\usepackage{graphicx}
\usepackage{multicol}
\usepackage{anysize}
\usepackage{fontspec}
\usepackage[fontset=none]{ctex}
\usepackage{tabularx}
\usepackage{longtable}
\PassOptionsToPackage{hyphens}{url}
\usepackage[breaklinks=true, colorlinks=true]{hyperref}
\expandafter\def\expandafter\UrlBreaks\expandafter{\UrlBreaks\do\a\do\b\do\c\do\d\do\e\do\f\do\g\do\h\do\i\do\j\do\k\do\l\do\m\do\n\do\o\do\p\do\q\do\r\do\s\do\t\do\u\do\v\do\w\do\x\do\y\do\z\do\A\do\B\do\C\do\D\do\E\do\F\do\G\do\H\do\I\do\J\do\K\do\L\do\M\do\N\do\O\do\P\do\Q\do\R\do\S\do\T\do\U\do\V\do\W\do\X\do\Y\do\Z}
% \let\oldurl\url
% \renewcommand{\url}[1]{\begin{sloppypar}\oldurl{#1}\end{sloppypar}}
\setlength\columnsep{30pt}
\marginsize{30pt}{30pt}{10pt}{20pt}
\setmainfont{TeX Gyre Bonum}
\setCJKmainfont[BoldFont=Noto Serif CJK SC Bold, ItalicFont=FandolKai]{Noto Sans CJK SC}
\setlength{\parindent}{0cm}
% \setCJKmonofont{Noto Sans CJK SC}
\begin{document}
\begin{center}
    \Huge\textbf{南哪大专醒前消息}
\end{center}
\vspace{4mm}
\hrule
\renewcommand\tabularxcolumn[1]{m{#1}}
\begin{tabularx}{\textwidth}{>{\hsize.2\hsize}X>{\hsize.6\hsize}X>{\hsize.2\hsize}X}
    \begin{flushleft}
        2024.10.25\, No.98
    \end{flushleft}
    &
    \begin{center}
        \textit{“克明峻德。”}
    \end{center}
    &
    \begin{flushright}
        \textbf{南京市栖霞区}
    \end{flushright}
\end{tabularx}
\vspace{-3.5mm}
\hrule
\vspace{4mm}
% HEAD END
\centerline{\huge\textbf{活动预告}}
\begin{multicols}{2}
    \section{订阅方式和加入编辑部}
编辑部招聘人才,用爱发电,工作轻松,详情可联系QQ:1329527951 客服小祥\\想订阅本消息或获取PDF版(便于查看超链接和往期),可加QQ群:\href{https://qm.qq.com/q/VXIW7fgsEe}{849644979}.
\section{Deadline Ongoing}
\setbox\colbbox\vbox{
\makeatletter\col@number\@ne
\begin{longtable}{|c|c|c|}
    \hline
    消息(未见ddl的,不刊) & 截止日期 & 刊载日期\\
    \hline\hline
    毓琇书院期中辅导 & 10.26、27 & 10.24\\
    历史学院羽毛球赛 & 10.26 & 10.22\\
    紫藤学刊征稿 & 12.15 & 10.22\\
    织围巾志愿者招募 & 10.26 & 10.20\\
    心协DIY活动 & 10.26 & 10.20\\
    心协流光影院 & 10.26 & 10.17\\
    校园今日说法大赛 & 10.26 & 10.17\\
    遵义精神宣讲团遴选 & 10.27 & 10.10\\
    青鸟剧场新戏招募 & 10.27 & 10.14\\
    体测 & 10.27 & 10.16\\
    鼓楼音乐跑 & 10.27 & 10.20\\
    普通话考试报名 & 10.28 & 10.14\\
    仙林校史馆招募讲解员 & 10.30 & 9.12\\
    本科生暑期课程评教 & 10.31 & 9.19\\
    黑匣招募 & 11.1 & 10.19\\
    学位英语考试报名 & 11.3 & 10.17\\
    校运会 & 11.8 & 10.21\\
    后革命鲁迅研究征文 & 11.10 & 10.8\\
    大创训练计划申报 & 11.18 & 9.24\\
    招生宣传创意征集大赛 & 11.18 & 10.21\\ 
    EBSCO数据库检索大赛 & 11.20 & 10.3\\
    文院征稿 & 11.20 & 10.20\\
    乐跑 & 12.8 & 10.12\\
    国际访学计划申报 & 11.22 & 10.22\\
    毓琇书院宿舍评比 & 10.31 & 10.22\\
    流光影院 & 10.26 & 10.23\\
    南大献血周 & 10.31 & 10.24\\
    南京马拉松志愿者招募 & 10.26 & 10.24\\
    信件盲盒活动 & 10.26 & 10.24\\

    \hline
\end{longtable}
\unskip
\unpenalty
\unpenalty}\unvbox\colbbox
\end{multicols}
\hrule
\pagebreak
\begin{multicols}{2}

\section{讲座}
\begin{tabular}{|c|c|c|}
    \hline
    往期讲座 & 开展日期 & 刊载日期\\
    \hline\hline
    《电池及电化学能...》 & 11.24 & 10.3\\
    《专利查新与规避...》 & 12.19 & 10.3\\
    《与<自然>编辑对...》 & 10.30 & 10.16\\
    图书馆系列讲座 & 12.3 & 10.20\\
    《志工人力资源的...》 & 11.4 & 10.23\\
    《华人社会工作的...》 & 11.4 & 10.23\\
    《困在历史中的卢...》 & 10.29 & 10.23\\
    《元明时代的中心...》 & 10.26 & 10.23\\
    《行知学堂……》 & 10.27 & 10.24\\
    《从诺奖看AI在科...》 & 10.26 & 10.24\\
    《下一代信任网络...》 & 10.26 & 10.24\\
    《大模型技术的发...》 & 10.31 & 10.25\\
    \hline
\end{tabular}

1.AI for Science 大模型技术的发展、实践及挑战(美团)\\
时间:2024年10月31日(周四) 19:00-20:00\\
地点:仙林校区计算机科学技术楼231教室\\
此次分享的主要内容是结合美团业务场景,介绍大模型关键技术的发展,分享大模型在不同业务场景下的实践经验和新的挑战。\\
讲座流程:\\
1. 技术大咖带你了解美团最新、最前沿的科技场景。\\
2. 校招HR解读美团技术校招的岗位和方向。\\
3. 现场Q\&A,与嘉宾一对一交流,尽情提问。\\
参与方式详见链接\url{https://mp.weixin.qq.com/s/AkQqn7UDnzjXm5kN7PXpaw}\\
到场即可获得精美伴手礼一份,讲座设置有奖问答,礼品多多!\\

\section{黑匣子 | 重写剧场史·南京}
是庆祝南京大学文学院建院110周年系列活动之一\\
中国当代表演艺术文献开放展\\
附:南京大学艺术硕士剧团回顾展\\
展览时间:2024.10.28-12.28\\
开幕:2024年10月28日15:00\\
地点:南京大学博物馆(星云楼一楼)\\
主办:南京大学文学院戏剧影视艺术系\\
协办:南京大学博物馆\\
展览具体内容详见\url{https://mp.weixin.qq.com/s/xCA2Qe7wDlXd2XnfmgniQA}

\section{物院奋进体能营}
活动时间:2024年10月28日 - 2024年12月8日\\
活动对象:物理学院全体师生\\
打卡项目:跑步、游泳、球类、健身、徒步\\
打卡方式:加入活动群,使用腾讯文档进行“运动打卡”\\
打卡群、具体打卡方式、奖励等见原文:\url{https://mp.weixin.qq.com/s/HPX6UFKlSoiVdRumUo0OyA}

\section{奋进跑}
活动时间:\\
10月25日——10月27日 \\
周五 21:30-22:10\\
周末 19:30-21:00\\
(如遇雨水天气或其他特殊情况暂停)\\
活动地点:\\
仙林校区炜华体育场\\
手环领取处:\\
仙林校区炜华体育场北侧入口\\
(手环数量限量100个,先到先得哦!)\\
\end{multicols}
\hrule
\vspace{4mm}
\centerline{\huge\textbf{参考消息}}
\begin{multicols}{2}
\section{严复《救亡决论》(一)}
天下理之最明而势所必至者,如今日中国不变法则必亡是已。然则变将何先?曰:莫亟于废八股。夫八股非自能害国也,害在使天下无人才。其使天下无人才奈何?曰:有大害三:\\

其一害曰:锢智慧。今夫生人之计虑智识,其开也,必由粗以入精,由显以至奥,层累阶级,脚踏实地,而后能机虑通达,审辨是非。方其为学也,必无谬悠影响之谈,而后其应事也,始无颠倒支离之患。何则?其所素习者然也。而八股之学大异是。垂髫童子,目未知菽粟之分,其入学也,必先课之以《学》《庸》《语》《孟》,开宗明义,明德新民,讲之既不能通,诵之乃徒强记。如是数年之后,行将执简操觚,学为经义,先生教之以擒挽之死法,弟子资之于剽窃以成章。一文之成,自问不知何语。迨夫观风使至,群然挟兔册,裹饼饵,逐队唱名,俯首就案,不违功令,皆足求售,谬种流传,羌无一是。如是而博一衿矣,则其荣可以夸乡里;又如是而领乡荐矣,则其效可以觊民社。至于成贡士,入词林,则其号愈荣,而自视也亦愈大。出宰百里,入主曹司,珥笔登朝,公卿跬步,以为通天地人之谓儒。经朝廷之宾兴,蒙皇上之亲策,是朝廷固命我为儒也。千万旅进,人皆铩羽,我独成龙,是冥冥中之鬼神,又许我为儒也。夫朝廷鬼神皆以我为儒,是吾真为儒,且真为通天地人之儒。从此天下事来,吾以半部《论语》治之足矣,又何疑哉!又何难哉!做秀才时无不能做之题,做宰相时自无不能做之事,此亦其所素习者然也。谬妄糊涂,其曷足怪?\\

其二害曰:坏心术。揆皇始创为经义之意,其主于愚民与否,吾不敢知。而天下后世所以乐被其愚者,岂不以圣经贤传,无语非祥,八股法行,将以“忠信廉耻”之说渐摩天下,使之胥出一途,而风俗亦将因之以厚乎?而孰知今日之科举,其事效反于所期,有断非前人所及料者。今姑无论试场大弊,如关节、顶替、倩枪、联号,诸寡廉鲜耻之尤,有力之家,每每为之,而未尝稍以为愧也。请第试言其无弊者,则孔子有言:“知之为知之,不知为不知,是知也”,故言止于所不知,固学者之大戒也。而今日八股之士,乃真无所不知。夫无所不知,非人之所能也。顾上既如是求之,下自当以是应之。应之奈何?剿说是已。夫取他人之文词,腆然自命为己出,此其人耻心所存,固已寡矣。苟缘是而侥幸,则他日掠美作伪之事愈忍为之,而不自知其为可耻。然此犹其临场然耳。至其平日用功之顷,则人手一编,号曰揣摩风气。即有一二聪颖子弟,明知时尚之日非,然去取所关,苟欲求售,势必俯就而后可。夫所贵于为士,与国家养士之深心,岂不以矫然自守,各具特立不诡随之风,而后他日登朝,乃有不苟得不苟免之概耶!乃今者,当其做秀才之日,务必使之习为剿窃诡随之事,致令羞恶是非之心,旦暮梏亡,所存濯濯。又何怪委贽通籍之后,以巧宦为宗风,以趋时为秘诀。否塞晦盲,真若一丘之貉。苟利一身而已矣,遑恤民生国计也哉!且其害不止此。每逢春秋两闱,其闱内外所张文告,使不习者观之,未有不欲股弁者。逮亲见其实事,乃不徒大谬不然,抑且变本加厉。此奚翅当士子出身之日,先教以赫赫王言,实等诸济窍飘风,不关人事,又何怪他日者身为官吏,刑在前而不栗,议在后而不惊。何则?凡此又皆所素习者然也。是故今日科举之事,其害不止于锢智慧,坏心术,其势且使国宪王章渐同粪土,而知其害者,果谁也哉?\\

其三害曰:滋游手。扬子云有言:“言,心声也;书,心画也。”故知言语文字二事,系生人必具之能。人不知书,其去禽兽也,仅及半耳。中国以文字一门专属之士,而西国与东洋则所谓四民之众,降而至于妇女走卒之伦,原无不识字知书之人类。且四民并重,从未尝以士为独尊,独我华人,始翘然以知书自异耳。至于西洋理财之家,且谓农工商贾皆能开天地自然之利,自养之外,有以养人,独士枵然,开口待哺。是故士者,固民之蠹也。唯其蠹民,故其选士也,必务精,而最忌广;广则无所事事,而为游手之民,其弊也,为乱为贫为弱。而中国则后车十乘,从者百人,孟子已肇厉阶。至于今日之士,则尚志不闻,素餐等消。十年之间,正恩累举,朝廷既无以相待,士子且无以自存。棫朴丛生,人文盛极。然若以孙文台杀荆州太守坐无所知者例之,则与当涂公卿,皆不容于尧舜之世者也。况夫益之以保举,加之以捐班,决疣溃痈,靡知所届。中国一大豕也,群虱总总,处其奎蹄曲隈,必有一日焉,屠人操刀,具汤沐以相待,至是而始相吊焉,固已晚矣。悲夫!
\end{multicols} 
\end{document}