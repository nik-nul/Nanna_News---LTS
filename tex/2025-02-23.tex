% HEAD BEGIN
\documentclass[letterpaper, 12pt]{article}
\newsavebox\colbbox
\usepackage{graphicx}
\usepackage{multicol}
\usepackage{anysize}
\usepackage{fontspec}
\usepackage[fontset=none]{ctex}
\usepackage{tabularx}
\usepackage{longtable}
\PassOptionsToPackage{hyphens}{url}
\usepackage[breaklinks=true, colorlinks=true]{hyperref}
\expandafter\def\expandafter\UrlBreaks\expandafter{\UrlBreaks\do\a\do\b\do\c\do\d\do\e\do\f\do\g\do\h\do\i\do\j\do\k\do\l\do\m\do\n\do\o\do\p\do\q\do\r\do\s\do\t\do\u\do\v\do\w\do\x\do\y\do\z\do\A\do\B\do\C\do\D\do\E\do\F\do\G\do\H\do\I\do\J\do\K\do\L\do\M\do\N\do\O\do\P\do\Q\do\R\do\S\do\T\do\U\do\V\do\W\do\X\do\Y\do\Z}
% \let\oldurl\url
% \renewcommand{\url}[1]{\begin{sloppypar}\oldurl{#1}\end{sloppypar}}
\setlength\columnsep{30pt}
\marginsize{30pt}{30pt}{10pt}{20pt}
\setmainfont{TeX Gyre Bonum}
\setCJKmainfont[BoldFont=Noto Serif CJK SC Bold, ItalicFont=FandolKai]{Noto Sans CJK SC}
\setlength{\parindent}{0cm}
% \setCJKmonofont{Noto Sans CJK SC}
\newCJKfontfamily\fan{FandolFang}
\begin{document}
\begin{center}
    \Huge\textbf{南哪大专醒前消息}
\end{center}
\vspace{4mm}
\hrule
\renewcommand\tabularxcolumn[1]{m{#1}}
\begin{tabularx}{\textwidth}{>{\hsize.2\hsize}X>{\hsize.6\hsize}X>{\hsize.2\hsize}X}
    \begin{flushleft}
        2025.2.23\, No.173
    \end{flushleft}
    &
    \begin{center}
        \textit{“秉中持正、求新博闻。”}
    \end{center}
    &
    \begin{flushright}
        \textbf{南京市栖霞区}
    \end{flushright}
\end{tabularx}
\vspace{-3.5mm}
\hrule
\vspace{4mm}
% HEAD END
\centerline{\huge\textbf{活动预告}}
\begin{multicols}{2}
    \section{订阅方式和加入编辑部}  
编辑部招聘人才,用爱发电,工作轻松,详情可联系QQ:1329527951 客服小祥\\想订阅本消息或获取PDF版(便于查看超链接和往期),可加QQ群:\href{https://qm.qq.com/q/VXIW7fgsEe}{849644979}.
\section{Deadline Ongoing}
\setbox\colbbox\vbox{
\makeatletter\col@number\@ne
\begin{longtable}{|c|c|c|}
    \hline
    消息(未见ddl的,不刊) & 截止日期 & 刊载日期\\
    \hline\hline
    南大版deepseek & / & 2.22\\
    天文台开放日 & / & 1.6\\
    原创剧本联合孵化报名 & 3.20 & 1.10\\
    njumun代表报名 & 3.2 & 1.16\\
    课程补退选 & 3.2 & 2.19\\
    南大育教新媒体招新 & 2.27 & 2.19\\
    本科生劳育实践 & 7.20 & 2.19\\
    医保零星报销 & 3.31 & 2.19\\
    第二届大学生阅读分享活动 & 3.7 & 2.21\\
    心理中心助理招新 & 2.28 & 2.20\\
    招办全媒体招新 & 3.5 & 2.20\\
    交响乐团招新 & 3.7 & 2.20\\
    歌魅剧务招募 & 2.26 & 2.21\\
    萌马音乐工作室招新 & 2.28 & 2.22\\
    秉文书院早晚自习 & 3.3 & 2.23\\
    图协招新 & 2.28 & 2.23\\
    \hline
\end{longtable}
\unskip
\unpenalty
\unpenalty}\unvbox\colbbox
\end{multicols}
\hrule
\pagebreak
\begin{multicols}{2}

\section{讲座}
\begin{tabular}{|>{\centering\arraybackslash}m{.3\textwidth}|m{.06\textwidth}|m{.06\textwidth}|}
    \hline
    讲座 & 开展时间 & 刊载时间\\
    \hline\hline
    人机协同背景下高等外语教育的守正创新 & 2.27 & 2.17\\\hline
    大陆的起源 & 3.4 & 2.17\\\hline
    年号勘文中所见日本的类书利用 & 2.24 & 2.20\\\hline
    中国中古《孙子算经》在日本的受容 & 2.25 & 2.20\\\hline
    南京“世界文学之都”的前世今生 & 2.27 & 2.20\\\hline
    复杂异构大数据治理与分析关键技术及应用 & 2.25 & 2.20\\\hline
    香港大学经管学院硕士课程高校专场线下宣讲会 & 2.27 & 2.20\\\hline
    电子平带材料中的关联与拓扑 & 2.25 & 2.21\\\hline
    因明与逻辑文化学 & 2.24 & 2.21\\\hline
    2025香港大学暑期课程宣讲会 & 2.26 & 2.21\\\hline
    在哲学与艺术之间 & 2.26 & 2.23\\\hline
\end{tabular}
1.南新读书会\\
时间:2025年2月26日(周三)19:00\\
地点:南京大学新闻传播学院311室\\
《在哲学与艺术之间》\\
分享人:闫炜炜 2024级硕士研究生\\
《Why Do People Sing?》\\
分享人:张青 南京大学新闻传播学院助理研究员\\


\section{秉文书院|“晨读晚修”}
1.晨读|一日之计在于晨\\
时间:3月3日起工作日7:00-10:00\\
地点:新教205\\
形式:来到教室自主诵读古诗文、英语课文或背诵思修知识\\
2.晚修 | 吹灭读书灯,一身皆是月\\
时间:3月3日起工作日18:30-21:20 (每50分钟统一休息10分钟)\\
地点:新教205\\
形式:晚修共有3节晚修课堂供同学们自由选择,课间允许到场或退场,晚修进行时原则上不允许随意进出教室\\
奖励机制\\
本次活动采取积分奖励制,早读每满30分钟积1分,晚修每参加一节完整的自习课堂积1分。期末将为积分榜名列前茅的同学颁发精美礼品!\\
报名表和QQ群见\url{https://mp.weixin.qq.com/s/9i7dc5-AnREHRXmbx7HBQw}

\section{秉文书院|英语四级学业辅导预告}
辅导老师:外国语学院英语专业硕士研究生\\
辅导内容:四节辅导课,涉及四大模块——听力、阅读、翻译和写作\\
具体时间另通知。更多需求和建议可填表格,详见\url{https://mp.weixin.qq.com/s/9i7dc5-AnREHRXmbx7HBQw}

\section{图协招新}
南京大学图书馆学生发展协会(简称“南大图协”)是挂靠于南京大学图书馆的大学生社团组织,由五个部门组成:内务管理部、新闻宣传部、活动策划部、综合联络部、服务维护部。\\
点击链接\url{https://mp.weixin.qq.com/s/guEQfcU89CR9emiY-UeUjw}扫码报名,2月28日截止\\
填写完成后加入QQ群:683487746\\


\section{排超赛程预告 | 2月23日排超季后赛} 
南京广电猫猫VS浙江德清\\
时间:2月23日19:30 周日\\
地点:南京大学方肇周体育馆
\end{multicols} 
\hrule
\vspace{4mm}
\centerline{\huge\textbf{参考消息}}
\begin{multicols}{2}
\section{南哪消息同学小文连载板块}
因收到小说投稿一篇,南哪消息现在开辟了同学小文连载板块。如想评论,可以发至邮箱:1329527951@qq.com,第二天会刊在此处。如想投稿渠道相同。
\section{《等待,遗忘》(2)}
金映樺\\
\fan
\\
  第一次和林望舒见面是在那家味道并不怎么样的F餐厅。除了价格高昂得好看以外一无是处。我匆匆走进餐厅时,她已经在预约好的位置坐下了。\\
  
“抱歉,久等了...”我坐下来,心想早知道就打个车了,公交车来得总是太慢。\\
“没事,是我来得太早,其实还有几分钟呢,”她点了点头,微微笑了笑,“我叫林望舒。”\\
“我是余淮。嗯,不是耿耿于怀的于怀,多余的余,三点水的淮。”她莞尔一笑,只是那笑容依然很浅,怎么说呢,像只够一口的水一样在杯中见底吧。\\

然后是沉默,沉默的尴尬,尴尬的沉默。\\

我灵光一闪, “那个,你念书的时候语文是不是学的不太好啊...”然而刚说出口我就后悔了,这也太失礼了。完了,全完了,搞砸了,明明出门前娘娘还叮嘱过我让我少说两句,至少说之前过过脑子。她果然惊讶地睁大了眼睛(其实也只是微微睁了一下),问我为什么这么说。\\

“因为你的名字呀,就是那个什么,忘书不就是背不下来书么......”我认命般地解释道,心里想着要是她生气了,一下就打道回府,我回去该怎么向娘娘交差。\\

但是她没有。她反而忍不住笑了,解释说,“不是那个忘书,是遥望的望,舒适的舒。”\\

“其实你这样笑比较好看。”我点点头,情不自禁,脱口而出。\\

她愣住了,像是从未想到过我会来这么一出(其实我也没想到),随即掩嘴,但我仍看到流泻出的笑容。不过这不重要,重要的是她笑起来真的很好看。怎么说呢,用我贫瘠的词汇来描述,和高中放学回家路上吹来的风一样美。\\

“谢谢你,”她说,“还没有人这么说过呢。真好,我喜欢她的语气词,上扬的尾音。我想这次赴约好像挺值,就算只当朋友也很好,林望舒看起来真的很像知心好友(我承认滤镜有那么一点重)。但是我知道我想要的不止于此,这是我为数不多可以确定的事情。\\

接下来就挺拘谨,无非就是吃吃那寸粒寸金的奢华午餐,寒暄几句生活日常,上什么学啦,学的什么专业啦,之后的打算啦...我不大敢直接看着她,她好像也不大像多说什么,若有若无地蹙眉,若隐若现的严肃,始终如一的沉默。其实我觉得是因为这菜色实在不入口,比不上娘娘做的酸菜鱼。\\

“你是不是也觉得...下次我们吃...酸菜鱼?”我脱口而出,但好在这次及时止损,没有不客气地点评餐厅,好险,好险。她这次倒没有之前那么诧异,或者说妥帖地收拾好了惊讶,只是轻轻点点头,“有空再约。”然后她问我是不是知道哪家店很不错,我有点不好意思,笑着应付过去了,心里想那娘娘的手艺当然一流,不过这话说出来有点尴尬。她也不在意,浅酌一口配餐里的红酒,那杯口亮晶晶的。我觉得她有心事。\\

我不只一次想过如果我当时直截了当地问她在想什么可不可以说给我听,一切会不会不一样?错在开头,我们不应该这样遇见,不应该这样交谈,不应该这样忽视一切。我唯一能确定的是,我当时没有那么做,于是我再也没法那么做。错过了的便会消失,消失了的便不会再回来。不回来。回来。\\

吃完饭她就走了。那时阳光明媚,明媚得让人有些昏沉,昏沉得让人睁不开眼。她将头发撩在耳后,回头朝我莞尔一笑,就像我刚来时所见的那样。她说下次再见,我说,好。然后她便消失在视线中,我想找到她,却怎么也找不到。\\

备注:我还没有加她微信。

\end{multicols} 

\end{document}