% HEAD BEGIN
\documentclass[letterpaper, 12pt]{article}
\newsavebox\colbbox
\usepackage{graphicx}
\usepackage{multicol}
\usepackage{anysize}
\usepackage{fontspec}
\usepackage[fontset=none]{ctex}
\usepackage{tabularx}
\usepackage{longtable}
\PassOptionsToPackage{hyphens}{url}
\usepackage[breaklinks=true, colorlinks=true]{hyperref}
\expandafter\def\expandafter\UrlBreaks\expandafter{\UrlBreaks\do\a\do\b\do\c\do\d\do\e\do\f\do\g\do\h\do\i\do\j\do\k\do\l\do\m\do\n\do\o\do\p\do\q\do\r\do\s\do\t\do\u\do\v\do\w\do\x\do\y\do\z\do\A\do\B\do\C\do\D\do\E\do\F\do\G\do\H\do\I\do\J\do\K\do\L\do\M\do\N\do\O\do\P\do\Q\do\R\do\S\do\T\do\U\do\V\do\W\do\X\do\Y\do\Z}
% \let\oldurl\url
% \renewcommand{\url}[1]{\begin{sloppypar}\oldurl{#1}\end{sloppypar}}
\setlength\columnsep{30pt}
\marginsize{30pt}{30pt}{10pt}{20pt}
\setmainfont{TeX Gyre Bonum}
\setCJKmainfont[BoldFont=Noto Serif CJK SC Bold, ItalicFont=FandolKai]{Source Han Sans SC}
\setlength{\parindent}{0cm}
% \setCJKmonofont{Noto Sans CJK SC}
\begin{document}
\begin{center}
    \Huge\textbf{南哪大专醒前消息}
\end{center}
\vspace{4mm}
\hrule
\renewcommand\tabularxcolumn[1]{m{#1}}
\begin{tabularx}{\textwidth}{>{\hsize.2\hsize}X>{\hsize.6\hsize}X>{\hsize.2\hsize}X}
    \begin{flushleft}
        2025.3.20\, No.196
    \end{flushleft}
    &
    \begin{center}
        \textit{“秉中持正、求新博闻。”}
    \end{center}
    &
    \begin{flushright}
        \textbf{南京市栖霞区}
    \end{flushright}
\end{tabularx}
\vspace{-3.5mm}
\hrule
\vspace{4mm}
% HEAD END
\centerline{\huge\textbf{活动预告}}
\begin{multicols}{2}
\section{订阅方式和加入编辑部}  
编辑部招聘人才,用爱发电,工作轻松,详情可联系QQ:1329527951 客服小千\\想订阅本消息或获取PDF版(便于查看超链接和往期),可加QQ群:\href{https://qm.qq.com/q/4HL41Nt3sQ}{466863272}.
\section{活动清单}
\setbox\colbbox\vbox{
\makeatletter\col@number\@ne
\begin{longtable}{|>{\centering\arraybackslash}m{.3\textwidth}|m{.06\textwidth}|m{.06\textwidth}|}
    \hline
    活动 & 开展时间 & 刊载时间\\
    \hline\hline
    南大版deepseek & / & 2.22\\
    悦读课程群 & / & 2.24\\
    eScience AI科研助手 & / & 3.11\\
    乐跑 & 5.16 & 3.10\\
    本科生劳育实践 & 7.20 & 2.19\\
    医保零星报销 & 3.31 & 2.19\\
    银星杯论文赛 & 4.22 & 2.27\\
    高教社杯 & 4.25 & 3.5\\
    大创报名 & 3.23 & 3.6\\
    银星杯论文竞赛 & 4.22 & 3.6\\
    南辩院系杯 & 4.12 & 3.6\\
    大文大理题目征集 & 期末 & 3.8\\
    5月免费上网 & ? & 3.9\\
    基础学科论坛 & 4.20 & 3.9\\
    四六级开始报名 & 3.18 & 3.11\\
    普通话测试 & 3.28 & 3.12\\
    外教社杯 & 5.27 & 3.12\\
    心理中心全媒体招新 & 3.25 & 3.14\\
    研会免费证件照拍摄 & 3.21 & 3.15\\
    Python比赛 & 4.6 & 3.16\\
    扎染志愿者招募 & 3.28 & 3.18\\
    心理中心征稿 & 3.31 & 3.18\\
    中美中心开放日 & 3.26 & 3.19\\
    红会成分献血 & 3.22 & 3.19\\
    扎染体验 & 3.23 & 3.20\\
    
    \hline
\end{longtable}
\unskip
\unpenalty
\unpenalty}\unvbox\colbbox
\end{multicols}
\begin{multicols}{2}
\pagebreak

\section{讲座}
\begin{tabular}{|>{\centering\arraybackslash}m{.3\textwidth}|m{.06\textwidth}|m{.06\textwidth}|}
    \hline
    讲座 & 开展时间 & 刊载时间\\
    \hline\hline
    春与死:格非《春尽江南》读书会 & 3.22 & 3.4\\\hline
    陶行知对中国教育现代化问题的探索 & 3.24 & 3.7\\\hline
    Vinaya Revival on Baohua Mountain in Ming–Qing China & 3.25 & 3.18 \\\hline
    文献检索与学术工具使用 & 3.22 & 3.18\\\hline
    中美关系百年史 & 3.23 & 3.19\\\hline
    蛇行变换的气温与生活 & 3.22 & 3.19\\\hline
    文学之都南京的前世今生 & 3.23 & 3.19\\\hline
    考古视角下建康城中轴线的形成与发展 & 3.21 & 3.20\\\hline
    CSMAR数据库的科研应用 & 3.21 & 3.20\\\hline
    如何高效使用GTF数据分析全球货物贸易变化及应用场景 & 3.21 & 3.20\\\hline
    思考他者:跨文化的对话与反思 & 3.21 & 3.20\\\hline
    从数字化、网络化到AI驱动的新趋势 & 3.23 & 3.20\\\hline
    研究兴趣与文献素养 & 3.21 & 3.20\\\hline
    校史微团课培训 & 3.22 & 3.20\\\hline
    Mobility Hubs,Resilience and Bike Station Location:Network and Accessibility Approaches & 3.21 & 3.20\\\hline
    What Can Ecological Spatiotemporal Indicators Tell Us about the Resilience to Economic Crisis & 3.25 & 3.20\\\hline
    Eigenvector Spatial Filtering in Areal and Origin-Destination Data & 3.25 & 3.20\\\hline
    世界睡眠日科普义诊 & 3.21 & 3.20\\\hline
\end{tabular}
%讲座预告写在这
\subsection{讲座详情}
1.考古视角下建康城中轴线的形成与发展\\
主讲人:陈大海 主持人:冷天\\
时间:3月21日(周五)14:00-16:00\\
地点:线下:鼓楼校区费彝民楼A-318 线上:腾讯会议号 924-614-819,会议密码 375600\\

2.人工智能时代下的实证研究新观点--CSMAR数据库的科研应用\\
时间:3月21日(周五)14:00-15:00\\
地点:线下:鼓楼校区商学院安中楼601室 线上:腾讯会议364-248-664\\

3.如何高效使用GTF数据,分析全球货物贸易变化及应用场景\\
时间:3月21日(周五)15:10-16:10\\
地点:线下:鼓楼校区商学院安中楼601室 线上:腾讯会议355-465-295\\

4.思考他者:跨文化的对话与反思\\
主讲人:珍妮·曼德 Jenny Mander(剑桥大学纽纳姆学院教授,全球人类迁移研究中心联合主任) 主持人:何成洲(南京大学外国语学院教授,南京大学全球人文研究院院长)\\
时间:3月21日(周五)18:00-20:00\\
地点:仙林校区外国语学院侨裕楼424会议室\\

5.从数字化、网络化到AI驱动的新趋势(文娱消费案例分享与启示)\\
主讲人:徐宁 江苏省文化投资管理集团有限公司原党委书记、董事长,正高级会计师,南京大学商学院企业管理专业博士\\
时间:3月23日(周日)14:00始\\
地点:鼓楼校区逸夫馆报告厅\\


6.研究兴趣与文献素养:本科生的科研入门课\\
时间:2025年3月21日(周五)18:30\\
地点:鼓楼校区田家炳楼多功能厅\\
主讲人:张秀娟 南京大学现代工程与应用科学学院副教授,博士生导师。主要从事人工微结构材料、拓扑物理、非厄米物理方向的研究,在Nature, Nature Physics,Nature Communications,PhysicaReview Letters,Advanced Science等国际期刊发表成果 30 余篇被引 2000 余次。\\
\url{https://mp.weixin.qq.com/s/UgJqSnHqR14C_8eWzRbnSg}\\

7.校史微团课培训报名\\
活动时间:2025年3月22日10:00-11:00\\
地点:鼓楼校区 逸夫馆Ⅰ-103\\
活动主题:讲好奋进故事,赓续百廿青春\\
主讲人:安邦书院助理辅导员陈石、校史系列精品微团课主讲人唐晨曦、校史文化社社长尹永哲\\
\url{https://mp.weixin.qq.com/s/RK1b_rNn9E5SinZu6PSkqw}\\

8.国际访问学者系列讲座\\
主题:Space-Time Modelling in Mobile Society 
\\(1)时间:3月21日 15:00-16:30
\\主题:Mobility Hubs,Resilience and Bike Station Location:Network and Accessibility Approaches
\\地点:昆山楼B537
\\腾讯会议:487-721-511
\\(2)时间:3月24日 9:00-10:30
\\主题:What Can Ecological Spatiotemporal Indicators Tell Us about the Resilience to Economic Crisis
\\地点:圣达楼321
\\腾讯会议:275-459-290
\\(3)时间:3月25日 10:10-12:00
\\地点:逸夫楼B209
\\主题:Eigenvector Spatial Filtering in Areal and Origin-Destination Data
\\腾讯会议:381-397-799

9.世界睡眠日科普义诊\\
时间:2025年3月21日 9:00-10:00
\\在线直播,扫码观看。
\\详见:\url{https://mp.weixin.qq.com/s/yYTW3EHbNqNhFK2X29ohww}






\section{2025年南京大学博士生讲师团苏州校区分团招新通知}
苏州分团由讲师团队和运营团队两部分组成。团长统筹协调分团工作,副团长协调讲师团队工作;运营团队含秘书部、活动部、宣传部三个部门。
\\苏州分团2025年计划招募团长一名、副团长一名,秘书部、活动部、宣传部部长各一名,硕士研究生、博士研究生均可报名。
\\招募对象:苏州校区在校硕士研究生、博士研究生
\\报名时间:2025年3月20日至4月2日
\\申报者需填写《2025年南京大学博士生讲师团苏州校区分团招新报名表》,于2025年4月2日前将该表格以附件形式发送至邮箱chengye08@nju.edu.cn
\\
\\详见:\url{https://mp.weixin.qq.com/s/G_fE3bdRt3-z3YCq_8ipYQ}
\section{扎染体验}
活动时间地点\\
仙林校区:敬文学生活动中心107活动房\\
3月23日 10:00-12:00\\
鼓楼校区:鼓楼南青格庐多功能教室\\
3月23日 15:00—17:00\\
报名链接请点击\url{https://mp.weixin.qq.com/s/0IFgms6zD0KKC4HWUIHPQg}\\

\section{院级活动}
\begin{tabular}{|>{\centering\arraybackslash}m{.3\textwidth}|m{.06\textwidth}|m{.06\textwidth}|}
\hline
    活动 & 开展时间 & 刊载时间\\
    \hline\hline
    文院剧本创作研讨会 & 9.30 & 3.2\\
    物院征集课程指南 & 6.15 & 3.3\\
    信地海环四院羽球赛 & 3.23 & 3.10\\
    电子学院淘宝简历面试指导 & 3.21 & 3.10\\
    电子学院腾讯简历面试指导 & 3.24 & 3.10\\
    地海征集春日影 & 6.15 & 3.14\\
    计院定向越野 & 3.22 & / \\
    智软手作集市 & 3.20 & 3.16\\
    秉文猫鼠游戏 & 3.23 & 3.18\\
    AI院影色舞 & 3.29 & 3.19\\
    商院羽球 & 3.29 & 3.19\\
    \hline
\end{tabular}
%这里是写院级活动的,院级活动就是只限某院学生参加的活动,和由某院某部门主办、主要针对某院学生的活动。不要把对全校学生开放的活动写在这里。
\subsection{秉文书院|第一期英语学业支持}
时间:3月22日(周六)下午14:30-15:30\\
地点:鼓楼校区北园教学楼101\\
活动内容:\\
· 英语四级备考及学习方法总览\\
· 英语四级写作专题讲解\\
· 现场练习与交流指导\\
\subsection{行知书院|经管大类分流朋辈咨询会\&新生“学科专业节”}
时间:2025年3月23日(周日)9:00-17:00\\
主会场:安中楼210(鸿意报告厅)\\
分会场(按系所):\\
安中楼402 经济产经专场\\
安中楼312 国贸金融专场\\
安中楼B03 工管人资专场\\
安中楼109 会计营电专场\\
具体内容见\url{https://mp.weixin.qq.com/s/ZXBv44FXaCzdViQLZQ15JA}
\subsection{智科每周实习速递(四)}
1.阿里巴巴淘天集团
\\2.联想
\\3.德宁资本
\\4.招商银行
\\5.华泰证券
\\6.南方基金
\\7.龙湖集团
\\更多信息关注公众号内容
\\详见:\url{https://mp.weixin.qq.com/s/N6oaRzT_LvHRJCrg_4suJA}
\section{社团活动}
\begin{tabular}{|>{\centering\arraybackslash}m{.3\textwidth}|m{.06\textwidth}|m{.06\textwidth}|}
    \hline
    社团活动 & 开展时间 & 刊载时间\\
    \hline\hline
    天文台开放日 & / & 1.6\\
    相声社春季专场 & 3.22 & 3.17\\
    鸿新社捐书活动 & 3.30 & 3.17\\
    CAC观影 & 3.22 & 3.17\\
    心协卡牌招募 & 3.22 & 3.17\\
    知行古案今判 & 3.23 & 3.19\\
    心协流光影院 & 3.22 & 3.19\\
    长歌行声演剧 & 3.29 & 3.19\\
    乒协抽奖 & 3.24 & 3 19\\
    \hline
\end{tabular}
%这里是写社团活动的,社团活动就是由社团主办、主要针对社团内部人员的活动。不要把非社团活动写在这里。
\subsection{2025年南大招协新一届各地区招生宣传理事长选拔通知}
1.本次仅面向各年级本科生,统一选拔各地区理事长,不包含中学理事、招生志愿者。这里的“地区”,江苏省内以地级市为单位,江苏省外以省(直辖市、自治区)为单位。
\\2. 本届理事长的任期自2025年4月起,至2026 年4月结束。期间,需要密切联络本地区同学,服务地区招生组,协助组织学生志愿者,参与到春季招生宣传、暑期招生咨询、秋季新生见面会、寒假“南星梦想计划”等活动中,完成招生组、招协交办的任务。
\\3. 作为招协(校十佳社团、五星级社团)正式成员,任期结束后由本科招生办公室颁发工作证明,有机会评选(或被推荐评选)优秀理事长、优秀学生社团骨干等荣誉,有机会通过选拔成为招协管理层成员,收获一段充实的团学工作经历。协助组织“南星梦想计划”等大型志愿活动,拥有参与志愿服务的丰富机会,收获与实际服务量相匹配的志愿时长。
\\4. 报名方式
\\1)自主报名点击“阅读原文”下载报名表,填写完成后扫描文中二维码。
\\2)参加面试
\\预计将于3月28日至30日之间组织面试,具体时间、形式待定,与各招生组商定确认后另行通知。
\\3)公布录取名单
\\预计4月2日左右公布录取名单。
\\b
\\详见:\url{https://mp.weixin.qq.com/s/dE1Dg4A6dgIFllJjCBSccg}

\subsection{南京大学2025“院系杯”网球团体赛}
赛事时间:2025年3月30日9:00
\\(8:45到场,结束时间视赛程定)
\\赛事地点:南京大学仙林校区网球场
\\参加对象:南京大学全体本科生、研究生、教职工
\\报名办法:以院系为单位报名,可多院联队
\\注:多院联队中至少含有两名同院系球员
\\(大一以书院涵盖的专业院系为依据自行选择队伍,例行知书院球员可加入商学院、法学院、政管院、信管院、社院的队伍参赛)
\\详见:\url{https://mp.weixin.qq.com/s/zmVzZq680Vv_mCSNUjAr6Q}
\end{multicols}
\end{document}
