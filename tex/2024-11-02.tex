% HEAD BEGIN
\documentclass[letterpaper, 12pt]{article}
\newsavebox\colbbox
\usepackage{graphicx}
\usepackage{multicol}
\usepackage{anysize}
\usepackage{fontspec}
\usepackage[fontset=none]{ctex}
\usepackage{tabularx}
\usepackage{longtable}
\PassOptionsToPackage{hyphens}{url}
\usepackage[breaklinks=true, colorlinks=true]{hyperref}
\expandafter\def\expandafter\UrlBreaks\expandafter{\UrlBreaks\do\a\do\b\do\c\do\d\do\e\do\f\do\g\do\h\do\i\do\j\do\k\do\l\do\m\do\n\do\o\do\p\do\q\do\r\do\s\do\t\do\u\do\v\do\w\do\x\do\y\do\z\do\A\do\B\do\C\do\D\do\E\do\F\do\G\do\H\do\I\do\J\do\K\do\L\do\M\do\N\do\O\do\P\do\Q\do\R\do\S\do\T\do\U\do\V\do\W\do\X\do\Y\do\Z}
% \let\oldurl\url
% \renewcommand{\url}[1]{\begin{sloppypar}\oldurl{#1}\end{sloppypar}}
\setlength\columnsep{30pt}
\marginsize{30pt}{30pt}{10pt}{20pt}
\setmainfont{TeX Gyre Bonum}
\setCJKmainfont[BoldFont=Noto Serif CJK SC Bold, ItalicFont=FandolKai]{Noto Sans CJK SC}
\setlength{\parindent}{0cm}
% \setCJKmonofont{Noto Sans CJK SC}
\begin{document}
\begin{center}
    \Huge\textbf{南哪大专醒前消息}
\end{center}
\vspace{4mm}
\hrule
\renewcommand\tabularxcolumn[1]{m{#1}}
\begin{tabularx}{\textwidth}{>{\hsize.2\hsize}X>{\hsize.6\hsize}X>{\hsize.2\hsize}X}
    \begin{flushleft}
        2024.11.2\, No.106
    \end{flushleft}
    &
    \begin{center}
        \textit{“既有令名、复求寿考、可兼得乎。”}
    \end{center}
    &
    \begin{flushright}
        \textbf{南京市栖霞区}
    \end{flushright}
\end{tabularx}
\vspace{-3.5mm}
\hrule
\vspace{4mm}
% HEAD END
\centerline{\huge\textbf{活动预告}}
\begin{multicols}{2}
    \section{订阅方式和加入编辑部}  
编辑部招聘人才,用爱发电,工作轻松,详情可联系QQ:1329527951 客服小祥\\想订阅本消息或获取PDF版(便于查看超链接和往期),可加QQ群:\href{https://qm.qq.com/q/VXIW7fgsEe}{849644979}.
\section{Deadline Ongoing}
\setbox\colbbox\vbox{
\makeatletter\col@number\@ne
\begin{longtable}{|c|c|c|}
    \hline
    消息(未见ddl的,不刊) & 截止日期 & 刊载日期\\
    \hline\hline
    紫藤学刊征稿 & 12.15 & 10.22\\
    学位英语考试报名 & 11.3 & 10.17\\
    校运会 & 11.8 & 10.21\\
    后革命鲁迅研究征文 & 11.10 & 10.8\\
    大创训练计划申报 & 11.18 & 9.24\\
    招生宣传创意征集大赛 & 11.18 & 10.21\\ 
    EBSCO数据库检索大赛 & 11.20 & 10.3\\
    文院征稿 & 11.20 & 10.20\\
    乐跑 & 12.6 & 10.12\\
    国际访学计划申报 & 11.22 & 10.22\\
    仙林草地音乐节 & 11.3 & 10.27\\
    普通话测试网络报名 & 11.12 & 10.29\\
    全球学习交流展 & 11.4 & 10.29\\
    健雄捡秋活动 & 11.3 & 10.29\\
    南大演说家 & 11.9 & 10.30\\
    导游志愿者招募 & 11.3 & 10.30\\
    南大演说家报名 & 11.9 & 10.30\\
    腾讯线下观影 & 11.3 & 10.30\\
    杜厦剧本杀 & 11.3 & 10.30\\
    秉文朋导分享会 & 11.3 & 10.31\\
    物院飞盘活动 & 11.3 & 10.31\\
    心协乐跑 & 11.3 & 10.31\\
    读书午餐会报名 & 11.6 & 11.1\\
    南大会征募会设 & 11.15 & 11.1\\
    新生午餐会报名 & 11.4 & 11.1\\
    马主义学术研讨会 & 11.3 & 11.2\\
    心协十一月征稿 & 11.10 & 11.2\\
    \hline
\end{longtable}
\unskip
\unpenalty
\unpenalty}\unvbox\colbbox
\end{multicols}
\hrule
\pagebreak
\begin{multicols}{2}

\section{讲座}
\begin{tabular}{|c|c|c|}
    \hline
    往期讲座 & 开展日期 & 刊载日期\\
    \hline\hline
    《电池及电化学能...》 & 11.24 & 10.3\\
    《专利查新与规避...》 & 12.19 & 10.3\\
    图书馆系列讲座 & 12.3 & 10.20\\
    《志工人力资源的...》 & 11.4 & 10.23\\
    《华人社会工作的...》 & 11.4 & 10.23\\
    《从全球视角探讨...》 & 11.4 & 10.28\\
    《瑞典电力和氢能...》 & 11.7 & 10.29\\
    《组织动员如何影...》 & 11.6 & 10.30\\
    《信息与现代信息...》 & 11.6 & 10.31\\
    《卢卡奇1919与19...》 & 11.8 & 11.2\\
    《青年卢卡奇论马...》 & 11.8 & 11.2\\
    \hline
\end{tabular}

1.卢卡奇1919与1923年的历史唯物主义研究所计划\\
主讲人:Rüdiger Dannemann(国际卢卡奇协会主席)\\
时间:11月8日9:00\\
地点:哲学学院402教室\\
摘要:基于对卢卡奇《历史与阶级意识》中的“历史唯物主义的功能变化”一文的深入考察,并与1931年霍克海默就任法兰克福社会研究所主任的就职演说“社会哲学的现状与社会研究所的任务”进行比较分析,以此展现卢卡奇在1919年至1923年间创设历史唯物主义研究所的构想,梳理卢卡奇对马克思主义认识的深化。\\

2.青年卢卡奇论马克思的物化概念\\
主讲人:Konstantinos Kavoulakos(国际卢卡奇协会副主席)\\
时间:11月8日10:30\\
地点:哲学学院402教室\\
摘要:基于卢卡奇的《历史与阶级意识》文本,重新发掘卢卡奇的“物化”概念,以此厘清“物化”概念的理论来源,呈现卢卡奇批判资本主义的理论逻辑,并把握“物化”概念的当代价值。\\


\section{马克思主义学术研讨会}
“马克思主义与中国式现代化——第四届马克思主义文献典藏与研究国际学术研讨会”将于11月3日在国际会议中心举行,9:00-12:00的地点为会议中心紫金厅,14:00-18:00的地点为思学厅、恒学厅、励学厅。会议议程本报不得而知。
\section{心协11月征稿}
10月份举行了第一期征稿,此为11月-第二期。\\
征稿主题:君子淡如水,岁久情愈真\\
①你与TA的初次相遇是怎样的情景呢?是日暮大道,江畔苇草还是都市喧嚣?请用一张图片定格那一刻的美丽瞬间,并分享属于你们的故事。\\
②记录下在人际交往中那些触动心灵、经住时间考验的美好回忆,让我们一同感受那份纯粹而深刻的情感,如何随着时间的推移变得愈发珍贵和难忘。\\
截止时间:2024年11月10日19:00\\
投稿方式:扫码填写问卷(见原文)\\
投稿经筛选后将在公众号发出,通过投票获得前十名的同学可获得神秘礼品一份,所有通过筛选的投稿者都将获得作品印制的定制明信片一张。\\
原文:\url{https://mp.weixin.qq.com/s/p9K-iOz9NFT1h14-H_5zBQ}


\end{multicols} 
\end{document}