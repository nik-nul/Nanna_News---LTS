% HEAD BEGIN
\documentclass[letterpaper, 12pt]{article}
\usepackage{graphicx}
\usepackage{multicol}
\usepackage{anysize}
\usepackage{fontspec}
\usepackage[fontset=none]{ctex}
\usepackage{tabularx}
\PassOptionsToPackage{hyphens}{url}
\usepackage[breaklinks=true, colorlinks=true]{hyperref}
\expandafter\def\expandafter\UrlBreaks\expandafter{\UrlBreaks\do\a\do\b\do\c\do\d\do\e\do\f\do\g\do\h\do\i\do\j\do\k\do\l\do\m\do\n\do\o\do\p\do\q\do\r\do\s\do\t\do\u\do\v\do\w\do\x\do\y\do\z\do\A\do\B\do\C\do\D\do\E\do\F\do\G\do\H\do\I\do\J\do\K\do\L\do\M\do\N\do\O\do\P\do\Q\do\R\do\S\do\T\do\U\do\V\do\W\do\X\do\Y\do\Z}
% \let\oldurl\url
% \renewcommand{\url}[1]{\begin{sloppypar}\oldurl{#1}\end{sloppypar}}
\setlength\columnsep{30pt}
\marginsize{30pt}{30pt}{10pt}{20pt}
\setmainfont{TeX Gyre Bonum}
\setCJKmainfont[BoldFont=Noto Serif CJK SC Bold, ItalicFont=FandolKai]{Noto Sans CJK SC}
\setlength{\parindent}{0cm}
% \setCJKmonofont{Noto Sans CJK SC}
\begin{document}
\begin{center}
    \Huge\textbf{南哪大专醒前消息}
\end{center}
\vspace{4mm}
\hrule
\renewcommand\tabularxcolumn[1]{m{#1}}
\begin{tabularx}{\textwidth}{>{\hsize.2\hsize}X>{\hsize.6\hsize}X>{\hsize.2\hsize}X}
    \begin{flushleft}
        2024.9.25\, No.71
    \end{flushleft}
    &
    \begin{center}
        \textit{“克明峻德。”}
    \end{center}
    &
    \begin{flushright}
        \textbf{南京市栖霞区}
    \end{flushright}
\end{tabularx}
\vspace{-3.5mm}
\hrule
\vspace{4mm}
% HEAD END
\centerline{\huge\textbf{活动预告}}
\begin{multicols}{2}
\section{编辑部招聘人才}
编辑部招聘人才,用爱发电,工作轻松,详情可联系QQ:1329527951 客服小祥\\想订阅本消息或获取PDF版(便于查看超链接),可加QQ群:\href{https://qm.qq.com/q/FGX1VYCrGS}{962626571}.
\section{Deadline Ongoing}
\begin{tabular}{|c|c|c|}
    \hline
    消息(未见ddl的,不刊) & 截止日期 & 刊载日期\\
    \hline\hline
    仙林校史馆招募讲解员 & 10.30 & 9.12\\
    管道宣传志愿团队遴选 & 9.27 & 9.12\\
    国优计划报名 & 10.7 & 9.19\\
    本科生暑期课程评教 & 10.31 & 9.19\\
    网易雷火大赛 & 10.7 & 9.22\\
    蓝鲸之材产融活动 & 9.26 & 9.22\\
    走近华为报名 & 9.26 & 9.23\\
    部分课程增加名额 & 9.27 & 9.24\\
    大创训练计划申报 & 9.27 & 9.24\\
    历史学院宣传技能培训 & 9.28 & 9.24\\
    苏州校区音乐会 & 10.19 & 9.25\\
    外院国庆摄影征集 & 10.7 & 9.25\\
    历史学院新疆项目 & 9.30 & 9.25\\
    生涯力提升课报名 & 9.27 & 9.25\\
    Flicker周常影映 & 9.28 & 9.25\\
    II剧剧场讨论 & 9.27 & 9.25\\

    
    \hline
\end{tabular}

\section{讲座}

\begin{tabular}{|c|c|c|}
    \hline
    往期讲座 & 开展日期 & 刊载日期\\
    \hline\hline
    《虚实相生读杜诗》 & 9.26 & 9.25\\
    《作为批评的文学史》 & 9.29 & 9.25\\
    \hline
\end{tabular}\\\\
1.《虚实相生读杜诗》
主讲人:刘奕 上海大学文学院教授\\
与谈人:徐涛 南京大学文学院副教授\\
时间:2024年9月26日(周四)19:00-21:00\\
地点:南京大学仙林校区 仙II-104教室\\
详见:\url{https://mp.weixin.qq.com/s/xSuZVqheTbpLim52_oWLYw}\\
\section{当代剧场讨论 |《海鸥》}
II剧现通知演出事宜。

9月27日周五18:30-22:00,在文学院441,开展Yury Butusov导演《海鸥》当代剧场讨论。
\section{南播玩招新}
南播玩工作室郑钢基金工作室(以下简称南播玩),隶属于南京南播玩文化传播有限公司,是立足于南京大学起步发展、并不断推广的公益性、服务性高校校园直播平台。

根据工资标准每月为各层级员工开具劳务;根据人事考核标准及劳务标准,给予阶段表现较为突出的员工额外奖励;为每一位合格员工颁发实习证明。

招新部门包括新媒体部、推广部、直播部、公益部、企管部、综合部、视频部。详见:\url{https://mp.weixin.qq.com/s/EfyNFkUFst8F1QG_IygHeg}
\section{Flicker影映社周常放映}
Flicker影映社将于9月28日18:30在鼓楼校区费彝民楼A栋410教室放映电影《霸王别姬》,无需报名,可直接到现场参与观影。
\section{仙林校区杜厦图书馆探索指南}
南大育教发布杜厦图书馆探索指南,内含区域介绍、借阅方法、开放时间等信息,详见链接\url{https://mp.weixin.qq.com/s/Yb_QtsZ6WQsj5DLgvAk5XQ}
\section{南京大学苏州校区开唱音乐会}
N/S乐队成立于2020年6月,由南京大学校友总会副秘书长、南京大学上海校友会副会长兼秘书长邱继良发起,联合南京大学、北京大学、中山大学、中国美术学院多所高校的校友组成。N/S乐队将带来18首经典单曲,用音乐讲述故事,让歌声传递情感。\\
时间:2024年10月19日(周六)晚7:30\\
地点:南京大学苏州校区运动场\\
具体情况可参考:\url{https://mp.weixin.qq.com/s/kosmihEouZcuySsr6zCNLA}
(注意此文面向毕业生,但没找到其它信源)
\section{国庆摄影征集}
为庆祝中华人民共和国成立75周年,外国语学院学生会文体部与新媒体工作部联合开展“国庆华章”摄影征集项目。\\
征集主题:邀请同学们用镜头记录下国庆节这个特殊时刻的美好瞬间\\
主题类别:壮丽山河;城乡风光;校园风采\\
活动流程:\\
1.9月26日至10月7日,面向全院师生(含24级秉文书院外语大类新生)征集与“国庆”有关的照片。\\
2.外国语学院团委新媒体中心将会同学生会文体部对提交的作品进行评选,并于国庆假期后在外新社公众号对获奖作品进行展示。\\
3.获奖作品拍摄者可获得本院精美周边一份。包括:\\
一等奖2名 | 奖品:北大楼建筑摆件\\
二等奖6名 | 奖品:“蓝鲸与猫”马克杯/梳妆镜、口红套装\\
三等奖若干名 | 奖品:陶瓷杯垫\\
提交方法和格式见原文:\url{https://mp.weixin.qq.com/s/p-iltVUsLbZt2NMSEu1ghA}

\section{生涯胜任力提升体验式公开课第二期报名}
【时间】9月27日(周五)14:00-16:00\\
【地点】南京大学仙林校区学生就业指导中心生涯规划体验馆\\
【报名方式】扫描附录中的二维码,填写报名问卷,课程名额有限(50人),先到先得。\\
报名截止时间9月27日(周五)10:00。\\
\section{历史学院新疆项目招新启事}
历史学院拟开展社会实践活动,现面向全校学生招募:

课程教学组:

1.工作态度端正细致,有较强的沟通联络、课件制作和文字写作能力,具有服务意识和奉献精神。

2.有党团课竞赛、各级各类讲师团经验者优先。

3.报名需上传宣讲课件或教学设计。\\
策展产出组:

1.具备文字功底,拥有一定历史地理或边疆考古的知识背景。

2.擅长文创设计、展板制作、视频剪辑、小程序搭建者优先,需上传个人作品。(条件1、2符合其一即可)\\
口述访谈组1.工作态度端正细致,有较强的文字写作能力,具有责任意识。2.有实践调研、人物访谈、宣传写作经验者优先,报名需上传个人作品。\\
详见:\url{https://mp.weixin.qq.com/s/hYYSYxqsu--SEdVVVqUBLg}

截止时间:9月30日17:00前填写报名表

\section{高研院“作为批评的文学史”工作坊}
主题:“作为批评的文学史”工作坊\\
时间:2024年9月29日(周日)下午13:30-17:30\\
地点:南京大学鼓楼校区逸夫馆9楼高研院报告厅\\

\end{multicols} 

\hrule
\vspace{4mm}
% APPENDIX BEGIN
\centerline{\huge\textbf{附录}}
\begin{figure}[htbp]
    \centering
    \begin{minipage}[b]{0.32\textwidth}
        \centering
         \includegraphics[width=0.5\textwidth]{640.png}
        \caption{生涯胜任力课程报名}
    \end{minipage}
\end{figure}
\end{document}