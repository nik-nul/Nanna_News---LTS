% HEAD BEGIN
\documentclass[letterpaper, 12pt]{article}
\newsavebox\colbbox
\usepackage{graphicx}
\usepackage{multicol}
\usepackage{anysize}
\usepackage{fontspec}
\usepackage[fontset=none]{ctex}
\usepackage{tabularx}
\usepackage{longtable}
\PassOptionsToPackage{hyphens}{url}
\usepackage[breaklinks=true, colorlinks=true]{hyperref}
\expandafter\def\expandafter\UrlBreaks\expandafter{\UrlBreaks\do\a\do\b\do\c\do\d\do\e\do\f\do\g\do\h\do\i\do\j\do\k\do\l\do\m\do\n\do\o\do\p\do\q\do\r\do\s\do\t\do\u\do\v\do\w\do\x\do\y\do\z\do\A\do\B\do\C\do\D\do\E\do\F\do\G\do\H\do\I\do\J\do\K\do\L\do\M\do\N\do\O\do\P\do\Q\do\R\do\S\do\T\do\U\do\V\do\W\do\X\do\Y\do\Z}
% \let\oldurl\url
% \renewcommand{\url}[1]{\begin{sloppypar}\oldurl{#1}\end{sloppypar}}
\setlength\columnsep{30pt}
\marginsize{30pt}{30pt}{10pt}{20pt}
\setmainfont{TeX Gyre Bonum}
\setCJKmainfont[BoldFont=Noto Serif CJK SC Bold, ItalicFont=FandolKai]{Noto Sans CJK SC}
\setlength{\parindent}{0cm}
% \setCJKmonofont{Noto Sans CJK SC}
\begin{document}
\begin{center}
    \Huge\textbf{南哪大专醒前消息}
\end{center}
\vspace{4mm}
\hrule
\renewcommand\tabularxcolumn[1]{m{#1}}
\begin{tabularx}{\textwidth}{>{\hsize.2\hsize}X>{\hsize.6\hsize}X>{\hsize.2\hsize}X}
    \begin{flushleft}
        2024.10.22\, No.95
    \end{flushleft}
    &
    \begin{center}
        \textit{“Keep watch with prayer, \\so that you may not be put to the test.”}
    \end{center}
    &
    \begin{flushright}
        \textbf{南京市栖霞区}
    \end{flushright}
\end{tabularx}
\vspace{-3.5mm}
\hrule
\vspace{4mm}
% HEAD END
\centerline{\huge\textbf{活动预告}}
\begin{multicols}{2}
    \section{订阅方式和加入编辑部}  
编辑部招聘人才,用爱发电,工作轻松,详情可联系QQ:1329527951 客服小祥\\想订阅本消息或获取PDF版(便于查看超链接和往期),可加QQ群:\href{https://qm.qq.com/q/VXIW7fgsEe}{849644979}.
\section{Deadline Ongoing}
\setbox\colbbox\vbox{
\makeatletter\col@number\@ne
\begin{longtable}{|c|c|c|}
    \hline
    消息(未见ddl的,不刊) & 截止日期 & 刊载日期\\
    \hline\hline
    历史学院羽毛球赛 & 10.26 & 10.22\\
    紫藤学刊征稿 & 12.15 & 10.22\\
    南新读书会 & 10.23 & 10.20\\
    歌魅放映会 & 10.23 & 10.20\\
    急救培训活动报名 & 10.24 & 10.17\\
    计院迎新晚会征集节目 & 10.25 & 10.12\\
    马院主题宣讲报名 & 10.25 & 10.5\\
    织围巾志愿者招募 & 10.26 & 10.20\\
    心协DIY活动 & 10.26 & 10.20\\
    心协流光影院 & 10.26 & 10.17\\
    校园今日说法大赛 & 10.26 & 10.17\\
    遵义精神宣讲团遴选 & 10.27 & 10.10\\
    青鸟剧场新戏招募 & 10.27 & 10.14\\
    体测 & 10.27 & 10.16\\
    鼓楼音乐跑 & 10.27 & 10.20\\
    普通话考试报名 & 10.28 & 10.14\\
    仙林校史馆招募讲解员 & 10.30 & 9.12\\
    本科生暑期课程评教 & 10.31 & 9.19\\
    黑匣招募 & 11.1 & 10.19\\
    学位英语考试报名 & 11.3 & 10.17\\
    校运会 & 11.8 & 10.21\\
    后革命鲁迅研究征文 & 11.10 & 10.8\\
    大创训练计划申报 & 11.18 & 9.24\\
    招生宣传创意征集大赛 & 11.18 & 10.21\\ 
    EBSCO数据库检索大赛 & 11.20 & 10.3\\
    文院征稿 & 11.20 & 10.20\\
    乐跑 & 12.8 & 10.12\\
    大创课题成员招募 & 10.24 & 10.22\\
    国际访学计划申报 & 11.22 & 10.22\\
    校园涂鸦快闪 & 10.24 & 10.22\\
    毓琇书院宿舍评比 & 10.31 & 10.22\\
    物院原子弹爆炸活动 & 10.24 & 10.22\\
    
    \hline
\end{longtable}
\unskip
\unpenalty
\unpenalty}\unvbox\colbbox
\end{multicols}
\hrule
\pagebreak
\begin{multicols}{2}

\section{讲座}
\begin{tabular}{|c|c|c|}
    \hline
    往期讲座 & 开展日期 & 刊载日期\\
    \hline\hline
        《聚合物的研发与...》 & 10.24 & 10.3\\
    《电池及电化学能...》 & 11.24 & 10.3\\
    《专利查新与规避...》 & 12.19 & 10.3\\
    《中国法律形象西...》 & 10.23 & 10.16\\
    《与<自然>编辑对...》 & 10.30 & 10.16\\
    《语言能力与前近...》 & 10.25 & 10.18\\
    《物理信息神经网...》 & 10.23 & 10.18\\
    《国家图书馆的古...》 & 10.24 & 10.18\\
    《HKMW与当代德国...》 & 10.23 & 10.20\\
    图书馆系列讲座 & 12.3 & 10.20\\
    《则天文字在日本...》 & 10.23 & 10.20\\
    《大众视角与历史...》 & 10.25 & 10.21\\
    《中国电影史中的...》 & 10.23 & 10.21\\
    《近代中国女性史...》 & 10.24 & 10.22\\
    《量子非互易性》 & 10.24 & 10.22\\
    《简谈Hopf代数..》 & 10.23 & 10.22\\
    《What happens on...》 & 10.23 & 10.22\\
    \hline
\end{tabular}

1、近代中国女性史研究\\
主讲人:游鉴明,台湾政治大学历史系学士\\
主持人:马俊亚\\
时间:2024.10.24(周四)14:30-16:30\\
地点:历史学院233会议室\\
2.物理学院学术报告会(第38期)\\
题   目:量子非互易性:新认知,新应用\\
报告人:景辉,湖南师范大学\\
时   间:2024年10月24日(周四)15:30\\
地   点:鼓楼校区唐仲英楼B501\\
\url{https://mp.weixin.qq.com/s/O3Piy93mCZhTOEU59gVGFw}\\
3.数学学院本科生论坛(学生系列第49讲)
题目: 简谈Hopf代数与张量范畴\\
报告人:李博文 (21级博)\\
时间:10月23日(星期三)16:00-17:30\\
地点:戊己庚四楼北\\
腾讯会议:870-7007-3326\\
\url{https://mp.weixin.qq.com/s/g92dNnVIJqF_f3j1UrCcFA}\\
4.数学学院青年学者论坛\\
题目:What happens on the limit of a sequence of models of ZFC\\
主讲人:游志兴\\
时间:10月23日(周三)下午4:30-5:30\\
地点:西大楼108报告厅\\
腾讯会议:790-952-969(密码:8392)\\
\url{https://mp.weixin.qq.com/s/-ehjyv7ai0Kd37Cn-2kARA}\\
5.志愿服务大讲堂进书院(文科专场)\\
时间:10月25日(周五)16:00-18:00\\
地点:线下-鼓楼校区南青格庐,线上-腾讯会议591-9065-0033\\
内容:解读志愿服务相关问题、志愿者经验分享\\
线下报名链接\url{
https://table.nju.edu.cn/dtable/forms/d4c92fbc-d123-411a-aa52-7af185a432e9/}\\
\section{历史学院羽毛球友谊赛}
时间:10月26日\\下午 14:00 — 16:00\\地点:方肇周体育馆羽毛球场\\对象:南京大学历史学院师生\\
分友谊赛和趣味赛两种,详见\url{https://mp.weixin.qq.com/s/OR2TXTv0rUSkGbefIkvBMw}\\
报名链接\url{https://www.wjx.cn/vm/rCdPWnU.aspx}
\section{南京大学《紫藤学刊》第19期征稿}
2024年10月15日起,第19期《紫藤学刊》正式面向校内外师生征稿,征稿须知如下:\\
征稿要求:1.本刊是历史学领域的综合性学术性刊物,稿件内容可涵盖中国史、世界史、专门史、国际关系史、外交史、考古学、文物与博物馆学等领域。\\
2.稿件不限题材,包括但不局限于论文、随笔札记、读书报告、大学生活相关的学术性文章。激励本科生在他们的学习过程中,积极捕捉那些瞬间的灵感,并将其转化为简短而有力的学术笔记。这些笔记应该专注于一至两个学术问题,只要与历史学相关,任何有可读性、有价值的内容都是受欢迎的。\\
3.体例格式请参照《历史研究》注释规范,篇幅遵循国际通例,文字重复率最高不得超过15\%,字数控制在2.5万字以内(含脚注)为宜,特别优秀的稿件在字数上可酌情放宽。所投稿件必须是作者独立研究完成的原创性作品,对他人的知识产权有充分的尊重。\\
详见\url{https://mp.weixin.qq.com/s/lue3z6Mf6R1DTFlwnB5L6w}\\
征稿截止日期:2024年12月15日。请于截止日期前将回执、作品文稿投送至紫藤学社公共邮箱zitengxueshe@126.com。
\section{大学生创新训练项目导师课题成员招募第二、第三期}
1、项目导师-张正堂教授\\
   课题名称:我国高校“非升即走”制度改革实施的成效、问题及优化;\\
   成员要求:对科研感兴趣,有团队精神,愿意投入时间和精力,具有较强学习能力的同学。部分名额以人力资源管理、应用心理学、教育学专业同学优先。\\
2、项目导师-冯欣副教授\\
   课题名称:信息经济学及其应用;\\
   成员要求:经济学基础或数学基础或计算机基础扎实者优先\\
3、项目导师-龙郑华副教授\\
   课题名称:ESG评分与企业风险管理能力的关系\\
   成员要求:两位本科生\\
4、项目导师-杨念副教授\\
   课题名称:机器学习方法在金融工程中的应用\\
   成员要求:有较好的金融、数学(概率统计)、计算机编程的基础\\
5、项目导师-孔祥年\\
   课题名称:强制环保审查,ESG绩效与资本配置\\
   成员要求:无明确要求\\
6、项目导师-贺伟教授\\
   课题名称:基于机器学习的个体情商评估模型研究\\
   成员要求:3名左右。1.本研究为管理学、心理学、人工智能交叉课题,团队成员需具备机器学习知识及能力,或具备心理学、人力资源管理学专业知识。2.专业要求:计算机、人工智能、人力资源管理、心理学相关专业。3.对该研究内容感兴趣,具有高度的内驱力以及目标感。4.具备熟练且独立的的英文文献搜索、阅读、整理及分析能力。5.具备开展实验研究、问卷调查的基础能力。\\
7、项目导师:王明\\
  课题名称:佛祖保佑:传统中国的佛教与公共品供给\\
  成员要求:经济学专业;具有python抓取数据的技能;掌握计量知识\\
8、报名须知:若想加入老师的课题组,请填写报名表提交到box链接\url{https://box.nju.edu.cn/u/d/9b185b1675544d5ab3cd/},报名表获取方式为进入相关推文点击链接\\
9、报名截止日期:10.24 晚24点\\
10、具体导师信息和报名表获取链接请看推文:\url{https://mp.weixin.qq.com/s/2DlrBGDZn_p21MWzGFg4CA}\\
\url{https://mp.weixin.qq.com/s/LnliR0hKr6Z3Z33MK_23Ig}\\
注:通过老师审核的同学会收到来自商学院满天星学术科创实践中心的短信通知,请注意查收。在11月18日晚24点即立项申报截止之前,请入选学生和老师自行联系,在系统上完成立项信息填报。\\
另:请感兴趣的同学加入QQ群(490874894)\\

\section{项目通知|关于申报南京大学“国际访问学者计划”的通知(2025年第一批次)}
申报形式:“国际访问学者计划”应由校内合作者在与访问学者达成合作意向后,报依托单位批准,向国际化工作处提交申请。请有意向申报“国际访问学者计划”(2025年第一批次)的校内合作者登录申报系统进行提交(申报网址见下)\\
本批次系统填报开放时间:2024年10月21日至2024年11月22日。评审工作预计将于2024年11月下旬开展。\\
具体见:\url{https://mp.weixin.qq.com/s/Z6kRuLDQY2bSrtE4Pr5mkA}\\
申报网址链接:\url{http://isa.nju.edu.cn}\\
\section{SRTP  学术文化活动概览}
南京大学学生会学术创新部与SRTP学社将联合整理校内讲座,并发布“饕餮大餐·学术文化活动概览”系列推送,频率为一周两次。以下为10月23日至10月25日学术讲座的汇总:\\
周三(10.23)\\
1.中国法律形象西语世界变迁与美国的中国法研究\\
2.如何进行国际出版?\\
3.武周女皇的东亚旅行——则天文字在日本及朝鲜半岛的流传\\
4.《马克思主义历史考证大辞典》(HKWM)与当代德国马克思主义\\
5.从自我确立到自我祛魅:中国电影史中的自反性研究\\
6.1934-1935年的“杂文问题”论战与鲁迅的抉择\\
7.物理信息神经网络(PINN)在MATLAB中的实现\\周四(10.24)
周四(10.24)\\
1.馆藏生命科学相关资源的检索与利用\\
2.有趣有料有用的案例大赛\\
3.制度中的“政府”与“市场”\\
4.2024南大建筑系列讲座——李立:在地的实践\\
周五(10.25)\\
1.语言能力与前近代中日文化交流\\
2.新东方多媒体学习库名师讲堂——教你一招致胜英语学习及雅思考试\\
3.大众视角与历史真相\\
详情见:\url{https://mp.weixin.qq.com/s/toBnRHgtWtaanOwlJFfCOg}\\
\section{校园涂鸦快闪}
Brushstroke笔触画社发布,笔触画社将举办校快闪活动\\
地点1:四五六食堂门口    主题:晚餐吃什么\\
地点2:十二食堂门口      主题:角落里的一只猫\\
时间:2024年10月24日~10月25日中午\\
举办方已准备马克笔和画纸\\
详情请查看:\url{https://mp.weixin.qq.com/s/AOhGv03NEaA6qGeYGZQVuQ}\\




\section{“生活影像诗”投票}
南京大学学生心理协会发布。邀请各位为以下链接内的共27个生活影像故事投票:\\
\url{https://mp.weixin.qq.com/s/rAMZXxv5NsudrPBV8W8fSA}\\
投票规则:通过文章末尾小程序为你支持的作品投票,每人最多可投10票,可重复投给同一作品。\\
截止时间:2024年10月26日20:00\\
\section{南大篮球明日赛程}
男篮院系杯\\
地科vs政管\\
12:30-14:00\\
地点:一组团篮球场\\
\section{物院“南雍慧聚”}
学长学姐经验分享:特邀大三优秀学长学姐,分享高效学习方法、时间管理技巧及备考攻略。\\
互动答疑环节:现场解答你的所有学习困惑,让难题迎刃而解。\\
活动时间:本周五(10月25日)下午4点45分,活动时长约一个小时\\
活动地点:仙ll-306\\
报名方式:复制下面链接到浏览器\url{https://table.nju.edu.cn/dtable/forms/1db72af6-0c26-4576-af22-a30c67665087/}


\section{“医脉相连,物启新程”南京大学\&南京中医药大学联谊活动}
南中医中医学院携手南大物理学院、天文与空间科学学院举办\\
活动1:参观南中医仙林校区药苑\\
时间:10月26日10:00——12:00\\
活动2:南中医义诊\\
时间:10月27日10:00——11:30\\
地点:南大仙林校区某处待定\\
活动注意事项:\\
1. 报名成功后请进群按提示进行预约\\
2. 10月26日10:00前请自行前往南京中医药大学仙林校区北门\\
详情与报名请查看:\url{https://mp.weixin.qq.com/s/YTHiuJjChI4nMjlY8KbOEQ}\\
\section{物理学院纪念我国第一颗原子弹爆炸60周年活动}
为纪念中国第一颗原子弹爆炸成功60周年,传承“两弹一星”精神,南京大学物理学院将于10月24日举办“‘两弹一星’,薪火相传”示范性主题团日活动\\
主要内容:邀请南京大学物理学院副院长、中国核物理学会常务理事许昌教授,开展核物理知识普及讲座;邀请2024级物理学直博生、物理学院“强国”宣讲团成员沈建宇同学,讲述相关历史故事\\
活动时间:10月24日 19:00\\
活动地点:主会场——仙林校区 南青报告厅 \\
分会场——鼓楼校区 物理楼233(同步线上直播)\\
报名链接:\url{https://table.nju.edu.cn/dtable/forms/7ac266c4-3ffe-4b03-b43a-736e69069544/}\\
详情请查看:\url{https://mp.weixin.qq.com/s/HrRor-wwrnWubsawhYm3bA}\\
\section{情暖冬“计”系列活动第一弹——螺钿胸针制作}
活动对象:南京大学计算机学院全体师生\\
时间:2024年10月28日 14:00-17:00\\
地点:计算机科学技术楼221会议室\\
报名方式:进入链接扫码报名\url{https://mp.weixin.qq.com/s/6D_AO0SZHHHgpyVcPnShuQ}


\section{庆祝文学院创建110周年征集活动}
南京大学文学院:2024年11月2日,南京大学文学院即将迎来一百一十周年院庆盛事。为更加全面、深入地展现文学院的发展历史、学术底蕴和学院生活,营造浓厚的院庆氛围,特面向广大师生、校友以及关心支持文学院的各界人士征集新诗、传统诗词、楹联等作品。文学院将精选部分作品在学院内的多个空间进行展示,在官方公众号进行推送,并拟择优编选110周年院庆诗联集。此外,回忆录、祝福语、学术展、照片视频、院史资料等内容的征集也在进行之中,欢迎大家投稿。\\
诗歌、楹联在线征集:\\
(一)大赛分为新诗、传统诗词与楹联三个组别。\\
(二)以“庆祝南京大学文学院创建110周年”为主题,选择与南大文学院历史、名家、风景、日常生活等相关的题材,创作诗歌或楹联。\\
(三)传统诗词的诗体、词牌不限,韵部不限。\\
征稿阶段:10月20日—11月20日\\
评选阶段:11月21日—11月30日\\
可在“知鸿蒙”微信小程序或app投稿,也可扫描原文二维码。\\
具体要求、提交方式以及其他征集内容相关详细参见原文:\url{https://mp.weixin.qq.com/s/gYn1c_8p96LsN1ZhiBIiyQ}\\

\section{毓琇书院文明宿舍评比活动}
活动地点:涵盖陶园二舍、陶园三舍及毓琇书院学生所在的各个宿舍\\
评比时间:10月25日至10月31日\\
活动内容:检查人员由宿舍楼长、层长和各班班委组成,他们将分组对寝室进行细致检查,客观评分。负责摄影的同学们也会记录下活动的精彩瞬间。\\
除了常规检查,本次活动还将设置“特色文化宿舍”若干个,各宿舍可以根据情况申请如下宿舍称号,也可自主命名“**型宿舍”:清新文雅型宿舍、博学求真型宿舍、阳光热情型宿舍、赤忱向党型宿舍、健康锻炼型宿舍、博艺广趣型宿舍……\\
奖励:“合格”宿舍每位同学获3小时劳育时长,“良好”宿舍获4小时,“优秀”宿舍获5小时。获评“优秀”宿舍的还将获得一份惊喜礼品。评比“不合格”的宿舍需整改,整改合格后计3个小时。\\
(院系混住宿舍中,非毓琇书院的同学不参与本次活动劳动时长计算。)\\
评分准则和宿舍名称申请链接见\url{https://mp.weixin.qq.com/s/iJm89MhcgOAi14FZJ9-5cA}
\end{multicols}
\end{document}