\documentclass[letterpaper, 12pt]{article}
\newsavebox\colbbox
\usepackage{graphicx}
\usepackage{multicol}
\usepackage{anysize}
\usepackage{fontspec}
\usepackage[fontset=none]{ctex}
\usepackage{tabularx}
\usepackage{longtable}
\PassOptionsToPackage{hyphens}{url}
\usepackage[breaklinks=true, colorlinks=true]{hyperref}
\expandafter\def\expandafter\UrlBreaks\expandafter{\UrlBreaks\do\a\do\b\do\c\do\d\do\e\do\f\do\g\do\h\do\i\do\j\do\k\do\l\do\m\do\n\do\o\do\p\do\q\do\r\do\s\do\t\do\u\do\v\do\w\do\x\do\y\do\z\do\A\do\B\do\C\do\D\do\E\do\F\do\G\do\H\do\I\do\J\do\K\do\L\do\M\do\N\do\O\do\P\do\Q\do\R\do\S\do\T\do\U\do\V\do\W\do\X\do\Y\do\Z}
% \let\oldurl\url
% \renewcommand{\url}[1]{\begin{sloppypar}\oldurl{#1}\end{sloppypar}}
\setlength\columnsep{30pt}
\marginsize{30pt}{30pt}{10pt}{20pt}
\setmainfont{TeX Gyre Bonum}
\setCJKmainfont[BoldFont=Noto Serif CJK SC Bold, ItalicFont=FandolKai]{Noto Sans CJK SC}
\setlength{\parindent}{0cm}
% \setCJKmonofont{Noto Sans CJK SC}
\begin{document}
\begin{center}
    \Huge\textbf{南哪大专醒前消息}
\end{center}
\vspace{4mm}
\hrule
\renewcommand\tabularxcolumn[1]{m{#1}}
\begin{tabularx}{\textwidth}{>{\hsize.2\hsize}X>{\hsize.6\hsize}X>{\hsize.2\hsize}X}
    \begin{flushleft}
        2024.11.26\, No.128
    \end{flushleft}
    &
    \begin{center}
        \textit{“秉中持正、求新博闻。”}
    \end{center}
    &
    \begin{flushright}
        \textbf{南京市栖霞区}
    \end{flushright}
\end{tabularx}
\vspace{-3.5mm}
\hrule
\vspace{4mm}
% HEAD END
\centerline{\huge\textbf{活动预告}}
\begin{multicols}{2}
    \section{订阅方式和加入编辑部}  
编辑部招聘人才,用爱发电,工作轻松,详情可联系QQ:1329527951 客服小祥\\想订阅本消息或获取PDF版(便于查看超链接和往期),可加QQ群:\href{https://qm.qq.com/q/VXIW7fgsEe}{849644979}.
\section{Deadline Ongoing}
\setbox\colbbox\vbox{
\makeatletter\col@number\@ne
\begin{longtable}{|c|c|c|}
    \hline
    消息(未见ddl的,不刊) & 截止日期 & 刊载日期\\
    \hline\hline
    紫藤学刊征稿 & 12.15 & 10.22\\
    乐跑 & 12.6 & 10.12\\
    DIY课程学术论坛征稿 & 11.30 & 11.13\\
    国风歌曲演唱赛 & 12.1 & 11.13\\
    牡丹亭庆演 & 12.1 & 11.13\\
    普通话测试网络报名 & 11.30 & 11.16\\
    安邦征稿 & 1.12 & 11.16\\
    秉文宿舍风采 & 12.1 & 11.17\\
    法学院征诗活动 & 12.2 & 11.20\\
    平安留学交流会 & 12.3 & 11.20\\
    七院相亲活动 & 11.30 & 11.21\\
    免费上网讲座 & 11.27 & 11.22\\
    猫鼠大战 & 12.15 & 11.22\\
    日本征文大赛 & 12.6 & 11.22\\
    萨勒姆的女巫 & 11.30 & 11.22\\
    防艾征集 & 12.10 & 11.22\\
    爱心义卖 & 11.30 & 11.22\\
    银杏叶制作之旅 & 11.29 & 11.23\\
    法学朋导招募 & 11.29 & 11.23\\
    南选问答集赞 & 11.27 & 11.23\\
    心协信件盲盒 & 12.7 & 11.23\\
    港澳台晚会招募 & 11.27 & 11.23\\
    配音大赛招募 & 12.7 & 11.23\\
    物院研会系列活动 & 12.21 & 11.23\\
    心理中心征稿 & 12.10 & 11.23\\
    防艾同伴教育 & 12.15 & 11.24\\
    南新读书会 & 11.27 & 11.24\\
    腾讯游戏开发赛报名 & 11.29 & 11.24\\
    新生午餐会报名 & 11.27 & 11.24\\
    物院摄影征集 & 12.9 & 11.24\\
    一南一度脱口秀 & 11.29 & 11.25\\
    天健志愿活动报名 & 11.31 & 11.25\\
    创意物理实验竞赛 & 12.21 & 11.15\\
    古声国风音乐活动 & 11.29 & 11.25\\
    南悦制作贺卡活动 & 11.30 & 11.25\\
    午餐读书会 & 11.27 & 11.25\\
    南商面对面 & 11.29 & 11.26\\
    期末解压活动营 & 11.30 & 11.26\\
    包饺子志愿招募 & 12.6 & 11.26\\
    法学院征稿 & 12.4 & 11.26\\
    金陵旧书市集 & 11.30 & 11.26\\
    仙林通宵自习室 & 1.12 & 11.26\\
    四六级耳机试音 & 12.6 & 11.26\\
    放克音乐活动 & 11.30 & 11.26\\
    心协团辅活动 & 11.30 & 11.26\\
    
    \hline
\end{longtable}
\unskip
\unpenalty
\unpenalty}\unvbox\colbbox
\end{multicols}
\hrule
\pagebreak
\begin{multicols}{2}

\section{讲座}
\begin{tabular}{|c|c|c|}
    \hline
    往期讲座 & 开展日期 & 刊载日期\\   \hline\hline
    《专利查新与规避...》 & 12.19 & 10.3\\
    图书馆系列讲座 & 12.3 & 10.20\\
    《学术写作入门...》& 11.21 & 11.18\\
    《Adobe AI 讲座》 & 11.27 & 11.22\\
    《马克思的经济全球化》 & 11.27 & 11.23\\
    《法学研究类型和方法》 & 11.29 & 11.23\\
    《前沿情报捕捉...》 & 11.29 & 11.23\\
    《AI在设计中的参与...》 & 12.2 & 11.23\\
    《决策规划算法》 & 11.28 & 11.24\\
    《鲁艺木刻工作团...》 & 11.28 & 11.25\\
    《如何利用超星系...》 & 11.27 & 11.25\\
    《校外文献互借指南》 & 11.28 & 11.25\\
    《简历简化与制作...》 & 11.27 & 11.25\\
    《数据资源入表...》 & 12.4 & 11.26\\
    《微表情与伪装...》 & 12.6 & 11.26\\
    《费马大定理》 & 11.27 & 11.26\\
    《Quantum mater...》 & 11.28 & 11.26\\
    《案例分析大赛...》 & 11.27 & 11.26\\
    \hline
\end{tabular}

1.南大MBA科技与人文系列讲座\\
讲座主题:数据资源入表相关会计问题研究\\
讲座介绍:梳理数据要素入表现状、问题与难点,结合现有的数据要素会计研究,提出数据资源入表可能的会计实现路径。\\
主讲嘉宾:何贤杰\\
安徽财经大学学术副校长、上海财经大学会计学院特聘教授\\
讲座时间:12月4日(周三)20:00\\
讲座地点:腾讯会议 线上讲座\\
讲座报名:扫描二维码(见附录),成功报名后,会议地址将于讲座开始前发送\\

2.潘菽心理学论坛第九十二期\\
讲演题目:微表情与伪装表情研究\\
主讲内容:本报告将简要说明我国构建的系列微表情数据库,并具体介绍四个伪装表情实验及其研究发现。\\
讲演人:傅小兰\\
中国科学院心理研究所研究员\\
讲座时间:12月6日(周五)下午3:30-5:30\\
讲座地点:仙林社会学院河仁楼401\\

3、星成长丨满天星商知讲堂\\
题目:案例分析大赛备赛指南:商赛经验与技巧分享\\
分享嘉宾:商学院22级本科生黄子珊、吕博晨、杨宗霖\\
时间:2024.11.27 18:30-19:40\\
地点:仙林校区 仙1-111\\
报名链接:\url{https://table.nju.edu.cn/dtable/forms/824a2d0b-8f6f-409d-a7fe-5780d0ff8e87/}\\
问题收集:\url{https://table.nju.edu.cn/dtable/forms/7918c709-47e5-4cad-ab26-ebda3805b854/}\\
注:此次活动可作为项目制课程的过程性学习\\
具体内容请见推文:\url{https://mp.weixin.qq.com/s/PkVBWqL4OAAiYMDDTpq3dA}\\

4.物理学院学术报告会(第42期)\\
题目:Quantum materials under extreme conditions\\
报告人:李世燕,复旦大学\\
时间:2024年11月28日(周四)15:30\\
地点:鼓楼校区唐仲英楼B501\\
讲座内容简介等见\url{https://mp.weixin.qq.com/s/USb-txctJasSt9gy9gJN4A}\\

5.数学学院本科生论坛(教师系列第89讲)--大一专场\\
题目:费马大定理\\
报告人:程创勋\\
时间:11月27日(星期三)16:00-17:30\\
地点:戊己庚四楼北\\
腾讯会议:870-7007-3326\\
简介:在这个报告里,报告人将向大家讲述费马大定理的起源、发展、以及相关的数学理论和数学故事,并介绍怀尔斯证明中的两个重要对象:椭圆曲线和模形式.\\
\url{https://mp.weixin.qq.com/s/x4EGyVnHnN9F9sm732rs8w}\\


\section{“南商面对面”第五期}
主题:微光闪烁,渺小启程\\
在享用精美餐点和茶歇时,与同辈、长辈分析优绩主义,分享“普通人”的故事,收获友谊知音。\\
时间:11月29日(周五)中午\\
地点:仙林学生公寓4栋1楼活动室\\
嘉宾:赵华 南京大学经济学系副教授、南京大学理论经济学博士后\\
报名:截至11月27日(周三)20:00,点击链接\url{https://table.nju.edu.cn/dtable/forms/23f9e0b0-358e-4da1-aabe-408173033fd7/}报名,报名后请加入活动咨询QQ群(群号:174573214;群二维码见附录)\\
注:该次活动会为参与同学录入五育时长\\




\section{角落里的文物征集}
南大文爱 鼎鬲觥斝甗匜 发布\\
对象说明:“角落里的文物”是指那些散落在城乡各个角落或博物馆中,未被充分重视、容易被忽视的具有一定历史文化价值的物品或建筑等。\\
征文要求(简):作品以照片和视频为主,可附带简短文字说明。文字部分要求介绍文物或建筑的基本信息、历史背景及其文化价值,不超过300字。所提交的作品必须为原创。\\
收录说明:被收录的作品将在南京大学文物爱好者协会微信公众号发表;被收录的作品或在未来相关展览展出。优秀的投稿者将有机会获得丰厚奖品。\\
其他活动信息、具体征文要求和投稿方式请查看\url{https://mp.weixin.qq.com/s/5XskK4-wKaznogfyXUdfOg}\\

\section{防艾讲座}
讲座时间:2024年11月27日 18:30\\
讲座地点:南京大学鼓楼校区逸夫馆报告厅\\
腾讯会议号:930-312-228\\
详见\url{https://mp.weixin.qq.com/s/exu2d59tvzQ9eCeXhpwiCw}

\section{放克音乐活动}
主办:passion街舞社
时间:2024.11.30周六(18:00签到,18:30开始)\\
地点:南京大学仙林校区敬文学生活动中心111室\\
音乐:Funk Music\\
报名、赛制等信息请查看\url{https://mp.weixin.qq.com/s/3AqNpsVmNYbtFjzA7hroTA}\\

\section{团辅活动招募}
活动时间:\\
2024年11月30日 10:00\\
活动地点:\\
鼓楼校区新教408\\
报名方式:\\
报名群聊见附录\\
详见\url{https://mp.weixin.qq.com/s/-znz3MkgQOUFL0SKm5PlZA}

\section{关于12月份全国大学英语四六级考试耳机试音及修理的通知}
为了确保我校2024年12月14日全国大学英语四、六级考试的顺利进行,大学外语部将为考生安排耳机听力试音,请各考生注意以下时间安排:\\
耳机试音时间:2024年12月6日09:00—16:00\\
1)仙林校区:教学楼和逸夫楼所有的考试教室(耳机接收选择开关拨至FM72.7兆赫)\\
2)鼓楼校区:教学楼一楼至四楼的考试教室(耳机接收选择开关拨至FM72.7兆赫)\\
3)苏州校区:南雍楼西一楼至五楼的考试教室(耳机接收选择开过拨至FM72.7兆赫)\\
2、教育技术中心将提供免费修理耳机服务,送耳机时间、地点及取耳机时间:\\
送耳机12月9日—12月10日(工作时间的9:00--16:00)。\\
仙林校区:教学楼Ⅱ-206室。\\
鼓楼校区:教学楼212室。\\
苏州校区:天枢楼101C室。\\
取耳机:12月12日—12月13日止(工作时间的9:00--16:00)\\
请考生注意:如在试音过程中发现放音问题,请在工作日的9:00-12:00,14:00-17:00之间反馈到本科生院,电话025-89681635。\\
具体链接:\url{https://jw.nju.edu.cn/16/7f/c26263a726655/page.htm}\\

\section{仙林校区通宵自习教室开放通知}
开放时间:2024年11月27日至2025年1月12日\\
开放教室:开放择善楼一楼仙Ⅰ-106、Ⅰ-107作为通宵自习教室。\\
(目前暂开放2间通宵自习教室,后续会根据同学们的实际学习需求作进一步的动态调整。下学期通宵自习教室的开放计划将结合本学期的使用情况和效果再另行通知。)\\
具体链接:\url{https://jw.nju.edu.cn/16/65/c26263a726629/page.htm}\\

\section{秉文书院趣味运动会}
1.活动主办:南京大学秉文书院团学联素质拓展部(体育方向)\\
2.活动形式:个人PK赛、团体巅峰赛\\
3.参与对象:个人PK赛面向南京大学新生学院全体新生开放报名,团体巅峰赛以秉文书院的班级为单位报名参与\\
4.地点:苏浙运动场\\
5.时间:预计为12月1日(周日)下午13:30-16:30;若遇不良天气情况则延期\\
详情见\url{https://mp.weixin.qq.com/s/R3FG5_TWiLexrxxHVcFHMw}\\
\section{金陵旧书市集}
时间:11月28日至11月30日10:00 —— 20:00
地点:仙林校区图书馆
大咖交流区:
胡阿祥:我与旧书的故事
时间:11月28日18:00~19:00
地点:杜厦图书馆一楼校友之家小报告厅

\section{法学院“云接力”征稿活动}
投稿要求:字数在100-300字之间,以树洞投稿形式,可自由选择是否附上照片,支持实名或匿名投稿,但务必附上个人真实姓名、学号、手机号,用于后续颁奖。\\
投稿方式如下,二选一即可。\\
1.邮件投稿:请各位同学将稿件发送至njufayan2024@163.com,邮件主题命名为“云接力征稿+姓名+手机号”。\\
2.问卷投稿:请各位同学通过链接或者扫描二维码,填写问卷报名\url{https://www.wjx.cn/vm/hhv9Vgx.aspx#} \\
时间:11月27日—12月4日\\
详情见\url{https://mp.weixin.qq.com/s/635U5nUPzeE6JSBz4hHk7Q}\\




\section{包饺子志愿者招募}
冬至时节,南大天健社与南京市贝贝儿童发展中心筹划举办一场包饺子活动,邀请自闭症小朋友们一同参与。\\
时间:12月6日\\
地点:建邺区南京市贝贝儿童发展中心\\
内容:讲解冬至小知识,组队进行包饺子大赛\\
报名方式:扫码加群,详见\url{https://mp.weixin.qq.com/s/wK9YaCExcedpYuaPlcNb6g}

\section{“前辈之声”访谈嘉宾招募}
招募对象:大四学生及研究生\\
如果愿意将自己对于成长的理解跟更多人交流,不论是专业思考、大学规划,还是生活兴趣,成长故事,都欢迎联系主办方。\\
访谈内容脱敏处理后将可公开内容汇集成《南大人的成长故事录》以记录每个人独一无二的成长经历(公开部分会经参与者审核)。\\
访谈嘉宾联系方式:ChrisWang\_20050621(微信号)\\
报名方式:扫码填写报名问卷,详见\url{https://mp.weixin.qq.com/s/7gzcow6axY4hxRC_EAn6QA}

\section{期末解压活动营}
时间:2024年11月30日(周六)上午10:00-12:30\\
地点:鼓楼校区校园内(报名通过后将通过邮箱发送具体地点)\\
带领人:吴思樾老师(心理中心专职教师、中国心理学会注册心理师、二级心理咨询师)\\
活动形式:知识讲解+团体活动\\
报名需填问卷,详见\url{https://mp.weixin.qq.com/s/OVpq5c3CPUKM-fNkCHLFKA}

\section{角落里的文物 征文活动}
征文对象:全体南大师生\\
征文主题:“角落里的文物”是指那些散落在城乡各个角落或博物馆中未被充分重视、容易被忽视的具有一定历史文化价值的物品或建筑等。这些文物虽然暂时未处于大众关注的焦点,但都凝聚着人类文明的智慧与心血,见证着过往的历史风云。\\
征文要求:作品以照片和视频为主,可以附带简短的文字说明。详见原文。\\
投稿方式:见原文。\\
优秀的投稿者将有机会获得丰厚奖品(精美书籍,博物馆文创等);征文入选的同学可获得来自故宫博物院或敦煌研究院的精美文创。
原文:\url{https://mp.weixin.qq.com/s/5XskK4-wKaznogfyXUdfOg}

\section{篮协赛程}
明日11.27:\\
男篮院系杯附加赛\\
政管vs海外\\
12:30-14:00\\
地点:方肇周体育馆
\section{11.27排球新生杯赛程预告}
11月27日 16:30-17:45\\
秉文书院—来都来了队\\
11月27日 18:00-19:15\\
行知书院队—无论输赢学长都要请吃饭队

\section{排超联赛}
南大排协 发布\\
时间:11月28日 19:30\\
南京广电猫猫vs浙江德清\\
转发原文推文至朋友圈集赞十个可扫码参与抽奖,礼品为小排球3份\\
11.28(周四)晚19:00开奖\\
原文:\url{https://mp.weixin.qq.com/s/7WvuNhIBDHmhZ8pQh-25cw}
地点:南京大学方肇周体育馆\\
观赛无需报名,集赞抽奖于11.28(周四)晚19:00开奖\\
其他详情请查看\url{https://mp.weixin.qq.com/s/7WvuNhIBDHmhZ8pQh-25cw}\\
\section{软件学院与法学院跨院系交流活动}
活动时间:12月1日15:00-17:00\\
活动地点:南京大学鼓楼校区\\
活动对象:南京大学软件学院、法学院学生\\
活动内容:趣味桌游、羽毛球赛、拼图与积木、飞盘活动\\
报名方式见链接\url{https://mp.weixin.qq.com/s/8iJi0yp7pn17-zkJZj76AQ}


\end{multicols} 
\hrule
\vspace{4mm}
\centerline{\huge\textbf{参考消息}}
\begin{multicols}{2}
\section{尸位素餐的小组作业负责人正在残害密斯卡托尼克大学学生的理智}

——密斯卡托尼克大学新闻学系 特约评论员\\

警告:此文本的构成涉及某种非人类智慧的隐秘介入,其背后逻辑或已被不可名状的意志扭曲。阅读者需谨慎,因其本质可能超出人类理智所能承受的界限。\\

密斯卡托尼克大学校园的某些角落正在悄然滋生一股无形的阴霾,这种阴霾并非来自神秘的旧神低语,而是由一些尸位素餐的学生负责人所引发的。这些自封为“小组项目之帝王”的角色,不仅正在摧残参与者的耐心,更似乎试图侵蚀他们的理智。\\

学生的呼声:逃离“至高祭司式”小组项目\\

根据多名学生在密斯卡托尼克大学校园社交平台上的自发陈述,这些令人绝望的“小组项目负责人”已成为社交媒体上最受诟病的话题之一。一位愤怒的学生发文表示:“当小组成员在深夜反复商讨实践方案时,负责人却宛如湮没于不可名状的虚无,久久不见踪影。三番催促之后,他们只投下一句似虚空回响般的言辞——‘我不想搞得太复杂’。那一刻,仿佛所有努力都被拖入无底的深渊,化作绝望的低语。”这句充满讽刺意味的话,成为了描述这些负责人行为的精准概括。\\

来自神秘学系、考古学系、生物学系等院系的学生们普遍反馈,这些负责人不仅在项目中“缺位”,更习惯性地用空洞的“权威式发言”掩盖自身的无能。一位参与者无奈地描述道:“当负责人高踞于至高的位置,似掌握了超越常人理解的权力,俨然化身为学术的秘仪祭司,课题的至高支配者。他们以指点江山的姿态激扬辞藻,睥睨群伦,俯视凡人般唯我独尊。那种掌控一切的幻觉,是否像触碰禁忌真理一般令人陶醉?”这种敷衍不仅延误了团队进度,更让整个项目陷入混乱。\\

理智受损:一个比比皆是的现实\\

在这些小组中,许多学生形容他们经历的痛苦已然超越学术难题,甚至开始质疑自身的存在意义。一名学生在群组中无数次@负责人,却只换来一句“消息太多看不了”,而负责人却能同时在社交媒体上发布长篇抒情文。这种“明忙实闲”的行为,宛如埃德加·戈登手稿中的隐匿恶魔,吞噬着团队的信任与理智。\\

更为令人愤怒的是,这些“负责人”不仅学术能力平平,态度恶劣,还常常以“学工忙”“课业多”为借口推卸责任。然而讽刺的是,他们却能依靠这些言辞塑造的表象轻松赢得校级奖项,成为师生口中值得尊敬的榜样。一名被压抑情绪淹没的学生控诉道:“昔日学者常言,‘唯有强者方可行于秘道’(陈力就列,不能者止),然此等无能之徒竟能行至高堂,受予荣耀,莫非世道已为深渊所反转?”\\

这种困惑与愤怒在密斯卡托尼克大学的学术殿堂中激荡,宛如无形的浪潮撼动着每一位求知者的心灵。有学生将其比作图书馆深处被禁忌锁链束缚的无名典籍,既受制于深邃奥秘的权威,又在规则中腐朽不堪。正如另一位学生所言:“若理智是学者探寻真知的明灯,那这类‘尸位素餐者’便是烛火中的暗影,所照之处,唯有腐败与虚伪。”\\

未来何在:破裂的信任与未知的深渊\\

密斯卡托尼克大学的氛围一向以开放与合作著称,然而,这些无能负责人的出现正逐渐破坏着学术共同体的根基。有学生形容,与这些人共事就像直面奈亚拉托提普的多面诡诈,似乎每一个决策都隐藏着深不见底的危险。\\

密斯卡托尼克大学的未来是否会因这些不称职的负责人而滑向未知的深渊?还是会在理性与努力的引导下重新回归正轨?目前尚不得而知。但可以确定的是,学生们已经开始为自己发声,试图抵抗这种逐渐吞噬校园的阴霾。\\

(本文由密斯卡托尼克大学新闻学系特约记者撰写,旨在揭露隐藏于表象之下的现实。)
\end{multicols} 

\end{document}