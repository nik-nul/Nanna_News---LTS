% HEAD BEGIN
\documentclass[letterpaper, 12pt]{article}
\newsavebox\colbbox
\usepackage{graphicx}
\usepackage{multicol}
\usepackage{anysize}
\usepackage{fontspec}
\usepackage[fontset=none]{ctex}
\usepackage{tabularx}
\usepackage{longtable}
\PassOptionsToPackage{hyphens}{url}
\usepackage[breaklinks=true, colorlinks=true]{hyperref}
\expandafter\def\expandafter\UrlBreaks\expandafter{\UrlBreaks\do\a\do\b\do\c\do\d\do\e\do\f\do\g\do\h\do\i\do\j\do\k\do\l\do\m\do\n\do\o\do\p\do\q\do\r\do\s\do\t\do\u\do\v\do\w\do\x\do\y\do\z\do\A\do\B\do\C\do\D\do\E\do\F\do\G\do\H\do\I\do\J\do\K\do\L\do\M\do\N\do\O\do\P\do\Q\do\R\do\S\do\T\do\U\do\V\do\W\do\X\do\Y\do\Z}
% \let\oldurl\url
% \renewcommand{\url}[1]{\begin{sloppypar}\oldurl{#1}\end{sloppypar}}
\setlength\columnsep{30pt}
\marginsize{30pt}{30pt}{10pt}{20pt}
\setmainfont{TeX Gyre Bonum}
\setCJKmainfont[BoldFont=Noto Serif CJK SC Bold, ItalicFont=FandolKai]{Noto Sans CJK SC}
\setlength{\parindent}{0cm}
% \setCJKmonofont{Noto Sans CJK SC}
\begin{document}
\begin{center}
    \Huge\textbf{南哪大专醒前消息}
\end{center}
\vspace{4mm}
\hrule
\renewcommand\tabularxcolumn[1]{m{#1}}
\begin{tabularx}{\textwidth}{>{\hsize.2\hsize}X>{\hsize.6\hsize}X>{\hsize.2\hsize}X}
    \begin{flushleft}
        2024.11.19\, No.121
    \end{flushleft}
    &
    \begin{center}
        \textit{“秉中持正、求新博闻。”}
    \end{center}
    &
    \begin{flushright}
        \textbf{南京市栖霞区}
    \end{flushright}
\end{tabularx}
\vspace{-3.5mm}
\hrule
\vspace{4mm}
% HEAD END
\centerline{\huge\textbf{活动预告}}
\begin{multicols}{2}
    \section{订阅方式和加入编辑部}  
编辑部招聘人才,用爱发电,工作轻松,详情可联系QQ:1329527951 客服小祥\\想订阅本消息或获取PDF版(便于查看超链接和往期),可加QQ群:\href{https://qm.qq.com/q/VXIW7fgsEe}{849644979}.
\section{Deadline Ongoing}
\setbox\colbbox\vbox{
\makeatletter\col@number\@ne
\begin{longtable}{|c|c|c|}
    \hline
    消息(未见ddl的,不刊) & 截止日期 & 刊载日期\\
    \hline\hline
    紫藤学刊征稿 & 12.15 & 10.22\\
    EBSCO数据库检索大赛 & 11.20 & 10.3\\
    文院征稿 & 11.20 & 10.20\\
    乐跑 & 12.6 & 10.12\\
    国际访学计划申报 & 11.22 & 10.22\\
    秉文心理短视频 & 11.25 & 11.3\\
    DIY课程学术论坛征稿 & 11.30 & 11.13\\
    国风歌曲演唱赛 & 12.1 & 11.13\\
    牡丹亭庆演 & 12.1 & 11.13\\
    创业集市摊位 & 11.22 & 11.14\\
    II剧 & 11.23 & 11.14\\
    潘高峰南大分享会 & 11.23 & 11.16\\
    普通话测试网络报名 & 11.30 & 11.16\\
    商院RPG活动 & 11.23 & 11.16\\
    安邦征稿 & 1.12 & 11.16\\
    南新读书会 & 11.20 & 11.17\\
    新传院迎新晚会征集 & 11.25 & 11.17\\
    秉文宿舍风采 & 12.1 & 11.17\\
    商院生涯论坛 & 11.24 & 11.18\\
    六朝博物馆访学 & 11.22 & 11.18\\
    秋日手作游园会 & 11.23 & 11.18\\
    新生午餐会报名 & 11.20 & 11.18\\
    黑匣招募 & 11.25 & 11.18\\
    邮局影映 & 11.21 & 11.18\\
    重唱诗社评诗会 & 11.24 & 11.19\\
    鼓楼中医义诊 & 11.23 & 11.19\\
    古琴社露天音乐会 & 11.24 & 11.19\\
    心协有奖征稿 & 11.25 & 11.19\\
    日俱影映 & 11.24 & 11.19\\
    
    \hline
\end{longtable}
\unskip
\unpenalty
\unpenalty}\unvbox\colbbox
\end{multicols}
\hrule
\pagebreak
\begin{multicols}{2}

\section{讲座}
\begin{tabular}{|c|c|c|}
    \hline
    往期讲座 & 开展日期 & 刊载日期\\
    \hline\hline
    《电池及电化学能...》 & 11.24 & 10.3\\
    《专利查新与规避...》 & 12.19 & 10.3\\
    图书馆系列讲座 & 12.3 & 10.20\\
    《解码黑猴背景音乐...》 & 11.21 & 11.14\\
    《量子计算的科普...》 & 11.22 & 11.16\\
    《学术写作入门...》& 11.21 & 11.18\\
    《中华伦理文明的...》& 11.22 & 11.18\\
    《谈当代中国水墨...》& 11.22 & 11.18\\
    《流体运动中的数...》 &11.20 & 11.18\\
    《文献检索交流分...》 & 11.21 & 11.18\\
    《超越Transformer》 & 11.23 & 11.19\\
    《人工神经网络...》 & 11.20 & 11.19\\
    《战争记忆的跨文...》 & 11.22 & 11.19\\
    《作为不莱梅商业...》 & 11.21 & 11.19\\
    《安永法证及诚信...》 & 11.20 & 11.19\\
    《量子材料》 & 11.20 & 11.19\\
    《芯片布局优化》 & 11.20 & 11.19\\
    《quantum metro...》 & 11.20 & 11.19\\
    《作为媒介的人工...》 & 11.22 & 11.19\\
    \hline
\end{tabular}

1.新一代大模型结构:超越Transformer\\
讲座内容:transformer的缺陷、RWKV-6和RWKV-7:核心机制、发展历史、多模态应用:视觉、音频、时间序列、RWKV的特点、RWKV带来的产品新机会、RWKV的生态发展\\
主讲人:
罗璇 元始智能(RWKV商业公司)联创及COO\\
侯皓文 新加坡国立大学博士\\
时间地点:11月23日 仙林校区\\
RWKV活动/报名群:653589877\\

2.诺贝尔物理学奖与人工神经网络的革命\\
主讲人:徐百乐(南京大学人工智能学院特聘助理研究员,主要研究方向为神经网络和机器学习)\\
讲座时间:2024年11月20日(周三)16:10-18:00\\
讲座地点:南京大学鼓楼校区馆1-205\\
讲座内容:本报告将简述神经网络与物理学的深厚关系,探讨神经网络在科学研究中的推动作用。通过这次报告,我们希望能够启发大家对人工智能与所学专业的联系进行思考,探索未来技术进步的可能性\\
报名详见\url{https://mp.weixin.qq.com/s/M54dgLUfA4YkpR2GCokRPw}

3.故事无界?\\
主题:故事无界?--战争记忆的跨文化转译与跨时空旅行\\
主讲人:李红涛\\
复旦大学信息与传播研究中心副主任、复旦大学新闻学院教授\\
时间:11月22日(周五)19:00-21:00\\
地点:仙林国际学院C308高研院报告厅\\
内容提要:讨论记忆框架共享度和文化接近性对记忆旅行的影响,反思(逆)全球化和平台化背景下跨文化记忆流动的“墙”、“界”乃至“壁”。\\
4.作为不莱梅商业学徒的恩格斯(1838年7月-1841年3月):恩格斯“双重生活”的开端\\
主讲人:卡尔-埃里希·福尔格拉夫 柏林勃兰登堡科学院\\
学术翻译:姜颖 刘佳 中共党史与文献研究院二级翻译\\
时间:11月21日(周四)9:00-11:30\\
地点:南京大学圣达楼213\\
5.多模态哈希学习与多视图学习中的优化方法\\
主讲人:李华雄\\
时间:11月20日(周三)14:00-16:00\\
地点:南京大学协鑫楼204\\

6.SRTP整理

周三(11.20)

1.安永法证及诚信合规服务交流分享会\\
2.流体运动中的数学\\
3.诺贝尔物理学奖与人工神经网络的革命\\
4.量子材料——从光说起\\
5.芯片布局优化:从2D到3D的转变\\
6.Quantum Metropolis Sampling via Weak Measurement\\
周四(11.21)\\
学术写作入门——巧用NE,EN,AI助力写作\\
周五(11.22)\\
1.作为媒介的人工智能\\
2.“孝道”与当代人类“代际伦理”的重建——中华伦理文明的现代阐释\\
详见\url{https://mp.weixin.qq.com/s/bLBYFwmLo26jK5bQNO3QdQ}

\section{重唱诗社匿名评诗会}
“我们忙于收获,去丰收进一颗果壳”\\
时间:2024年11月24日(周日)14:00\\
地点:新教405(南京大学鼓楼校区)\\
投稿邮箱:aichongchang@163.com(备注匿名评诗会,限投一首)\\
截稿日期:2024年11月23日(周六)12:00\\
详见\url{h