% HEAD BEGIN
\documentclass[letterpaper, 12pt]{article}
\newsavebox\colbbox
\usepackage{graphicx}
\usepackage{multicol}
\usepackage{anysize}
\usepackage{fontspec}
\usepackage[fontset=none]{ctex}
\usepackage{tabularx}
\usepackage{longtable}
\PassOptionsToPackage{hyphens}{url}
\usepackage[breaklinks=true, colorlinks=true]{hyperref}
\expandafter\def\expandafter\UrlBreaks\expandafter{\UrlBreaks\do\a\do\b\do\c\do\d\do\e\do\f\do\g\do\h\do\i\do\j\do\k\do\l\do\m\do\n\do\o\do\p\do\q\do\r\do\s\do\t\do\u\do\v\do\w\do\x\do\y\do\z\do\A\do\B\do\C\do\D\do\E\do\F\do\G\do\H\do\I\do\J\do\K\do\L\do\M\do\N\do\O\do\P\do\Q\do\R\do\S\do\T\do\U\do\V\do\W\do\X\do\Y\do\Z}
% \let\oldurl\url
% \renewcommand{\url}[1]{\begin{sloppypar}\oldurl{#1}\end{sloppypar}}
\setlength\columnsep{30pt}
\marginsize{30pt}{30pt}{10pt}{20pt}
\setmainfont{TeX Gyre Bonum}
\setCJKmainfont[BoldFont=Noto Serif CJK SC Bold, ItalicFont=FandolKai]{Source Han Sans SC}
\setlength{\parindent}{0cm}
% \setCJKmonofont{Noto Sans CJK SC}
\begin{document}
\begin{center}
    \Huge\textbf{南哪大专醒前消息}
\end{center}
\vspace{4mm}
\hrule
\renewcommand\tabularxcolumn[1]{m{#1}}
\begin{tabularx}{\textwidth}{>{\hsize.2\hsize}X>{\hsize.6\hsize}X>{\hsize.2\hsize}X}
    \begin{flushleft}
        2025.3.28\, No.204
    \end{flushleft}
    &
    \begin{center}
        \textit{“秉中持正、求新博闻。”}
    \end{center}
    &
    \begin{flushright}
        \textbf{南京市栖霞区}
    \end{flushright}
\end{tabularx}
\vspace{-3.5mm}
\hrule
\vspace{4mm}
% HEAD END
\centerline{\huge\textbf{活动预告}}
\begin{multicols}{2}
\section{订阅方式和加入编辑部}  
编辑部招聘人才,用爱发电,工作轻松,详情可联系QQ:1329527951 客服小千\\想订阅本消息或获取PDF版(便于查看超链接和往期),可加QQ群:\href{https://qm.qq.com/q/4HL41Nt3sQ}{466863272}.
\section{活动清单}
\setbox\colbbox\vbox{
\makeatletter\col@number\@ne
\begin{longtable}{|>{\centering\arraybackslash}m{.3\textwidth}|m{.06\textwidth}|m{.06\textwidth}|}
    \hline
    活动 & 开展时间 & 刊载时间\\
    \hline\hline
    南大版deepseek & / & 2.22\\
    悦读课程群 & / & 2.24\\
    eScience AI科研助手 & / & 3.11\\
    地科博物馆开放安排 & / & 3.22\\ 
    乐跑 & 5.16 & 3.10\\
    本科生劳育实践 & 7.20 & 2.19\\
    医保零星报销 & 3.31 & 2.19\\
    银星杯论文赛 & 4.22 & 2.27\\
    高教社杯 & 4.25 & 3.5\\
    南辩院系杯 & 4.12 & 3.6\\
    大文大理题目征集 & 期末 & 3.8\\
    基础学科论坛 & 4.20 & 3.9\\
    普通话测试 & 4.11 & 3.25\\
    外教社杯 & 5.27 & 3.12\\
    Python比赛 & 4.6 & 3.16\\
    本科生院征集大鸣大放 & 4.4 & 3.21\\
    两会知识竞赛 & 3.30 & 3.21\\
    纸鸢工作坊 & 4.3 & 3.22\\
    南大博篆刻体验课 & 4.2 & 3.23\\
    粤歌赛 & 4.12 & 3.24\\
    外词杯 & 3.31 & 3.25\\
    江苏创青春赛事 & 4.30 & 3.26\\
    石膏绘画活动 & 3.29 & 3.26\\
    悦读测试 & 4.6 & 3.27\\
    南大数学竞赛 & 4.15 & 3.27\\
    新生午餐会 & 3.30 & 3.28\\
    中美中心开放日报名&4.6&3.29\\
    中美中心开放日&4.9&3.29\\
    全球科考项目招募&4.6&3.29\\
    \hline
\end{longtable}
\unskip
\unpenalty
\unpenalty}\unvbox\colbbox
\end{multicols}
\begin{multicols}{2}
\pagebreak

\section{讲座}
\begin{tabular}{|>{\centering\arraybackslash}m{.3\textwidth}|m{.06\textwidth}|m{.06\textwidth}|}
    \hline
    讲座 & 开展时间 & 刊载时间\\
    \hline\hline
    如何影响消费者动机与心理健康支持 & 4.1 & 3.26\\\hline
    当本科生按下AI启动键 & 3.30 & 3.28\\\hline
    移民与流动研究的时间、情感和日常转向 & 4.1 & 3.28\\\hline
    AI: The Destruction of the Imagination? & 4.2 & 3.28\\\hline
\end{tabular}
%讲座预告写在这。用subsection
\subsection{ 秦汉乡里民众居住形态与人口迁移流动治理}
主讲人:卜宪群 教授
\\主持人:罗晓翔 教授
\\时间:2025年3月30日(周日)9:30
\\地点:南京大学历史学院223会议室
\\详见:\url{https://mp.weixin.qq.com/s/tj1LgPGH_hw3AwmGCu3G6A}

%此处写校级活动,请不要把讲座、院级活动和社团活动写在这里orz orz orz
\section{中美中心春季开放日}
活动时间: 2025年4月9日(星期三)14:00-15:00 线下宣讲  15:00-16:00 校园参观
\\主讲人:牛小虎(中美中心招生就业部主管)
\\南京大学中美文化研究中心将迎来2025年春季开放日活动。参加开放日活动,你将有机会了解中心的招生政策、课程设置以及就业服务。届时,你还可以参观校园,沉浸式体验书院式生活,以及中心“跨语言、跨文化、跨学科、跨国界”的独特氛围。欢迎广大对南京大学中美文化研究中心硕士项目、证书项目感兴趣的同学们、老师们、朋友们报名参与!参与方式:请扫描推文内二维码报名,4月6日截止报名。
\\详见:\url{https://mp.weixin.qq.com/s/Ib50yd0vsg7EwAUjVo6RzA}

\section{大数据产业发展调研实践——南京大学本科生全球科考项目成员招募通知}
一、项目主题与主要内容
\\课程学习(日本相关大学将为学生提供丰富的课程,讲座,讨论会)、企业参观(访问代表型大数据企业)、实地调研(对日本社会中大数据的应用进行深入调研)以及研讨交流等(组织学生围绕大数据产业发展进行讨论)。
\\二、项目实施时间
\\2025年7月下旬到8月上旬,历时10天左右。
\\三、项目费用
\\国际旅费、当地交通、住宿、保险费基本由项目承担。
\\项目实施期间的伙食费等其他费用由学生自理。
\\四、招募方式
\\● 招募对象:
\\南京大学在校本科一、二、三年级学生,12名左右。
\\● 申请要求:
\\1.政治立场正确,具有良好的政治素质和道德品行,组织纪律观念强。
\\2.对数据科学研究具有一定的知识储备与项目经历。
\\3.具有过往的社会实践或科考项目经历。
\\4.具备正常外出实地调研的身体条件和心理素质。
\\5.拥有较强的团队协作能力、善于沟通表达。
\\● 申请加分项:
\\英语、日语口语能力强;拥有公众号运营经验;具有中英文学术写作能力等。
\\● 招募方式:初筛 + 面试
\\● 申请方式:问卷星
\\● 咨询方式:025-89680235
\\● 截止时间:2025年4月6日晚12:00
\\● 申请结果:
\\4月10日前邮件回复。请获得面试通知的同学,请按通知的时间、地点参加线下面试。
\\报名方式即具体时间安排见原推
\\详见:\url{https://mp.weixin.qq.com/s/_8Yg7EDvhRZ6mjHHWlAtkg}

\section{“案牍存赤心——从援疆故事读懂援疆工作” ——南京大学本科生全球科考项目成员招募通知}
项目实施时间
\\2025年7月上旬到7月下旬,历时10天左右。
\\项目费用
\\当地交通、住宿、考察费用、保险费基本由项目承担。项目实施期间的伙食费等其他费用由学生自理。
\\招募方式
\\● 招募对象:
\\南京大学在校本科一、二、三年级学生,12名左右。
\\● 申请要求:
\\1. 政治立场正确,具有良好的政治素质和道德品行,组织纪律观念强。
\\2. 对援疆文化有浓厚兴趣,对援疆政策与援疆工作史有一定的了解;
\\3. 身心健康,具有过往的社会实践或科考项目经历;
\\4. 拥有较强的团队协作能力、善于沟通表达。
\\● 招募方式:初筛+面试
\\● 申请方式:问卷星
\\● 咨询方式:025-89680235
\\● 截止时间:2025年4月6日晚12:00
\\● 申请结果:
\\4月10日前邮件/短信回复。请获得面试通知的同学,按通知的时间、地点参加线下面试。
\\项目内容详见原推
\\详见:\url{https://mp.weixin.qq.com/s/cK-DK31-giCiAAj02M_7mQ}


\section{院级活动}
\begin{tabular}{|>{\centering\arraybackslash}m{.3\textwidth}|m{.06\textwidth}|m{.06\textwidth}|}
\hline
    活动 & 开展时间 & 刊载时间\\
    \hline\hline
    文院剧本创作研讨会 & 9.30 & 3.2\\
    物院征集课程指南 & 6.15 & 3.3\\
    地海征集春日影 & 6.15 & 3.14\\
    社院学术节 & 4.18 & 3.25\\
    生科栽培 & 3.30 & 3.25\\
    有训行知集体生日会 & 3.30 & 3.26\\
    电院趣运会 & 3.30 & 3.27\\
    马院春日音 & 3.30 & 3.28\\
    
    \hline
\end{tabular}


\section{社团活动}
\begin{tabular}{|>{\centering\arraybackslash}m{.3\textwidth}|m{.06\textwidth}|m{.06\textwidth}|}
    \hline
    社团活动 & 开展时间 & 刊载时间\\
    \hline\hline
    天文台开放日 & / & 1.6\\
    鸿新社捐书活动 & 3.30 & 3.17\\
    雁行南大x以伴云陪伴线上志愿者招募 & 3.30 & 3.24\\
    林泉CAC流行音乐会 & 3.30 & 3.26\\
    孤独症主题活动 & 3.30 & 3.27\\
    二剧招募 & 4.1 & 3.28\\
    悦读书社春季招新 & 3.30 & 3.28\\
    \hline
\end{tabular}
%这里是写社团活动的,社团活动就是由社团主办、主要针对社团内部人员的活动。不要把非社团活动写在这里。
\subsection{2025 Popping Crew Battle|谁是仙林震主}
时间:3.30 周日 19:30-21:00
\\地点:仙林校区方肇周体育馆健美操馆
\\详见:\url{https://mp.weixin.qq.com/s/IG6j-q81HA30shGXpZXe8g}
\subsection{今日战报\&明日赛程}
男篮院系杯小组赛
\\数学   :   软院
\\36   :   43
\\工程   :   地科
\\45   :   32
\\
\\女篮院系杯小组赛
\\软院   :   生科
\\35   :   08
\\
\\今日得分王
\\地科 6号 黄卓群 19分
\\软院 0号 冯缘 21分
\\
\\明日赛程
\\女篮院系杯小组赛
\\政管 vs 地海
\\16:00 - 17:00
\\地点:一组团篮球场
\\数理 vs 新传
\\19:00 - 20:00
\\地点:一组团篮球场
\\新生杯小组赛
\\工式 vs 化生
\\11:00 - 12:30
\\地点:鼓楼北园篮球场
\\软院 vs 匡院
\\13:00 - 14:30
\\地点:鼓楼北园篮球场
\\健雄 vs 医学
\\14:30 - 16:00
\\地点:鼓楼北园篮球场
\\地科 vs 电子
\\16:00 - 17:30
\\地点:鼓楼北园篮球场
\\行知 vs 数理
\\17:30 - 19:00
\\地点:鼓楼北园篮球场
\\文院 vs 计科
\\19:00 - 20:30
\\地点:鼓楼北园篮球场
\\详见:\url{https://mp.weixin.qq.com/s/viDXtvypd1b-wmCcg0n7FA}
\end{multicols}
\end{document}
