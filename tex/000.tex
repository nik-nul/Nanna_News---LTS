% HEAD BEGIN
\documentclass[letterpaper, 12pt]{article}
\newsavebox\colbbox
\usepackage{graphicx}
\usepackage{multicol}
\usepackage{anysize}
\usepackage{fontspec}
\usepackage[fontset=none]{ctex}
\usepackage{tabularx}
\usepackage{longtable}
\PassOptionsToPackage{hyphens}{url}
\usepackage[breaklinks=true, colorlinks=true]{hyperref}
\expandafter\def\expandafter\UrlBreaks\expandafter{\UrlBreaks\do\a\do\b\do\c\do\d\do\e\do\f\do\g\do\h\do\i\do\j\do\k\do\l\do\m\do\n\do\o\do\p\do\q\do\r\do\s\do\t\do\u\do\v\do\w\do\x\do\y\do\z\do\A\do\B\do\C\do\D\do\E\do\F\do\G\do\H\do\I\do\J\do\K\do\L\do\M\do\N\do\O\do\P\do\Q\do\R\do\S\do\T\do\U\do\V\do\W\do\X\do\Y\do\Z}
% \let\oldurl\url
% \renewcommand{\url}[1]{\begin{sloppypar}\oldurl{#1}\end{sloppypar}}
\setlength\columnsep{30pt}
\marginsize{30pt}{30pt}{10pt}{20pt}
\setmainfont{TeX Gyre Bonum}
\setCJKmainfont[BoldFont=Noto Serif CJK SC Bold, ItalicFont=FandolKai]{Noto Sans CJK SC}
\setlength{\parindent}{0cm}
% \setCJKmonofont{Noto Sans CJK SC}
\begin{document}
\begin{center}
    \Huge\textbf{南哪大专醒前消息}
\end{center}
\vspace{4mm}
\hrule
\renewcommand\tabularxcolumn[1]{m{#1}}
\begin{tabularx}{\textwidth}{>{\hsize.2\hsize}X>{\hsize.6\hsize}X>{\hsize.2\hsize}X}
    \begin{flushleft}
        2024.12.10\, No.140
    \end{flushleft}
    &
    \begin{center}
        \textit{“秉中持正、求新博闻。”}
    \end{center}
    &
    \begin{flushright}
        \textbf{南京市栖霞区}
    \end{flushright}
\end{tabularx}
\vspace{-3.5mm}
\hrule
\vspace{4mm}
% HEAD END
\centerline{\huge\textbf{活动预告}}
\begin{multicols}{2}
    \section{订阅方式和加入编辑部}  
编辑部招聘人才,用爱发电,工作轻松,详情可联系QQ:1329527951 客服小祥\\想订阅本消息或获取PDF版(便于查看超链接和往期),可加QQ群:\href{https://qm.qq.com/q/VXIW7fgsEe}{849644979}.
\section{Deadline Ongoing}
\setbox\colbbox\vbox{
\makeatletter\col@number\@ne
\begin{longtable}{|c|c|c|}
    \hline
    消息(未见ddl的,不刊) & 截止日期 & 刊载日期\\
    \hline\hline
    紫藤学刊征稿 & 12.15 & 10.22\\
    安邦征稿 & 1.12 & 11.16\\
    猫鼠大战 & 12.15 & 11.22\\
    创意物理实验竞赛 & 12.21 & 11.15\\
    仙林通宵自习室 & 1.12 & 11.26\\
    防艾同伴教育 & 12.15 & 11.29\\
    南京高校戏曲交流 & 12.15 & 12.2\\
    全国大学生家史大赛 & 1.31 & 12.2\\
    25年南大会学团报名 & 12.11 & 12.4\\
    女性劳动策展工坊 & 12.16 & 12.4\\
    防艾剧本杀 & 12.15 & 12.5\\
    金融消费者大赛 & 12.31 & 12.5\\
    药丸周边征稿 & 12.15 & 12.6\\
    四六级准考证打印 & 12.14 & 12.6\\
    花旗杯报名 & 1.3 & 12.6\\
    挑杯校园双选会 & 12.15 & 12.7\\
    南新读书会 & 12.11 & 12.7\\
    朋辈数模分享 & 12.15 & 12.7\\
    羊山公园环保市集 & 12.15 & 12.7\\
    C Through代码大赛 & 12.15 & 12.8\\
    数学公共课答疑报名 & 12.11 & 12.8\\
    教师实习经验分享会 & 12.14 & 12.9\\
    澳门回归对谈会 & 12.15 & 12.9\\
    午餐读书会 & 12.11 & 12.9\\
    健雄朋导分享会 & 12.11 & 12.9\\
    西安史学论坛征稿 & 3.20 & 12.9\\
    地海暖冬活动报名 & 12.11 & 12.9\\
    苗族民俗活动 & 12.11 & 12.10\\
    歌魅放映 & 12.14 & 12.10\\
    南苏新年晚会 & 12.15 & 12.10\\
    遇难同胞纪念活动 & 12.13 & 12.10\\
    民族文化分享会 & 12.12. & 12.10\\
    鼓楼旧书市集 & 12.14 & 12.10\\
    重唱英文评诗会 & 12.21 & 12.10\\
    叶嘉莹纪念征稿 & 12.25 & 12.10\\
    \hline
\end{longtable}
\unskip
\unpenalty
\unpenalty}\unvbox\colbbox
\end{multicols}
\hrule
\pagebreak
\begin{multicols}{2}

\section{讲座}
\begin{tabular}{|c|c|c|}
    \hline
    往期讲座 & 开展日期 & 刊载日期\\
    \hline\hline
    《全球视野中的加...》& 12.13 & 12.10\\
    《专利查新与规避...》 & 12.19 & 10.3\\
    basics on ... & 12.11 & 12.2\\
    《在线时尚零售的...》 & 12.11 & 12.6\\
    《中国文化中的点心》& 12.11 & 12.8\\
    《比较逻辑与社会...》 & 12.11 & 12.8\\
    《读雷吉斯•德布雷的媒介学》 & 12.13 & 12.9\\
    《佛教量论因明学...》 & 12.11 & 12.9\\
    《英帝国及其抒情...》 & 12.11 & 12.9\\
    《waveguide...》 & 12.12 & 12.9\\
    《数学研究中的what...》 & 12.11 & 12.9\\
    《读雷吉斯德布雷...》 & 12.13 & 12.9\\
    《GPU coroutines...》 & 12.12 & 12.10\\
    《前向推理的一阶...》 & 12.18 & 12.10\\
    《现代诗创作漫谈》 & 12.11 & 12.10\\
    \hline
\end{tabular}

1.书写的力量:全球视野中的加勒比文学\\
讲座嘉宾:苏娉,华南理工大学教授、博士生导师\\
主持人:吴维忆,南京大学艺术学院副教授\\
时间:2024年12月13日16:10-18:00\\
地点:鼓楼校区逸夫馆3-203\\
主办单位:南京大学艺术学院(讲座编辑:西野明日风)\\

2.GPU Coroutines for Flexible Splitting and Scheduling of Rendering Tasks\\
报告人:郑少锟 清华大学计算机系博士生\\
时间:2024年12月12日(星期四)13:30\\
腾讯会议ID:365-196-288\\

3.前向推理的一阶可定义性\\
报告人:刘达欣 南京大学人工智能学院助理教授\\
时间:2024年12月18日(星期三)12:30\\
地点:计算机科学与技术楼111室\\
腾讯会议ID:928-345-478\\

4.指向月亮的那根“手指”:再说媒介学 ——读雷吉斯•德布雷的媒介学\\
主题:指向月亮的那根“手指”:再说媒介学——读雷吉斯•德布雷的媒介学\\
主讲人:黄旦 浙江大学文科资深教授 浙江大学数字沟通研究中心主任\\
主持人:孙   玮 复旦大学新闻学院教授 复旦大学信息与传播研究中心主任\\
时间:2024年12月13日(周五)上午10:00-12:00\\
地点:仙林校区新闻传播学院紫金楼215\\
具体链接:\url{https://mp.weixin.qq.com/s/B3eAMDSkbwFqSkKnVInyzQ}\\

5.冯娜:一首诗在哪里现身?——现代诗创作漫谈\\
时间:2024年12月11日下午15:00\\
地点:南京大学文学院323会议室\\
主持人:刘驰(诗人,南京大学特任副研究员)\\
与谈人:杨文佳(南京大学重唱诗社社长)、姚星宇(南京大学重唱诗社副社长)\\
活动详情见推文。\url{https://mp.weixin.qq.com/s/Z3moTmz2RZH5hHYOqOk9w}\\
\section{活动预告 | 毽舞绣韵味悠长(苗毽体验、绣球制作)}
苗族手毽体验活动:\\
活动时间:12月11日(周三)下午16:00-17:30\\
活动地点:南京大学鼓楼校区羽毛球馆、乒乓球房\\
绣球制作活动:\\
活动时间:12月11日(周三)晚18:30-20:30\\
活动地点:鼓楼校区新教学楼501\\
活动详情及报名二维码:\url{https://mp.weixin.qq.com/s/8SUlSjZGIoOOxQczgMtFzA}

\section{放映会 | 《汉密尔顿》}
歌声魅影\\
时间:12月14日 19:30\\
地点:仙1-115

\section{南苏新年晚会}
晚会时间:2024年12月15日(周日)19:00\\
晚会地点:南雍楼多功能厅\\
领票时间:12月12日 12:00\\
领票地点:16食堂一楼大厅\\
节目预告、抽奖信息及详情:\url{https://mp.weixin.qq.com/s/G00yQQjTPhc1yP0SXu-4sQ}\\

\section{“悼殇感怀,抚念承志”南京大屠杀死难同胞丛葬地纪念活动}
时间:12月13日(周五)下午13:00。\\
集合地点: 南京大学仙林校区南门。\\
交通方式:集中坐地铁前往起点。徒步路线全程约4公里。\\
活动路线:将沿“北极阁纪念碑-金陵大学丛葬纪念碑-拉贝故居”路线进行公祭献花和缅怀,以表达对南京大屠杀遇难同胞的深切哀思。\\
报名方式:详见\url{https://mp.weixin.qq.com/s/-Ao6VAj_gbBo80GTgJdrUg}\\
特别提醒:寻访当日请着装深色(以黑色为宜)\\

\section{民族文化社会实践分享会}
在中华文化节暨民族团结进步宣传月期间,学生会特邀“悦云同枝”“石榴籽奋进行”“苏阳耀新,蜀星映疆”三支优秀实践团队,讲述他们的实践故事,进一步铸牢中华民族共同体意识。\\
活动时间:12月12日19:00-20:00\\
活动地点:南京大学鼓楼校区新教学楼405\\
活动详情及报名:\url{https://mp.weixin.qq.com/s/ii9nvfpR2g8VFZkFNeBbrg}\\
\section{金陵旧书市集}
时间:12月12日(周四)至12月14日(周六)10:00——17:30\\
地点:鼓楼校区图书馆\\

\section{NEC×重唱诗社|英文评诗会}
时间:12月21日 周六14:00\\
地点:请加入活动群后续将发布通知\\
主讲人:外国语学院 孙红卫老师\\
参与对象:对谈人(从投稿者中选出)和与谈人(30人)\\
投稿邮箱:aichongchang@163.com,主题请注明“英文评诗会+姓名+院系年级”\\
截稿时间:12月18日 24:00\\
活动详情、投稿流程详见推文。\url{https://mp.weixin.qq.com/s/9z_GJ2JNmnKpr1HhVwAWsg}\\


\section{关于在线征集纪念叶嘉莹先生诗词、楹联、文章的通知}
征稿时间:11月25日—12月25日\\
展示时间:11月25开始,所有作品将在空间实时展示;优秀作品,将推荐至相关刊物发表。\\
活动详情、投稿要求及方式详见推文。\url{https://mp.weixin.qq.com/s/B8yNcXmcJ8PgXcq4NNMCHg}(消息编辑:等离子的王N)\\

\section{数学学院师生“饺香满屋,岁暖冬临”包饺子活动}
时间:12月23日(周一)16:00-18:00\\
地点:鼓楼校区南芳园餐厅(学生一、二、三食堂楼上)\\
报名对象:数学学院全体教师、本研学生\\
报名截止时间:12月17日 24:00前\\
进入原文扫码报名\url{https://mp.weixin.qq.com/s/1FqQFid5PG1R8JxfZxmNLg}

\section{“天问镜象”系列主题摄影征稿比赛}
活动对象:天文与空间科学学院全体师生\\
作品征集:2024年12月9日至2024年12月17日\\
主题: “天问邀请函”\\
摄影内容:星空摄影、科研相关照片、生活场景等,一切与“天文人生活”相关的内容均可。\\
优秀作品将于下学期在天空院院楼大厅进行展板展览。\\
奖品包括雨伞、抱枕、帆布袋、钥匙扣等。\\
详细要求和奖品等见原文\url{https://mp.weixin.qq.com/s/HeBR0jmagHqbLM6toUY0Qw}

\section{行知书院|“人工智能实践应用”课程任务暨首届文献综述大赛}
"与人工智能通识核心课相配套,学校同时开设《人工智能实践应用》课程,鼓励本科新生充分运用人工智能工具,同自身专业领域相结合"\\
1.课程任务开展指导\\
时间:12月11日(周三)下午16:10\\
地点:鼓楼校区新教学楼207\\
主讲人:段巍  南京大学商学院产业经济学系副教授\\
于京东  南京大学政府管理学院副教授\\
宋宁远  南京大学信息管理学院准聘助理教授\\
内容:讲授“AI+社会科学”相关知识\\
2.行知书院:请各位同学选择下列的一个主题,仔细阅读主题要求,撰写一篇不少于1500字的文献综述作为课程作业提交\\
主题1(商学院)人工智能与劳动收入分配\\
主题2(法学院)规制“数字利维坦”:人工智能的立法路径选择\\
主题3(政府管理学院)人工智能与公共治理的关系\\
主题4(信息管理学院)数字人文:数据与智能环境下的人文社科发展转向\\
主题5(社会学院)从技术到伦理:生成式人工智能的多元社会影响\\
请全体2024级同学于2025年1月12日(周日)24:00前在Table链接中上交作业\\
寒假期间,同学们可以进一步对文献综述进行完善。有意愿参与文献综述评优的同学,请于2025年2月28日(周五)24:00前,将作品上传至Box链接(顺利评优有证书和奖品)\\
综述要求和提交链接详见\url{https://mp.weixin.qq.com/s/9ZeLV0vXodEZJ4LpBq-stw}(小编:草昌思)\\



\end{multicols} 

\end{document}