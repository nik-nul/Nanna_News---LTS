% HEAD BEGIN
\documentclass[letterpaper, 12pt]{article}
\usepackage{graphicx}
\usepackage{multicol}
\usepackage{anysize}
\usepackage{fontspec}
\usepackage[fontset=none]{ctex}
\usepackage{tabularx}
\PassOptionsToPackage{hyphens}{url}
\usepackage[breaklinks=true, colorlinks=true]{hyperref}
\expandafter\def\expandafter\UrlBreaks\expandafter{\UrlBreaks\do\a\do\b\do\c\do\d\do\e\do\f\do\g\do\h\do\i\do\j\do\k\do\l\do\m\do\n\do\o\do\p\do\q\do\r\do\s\do\t\do\u\do\v\do\w\do\x\do\y\do\z\do\A\do\B\do\C\do\D\do\E\do\F\do\G\do\H\do\I\do\J\do\K\do\L\do\M\do\N\do\O\do\P\do\Q\do\R\do\S\do\T\do\U\do\V\do\W\do\X\do\Y\do\Z}
% \let\oldurl\url
% \renewcommand{\url}[1]{\begin{sloppypar}\oldurl{#1}\end{sloppypar}}
\setlength\columnsep{30pt}
\marginsize{30pt}{30pt}{10pt}{20pt}
\setmainfont{TeX Gyre Bonum}
\setCJKmainfont[BoldFont=Noto Serif CJK SC Bold, ItalicFont=FandolKai]{Noto Sans CJK SC}
\setlength{\parindent}{0cm}
% \setCJKmonofont{Noto Sans CJK SC}
\begin{document}
\begin{center}
    \Huge\textbf{南哪大专醒前消息}
\end{center}
\vspace{4mm}
\hrule
\renewcommand\tabularxcolumn[1]{m{#1}}
\begin{tabularx}{\textwidth}{>{\hsize.2\hsize}X>{\hsize.6\hsize}X>{\hsize.2\hsize}X}
    \begin{flushleft}
        2024.10.5\, No.81
    \end{flushleft}
    &
    \begin{center}
        \textit{“克明峻德。”}
    \end{center}
    &
    \begin{flushright}
        \textbf{苏州市高新区}
    \end{flushright}
\end{tabularx}
\vspace{-3.5mm}
\hrule
\vspace{4mm}
% HEAD END
\centerline{\huge\textbf{活动预告}}
\begin{multicols}{2}
\section{Deadline Ongoing}
\begin{tabular}{|c|c|c|}
    \hline
    消息(未见ddl的,不刊) & 截止日期 & 刊载日期\\
    \hline\hline
    仙林校史馆招募讲解员 & 10.30 & 9.12\\
    国优计划报名 & 10.7 & 9.19\\
    本科生暑期课程评教 & 10.31 & 9.19\\
    网易雷火大赛 & 10.7 & 9.22\\
    大创训练计划申报 & 11.18 & 9.24\\
    苏州校区音乐会 & 10.19 & 9.25\\
    外院国庆摄影征集 & 10.7 & 9.25\\
    港澳台生中华文化大赛 & 10.9 & 9.26\\
    心理中心征稿 & 10.10 & 9.28\\
    周末剧场 & 10.10 & 9.28\\
    历史学院国庆活动 & 10.7 & 9.28\\
    台湾地区交换项目 & 10.7 & 9.29\\
    第十九届大挑 & 10.15 & 9.30\\
    声谷创新基金 & 10.18 & 9.30\\
    软院国庆桌游会 & 10.7 & 9.30\\
    国际化处全媒体招新 & 10.8 & 9.30\\
    午餐读书会 & 10.10 & 9.30\\
    鹰角校招宣讲 & 10.15 & 10.2\\
    软院国庆活动 & 10.7 & 10.2\\
    计院文创设计 & 10.8 & 10.2\\
    大专戏曲知识竞赛 & 10.20 & 10.2\\
    MathGlue导员招募 & 10.7 & 10.2\\
    健雄书院征集主题作品 & 10.7 & 10.3\\
    EBSCO数据库检索大赛 & 11.20 & 10.3\\
    Passion街舞公开课 & 10.9 & 10.3\\
    \hline
\end{tabular}
\begin{tabular}{|c|c|c|}
    \hline
    消息(未见ddl的,不刊) & 截止日期 & 刊载日期\\
    \hline\hline
    乒乓球新生杯报名 & 10.8 & 10.4\\
    炜华音乐跑 & 12.8 & 10.4\\
    台港澳交流协会招新 & 10.10 & 10.5\\
    马院主题宣讲报名 & 10.25 & 10.5\\
        \hline
\end{tabular}
\section{订阅方式和加入编辑部}
编辑部招聘人才,用爱发电,工作轻松,详情可联系QQ:1329527951 客服小祥\\想订阅本消息或获取PDF版(便于查看超链接),可加QQ群:\href{https://qm.qq.com/q/FGX1VYCrGS}{962626571}.
\section{讲座}
\begin{tabular}{|c|c|c|}
    \hline
    往期讲座 & 开展日期 & 刊载日期\\
    \hline\hline
    《聚合物的研发与...》 & 10.24 & 10.3\\
    《电池及电化学能...》 & 11.24 & 10.3\\
    《专利查新与规避...》 & 12.19 & 10.3\\
    《ChatGPT和生成...》 & 10.9 & 10.3\\
      \hline
\end{tabular}\\\\

\section{【研究生】2024秋季四院联合篮球赛}
参赛对象:南京大学地理与海洋科学学院、地球科学与工程学院、电子科学与工程学院、计算机学院全体研究生。\\
比赛形式:各学院最多派出3支队伍。每场比赛每支队伍同时至少有两位新生在场上。赛制采取小组赛+淘汰赛的形式,淘汰赛进行单败淘汰制,总计比赛15场。中场时段每队出女生定点投球,共30球,进一球得一分,计入球队总成绩。\\
比赛时间:2024年10月15-29日(暂定)\\
比赛地点:南京大学仙林校区一组团篮球场\\
报名截止日期为:10月10日晚20:00\\
报名方式:进入链接中找到报名问卷填写,或联系各院系负责人登记\url{https://mp.weixin.qq.com/s/GTEk5mJ6IUVI35Km2OjIwQ}
\section{【活动预告】选调校友经验谈:北京市选调专场}
南大基层研究会将于10月7日(下周一)晚19点在线上举办“选调校友经验谈——北京市选调专场”分享会,主要面向有意向报考北京选调的同学进行经验分享。\\
嘉宾1:2024届社会学院毕业生\\
在校经历:作为社会学院代表参与南京大学第五次研究生代表大会;多次获评校级荣誉。\\
嘉宾2:2024届物理学院毕业生\\
在校经历:多次获评校级荣誉;曾获国家奖学金。\\
嘉宾3:2024届建筑与城市规划学院毕业生\\
在校经历:曾获评校级荣誉。\\
报名二维码及微信群见链接:\url{https://mp.weixin.qq.com/s/ben36TagMTreItAeqWNH_Q}
\section{台港澳交流协会招新}
南京大学台港澳交流协会是在台港澳事务办公室指导下,以台港澳籍的交流生、学籍生、短期访问学生等为主要服务对象,旨在促进大陆学子与台港澳青年密切交流、共同学习、互助成长,助力海峡两岸暨港澳地区文化的交流与融合,传承与创新。\\

招新部门:

1.秘书处:负责协会的日常运营工作,包括社团年度计划安排、日常活动筹备、重点文案策划、日常会议、月报,社团年审等。

2.活动部:负责台港澳交流协会活动策划和执行,记录协会个人表现等日常性事务。

3.外联部:负责对外联系和拓展校内外合作平台,策划开展大小规模的座谈及联谊活动、保持与兄弟院校良好关系等。

4.新媒体中心:运营公众号“南大台港澳交流”,协助台港澳事务办公室及台港澳交流协会宣传工作,旨在整合并发布交流信息,聚焦校园活动等。\\

报名截止时间:10月10日晚8点

详见\url{https://mp.weixin.qq.com/s/FTmupMDExM87Owl02c-zPg}
\section{马院主题宣讲比赛报名}
南京大学马克思主义学院将举办“七十五载迎盛世,青年奋进正当时”主题宣讲比赛,马院全体党支书、团支书、青年宣讲团讲师必须参与,报名截止日期为10月25日。因仅允许马院学生参与,故不予详载,详阅:\url{https://mp.weixin.qq.com/s/K0UAM0nwLAscCfzhynMePQ}。
\end{multicols} 
\end{document}