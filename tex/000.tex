% HEAD BEGIN
\documentclass[letterpaper, 12pt]{article}
\newsavebox\colbbox
\usepackage{graphicx}
\usepackage{multicol}
\usepackage{anysize}
\usepackage{fontspec}
\usepackage[fontset=none]{ctex}
\usepackage{tabularx}
\usepackage{longtable}
\PassOptionsToPackage{hyphens}{url}
\usepackage[breaklinks=true, colorlinks=true]{hyperref}
\expandafter\def\expandafter\UrlBreaks\expandafter{\UrlBreaks\do\a\do\b\do\c\do\d\do\e\do\f\do\g\do\h\do\i\do\j\do\k\do\l\do\m\do\n\do\o\do\p\do\q\do\r\do\s\do\t\do\u\do\v\do\w\do\x\do\y\do\z\do\A\do\B\do\C\do\D\do\E\do\F\do\G\do\H\do\I\do\J\do\K\do\L\do\M\do\N\do\O\do\P\do\Q\do\R\do\S\do\T\do\U\do\V\do\W\do\X\do\Y\do\Z}
% \let\oldurl\url
% \renewcommand{\url}[1]{\begin{sloppypar}\oldurl{#1}\end{sloppypar}}
\setlength\columnsep{30pt}
\marginsize{30pt}{30pt}{10pt}{20pt}
\setmainfont{TeX Gyre Bonum}
\setCJKmainfont[BoldFont=Noto Serif CJK SC Bold, ItalicFont=FandolKai]{Noto Sans CJK SC}
\setlength{\parindent}{0cm}
% \setCJKmonofont{Noto Sans CJK SC}
\begin{document}
\begin{center}
    \Huge\textbf{南哪大专醒前消息}
\end{center}
\vspace{4mm}
\hrule
\renewcommand\tabularxcolumn[1]{m{#1}}
\begin{tabularx}{\textwidth}{>{\hsize.2\hsize}X>{\hsize.6\hsize}X>{\hsize.2\hsize}X}
    \begin{flushleft}
        2024.11.20\, No.122
    \end{flushleft}
    &
    \begin{center}
        \textit{“Vis ex acta Deus ex machina.”\\“新闻赋能机械降神”}
    \end{center}
    &
    \begin{flushright}
        \textbf{南京市栖霞区}
    \end{flushright}
\end{tabularx}
\vspace{-3.5mm}
\hrule
\vspace{4mm}
% HEAD END
\centerline{\huge\textbf{活动预告}}
\begin{multicols}{2}
    \section{订阅方式和加入编辑部}  
编辑部招聘人才,用爱发电,工作轻松,详情可联系QQ:1329527951 客服小祥\\想订阅本消息或获取PDF版(便于查看超链接和往期),可加QQ群:\href{https://qm.qq.com/q/VXIW7fgsEe}{849644979}.
\section{Deadline Ongoing}
\setbox\colbbox\vbox{
\makeatletter\col@number\@ne
\begin{longtable}{|c|c|c|}
    \hline
    消息(未见ddl的,不刊) & 截止日期 & 刊载日期\\
    \hline\hline
    紫藤学刊征稿 & 12.15 & 10.22\\
    乐跑 & 12.6 & 10.12\\
    国际访学计划申报 & 11.22 & 10.22\\
    秉文心理短视频 & 11.25 & 11.3\\
    DIY课程学术论坛征稿 & 11.30 & 11.13\\
    国风歌曲演唱赛 & 12.1 & 11.13\\
    牡丹亭庆演 & 12.1 & 11.13\\
    创业集市摊位 & 11.22 & 11.14\\
    II剧 & 11.23 & 11.14\\
    潘高峰南大分享会 & 11.23 & 11.16\\
    普通话测试网络报名 & 11.30 & 11.16\\
    商院RPG活动 & 11.23 & 11.16\\
    安邦征稿 & 1.12 & 11.16\\
    新传院迎新晚会征集 & 11.25 & 11.17\\
    秉文宿舍风采 & 12.1 & 11.17\\
    商院生涯论坛 & 11.24 & 11.18\\
    六朝博物馆访学 & 11.22 & 11.18\\
    秋日手作游园会 & 11.23 & 11.18\\
    黑匣招募 & 11.25 & 11.18\\
    邮局影映 & 11.21 & 11.18\\
    重唱诗社评诗会 & 11.24 & 11.19\\
    鼓楼中医义诊 & 11.23 & 11.19\\
    古琴社露天音乐会 & 11.24 & 11.19\\
    心协有奖征稿 & 11.25 & 11.19\\
    日俱影映 & 11.24 & 11.19\\
    南红会衣物捐赠 & 11.22 & 11.20\\
    法学院征诗活动 & 12.2 & 11.20\\
    招生宣传大赛投票 & 11.21 & 11.20\\
    高研院工作坊 & 11.23 & 11.20\\
    计院乒赛 & 11.26 & 11.20\\
    心理中心开放日 & 11.23 & 11.20\\
    歌魅影映 & 11.24 & 11.20\\
    平安留学交流会 & 12.3 & 11.20\\
    社计院联议 & 11.24 & 11.20\\
    音乐史分享会 & 11.24 & 11.20\\
    德俱口语角 & 11.23 & 11.20\\
    \hline
\end{longtable}
\unskip
\unpenalty
\unpenalty}\unvbox\colbbox
\end{multicols}
\hrule
\pagebreak
\begin{multicols}{2}

\section{讲座}
\begin{tabular}{|c|c|c|}
    \hline
    往期讲座 & 开展日期 & 刊载日期\\
    \hline\hline
    《电池及电化学能...》 & 11.24 & 10.3\\
    《专利查新与规避...》 & 12.19 & 10.3\\
    图书馆系列讲座 & 12.3 & 10.20\\
    《解码黑猴背景音乐...》 & 11.21 & 11.14\\
    《量子计算的科普...》 & 11.22 & 11.16\\
    《学术写作入门...》& 11.21 & 11.18\\
    《中华伦理文明的...》& 11.22 & 11.18\\
    《谈当代中国水墨...》& 11.22 & 11.18\\
    《文献检索交流分...》 & 11.21 & 11.18\\
    《超越Transformer》 & 11.23 & 11.19\\
    《战争记忆的跨文...》 & 11.22 & 11.19\\
    《作为不莱梅商业...》 & 11.21 & 11.19\\
    《作为媒介的人工...》 & 11.22 & 11.19\\
    《日本侵华与中国...》 & 11.24 & 11.20\\
    《文献资源检索...》 & 11.24 & 11.20\\
    
    \hline
\end{tabular}

1.徐勇、臧运祜:日本侵华与中国抗战 双人对谈
主讲人:徐勇:日本侵华史研究若干问题臧运祜:关于抗战时期沦陷区的研究\\
主持人:张生 南京大学历史学院院长、教授\\
时 间:11月24日(周日)上午9:00-11:00\\
地 点:南京大学历史学院133会议室\\

2.启明丛谈:文献资源检索、获取与利用\\
时间:2024年11月24日(周日)15:00\\
地点:南京大学鼓楼校区逸夫馆Ⅰ-204教室\\
主讲人:李铭锐(文学院中国古典文献学专业2024级硕士研究生,本科就读于汉语言文学(古文字学方向)专业,曾任悦读书社社长(2022-2023学年),学术兴趣为中国古典文献学、唐宋文献)\\
邵镕(新闻传播学院新闻学专业2021级本科生,保研至清华大学新闻与传播学院传播学专业,曾获基础学科论坛三等奖,学术兴趣为健康传播、新媒体传播。)\\
分享主题:古典文献的现代利用、写好论文,从找好文献开始\\
QQ群和问卷二维码见\url{https://mp.weixin.qq.com/s/LTXsrijAKH_Q8JhNb8UDhQ}



\section{衣物捐赠}
南京大学红十字会联合南京大学商学院举办2024年“冬衣送暖”衣物捐赠活动。同学们捐赠的冬衣,会在经过统一整理后寄送给对口支教地的孩子们。\\
捐赠时间地点:\\
鼓楼校区:\\
地点:南青格庐多功能教室\\
时间:11月22日、23日、24日 18-21点\\
仙林校区:\\
地点:敬文学生活动中心三楼校红会工位\\
时间:11月22日、23日、24日 18-21点\\
浦口校区:\\
地点:浦1-212\\
时间:11月22日、23日、24日 18-21点\\
苏州校区:\\
地点:南雍楼东116\\
时间:11月23日、24日 18-21点\\
在捐赠现场,每一位同学都可以领取明信片,将祝福远寄给未曾会面的朋友。\\
明信片将给予志愿时长为0.5h,在活动结束后,会根据捐赠者捐赠的冬衣质量、明信片制作等情况增加志愿时长。凡参与本次“冬衣送暖”捐赠活动的小伙伴都有精美小礼品。\\
详见\url{https://mp.weixin.qq.com/s/jWbLdwkGQjoVx2_htzUoHg}


\section{法学院秋日三行诗征集活动}
活动对象:南京大学法学院全体学生
活动时间2024年11-12月
活动规则:作品以三行诗格式呈现,内容积极向上,任选以下三个主题之一进行创作:校园秋影绘韶华、法学逐梦谱新章、思想光辉润心田\\
提交时间:12月2日18:00前\\
提交规则:将作品以word文件形式提交至box链接\url{https://box.nju.edu.cn/u/d/9daaf55c4a5d47139c36/},并注明作品名称、姓名、联系方式。\\
详情见\url{https://mp.weixin.qq.com/s/SaQ73OhEsaIYpBa61bKdsg}
\section{招生宣传创意征集大赛投票}
“点亮南星”招生宣传创意征集大赛自10月21日启动以来,共收集到91件优秀的创意作品,其中“讲好南大故事”主赛道87件,“寻根性办学”专项赛道4件;影音图像组25件,视觉设计组22件,综合资料组29件,绘画写作组14件,智能开发组1件。\\
作品将按照专业评审70%、网络投票30%的比例计算作品最终得分,并遴选10件左右的特等奖作品进入现场决赛。\\
投票说明:\\
1、每人至多可投8票,不区分赛道;\\
2、投票截止时间:2024年11月21日24:00。\\
投票链接详见\url{https://mp.weixin.qq.com/s/wZw6TaDB1rg23ksxqJYqpA}

\section{主题: “超越见证:数字记忆的情境伦理”工作坊}
总召集人:孙玮 复旦大学信息与传播研究中心主任 复旦大学新闻学院教授\\
召集人:李红涛复旦大学信息与传播研究中心副主任 复旦大学新闻学院教授\\
       刘于思 南京大学新闻传播学院教授 复旦大学信息与传播研究中心研究员\\
时间:2024年11月23日(周六)8:40-16:40\\
地点:鼓楼校区逸夫馆9楼高研院报告厅\\
\section{社、计院联谊活动}
 “共话人生,‘社’‘计’未来”社会学院与计算机学院联谊活动\\
计算机学院与社会学院携手举办“i人友好”的联谊活动。\\
时间:11月24日(周日)14:00开始全程大约2-3小时\\
地点:(具体地点在活动群另行通知)\\
1.招募执行导演(5名左右)\\
2.本次活动将与"南大招生小蓝鲸视频运营部"合作推出微综艺节目\\
报名方式见推文:\url{https://mp.weixin.qq.com/s/rmdqf9_iscvZM_z3L-2JxQ}
\section{计算机学院第三届“师生杯”乒乓球赛}
比赛时间:\\
11月26日 13:00-18:00\\
比赛地点:\\
方肇周体育馆乒乓球馆\\
面向对象:\\
计算机学院全体师生\\
比赛分为竞技赛和趣味赛\\
现招募选手和裁判志愿者\\
报名截止日期:【2024年11月24日12点前】\\
报名方式和交流群见链接\url{https://mp.weixin.qq.com/s/3Fjwph13C8ok326xbrzt2w}

\section{心理中心开放日活动}
时间:2024年11月23日10:00-15:00\\
地点:仙林校区敬文活动中心9-10层\\
南京大学心理健康教育与研究中心\\
开放内容:\\
深入参观:近距离接触并了解我校心理健康教育工作的实施情况,包括心理咨询室的布局、功能及日常运作。\\
亲身体验:通过参与特色体验活动,如手工落叶制作、漆扇制作、沙盘等,同学们将亲身体验到心理健康服务的魅力与实效。\\
原文:\url{https://mp.weixin.qq.com/s/e9p0hOxfkkxEM_X1gP_PYw}

\section{放映会 |《马戏之王》}
主办方:歌声魅影\\
11月24日 19:00-21:00\\
仙I-116\\
简介:《马戏之王》(The Greatest Showman)是一部由迈克尔·格雷西执导,休·杰克曼主演的音乐剧电影,讲述了美国马戏团大亨菲尼尔斯·泰勒·巴纳姆,如何从穷困潦倒到开创马戏团、成为“马戏之王”的传奇故事。这部电影以令人印象深刻的音乐制作、大牌明星的倾力演出、歌舞的精心编排,一举获得奥斯卡多项提名,收获了金球奖“最佳原创歌曲”奖项。

\section{师生平安留学系列交流会}
为全面贯彻落实党的二十大精神和全国教育大会精神,进一步增强出国留学人员的安全防范意识及风险应对能力,帮助在外学子以充分的准备和积极的心态迎接即将到来的留学生活,南京大学国际化工作处、教育部出国留学培训与研究中心、物理学院、医学院及工程管理学院决定将于2024年12月3日联合举办“留学梦,报国志”南京大学师生平安留学系列交流会。\\
时间:2024年12月3日 16:00\\
地点:南京大学鼓楼校区唐仲英楼 B501\\
主讲人:戴者华,南京大学教育部出国留学培训与研究中心主任,中国驻加拿大大使馆原教育外交官。\\
点评人:陈志云,南京大学工程管理学院党委书记,中国驻俄罗斯大使馆原教育外交官。\\
请各位有意向参加的老师同学扫描二维码进行报名登记。\\
二维码见原文:\url{https://mp.weixin.qq.com/s/Bwe76BVUyBWpem8B2NJFPA}

\section{“口语角“活动}
SIECA和德语俱乐部社团将举办“口语角“活动,同学们可以在这里自由交谈,练习口语,了解不同的文化,认识更多外国朋友!\\
时间:11月23日(本周六)下午3点至下午5点左右。\\
地点:在仙林校区(具体地点稍后更新)\\
活动内容:\\
1.您可以选择英语组或德语组。每个组都有几个小组(3-5人一组,至少有一名交换生)。\\
2.给出十个话题,选择一个你和你的队友喜欢谈论的话题,并准备在活动中讨论它。\\
3.讨论部分将分为两个部分:外语时间(英语/德语)和中文时间,这样每个人都有机会练习!\\
4.在讨论结束后,请向我们反馈参加活动的心得和对于活动的建议以便我们以后能将这个活动办的更好,持续每周举办。\\
门:每位参与者都将获得一个小礼物!

\section{音乐史分享会}
标题:30年代中国音乐史\\
时间:11月24日 19:30\\
地点:仙林校区(待定)\\
简介:“文章合为时而著 歌诗合为事而作”,本分享会将回顾上世纪20-30年代中国左翼音乐发展道路,追寻艺术与时代的内在关联。分享会由同学自发举办。\\
详情可加入QQ群了解。QQ群:955123067

\section{乐跑}
从明日(11月21日)算起,还有16次乐跑机会。\\
也就是说,若还有乐跑一次未跑的新生同学,若后续不全勤,将注定无法跑完16次。\\
\end{multicols} 
\end{document}