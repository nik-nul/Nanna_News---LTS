% HEAD BEGIN
\documentclass[letterpaper, 12pt]{article}
\newsavebox\colbbox
\usepackage{graphicx}
\usepackage{multicol}
\usepackage{anysize}
\usepackage{fontspec}
\usepackage[fontset=none]{ctex}
\usepackage{tabularx}
\usepackage{longtable}
\PassOptionsToPackage{hyphens}{url}
\usepackage[breaklinks=true, colorlinks=true]{hyperref}
\expandafter\def\expandafter\UrlBreaks\expandafter{\UrlBreaks\do\a\do\b\do\c\do\d\do\e\do\f\do\g\do\h\do\i\do\j\do\k\do\l\do\m\do\n\do\o\do\p\do\q\do\r\do\s\do\t\do\u\do\v\do\w\do\x\do\y\do\z\do\A\do\B\do\C\do\D\do\E\do\F\do\G\do\H\do\I\do\J\do\K\do\L\do\M\do\N\do\O\do\P\do\Q\do\R\do\S\do\T\do\U\do\V\do\W\do\X\do\Y\do\Z}
% \let\oldurl\url
% \renewcommand{\url}[1]{\begin{sloppypar}\oldurl{#1}\end{sloppypar}}
\setlength\columnsep{30pt}
\marginsize{30pt}{30pt}{10pt}{20pt}
\setmainfont{TeX Gyre Bonum}
\setCJKmainfont[BoldFont=Noto Serif CJK SC Bold, ItalicFont=FandolKai]{Noto Sans CJK SC}
\setlength{\parindent}{0cm}
% \setCJKmonofont{Noto Sans CJK SC}
\begin{document}
\begin{center}
    \Huge\textbf{南哪大专醒前消息}
\end{center}
\vspace{4mm}
\hrule
\renewcommand\tabularxcolumn[1]{m{#1}}
\begin{tabularx}{\textwidth}{>{\hsize.2\hsize}X>{\hsize.6\hsize}X>{\hsize.2\hsize}X}
    \begin{flushleft}
        2024.11.6\, No.110
    \end{flushleft}
    &
    \begin{center}
        \textit{“克明峻德。”}
    \end{center}
    &
    \begin{flushright}
        \textbf{南京市栖霞区}
    \end{flushright}
\end{tabularx}
\vspace{-3.5mm}
\hrule
\vspace{4mm}
% HEAD END
\centerline{\huge\textbf{活动预告}}
\begin{multicols}{2}
    \section{订阅方式和加入编辑部}  
编辑部招聘人才,用爱发电,工作轻松,详情可联系QQ:1329527951 客服小祥\\想订阅本消息或获取PDF版(便于查看超链接和往期),可加QQ群:\href{https://qm.qq.com/q/VXIW7fgsEe}{849644979}.
\section{Deadline Ongoing}
\setbox\colbbox\vbox{
\makeatletter\col@number\@ne
\begin{longtable}{|c|c|c|}
    \hline
    消息(未见ddl的,不刊) & 截止日期 & 刊载日期\\
    \hline\hline
    紫藤学刊征稿 & 12.15 & 10.22\\
    校运会 & 11.8 & 10.21\\
    后革命鲁迅研究征文 & 11.10 & 10.8\\
    大创训练计划申报 & 11.18 & 9.24\\
    招生宣传创意征集大赛 & 11.18 & 10.21\\ 
    EBSCO数据库检索大赛 & 11.20 & 10.3\\
    文院征稿 & 11.20 & 10.20\\
    乐跑 & 12.6 & 10.12\\
    国际访学计划申报 & 11.22 & 10.22\\
    普通话测试网络报名 & 11.12 & 10.29\\
    南大演说家报名 & 11.9 & 10.30\\
    南大会征募会设 & 11.15 & 11.1\\
    心协十一月征稿 & 11.10 & 11.2\\
    秉文心理短视频 & 11.25 & 11.3\\
    扭泵音乐节 & 11.8 & 11.3\\
    法学主题参会 & 11.11 & 11.4\\
    心协香囊活动 & 11.10 & 11.4\\
    BRAVO草地音乐节 & 11.9 & 11.4\\
    高校联合徒步报名 & 11.10 & 11.5\\
    天文台车赛报名 & 11.12 & 11.5\\
    南大模联校内会报名 & 11.11 & 11.5\\
    公务员面试大赛报名 & 11.12 & 11.6\\
    炜华音乐会(地海院) & 11.10 & 11.6\\
    \hline
\end{longtable}
\unskip
\unpenalty
\unpenalty}\unvbox\colbbox
\end{multicols}
\hrule
\pagebreak
\begin{multicols}{2}

\section{讲座}
\begin{tabular}{|c|c|c|}
    \hline
    往期讲座 & 开展日期 & 刊载日期\\
    \hline\hline
    《电池及电化学能...》 & 11.24 & 10.3\\
    《专利查新与规避...》 & 12.19 & 10.3\\
    图书馆系列讲座 & 12.3 & 10.20\\
    《瑞典电力和氢能...》 & 11.7 & 10.29\\
    《卢卡奇1919与19...》 & 11.8 & 11.2\\
    《青年卢卡奇论马...》 & 11.8 & 11.2\\
    《比较文化研究与...》 & 11.8 & 11.3\\
    《中国历代龙形象...》 & 11.7 & 11.4\\
    《打开人文社科研...》 & 11.7 & 11.4\\
    《卢卡奇遗产中的...》 & 11.10 & 11.4\\
    《Predictive M...》 & 11.7 & 11.5\\
    《史料场与问题域...》 & 11.8 & 11.5\\
    《一位晚清驻防旗...》& 11.7 & 11.5\\
    《从微观数据到宏...》& 11.11 & 11.5\\
    《健雄学科认知分享》 & 11.8 & 11.6\\
    《如何写好调研报...》 & 11.7 & 11.6\\
    \hline
\end{tabular}

1.星成长 | 满天星实践讲堂\\
主题:首届满天星报告大赛参赛辅导:如何写好调研报告和决策咨询报告\\
主讲人:徐宁,南京大学产业经济学博士,南京大学长江产业发展研究院副院长\\
时间:2024.11.7(周四)18:30-20:00\\
地点:南京大学仙林校区 仙二104\\
报名方式:\url{https://table.nju.edu.cn/dtable/forms/57da5eac-83bd-4056-84b2-5e28c643e09c/},请进入链接报名\\
注:本次活动可作为选择项目制课程同学的过程性学习\\
详情请见推文\url{https://mp.weixin.qq.com/s/r3Z8OFm2FSNIL99FSCbuYw}\\

3.《南天学堂》云讲堂第十八期:伽玛射线暴--宇宙中最猛烈的爆炸\\
讲座专家:王发印 南京大学天文与空间科学学院教授、博导\\
云讲堂摘要\\       
伽玛射线暴是在宇宙深处发生的爆发现象,是宇宙中最剧烈的爆炸,1973年被宣布发现,然而至今仍是未解之谜。它们与原子弹爆炸有怎样联系?与引力波有什么联系?为什么会引起恐龙灭绝?如何利用它预测我们宇宙的命运?本报告将回答以上问题,并介绍南京大学在伽玛暴领域的贡献等。\\
本条为视频讲座,可直接进入链接观看学习\url{https://mp.weixin.qq.com/s/lWWyPU0dEVFb8on7_qvpoA}\\
4.“朋辈导学”系列分享第二期:健雄书院学科认知分享会\\
主讲人:22级集成电路学院 任一帆\\
曾获南京大学优秀朋辈导师、少数民族优秀学生等荣誉以及亨通奖学金\\
活动时间:2024年11月8日晚19:00\\
活动地点:鼓楼校区南青格庐\\
活动对象:健雄书院全体同学\\
报名链接见\url{https://mp.weixin.qq.com/s/5F7uX2Qk1j5pfNRo6JlzSQ}

\section{2024-2025学年第一学期数学期中考试安排}
考试时间:2024年11月16日(周六)\\
具体考试科目时间和考场安排\url{https://jw.nju.edu.cn/05/f8/c26263a722424/page.htm}\\

\section{鼓楼校区通宵自习教室调整}
调整时间:11月11日开始\\
调整教室:开放南教学楼一楼101作为通宵自习教室,关闭原教学楼(郑钢楼)的通宵教室。\\
(目前根据通宵教室的实际使用情况,暂开南教-101,后续会根据同学们的实际学习需求,适时增开通宵自习教室。)\\

\section{日语俱乐部观影会预告}
时间和影片依投票决定。投票二维码见原文\url{https://mp.weixin.qq.com/s/Q_AdBr6QDG9IWoxcL81FxQ}\\
备选方案包括:《言叶之庭》《未来的未来》《垫底辣妹》《余命10年》

\section{<SRTP整理>11.7-11.8学术文化活动概览}
文科\\
周四(11.7)\\
1.我们可以不装腔作势吗\\
2.Gale scholar: 打开人文社科研究新视界\\
3.中国历代龙形象与龙文化艺术讲座\\
周五(11.8)\\
1.卢卡奇1919与1923年历史唯物主义研究所计划\\
2.比较文化研究与数据科学\\
理科\\
周四(11.7)\\
1.Swedish electricity and hydrogen market: an outlook\\
2.高校图书馆赋能青年科研人才创新发展的探索与实践\\
详见\url{https://mp.weixin.qq.com/s/60VobhyIF2-rOZVshHPVsg}
\section{黑匣子剧场|《歌》}
时间:2024年11月9日(周六)19:30\\
地点:仙林校区敬文学生活动中心三楼黑匣子剧场\\
报名方式:扫描推文中二维码\url{https://mp.weixin.qq.com/s/-YJmKYqpbxPtxVzG6HTKbA}

\section{南京大学首届“政管杯”公务员面试模拟大赛}
时间:11-12月(初赛11月23日(周六),复赛11月末,决赛12月中旬)\\
地点:仙林校区圣达楼\\
报名时间:即日起至11月12日24:00\\
报名条件:南京大学本、硕、博在校生\\
奖项设置:\\
\textbf{一等奖一名:}荣誉证书+2000元梦想基金\\
\textbf{二等奖两名:}荣誉证书+1500元梦想基金\\
\textbf{三等奖三名:}荣誉证书+1000元梦想基金\\
\textbf{优秀奖三名:}荣誉证书+500元梦想基金\\
比赛详情及报名方式请扫描推文中二维码\url{https://mp.weixin.qq.com/s/iRo4CqmH0Pnt4HS7QvtbIg}

\section{<地海院>炜华音乐会}
活动时间:11月10日(周日)16:00\\
活动地点:炜华体育场\\
活动简介:这是一场汇聚舞蹈、音乐、欢笑与友谊的盛会,精彩节目轮番上演,绝对不容错过。无论你是参与者,还是旁观者,只需在活动当天到场,就有机会获得丰厚的奖品。\\
活动详情参见\url{https://mp.weixin.qq.com/s/6VtweSZYWYALuo4pPpOyhQ}\\
\end{multicols} 
\hrule
\vspace{4mm}
\centerline{\huge\textbf{参考消息}}
\begin{multicols}{2}
\section{南京大学校园大盗已锁定}
现转载如下:

“警察于今日下午来到南京大学保卫处,告知我二人已找到,是南京大学地科楼负责施工的民工。男子是女子上司,二人在候车厅有亲密行为,女子坐在了男子身上,但并非夫妻。

二人声称‘10月31日中午在侯车大厅休息,男士已醉酒,神志不清,腰椎不好,想拿我的球拍挥一挥,活动一下,手感不错,就拿走了’。警方认为此说辞没有问题,并且候车厅属于公共区域,不是私人区域,本次案件定性为‘遗失’,‘不予立案’,不能定性为“盗窃”。若我接受调解,则结案,此二人将不被判刑,依然可自由活动。

而后,我指出,二人多次来到校园“疑似盗窃”,我想不会那么巧只有我一个人丢东西了只,是大家或由于失窃物品价格较低并未报案。并且此二人仅10月31日一天,便在候车厅左顾右盼多次,顺走好几样物品。只可惜南京大学安保系统疏漏,候车厅偌大的地方,竟没有摄像头可清晰照到,顺走物品的瞬间无法看清,因而不能定性为盗窃。还望有同样在候车厅丢失物品的同学,能够站出来,为南大安全尽一份力!

当然了,已经为此事努力许久的我,不可能就此接受调解,即便警方相信他的说辞,我也不会相信。

在与民警的沟通中,他们一定要我拿出他们偷别的东西的证据,才定性为盗窃。我指出,10月29日此二人偷过一只有黑色拍套的Wilson网球拍。此事是在我调取监控室,通过学校保卫处人脸识别系统发现的。试想,那名女子是53岁农民工,不识字,还有抑郁症,会一个人拿一把Wilson拍子去打网球的概率有多大?我因此推断很可能是偷的,并将这一线索提供给了警方,让他们继续侦破。”
\end{multicols} 
\end{document}