% HEAD BEGIN
\documentclass[letterpaper, 12pt]{article}
\newsavebox\colbbox
\usepackage{graphicx}
\usepackage{multicol}
\usepackage{anysize}
\usepackage{fontspec}
\usepackage[fontset=none]{ctex}
\usepackage{tabularx}
\usepackage{longtable}
\PassOptionsToPackage{hyphens}{url}
\usepackage[breaklinks=true, colorlinks=true]{hyperref}
\expandafter\def\expandafter\UrlBreaks\expandafter{\UrlBreaks\do\a\do\b\do\c\do\d\do\e\do\f\do\g\do\h\do\i\do\j\do\k\do\l\do\m\do\n\do\o\do\p\do\q\do\r\do\s\do\t\do\u\do\v\do\w\do\x\do\y\do\z\do\A\do\B\do\C\do\D\do\E\do\F\do\G\do\H\do\I\do\J\do\K\do\L\do\M\do\N\do\O\do\P\do\Q\do\R\do\S\do\T\do\U\do\V\do\W\do\X\do\Y\do\Z}
% \let\oldurl\url
% \renewcommand{\url}[1]{\begin{sloppypar}\oldurl{#1}\end{sloppypar}}
\setlength\columnsep{30pt}
\marginsize{30pt}{30pt}{10pt}{20pt}
\setmainfont{TeX Gyre Bonum}
\setCJKmainfont[BoldFont=Noto Serif CJK SC Bold, ItalicFont=FandolKai]{Noto Sans CJK SC}
\setlength{\parindent}{0cm}
% \setCJKmonofont{Noto Sans CJK SC}
\begin{document}
\begin{center}
    \Huge\textbf{南哪大专醒前消息}
\end{center}
\vspace{4mm}
\hrule
\renewcommand\tabularxcolumn[1]{m{#1}}
\begin{tabularx}{\textwidth}{>{\hsize.2\hsize}X>{\hsize.6\hsize}X>{\hsize.2\hsize}X}
    \begin{flushleft}
        2024.11.13\, No.116
    \end{flushleft}
    &
    \begin{center}
        \textit{“秉中持正、求新博闻。”}
    \end{center}
    &
    \begin{flushright}
        \textbf{南京市栖霞区}
    \end{flushright}
\end{tabularx}
\vspace{-3.5mm}
\hrule
\vspace{4mm}
% HEAD END
\centerline{\huge\textbf{活动预告}}
\begin{multicols}{2}
    \section{订阅方式和加入编辑部}  
编辑部招聘人才,用爱发电,工作轻松,详情可联系QQ:1329527951 客服小祥\\想订阅本消息或获取PDF版(便于查看超链接和往期),可加QQ群:\href{https://qm.qq.com/q/VXIW7fgsEe}{849644979}.
\section{Deadline Ongoing}
\setbox\colbbox\vbox{
\makeatletter\col@number\@ne
\begin{longtable}{|c|c|c|}
    \hline
    消息(未见ddl的,不刊) & 截止日期 & 刊载日期\\
    \hline\hline
    紫藤学刊征稿 & 12.15 & 10.22\\
    大创训练计划申报 & 11.18 & 9.24\\
    招生宣传创意征集大赛 & 11.18 & 10.21\\ 
    EBSCO数据库检索大赛 & 11.20 & 10.3\\
    文院征稿 & 11.20 & 10.20\\
    乐跑 & 12.6 & 10.12\\
    国际访学计划申报 & 11.22 & 10.22\\
    南大会征募会设 & 11.15 & 11.1\\
    秉文心理短视频 & 11.25 & 11.3\\
    简历大赛 &11.17 & 11.7\\
    医保补参保 & 11.17 & 11.8\\
    NCQM2024报名 & 11.15 & 11.9\\
    博洽书会 & 11.15 & 11.9\\
    AI爱情主题辩论赛 & 11.16 & 11.11\\
    NCA分享会 & 11.17 & 11.11\\
    流光《心灵捕手》 & 11.16 & 11.12\\
    心协小圆桌 & 11.17 & 11.13\\
    社调loser杯报名 & 11.15 & 11.13\\
    DIY课程学术论坛征稿 & 11.30 & 11.13\\
    羊城晚报宣讲会 & 11.17 & 11.13\\
    国风歌曲演唱赛 & 12.1 & 11.13\\
    长颈鹿助人训练营 & 11.15 & 11.13\\
    南说开放麦 & 11.15 & 11.13\\
    牡丹亭庆演 & 12.1 & 11.13\\
    现象学工作坊 & 11.16 & 11.13\\
    \hline
\end{longtable}
\unskip
\unpenalty
\unpenalty}\unvbox\colbbox
\end{multicols}
\hrule
\pagebreak
\begin{multicols}{2}

\section{讲座}
\begin{tabular}{|c|c|c|}
    \hline
    往期讲座 & 开展日期 & 刊载日期\\
    \hline\hline
    《电池及电化学能...》 & 11.24 & 10.3\\
    《专利查新与规避...》 & 12.19 & 10.3\\
    图书馆系列讲座 & 12.3 & 10.20\\
    《教室性别结构对...》 & 11.14 & 11.7\\
    《Decouple electron...》 & 11.14 & 11.11\\
    《大数据与传统数...》 & 11.15 & 11.12\\
    《自传经典比较研...》 & 11.14 & 11.12\\
    《水下考古漫谈...》 & 11.14 & 11.12\\
    《中华文明现代化...》 & 11.15 & 11.13\\
    《太平天国败亡...》& 11.15 & 11.13\\
    《德性知识论的两...》 & 11.15 & 11.13\\
    《Quid pro quo...》 & 11.15 & 11.13\\
    \hline
\end{tabular}

1.中华文明现代化理据一一兼谈学术论文选题\\
主讲人:项久雨(教育部“长江学者”特聘教授)\\
时间:11月15日 周五 10:00\\
地点:仙Ⅱ-105\\

2.夏春涛 太平天国败亡 160周年之反思\\
主讲人:夏春涛 中国社会科学院近代史研究所所长、研究员\\
主持人:张  生 南京大学历史学院院长、教授\\
与谈人:李  玉 南京大学历史学院教授\\
讲座地点:南京大学历史学院133会议室\\
讲座时间:2024.11.15(周五)下午 15:30—17:30\\

3.德性知识论的两难困境及其消解\\
主讲人:赵海丞(厦门大学哲学系副教授)\\
与谈人:张昱顾(南京大学)\\
时间:11月15日(周五)10:00\\
地点:哲学学院218\\
摘要详阅:\url{https://mp.weixin.qq.com/s/o-1SZK0Tq-CZxnNHfYfc6A}。\\

4. Quid pro quo in online medical consultation? Investigating the effects of small monetary gifts from patients

报告人 郭熙铜 教授

主持人 肖条军 教授

时间 11月15日(周五)14:30-16:00

地点 协鑫楼204

腾讯会议 725-761-010
\section{心理学小圆桌}
本期讨论书目:《我们内心的冲突》\\
时间:11月17日14:00\\
地点:鼓楼校区新教学楼501\\
报名方式:扫描推文中二维码,加群报名\\
转发推文并收集20个赞便可在活动上领取一整套心协文创书签\\
详见\url{https://mp.weixin.qq.com/s/DbmMrgMMVyxbEk9fC52eEw}

\section{社调LoSeR杯田野大赛}
报名截止:11月15日20:00\\
活动时间:11月16日18:30\\
活动地点:南大社院某间教室(将于11月15日20:00后陆续通知)\\
不论专业为何,只要经历过较为长时间的调研、亲历与某个社群或个体的相处、有过在田野里发问与反思的时刻、且认为自己的田野“不够充分、不够好”,都可参加\\
参与方式:提交报名表,详见\url{https://mp.weixin.qq.com/s/zFlVG4rQMVemvkiwRft-Xw}

\section{DIY研读研究课程学术论坛}
会议时间:2024年12月7日(周六)\\
会议地点:仙林校区圣达楼102室\\
论坛拟聚焦以下议题:\\
1. 文学中有关政治共同体的书写与想象;\\
2. 国家、民族与全球化之间的张力关系;\\
3. 从共同体到现代社会的叙事及其嬗变。\\
征稿对象:面向南京大学“世界主义与民族国家”“文学与政治:德意志浪漫派的共同体想象”DIY研读研究课程的全体选课同学征稿,亦欢迎其他校内本科生同学赐稿。\\
征稿时间:2024年11月13日至2024年11月30日24:00\\
投稿方式:投稿邮件请命名为【DIY课程学术论坛投稿+姓名+论文标题】,发送至邮箱:irs@nju.edu.cn\\
详见\url{https://mp.weixin.qq.com/s/u_u7aPv-JLbwCAyOFSvP_Q}

\section{爱心书屋及志愿者招募}
爱心书屋开放时间:每天14:00—17:00  18:30—21:30\\
活动地点:杜厦图书馆C305\\
参与方式:加入爱心书屋值班群或加入鸿新社,详见\url{https://mp.weixin.qq.com/s/KQt9OSstn00zVMjJx1QWlw}

\section{羊城晚报宣讲会}
羊城晚报报业集团一行将于11月17日光临我校,当日9:00-12:30在方肇周体育馆二楼乒羽馆举行双选会,15:00-17:00在新闻传播学院201室举办宣讲交流会,欢迎同学们参与。
\section{国风歌曲演唱比赛}
时间:12月1日(周日)14:00-17:00\\
地点:仙林校区敬文学生活动中心多功能厅\\
报名方式:搜索qq群号:977208042(或扫描推文中二维码)进入海选群,填写问卷并发送音频文件至邮箱528755053@qq.com报名参加比赛\\
报名截止:11月21日(周四)24:00\\
详见\url{https://mp.weixin.qq.com/s/omutbH1g-Voswo6QmnwD7Q}

\section{长颈鹿助人技能训练营}
南京大学心理健康教育与研究中心现开展“长颈鹿助人技能训练营”,现面向本校心理学爱好者及专业人士开展报名工作。完成培训后,学员将参加南京大学热线咨询师考核,通过考核者方可获得培训结业证书,并有机会成为南京大学心理热线接线员,进行心理热线服务,并获得相应的热线服务时长证明。\\
时间:2024年11月--2024年12月,每周六下午。\\
地点:南京大学南京各校区内,具体地点待定。线下授课,不超过50人。\\
报名截止时间:2024年11月15日14:00

报名方式及要求:见\url{https://mp.weixin.qq.com/s/WSmX3ahoqxXoWQZVvAZKjA}\\
\section{南说喜剧开放麦}
南说戏剧邀请您来听段子啦!这里是开放麦新演员练习胆量,老演员打磨段子的舞台,在这里,你会看到首次亮相的新面孔,听到全新的搞笑段子!\\
时间:11.15周五 19:00\\
地点:敬文学生活动中心9楼\\
报名方式:见\url{https://mp.weixin.qq.com/s/j5qPwpZ0Y8tGtx_YVTdkNQ}
\section{创新论坛暨团队成果汇报会}
主题:“深化校地融合,推动协同育人”创新论坛暨“拉贝日记与和平城市”团队成果汇报会\\
时间:2024年11月15日(周五)\\
地点:南京大学国际会议中心以行厅\\
详见\url{https://mp.weixin.qq.com/s/k-8dVBbB3UkoH-uTTM5RQQ}
\section{青春版《牡丹亭》重回南大}
白先勇将携青春版《牡丹亭》重返南大,并于2024年11月29日至12月1日进行首演二十周年庆演\\
青春版《牡丹亭》二十周年庆演:\\
2024年11月29日(周五)19:00(上本《梦中情》)\\
2024年11月30日(周六)19:00(中本《人鬼情》)\\
2024年12月1日(周日)19:00(下本《人间情》)\\
演出地点:\\
荔枝大剧院·南京大学专场(北京东路2号荔枝广场3层)\\
“青春版《牡丹亭》‘西游记’”专场讲座:\\
时间:11月28日 晚7点\\
地点:仙林国际会议中心三江厅\\
详情信息、票务获取及抽奖福利,详见:\url{https://mp.weixin.qq.com/s/wUn5sfw2653NnzNrGV0XtQ}
\section{行知书院2024年数学期中考前辅导}
简明微积分
时间:11月15日(周五)18:30-20:30\\
地点:教学楼213\\
主讲人:数学学院 唐文星\\
其他详见\url{https://mp.weixin.qq.com/s/gHG3fR3U-_q-nOobXZoVxQ}
\section{“时间与他者”工作坊预告}
哲学学院与现象学研究所将于11月16日-11月17日在哲学学院218举办“时间与他者”欧陆哲学工作坊,并研讨王恒老师著作《时间性:自身与他者——从胡塞尔、海德格尔到列维纳斯》。时间表详阅:\url{https://mp.weixin.qq.com/s/o55Ves1kN1ExPhx1VPwo3A}。
\section{明日赛程}
女篮院系杯小组赛\\
新传vs材料 19:30-21:00\\
地点:一组团篮球场\\
外院vs地海 20:40-22:00\\
地点:方肇周体育馆
\end{multicols} 
\end{document}