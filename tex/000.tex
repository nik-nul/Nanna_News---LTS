% HEAD BEGIN
\documentclass[letterpaper, 12pt]{article}
\newsavebox\colbbox
\usepackage{graphicx}
\usepackage{multicol}
\usepackage{anysize}
\usepackage{fontspec}
\usepackage[fontset=none]{ctex}
\usepackage{tabularx}
\usepackage{longtable}
\PassOptionsToPackage{hyphens}{url}
\usepackage[breaklinks=true, colorlinks=true]{hyperref}
\expandafter\def\expandafter\UrlBreaks\expandafter{\UrlBreaks\do\a\do\b\do\c\do\d\do\e\do\f\do\g\do\h\do\i\do\j\do\k\do\l\do\m\do\n\do\o\do\p\do\q\do\r\do\s\do\t\do\u\do\v\do\w\do\x\do\y\do\z\do\A\do\B\do\C\do\D\do\E\do\F\do\G\do\H\do\I\do\J\do\K\do\L\do\M\do\N\do\O\do\P\do\Q\do\R\do\S\do\T\do\U\do\V\do\W\do\X\do\Y\do\Z}
% \let\oldurl\url
% \renewcommand{\url}[1]{\begin{sloppypar}\oldurl{#1}\end{sloppypar}}
\setlength\columnsep{30pt}
\marginsize{30pt}{30pt}{10pt}{20pt}
\setmainfont{TeX Gyre Bonum}
\setCJKmainfont[BoldFont=Noto Serif CJK SC Bold, ItalicFont=FandolKai]{Source Han Sans SC}
\setlength{\parindent}{0cm}
% \setCJKmonofont{Noto Sans CJK SC}
\begin{document}
\begin{center}
    \Huge\textbf{南哪大专醒前消息}
\end{center}
\vspace{4mm}
\hrule
\renewcommand\tabularxcolumn[1]{m{#1}}
\begin{tabularx}{\textwidth}{>{\hsize.2\hsize}X>{\hsize.6\hsize}X>{\hsize.2\hsize}X}
    \begin{flushleft}
        2025.4.16\, No.222
    \end{flushleft}
    &
    \begin{center}
        \textit{“秉中持正、求新博闻。”}
    \end{center}
    &
    \begin{flushright}
        \textbf{南京市栖霞区}
    \end{flushright}
\end{tabularx}
\vspace{-3.5mm}
\hrule
\vspace{4mm}
% HEAD END
\centerline{\huge\textbf{活动预告}}
\begin{multicols}{2}
\section{订阅方式和加入编辑部}  
编辑部招聘人才,用爱发电,工作轻松,详情可联系QQ:1329527951 客服小千\\想订阅本消息或获取PDF版(便于查看超链接和往期),可加QQ群:\href{https://qm.qq.com/q/4HL41Nt3sQ}{466863272}.
\section{活动清单}
\setbox\colbbox\vbox{
\makeatletter\col@number\@ne
\begin{longtable}{|>{\centering\arraybackslash}m{.3\textwidth}|m{.06\textwidth}|m{.06\textwidth}|}
    \hline
    活动 & 开展时间 & 刊载时间\\
    \hline\hline
    南大版deepseek & / & 2.22\\
    悦读课程群 & / & 2.24\\
    eScience AI科研助手 & / & 3.11\\
    地科博物馆开放安排 & / & 3.22\\ 
    2025年分流和转专业政策通知 & / & 4.7\\
    乐跑 & 5.16 & 3.10\\
    本科生劳育实践 & 7.20 & 2.19\\
    银星杯论文赛 & 4.22 & 2.27\\
    高教社杯 & 4.25 & 3.5\\
    大文大理题目征集 & 期末 & 3.8\\
    5月免费上网 & ? & 3.9\\
    基础学科论坛 & 4.20 & 3.9\\
    外教社杯 & 5.27 & 3.12\\
    江苏创青春赛事 & 4.30 & 3.26\\
    浦口音乐跑 & 5.30 & 3.31\\
    红会暑期项目招募 & 4.12 & 4.1\\
    程设大赛 & 4.26 & 4.2\\
    瑞声杯 & 4.20 & 4.4\\
    仙林校区志愿法律咨询 & / & 4.4\\
    青春活力大赛 & 5.17 & 4.7\\
    在校生自愿体检 & 6.20 & 4.8\\
    南大购买WPS & / & 4.8\\
    24级程设大赛 & 4.27 & 4.11\\
    法治情景剧策划大赛 & 4.23 & 4.11\\
    仙林猫鼠游戏 & 4.19 & 4.12\\
    EL程设大赛 & 4.27 & 4.13\\
    全国行研大赛 & 4.20 & 4.13\\
    全国ESG大赛 & 4.17 & 4.13\\
    南大网双公开赛报名 & 4.18 & 4.13\\
    中美中心2025年证书项目 & 5.24 & 4.14\\
    心里剧本创作大赛 & 4.20 & 4.14\\
    体育月舞蹈教学2 & 4.18 & 4.15\\
    春季学期创新训练计划结题考核通知 & 4.28 & 4.15\\
    竺可桢讲师团义卖 & 4.18 & 4.15\\
    心理中心开放日 & 4.19 & 4.16\\
    植物微景观活动 & 4.23 & 4.16\\
    \hline
\end{longtable}
\unskip
\unpenalty
\unpenalty}\unvbox\colbbox
\end{multicols}
\begin{multicols}{2}
\pagebreak

\section{讲座}
\begin{tabular}{|>{\centering\arraybackslash}m{.3\textwidth}|m{.06\textwidth}|m{.06\textwidth}|}
    \hline
    讲座 & 开展时间 & 刊载时间\\
    \hline\hline
    社交媒体分享实践的语用学研究 & 4.18 & 4.9\\\hline
    Deepseek现象中的管理学 & 4.18 & 4.10\\\hline
    智能时代的中国式养老:理论与实践”学术研讨会 & 4.18-20 & 4.10\\\hline
    从感知到疗愈:人脑音乐加工机制 & 4.25 & 4.11\\\hline
    Accretion-generated rings:coplaner and polar structures & 4.18 & 4.11\\\hline
    Ionizing spotlight of Active Galactic Nucleus & 4.23 & 4.11\\\hline
    人智协同式内容创作方法初探 & 4.17 & 4.15\\\hline
    在记忆星丛中走向他人 & 4.18 & 4.15\\\hline
    量子模型中一类新的超扩散机制 & 4.17 & 4.15\\\hline
    软件挖掘:机遇与挑战 & 4.17 & 4.15\\\hline
    A Statistical Theory of Contrastive Pre-training and Multimodal Generative AI & 4.17 & 4.16\\\hline
    Can Artificial Intelligence Improve Gender Equality? Evidence from a Natural Experiment & 4.17 & 4.16\\\hline
    人工智能与大模型的应用方法 & 4.18 & 4.16\\\hline
    科技之巅:人机共智2030 & 4.22 & 4.16\\\hline
    软件发展与技术漫谈 & 4.29 & 4.16\\\hline
    DeepSeek: 从人工智能到大模型及应用思考 & 4.24 & 4.16\\\hline
    从语言到智能 ⸺ 大语言模型的奥秘与应用 & 5.6 & 4.16\\\hline
\end{tabular}
%讲座预告写在这。用subsection
\subsection{4月17日 学术报告 - 蔡榆杭 博士生}
A Statistical Theory of Contrastive Pre-training and Multimodal Generative AI
\\时间:2025年4月17日(星期四) 10:30
\\Zoom: 86858521216
\\Password: 0417
\\蔡榆杭, 博士生
\\University of California, Berkeley
\\详见:\url{https://mp.weixin.qq.com/s/bJdqaQk3L1Y4wgQfpK5rYg}

\subsection{科技之巅:人机共智2030}
报告人 田丰 院长 快思慢想研究院
\\主持人 俞红海 院长 南京大学工程管理学院
\\时间 4月22日(周二)14:30-16:00
\\地点 工程管理学院104报告厅
\\详见:\url{https://mp.weixin.qq.com/s/TJ9K464nuK1nD5gunWz8Pw}

\subsection{Can Artificial Intelligence Improve Gender Equality? Evidence from a Natural Experiment}
主讲人 黄棣芳  助理研究员
\\主持人 杨学伟  教授
\\时间 4月17日(周四)10:00-12:00
\\地点 北楼105报告厅
\\详见:\url{https://mp.weixin.qq.com/s/GT4CcRJJ0fJ0X0hvPCgpwQ}



\subsection{人工智能与大模型的应用方法}
时间:4月18日(周五)14:00--16:00
\\地点:仙林校区侨裕楼(外国语学院大楼)301
\\主持人:杨学伟(南京大学工程管理学院教授,社科处副处长)
\\分享人:范玉顺(清华大学)
\\本报告面向非人工智能专业的人员。首先介绍人工智能中机器学习的基本原理和大模型的技术架构,基本原理和训练方法,介绍Word2Vec和Trasformer的基本原理DeepSeek的技术架构和优势来源。本报告结合实际应用案例和大模型操作应用过程介绍大模型应用的十种方法,包括:文本生成和检索;图像和视频生成;代码生成、分析推理、学习辅导;方案制定和优化;应用商店(AIGC):API调用;模型微调;模型蒸馏;知识图谱和知识增强应用;DeepseeK本地部署。
\\详见:\url{https://mp.weixin.qq.com/s/Rf6lCRFujOWZx2ZWWNbipg}

\subsection{“诚计划”第146期至149期讲座预告合集}
微信扫码识别,前往南大“暾学堂”移动端或微信视频号观看直播。
\\一、2025年04月17日(周四) 19:30-21:00 《软件挖掘:机遇与挑战》
\\二、2025年04月24日(周四) 19:30-21:00《DeepSeek:从人工智能到大模型及应用思考》
\\三、2025年04月29日(周二) 19:30-21:00《软件发展与技术漫谈》
\\四、2025年05月06日(周二) 19:30-21:00《从语言到智能——大语言模型的奥秘与应用》
\\详见:\url{https://mp.weixin.qq.com/s/B2lLJ9oOao21Ej1XGCIvuQ}
\section{植物微景观活动}
活动内容:制成多肉景观
\\活动时间:4月23日(周三)18:30-19:30 
\\活动地点:生命科学学院A430
\\报名方式见原推
\\详见:\url{https://mp.weixin.qq.com/s/26_SWsm5ZobgccIWmelFcg}


\section{全民国家安全教育 走深走实十周年主题展}
2025 4/15-30
\\仙林校区四五六食堂对面橱窗
\\鼓楼校区南苑入口处西侧橱窗
\\浦口校区教学楼西侧橱窗
\\苏州校区仁园学生社区
\\详见:\url{8bd35e678b4fede3d107d452eec52160.jpg}

\section{【心理中心开放日活动】探索蓝鲸消失处}
你不能错过的打卡理由:
\\【心理学科普】心理学知识大揭秘,原来心理学就在我们的身边?
\\【美好分享墙】用灵魂涂鸦记录今日心情,拍照发朋友圈收割满满好评
\\【减压疗愈室】躲进香薰环绕的私密空间,5分钟感官SPA让焦虑瞬间融化
\\【温暖工作坊】手残党也能做的国风香囊+代寄感谢信,把温暖送给重要的人
\\【前方烧脑预警】心理剧本杀体验(打碎镜子/于蓝鲸中迷失),社交组队模拟真实心理博弈过程
\\【惊喜蓝鲸】每一个活动区域将会隐藏着一枚“蓝鲸”,收集全部“蓝鲸”可以兑换惊喜礼品!
\\
\\传送门:南大仙林校区敬文学生活动中心9、10楼
\\开放时空:2025年4月19日 13:00-17:00
\\(建议拉上室友/搭子组队探索)
\\详见:\url{https://mp.weixin.qq.com/s/eAbXX0-tWDO9GIWl53_UCA}



\section{食堂减脂餐套餐上线}
为响应国家体重管理年号召,即日起启动“春日减脂计划”,精心推出4款“减脂餐套餐”,水煮低脂烹调、主食粗细搭配,优质蛋白保证、蔬菜丰富健康,助力师生们轻松实现体重管理!
\\
\\供应时间:即日起指定窗口中餐、晚餐时间段供应
\\
\\供应地点:
\\鼓楼校区学生第二食堂19号窗口
\\仙林校区学生第六食堂05号窗口
\\浦口校区学生第十五食堂06号窗口
\\
\\供应方式:
\\主食自由选:杂粮饭、玉米段、荞麦面三个优质主食任选一样,碳水控量更科学。
\\时蔬随心配:香菇、青菜、胡萝卜、西兰花、紫甘蓝、娃娃菜等多种蔬菜任选四道,膳食纤维不可少。
\\蘸料按需取:三油汁、油醋汁、沙拉酱、番茄酱、芝麻沙拉汁等多款低卡蘸料,师生们按需自取。
\\
\\具体四种套餐详见链接
\\详见:\url{https://mp.weixin.qq.com/s/rzlClFv2DN44Ksc-O5iMKA}


\section{NJU四小只文创新鲜出炉}
南京大学图书馆与南大记忆联合推出南京大学NJU四小只文件夹/挂件/PP夹!
\\购买地点:鼓楼校区南大记忆文创店、仙林校区南大记忆文创店、仙林校区图书馆大厅自动贩售机
\\4月16日-4月18日在仙林图书馆大厅和鼓楼图书馆大厅购买文创,送明信片,购买三件以上(含三件)文创,送帆布包
\\
\\详见:\url{https://mp.weixin.qq.com/s/q5BPwgGiuM8wZnHPL37QOA}



\section{2025年“贝恩杯”咨询启航案例大赛即将开启}
贝恩杯咨询启航案例大赛由北京大学咨询学会主办,要求参赛成员为具有学籍的全日制在读本科生或硕士研究生,其中不包括MBA、博士生和直接攻博生。报名方法较为繁复,具体报名方式以及咨询群等可见原推。报名将于4月25日23:59截止。
\\详见:\url{https://mp.weixin.qq.com/s/BN8duoaNphDa31MwXbBXEQ}






\section{院级活动}
\begin{tabular}{|>{\centering\arraybackslash}m{.3\textwidth}|m{.06\textwidth}|m{.06\textwidth}|}
\hline
    活动 & 开展时间 & 刊载时间\\
    \hline\hline
    文院剧本创作研讨会 & 9.30 & 3.2\\
    物院征集课程指南 & 6.15 & 3.3\\
    地海征集春日影 & 6.15 & 3.14\\
    社院学术节 & 4.18 & 3.25\\
    五院乒乓球赛 & 4.19 & 3.31\\
    法院党建征文 & 5.20 & 4.2\\
    地学乒赛 & 4.19 & 4.2\\
    软院征集 & 4.20 & 4.4\\
    地学趣运会 & 4.26 & 4.9\\
    四院音乐节 & 5.11 & 4.7\\
    商院征集 & 5.5 & 4.8\\
    毓秀羽球 & 4.20 & 4.8\\
    大气设计 & 4.18 & 4.8\\
    文院诗歌 & 4.18 & 4.8\\
    物院运动打卡 & 5.14 & 4.12\\
    地学定向越野 & 4.19 & 4.12\\
    物院开放日 & 4.19 & 4.16\\
    地海图书漂流 & 4.23 & 4.16\\
    社院学术节 & 4.18 & 4.16\\
    商院影映 & 4.18 & 4.16\\
    \hline
\end{tabular}
\subsection{物理学院课题组开放日}
时间:2025年4月19日
\\宣讲地点:鼓楼校区逸夫馆报告厅
\\宣讲时间:09:30 - 12:08,14:00 - 17:00(12:08原文如此,原样摘录)
\\直播地址:https://meeting.tencent.com/l/VwG3F40cSozr
\\共43个课题组参加活动
\\具体日程安排见原推
\\详见:\url{https://mp.weixin.qq.com/s/Y7xRdvPoJKFHzToIzenkgg}

\subsection{地理与海洋科学学院第四届图书漂流活动}
为积极响应“推动全民阅读,建设书香社会”的号召,迎接即将到来的世界读书日,地理与海洋科学学院团委实践与志愿者工作部将面向全院师生推出第四届图书漂流活动——“图书漂流智慧共享,书香社会你我共创”。
\\活动时间:4月23日(周三)中午11:00-13:00、下午16:00-18:00
\\活动地点:地理与海洋科学学院(昆山楼)一楼大厅
\\活动对象:地理与海洋科学学院全体师生
\\详见:\url{https://mp.weixin.qq.com/s/XirSfYWUyq0JSHPqtJlC5Q}
\subsection{【心灵绿洲计划】月光影院第一期放映活动将于4.18(本周五)启动!}
时间:4月18号19: 00-21: 00
\\地点:仙1-303
\\影片选择:《肖申克的救赎》《怦然心动》《飞屋环游记》《少年派的奇幻漂流》《头脑特工队》《小妇人》《小小的我》《给桃子的信》《熊出没之逆转时空》
\\投票链接:https://table.nju.edu.cn/dtable/forms/4a14b0c4-cbc7-4af4-98ba-44e6fc0d4c9d/
\\票数最高的影片将在活动现场放映
\\QQ群号:675801347
\\详见:\url{https://mp.weixin.qq.com/s/O2ME7oJZlNrp6e61XohZhg}
\subsection{社会学院首届本科生学术节议程}
时间:2025年4月18日(周五)13:30-17:30
\\地点:南京大学仙林校区社会学院河仁楼401合美堂
\\主题讲演:经验感与想象力:社会学的气质与魅力
\\讲演嘉宾:孟庆延,中国政法大学社会学院教授,博士生导师
\\除主题讲演外,还有本科生学术论坛、闭幕式、与行知书院师生交流座谈会等流程
\\详见:\url{https://mp.weixin.qq.com/s/rUYNymtYxNlokmw2k4tzfg}
\section{社团活动}
\begin{tabular}{|>{\centering\arraybackslash}m{.3\textwidth}|m{.06\textwidth}|m{.06\textwidth}|}
    \hline
    社团活动 & 开展时间 & 刊载时间\\
    \hline\hline
    天文台开放日 & / & 1.6\\
    重唱诗歌奖征稿 & 4.30 & 3.31\\
    轮滑社体验 & 4.17 & 4.1\\
    拳击社体验 & 4.22 & 4.1\\
    轮滑社体验 & 4.22 & 4.1\\
    定向赛 & 4.20 & 4.1\\
    体育舞蹈教学 & 4.25 & 4.1\\
    吉他社歌手招募 & 4.20 & 4.4\\
    吉他社春日音 & 4.26 & 4.4\\
    天健捐衣 & 4.20 & 4.13\\
    毽球趣味赛 & 4.19 & 4.15\\
    南说喜剧开放麦 & 4.18 & 4.16\\
    流光影映 & 4.19 & 4.16\\
    \hline
\end{tabular}
%这里是写社团活动的,社团活动就是由社团主办、主要针对社团内部人员的活动。不要把非社团活动写在这里。
\subsection{南说喜剧第六次开放麦}
在这里,你会看到首次亮相的新面孔,听到全新的搞笑段子!
\\时间:4月18日(周五)晚 19:00
\\地点:新教101
\\详见:\url{https://mp.weixin.qq.com/s/xPvIwMLlDhY4dXfovxscOg}

\subsection{流光影院|《爱乐之城》:在星光之城的斑斓画卷里,旋律是梦想与爱情的无声对白}
活动时间\&地点:
\\时间:
\\4月19日(周六)19:00
\\地点:
\\南京大学仙林校区心理中心(大活9楼)
\\南京大学鼓楼校区地点待定(地点变更与否请加入活动群等待通知)
\\
\\
\\详见:\url{https://mp.weixin.qq.com/s/V4yzcNDiGkQpnxT9BPcN_w}
\end{multicols}
\end{document}
