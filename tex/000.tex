% HEAD BEGIN
\documentclass[letterpaper, 12pt]{article}
\newsavebox\colbbox
\usepackage{graphicx}
\usepackage{multicol}
\usepackage{anysize}
\usepackage{fontspec}
\usepackage[fontset=none]{ctex}
\usepackage{tabularx}
\usepackage{longtable}
\PassOptionsToPackage{hyphens}{url}
\usepackage[breaklinks=true, colorlinks=true]{hyperref}
\expandafter\def\expandafter\UrlBreaks\expandafter{\UrlBreaks\do\a\do\b\do\c\do\d\do\e\do\f\do\g\do\h\do\i\do\j\do\k\do\l\do\m\do\n\do\o\do\p\do\q\do\r\do\s\do\t\do\u\do\v\do\w\do\x\do\y\do\z\do\A\do\B\do\C\do\D\do\E\do\F\do\G\do\H\do\I\do\J\do\K\do\L\do\M\do\N\do\O\do\P\do\Q\do\R\do\S\do\T\do\U\do\V\do\W\do\X\do\Y\do\Z}
% \let\oldurl\url
% \renewcommand{\url}[1]{\begin{sloppypar}\oldurl{#1}\end{sloppypar}}
\setlength\columnsep{30pt}
\marginsize{30pt}{30pt}{10pt}{20pt}
\setmainfont{TeX Gyre Bonum}
\setCJKmainfont[BoldFont=Noto Serif CJK SC Bold, ItalicFont=FandolKai]{Noto Sans CJK SC}
\setlength{\parindent}{0cm}
% \setCJKmonofont{Noto Sans CJK SC}
\begin{document}
\begin{center}
    \Huge\textbf{南哪大专醒前消息}
\end{center}
\vspace{4mm}
\hrule
\renewcommand\tabularxcolumn[1]{m{#1}}
\begin{tabularx}{\textwidth}{>{\hsize.2\hsize}X>{\hsize.6\hsize}X>{\hsize.2\hsize}X}
    \begin{flushleft}
        2024.11.24\, No.126
    \end{flushleft}
    &
    \begin{center}
        \textit{“秉中持正、求新博闻。”}
    \end{center}
    &
    \begin{flushright}
        \textbf{南京市栖霞区}
    \end{flushright}
\end{tabularx}
\vspace{-3.5mm}
\hrule
\vspace{4mm}
% HEAD END
\centerline{\huge\textbf{活动预告}}
\begin{multicols}{2}
    \section{订阅方式和加入编辑部}  
编辑部招聘人才,用爱发电,工作轻松,详情可联系QQ:1329527951 客服小祥\\想订阅本消息或获取PDF版(便于查看超链接和往期),可加QQ群:\href{https://qm.qq.com/q/VXIW7fgsEe}{849644979}.
\section{Deadline Ongoing}
\setbox\colbbox\vbox{
\makeatletter\col@number\@ne
\begin{longtable}{|c|c|c|}
    \hline
    消息(未见ddl的,不刊) & 截止日期 & 刊载日期\\
    \hline\hline
    紫藤学刊征稿 & 12.15 & 10.22\\
    乐跑 & 12.6 & 10.12\\
    秉文心理短视频 & 11.25 & 11.3\\
    DIY课程学术论坛征稿 & 11.30 & 11.13\\
    国风歌曲演唱赛 & 12.1 & 11.13\\
    牡丹亭庆演 & 12.1 & 11.13\\
    普通话测试网络报名 & 11.30 & 11.16\\
    安邦征稿 & 1.12 & 11.16\\
    新传院迎新晚会征集 & 11.25 & 11.17\\
    秉文宿舍风采 & 12.1 & 11.17\\
    黑匣招募 & 11.25 & 11.18\\
    心协有奖征稿 & 11.25 & 11.19\\
    法学院征诗活动 & 12.2 & 11.20\\
    计院乒赛 & 11.26 & 11.20\\
    平安留学交流会 & 12.3 & 11.20\\
    七院相亲活动 & 11.30 & 11.21\\
    青山线上支教报名 & 11.25 & 11.21\\
    计科工作坊 & 11.26 & 11.21\\
    黑匣工作坊 & 11.26 & 11.21\\
    免费上网讲座 & 11.27 & 11.22\\
    猫鼠大战 & 12.15 & 11.22\\
    日本征文大赛 & 12.6 & 11.22\\
    萨勒姆的女巫 & 11.30 & 11.22\\
    案例分析大赛 & 11.25 & 11.22\\
    防艾征集 & 12.10 & 11.22\\
    爱心义卖 & 11.30 & 11.22\\
    银杏叶制作之旅 & 11.29 & 11.23\\
    法学朋导招募 & 11.29 & 11.23\\
    青山同行支教招募 & 11.25 & 11.23\\
    南选问答集赞 & 11.27 & 11.23\\
    心协信件盲盒 & 12.7 & 11.23\\
    港澳台晚会招募 & 11.27 & 11.23\\
    配音大赛招募 & 12.7 & 11.23\\
    物院研会系列活动 & 12.21 & 11.23\\
    心理中心征稿 & 12.10 & 11.23\\
    防艾同伴教育 & 12.15 & 11.24\\
    南新读书会 & 11.27 & 11.24\\
    腾讯游戏开发赛报名 & 11.29 & 11.24\\
    高研院午餐会抽签报名 & 11.27 & 11.24\\
    物院摄影征集 & 12.9 & 11.24\\
    \hline
\end{longtable}
\unskip
\unpenalty
\unpenalty}\unvbox\colbbox
\end{multicols}
\hrule
\pagebreak
\begin{multicols}{2}

\section{讲座}
\begin{tabular}{|c|c|c|}
    \hline
    往期讲座 & 开展日期 & 刊载日期\\
    \hline\hline
    《专利查新与规避...》 & 12.19 & 10.3\\
    图书馆系列讲座 & 12.3 & 10.20\\
    《学术写作入门...》& 11.21 & 11.18\\
    《Adobe AI 讲座》 & 11.27 & 11.22\\
    《马克思的经济全球化》 & 11.27 & 11.23\\
    《法学研究类型和方法》 & 11.29 & 11.23\\
    《前沿情报捕捉...》 & 11.29 & 11.23\\
    《AI在设计中的参与...》 & 12.2 & 11.23\\
    决策规划算法 & 11.28 & 11.24\\
    漫谈学业与人生 & 11.25 & 11.24\\
    自动化装备运动控制 & 11.25 & 11.24\\
    大英博物馆文物历史 & 11.26 & 11.24\\
    摩尔条纹与超导性 & 11.26 & 11.24\\
    Design 2050 & 11.25 & 11.24\\
    \hline
\end{tabular}

\subsection{决策规划算法——从科研到工业运用}
时间:2024.11.28下午14:00\\
地点:南雍楼西区124会议室\\
内容:本次讲座邀请到了香港科技大学郑家纯机器人研究所的在读博士陈芯仪,为我们讲授自动移动机器人的拓补建图、考虑感知的运动规划和快速自主探索。以及香港科技大学工学院电子与计算机工程(ECE)在读博士吴易霖,为我们讲授决策规划算法以及工业应用。\\

\subsection{漫谈学业与人生}
主讲人:江伟(南京大学现代工程与应用科学学院教授,博士生导师)\\
时间:11月25日(周一)18:30\\
地点:南京大学鼓楼校区新教405\\
参加方式二维码请见:\url{https://mp.weixin.qq.com/s/7kHiDEcBn3Ffg80sfisMEA}\\

\subsection{面向新一代自动化装备的高速高精运动控制技术}
主讲人:陈思鲁(中国科学院宁波材料技术与工程研究所研究员,中国科学院大学宁波材料工程学院机械工程教研室副主任,博士生导师,浙江省特聘专家)\\
时间:11月25日(周一)16:00\\
地点:南京大学鼓楼校区工程管理学院平仓巷北楼101\\

\subsection{大英博物馆百件文物中的世界历史}
主讲人:刘超(南京大学文学院教授、博士生导师)\\
时间:11月26日(周二)19:30-21:00\\
参加方式二维码请见:\url{https://mp.weixin.qq.com/s/7kHiDEcBn3Ffg80sfisMEA}\\

\subsection{江苏省青年物理学家论坛——石墨烯莫尔条纹和无莫尔条纹系统中的超导性}
报告人:刘晓雪(上海交通大学李政道研究所,任长聘教轨副教授,博士生导师,入选国家高层次人才计划青年项目)
报告时间:11月26日(周二)中午12:00\\
报告地点:南京大学鼓楼校区唐仲英楼B501\\  
详情请见:\url{https://mp.weixin.qq.com/s/7cFG8m6feMwunYodWJirQA}\\
\subsection{Design 2050: 沉浸式设计的过去、现在和未来}
时间:11.25 9:00-12:00\\
地点:仙I-310\\
主讲人:Kyle Li, 美国帕森斯设计与科技研究生学院院长
\section{乐跑}
从明日(11月25日)算起,还有12次乐跑机会。
\section{防艾周同伴教育}
南京大学红十字会将在防艾周期间开展同伴教育主题系列活动,让同伴教育走进校园,让同学们在寓教于乐的氛围中了解更多生命健康知识。\\
时间地点:\\
鼓楼:12.7、12.14\\
仙林:12.8、12.15\\
每场人数限制:20人\\
报名请加QQ群,见原文:\url{https://mp.weixin.qq.com/s/GO-fKMsLLW-O0H8O4u-7IQ}


\section{南新读书会|下周预告}
下周的南新读书会将于11月27日(周三)19:00在新闻传播学院311室举行,23硕刘静将分享米切尔《图像理论》,24硕朱梓鹏将分享以赛亚·柏林《浪漫主义的根源》。

\section{高研院“新生午餐会” 第46场}
题目:出土战国秦汉简牍漫谈\\
谈话人:曹建国 高研院2024年度访问学者 绍兴文理学院鲁迅人文学院\\
时间:2024年11月28日(周四)12:20-13:20\\
地点:鼓楼校区逸夫馆9楼高研院报告厅\\
抽签开始时间:11月26日12:30\\
抽签结束时间:11月27日12:30\\
注:人数限额为30人,先到先得。\\


\section{2024腾讯游戏极限开发赛}
如果只给你72个小时,你是否能够激发出自己的潜能,将脑海中的创意转化为栩栩如生的游戏作品?\\
参与对象\\
南京及周边高校在校学生(包括本科生、硕士生和博士生)\\
赛事日程\\
报名截止:11月29日\\
大赛时间:11月29日14:00-12月2日14:00(共72小时)\\
参赛指引\\
报名链接:\url{https://www.wjx.cn/vm/YhhSAxJ.aspx#}\\
可以独立参赛,可以组队参与,也欢迎跨校组队\\
即使没找到队友,也可以在报名后加入赛事QQ群:754127282,在群内寻找志同道合的小伙伴组队\\
更多详细内容见\url{https://mp.weixin.qq.com/s/aui127WIklg-eeQ3CSvrDQ}


\section{物理学院摄影展}
物理学院第一届“悟理镜界”摄影展,现面向物理学院全体师生征集摄影展作品。\\
活动时间:2024年11月24日至2024年12月18日\\
投稿截止:2024年12月9日20:00\\
作品展示:将挑选优秀获奖作品在鼓楼校区物理楼进行展示\\
活动受众:物理学院全日制在读本科生、硕士/博士研究生、教职工和校友。\\
投稿方式:将作品与文字描述打包作为附件投稿至物院研会邮箱njuphysics520@163.com,投稿作品命名格式:《作品标题》+姓名,例如:《球状闪电》+南小鲸\\
评选方式:作品整理后,将在“南雍悟理”公众号进行作品展示并开启投票;同时开展评委投票,投票时间限定为3天。\\
详见\url{https://mp.weixin.qq.com/s/mju7kDJXD1e-dHZfB-_NiA}
\section{srtp讲座整理}
周日(11.24)\\
日本侵华和中国抗战 双人对谈\\
周一(11.25)\\
1.漫谈学业与人生\\
2.面向新一代自动化装备的高速高精运动控制技术\\
周二(11.26)\\
大英博物馆百件文物中的世界历史\\
详见原文:\url{https://mp.weixin.qq.com/s/7kHiDEcBn3Ffg80sfisMEA}

\end{multicols} 

\end{document}
