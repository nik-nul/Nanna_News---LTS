% HEAD BEGIN
\documentclass[letterpaper, 12pt]{article}
\newsavebox\colbbox
\usepackage{graphicx}
\usepackage{multicol}
\usepackage{anysize}
\usepackage{fontspec}
\usepackage[fontset=none]{ctex}
\usepackage{tabularx}
\usepackage{longtable}
\PassOptionsToPackage{hyphens}{url}
\usepackage[breaklinks=true, colorlinks=true]{hyperref}
\expandafter\def\expandafter\UrlBreaks\expandafter{\UrlBreaks\do\a\do\b\do\c\do\d\do\e\do\f\do\g\do\h\do\i\do\j\do\k\do\l\do\m\do\n\do\o\do\p\do\q\do\r\do\s\do\t\do\u\do\v\do\w\do\x\do\y\do\z\do\A\do\B\do\C\do\D\do\E\do\F\do\G\do\H\do\I\do\J\do\K\do\L\do\M\do\N\do\O\do\P\do\Q\do\R\do\S\do\T\do\U\do\V\do\W\do\X\do\Y\do\Z}
% \let\oldurl\url
% \renewcommand{\url}[1]{\begin{sloppypar}\oldurl{#1}\end{sloppypar}}
\setlength\columnsep{30pt}
\marginsize{30pt}{30pt}{10pt}{20pt}
\setmainfont{TeX Gyre Bonum}
\setCJKmainfont[BoldFont=Noto Serif CJK SC Bold, ItalicFont=FandolKai]{Noto Sans CJK SC}
\setlength{\parindent}{0cm}
% \setCJKmonofont{Noto Sans CJK SC}
\begin{document}
\begin{center}
    \Huge\textbf{南哪大专醒前消息}
\end{center}
\vspace{4mm}
\hrule
\renewcommand\tabularxcolumn[1]{m{#1}}
\begin{tabularx}{\textwidth}{>{\hsize.2\hsize}X>{\hsize.6\hsize}X>{\hsize.2\hsize}X}
    \begin{flushleft}
        2025.2.24\, No.174
    \end{flushleft}
    &
    \begin{center}
        \textit{“秉中持正、求新博闻。”}
    \end{center}
    &
    \begin{flushright}
        \textbf{南京市栖霞区}
    \end{flushright}
\end{tabularx}
\vspace{-3.5mm}
\hrule
\vspace{4mm}
% HEAD END
\centerline{\huge\textbf{活动预告}}
\begin{multicols}{2}
    \section{订阅方式和加入编辑部}  
编辑部招聘人才,用爱发电,工作轻松,详情可联系QQ:1329527951 客服小祥\\想订阅本消息或获取PDF版(便于查看超链接和往期),可加QQ群:\href{https://qm.qq.com/q/VXIW7fgsEe}{849644979}.
\section{Deadline Ongoing}
\setbox\colbbox\vbox{
\makeatletter\col@number\@ne
\begin{longtable}{|c|c|c|}
    \hline
    消息(未见ddl的,不刊) & 截止日期 & 刊载日期\\
    \hline\hline
    南大版deepseek & / & 2.22\\
    天文台开放日 & / & 1.6\\
    悦读课程群 & / & 2.25\\
    原创剧本联合孵化报名 & 3.20 & 1.10\\
    njumun代表报名 & 3.2 & 1.16\\
    课程补退选 & 3.2 & 2.19\\
    南大育教新媒体招新 & 2.27 & 2.19\\
    本科生劳育实践 & 7.20 & 2.19\\
    医保零星报销 & 3.31 & 2.19\\
    第二届大学生阅读分享活动 & 3.7 & 2.21\\
    心理中心助理招新 & 2.28 & 2.20\\
    招办全媒体招新 & 3.5 & 2.20\\
    交响乐团招新 & 3.7 & 2.20\\
    歌魅剧务招募 & 2.26 & 2.21\\
    萌马音乐工作室招新 & 2.28 & 2.22\\
    秉文书院早晚自习 & 3.3 & 2.23\\
    图协招新 & 2.28 & 2.23\\
    “核真录”招新 & 3.2 & 2.25\\
    菁菁南数招募讲师 & 3.9 & 2.25\\
    四六级查分 & 2.27 & 2.25\\
    \hline
\end{longtable}
\unskip
\unpenalty
\unpenalty}\unvbox\colbbox
\end{multicols}
\hrule
\pagebreak
\begin{multicols}{2}

\section{讲座}
\begin{tabular}{|>{\centering\arraybackslash}m{.3\textwidth}|m{.06\textwidth}|m{.06\textwidth}|}
    \hline
    讲座 & 开展时间 & 刊载时间\\
    \hline\hline

  人机协同背景下高等外语教育的守正创新 & 2.27 & 2.17\\\hline
    大陆的起源 & 3.4 & 2.17\\\hline
    中国中古《孙子算经》在日本的受容 & 2.25 & 2.20\\\hline
    南京“世界文学之都”的前世今生 & 2.27 & 2.20\\\hline
    复杂异构大数据治理与分析关键技术及应用 & 2.25 & 2.20\\\hline
    香港大学经管学院硕士课程高校专场线下宣讲会 & 2.27 & 2.20\\\hline
    电子平带材料中的关联与拓扑 & 2.25 & 2.21\\\hline
    2025香港大学暑期课程宣讲会 & 2.26 & 2.21\\\hline
    在哲学与艺术之间 & 2.26 & 2.23\\\hline
    Nonlinear magneto-optical effect in 2D magnets & 2.27&2.24\\\hline
    社会分层视野下的教育失配研究:博士论文的写作、拓展与反思 & 2.26 & 2.25\\\hline
    MIMOS: the asynchronous paradigm and tools for safety-critical software design and updates & 2.27 & 2.25\\\hline
    长江文明系列讲座第一期 & 2.27 & 2.25\\\hline
    \hline
\end{tabular}
1.合美博士餐叙第七十五期\\
讲座题目:社会分层视野下的教育失配研究:博士论文的写作、拓展与反思\\
摘要:教育失配反映个体的学历文凭和职业岗位的不匹配状况。主讲人将以博士论文为起点,深入分析教育失配的趋势与影响,指出教育失配的根源。\\
主讲人:李晓光 西安交通大学社会学系教授、博士生导师,实证社会科学研究所研究员\\
讲座时间:2月27日12:00-14:00\\
讲座地点:仙林校区社会学院河仁楼401室\\
报名方式:现场免费提供50份简餐,欲就餐者请发送短信“合美餐叙+姓名+学号/工号”至13408642952报名(截至2月26日下午17:00),以收到短信先后为序,社会学院教师与博士生优先。\\


2.物理学院学术报告会\(第50期\)\\
题目:Nonlinear magneto-optical effect in 2D magnets\\
报告人:Shiwei Wu, Fudan University\\
时 间:2025年2月27日\(周四\)15:30\\
地 点:鼓楼校区唐仲英楼B501\\
\url{https://mp.weixin.qq.com/s/xo1H5w4G8JA6u6jZwQwU0Q}\\

3.CSAI 卓越科学家大讲堂\\
题目:MIMOS: the asynchronous paradigm and tools for safety-critical software design\(\& updates\)\\
主讲人:Prof. Wang Yi 欧洲科学院院士 IEEE/ACM Fellow Uppsala University\\
时间:2月27日\(星期四\)11:00\\
地点:计算机科学技术楼230室\\
\url{https://mp.weixin.qq.com/s/FTgkxX4uF9kTMxVoKy7PWg}\\

4.长江文明系列讲座第一期:考古学与长江文化\\
作为长江流域重要的史前至青铜时代的聚落,磨盘山遗址的文化内涵对于认识长江流域的文明起源、文明演进和多元一体中华文明的形成过程提供了丰富的信息和独特的视角。鉴于此,南京大学博物馆将推出长江文明系列讲座。\\
主持人:赵东升\\
主讲人:贺云翱\\
时间:2025.02.27(周四)14:30-16:30\\
地点:南京大学仙林校区星云楼1F\\
线下参与方式:因展厅空间有限,落座观众限报60人,先到先得,报满即止,报名二维码详见原文。\\
线上同步直播:届时,本次导览将在南京大学博物馆视频号上进行现场直播,无法到场的小伙伴也可在线观看。\\
\url{https://mp.weixin.qq.com/s/eug_AjjGUXKRRPh7vNsjMA}
\section{部分课程增加名额的通知(二)}
链接:\url{https://jw.nju.edu.cn/5d/74/c26263a744820/page.htm}\\

\section{仙林校区图书馆“小蓝鲸”义工岗招募}
本次义工岗只针对仙林校区的同学进行招募,工作地点为杜厦图书馆,招募职位为“图书馆见习管理员”和“图书馆座位清理员”。具体工作内容、工作时间及其他注意事项详见“南京大学学生会”公众号文章。扫描二维码填写报名问卷,人数满即止。\\
链接:\url{https://mp.weixin.qq.com/s/EU2mtAXoIquSIR3O_SqcBg}\\

\section{25春悦读导读班课程群及助教信息}
链接:\url{https://jw.nju.edu.cn/5d/4f/c26263a744783/page.htm}\\


\section{通知 | 心理咨询网络预约指南}
南大心理中心特别推出心理咨询网络预约咨询服务,详细操作指南见下:\\
\url{https://mp.weixin.qq.com/s/yA3bE4VniZ1fwpmCwQ6yYQ}\\

\section{赛事预告丨KOOK杯英雄联盟友谊赛}
赛事时间:2月25日-3月10日\\
链接:\url{https://mp.weixin.qq.com/s/nXMvmk2YapSk3RFIWb5PIg}\\


\section{“核真录”招新}
“核真录”是一个结合新闻实践、内容运营为一体的院媒平台,也是南京大学新闻传播学院《事实核查》实务课程的实践平台。我们希望,你也愿意讲述“后真相”时代新闻行业“事实核查”的理念与实践,审视新闻专业主义在这个时代的价值;我们希望,你也期待着追寻新闻理想、以不变的热忱文字拥抱涓涓不壅的变化。最近一年来,核真录继续坚持开展事实核查,持续追踪热点,产出优质稿件。公众号累计发布原创稿件42篇,稿件阅读总次数近13万次,单篇最高达4.5万次。多篇文章被虎嗅、腾讯新闻、澎湃新闻等平台转载。本次招新岗位包括社会时政组的国际新闻方向、国内新闻方向、跨学科方向。招新人数、应聘条件和报名方式详见“核真录”公众号文章。报名截止日期为3月2日24:00\\
链接:\url{https://mp.weixin.qq.com/s/giSazS0bcXe7YQdw4k6ucw}\\

\section{暑期项目 | 政治大学国际夏日学院课程}
课程时间2025年6月30日-8月8日\\
授课语言:全英文\\
课程一览:\\
《东亚的历史、文化与可持续社会》\\
《台湾人工智能、市场营销与半导体的当代发展》\\
《普通话及其他:沉浸在台湾的语言和传统中》\\
《台湾人工智能、市场及半导体的当代发展》\\
名额:\\
免学费名额:5名\\
自费名额:若干\\
报名截止时间:3月2日24点\\
详见原文:\url{https://mp.weixin.qq.com/s/iD5-NwJIivyKZ2BFucsTTg}

\section{首届全球阅读大奖暨“未来知识领主”计划征集活动}
本次活动在“知鸿蒙”平台“全球阅读大奖赛”专区进行,分为阅读空间的运营与阅读帖子的创作两个部分。进入知鸿蒙小程序或APP,点击“全球阅读大奖”海报即可参与。\\
分为学校类阅读空间和领域类阅读空间。\\
活动举办:2025年1月22日——2025年10月15日。\\
年度奖项评选:2025年10月16日——2025年12月25日。\\
年度奖项公示:2025年12月25日——2025年12月30日。\\
具体情况详见:\url{https://mp.weixin.qq.com/s/1inR-aWUySuv7TXnVjQaHw}

\section{“菁菁南数”讲师团招募}
讲师团授课内容:数学分析、高等代数、微积分II(第一层次)、线性代数(第一层次)、微积分II与线性代数(第二层次),授课时长每次2课时左右\\
1、招募对象\\
全体在读本科生、研究生(专业不限)\\
2、服务人群\\
以全校大一新生为主,形式线上or线下\\
3、技能要求\\
熟悉相关课程知识(难度不大,只要愿意“讲”,都欢迎加入)\\
4、综合素质\\
有兴趣、有热情、有耐心,乐于为同学们分享数学知识,愿意为同学们答疑解惑,认真对待每一堂讲座课程\\
报名截止时间:2025年3月9日\\
报名方式等见原文\url{https://mp.weixin.qq.com/s/1Y5Mp82hntrK9D1bLSEmJw}\\

\section{江苏省心理学会2025年学术年会暨征稿通知(第一轮)}
会议主题:心理学与新兴技术的融合:创新、应用与发展\\会议时间及地点:2025年10-11月(具体日期待定)南京大学仙林校区\\会议征稿:截止日期为2025年8月31日\\会议内容、征稿类型、投稿方式等详情请见\url{https://mp.weixin.qq.com/s/k6W7sB0y9BvQ7RgzYegZzw}\\

\section{24年冬大学英语四六级考试成绩查询}
24年冬季举行的大学英语四六级考试将于本周三早6点开放分数查询。可在中国教育考试网(\url{http://cet.neea.edu.cn/cet})或中国教育考试网微信小程序查询分数。申请纸质证明者领取方式会另行通知。

\section{首届全球阅读大奖暨“未来知识领主”计划征集活动}
活动流程\\
(一)活动举办:\\
2025年1月22日——2025年10月15日。\\
(二)年度奖项评选:\\
2025年10月16日——2025年12月25日。\\
(三)年度奖项公示:\\
2025年12月25日——2025年12月30日。\\
参赛须知\\
(一)主办方对所有参赛作品(含图、文、视频或其他艺术类作品)拥有发表、出版、展览、开发文创等使用权,不另计稿酬,参赛即视为同意。\\
(二)作者享有署名权。\\
(三)因此活动为首届举办的原创性活动,相关活动细则如有未尽之处,组委会将在后续通知中补充或调整。\\
(四)活动最终解释权归大赛组委会。\\
报名相关事宜以及更多详情参见\url{https://mp.weixin.qq.com/s/1inR-aWUySuv7TXnVjQaHw}\\

\section{秉文书院|鼓楼医院志愿服务招募}
活动时间:每周二、四14:00-17:00\\
活动地点:南京大学医学院附属鼓楼医院门诊大厅\\
参与人员:秉文书院2024级学生\\
人员设置:每次6-10人\\
活动准备:志愿者需参加医院组织的岗前培训,将分为导诊组和自助设备协助组,并指定小组长负责协调工作。\\
活动流程、志愿服务内容、报名QQ群详见\url{https://mp.weixin.qq.com/s/dyrb7XmELyZOh7cE9VMx2Q}

\end{multicols} 
\hrule
\vspace{4mm}
\centerline{\huge\textbf{参考消息}}
\begin{multicols}{2}
\section{南哪消息同学小文连载板块}
因收到小说投稿一篇,南哪消息现在开辟了同学小文连载板块。如想评论,可以发至邮箱:1329527951@qq.com,第二天会刊在此处。如想投稿渠道相同。
\section{《等待,遗忘》(2)}
金映樺\\

第二天\\
\newCJKfontfamily\fan{FandolFang}\fan
睡着了总会觉得很冷,但她今天做了一个炽热的梦,没等闹钟响起便醒来,转眼又忘记。悄无声息地走下楼,但屋里其实并没有能被吵醒的东西,只有稀疏的黑暗。她像往常一样打开信箱,却发现里面空荡荡的,甚至连灰都没有。好奇怪,好像有什么东西被打破了。她闭上眼,努力回想是否提前收到过讯息,依稀想起昨天送报员问她能不能一星期一送,“附近只有你们一家订报纸”。她点点头,莫名感到很抱歉。\\

好像总是这样,她总是不断地感到抱歉。从前在苛责或爱护中为自己流下的眼泪而抱歉,现在为麻烦他人或是自己的先入为主而抱歉。小时候朋友送过一个八音盒,翻开木盖盒子里的女孩便开始跳舞,她想,如果从盒子中站起来的是自己,可能只会不断地重复对不起。她想起自己去过的寺庙、道观和教堂,她不知道那里是否有神灵栖居,天地圣众歆享了牲醴和香烟之后又是否会给予人们无限的幸福。或许运气传播的介质是烟,所以大家上香之后都有一种被庇护的安心感。她唯一能够确认的是,生活的不幸是会自动繁衍的。人没有办法成长为自我能量充足者,或许就如拉康所谓“理想自我”,只是想象中的构想,渴望成为它,但永远无法完全达到。或许,将其不断建诸于他者的期许、凝视、欲望之上,才是唯一的答案。就像她从前做的那样。\\

她坐在沙发上,客厅没有开灯,太阳也还未升起,只能依稀看见事物的轮廓,总归看不真切。她相信灵魂的存在,因为此刻它正在飘远,想离开,永远不再回来。但是闹钟响起,于是它又被拽回,注定无法走远。她将闹钟关掉,屏幕正中间的数字提示着现在是早上六点整,漫长的一天才刚刚开始。关掉勿扰模式,许多消息跳出来,应用图标左上角的小红点令人难受。点开一看,皆是母亲的消息,一条又一条问着昨天的那个男生。她点开对话框,想敷衍地回一句“不错”,却想到他昨天尴尬的玩笑,傻傻的笑脸;她不禁弯起嘴角,“很有意思”。母亲居然秒回,这让她有些不安,好像走在森林里却被猎人盯上,目光如影随形。她无心再看,关掉手机,仰头坐在沙发上。余淮,他叫余淮,这是个很忧郁的名字,但是他却阳光开朗明媚,好像所有褒义词放在他身上都再合适不过。那天或许应该留一下联系方式的,她想,不过总会再见的。她想起那天他生疏拙劣的借口,不知怎的隐隐有些期待,上一次吃酸菜鱼是在什么时候?她想不起来。\\

她隐隐约约想起昨晚的梦。在梦中她和他人拥抱,很久很用力遂感到窒息的拥抱。他们聊天,只是聊天,没有目的地聊天。谈论生命的意义和起源,说到死亡、性和遥远的星系,外星人的存在,最喜欢的音乐和电影,最害怕的事物,说过的最大的谎言,童年、父母,最喜欢的颜色,最珍贵的记忆,最喜欢的动物,最向往的生活。\\

她听见自己说,“我不了解我自己,却希望你能了解我。”\\

她听见自己祈求,“爱我,只爱我一个,爱我的所有,我的阴影与扭曲;看见我人性中最恶劣最不敢示人的一面,但还是控制不住地爱我,想要和我在一起,不会转身不会离开不会消失。爱我,因为我是我,不是其他任何人,没有理由地爱我。”\\

但是,她无论如何也想不起那人是谁,他的脸被雾笼罩着,努力了也看不清。解梦的网站说梦见他人意味着可能会找回已经失去的钱财或损失,梦见与人愉快地交谈,预示着工作上的顺利,但家庭生活会出现罅隙——真是能量守恒的说法。\\

她愣了一会,鬼使神差地拿起手机,打开微信,看到有人加她好友,验证消息写着,\\

“你好,我是余淮。”\\

她想起来了。那是余淮的脸。\\

\end{multicols} 

\end{document}