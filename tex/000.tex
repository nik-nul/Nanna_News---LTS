% HEAD BEGIN
\documentclass[letterpaper, 12pt]{article}
\newsavebox\colbbox
\usepackage{graphicx}
\usepackage{multicol}
\usepackage{anysize}
\usepackage{fontspec}
\usepackage[fontset=none]{ctex}
\usepackage{tabularx}
\usepackage{longtable}
\PassOptionsToPackage{hyphens}{url}
\usepackage[breaklinks=true, colorlinks=true]{hyperref}
\expandafter\def\expandafter\UrlBreaks\expandafter{\UrlBreaks\do\a\do\b\do\c\do\d\do\e\do\f\do\g\do\h\do\i\do\j\do\k\do\l\do\m\do\n\do\o\do\p\do\q\do\r\do\s\do\t\do\u\do\v\do\w\do\x\do\y\do\z\do\A\do\B\do\C\do\D\do\E\do\F\do\G\do\H\do\I\do\J\do\K\do\L\do\M\do\N\do\O\do\P\do\Q\do\R\do\S\do\T\do\U\do\V\do\W\do\X\do\Y\do\Z}
% \let\oldurl\url
% \renewcommand{\url}[1]{\begin{sloppypar}\oldurl{#1}\end{sloppypar}}
\setlength\columnsep{30pt}
\marginsize{30pt}{30pt}{10pt}{20pt}
\setmainfont{TeX Gyre Bonum}
\setCJKmainfont[BoldFont=Noto Serif CJK SC Bold, ItalicFont=FandolKai]{Noto Sans CJK SC}
\setlength{\parindent}{0cm}
% \setCJKmonofont{Noto Sans CJK SC}
\begin{document}
\begin{center}
    \Huge\textbf{南哪大专醒前消息}
\end{center}
\vspace{4mm}
\hrule
\renewcommand\tabularxcolumn[1]{m{#1}}
\begin{tabularx}{\textwidth}{>{\hsize.2\hsize}X>{\hsize.6\hsize}X>{\hsize.2\hsize}X}
    \begin{flushleft}
        2024.12.9\, No.139
    \end{flushleft}
    &
    \begin{center}
        \textit{“秉中持正、求新博闻。”}
    \end{center}
    &
    \begin{flushright}
        \textbf{南京市栖霞区}
    \end{flushright}
\end{tabularx}
\vspace{-3.5mm}
\hrule
\vspace{4mm}
% HEAD END
\centerline{\huge\textbf{活动预告}}
\begin{multicols}{2}
    \section{订阅方式和加入编辑部}  
编辑部招聘人才,用爱发电,工作轻松,详情可联系QQ:1329527951 客服小祥\\想订阅本消息或获取PDF版(便于查看超链接和往期),可加QQ群:\href{https://qm.qq.com/q/VXIW7fgsEe}{849644979}.
\section{Deadline Ongoing}
\setbox\colbbox\vbox{
\makeatletter\col@number\@ne
\begin{longtable}{|c|c|c|}
    \hline
    消息(未见ddl的,不刊) & 截止日期 & 刊载日期\\
    \hline\hline
    紫藤学刊征稿 & 12.15 & 10.22\\
    安邦征稿 & 1.12 & 11.16\\
    猫鼠大战 & 12.15 & 11.22\\
    防艾征集 & 12.10 & 11.22\\
    心理中心征稿 & 12.10 & 11.23\\
    创意物理实验竞赛 & 12.21 & 11.15\\
    仙林通宵自习室 & 1.12 & 11.26\\
    防艾同伴教育 & 12.15 & 11.29\\
    防艾文艺作品征集 & 12.10 & 11.29\\
    南京高校戏曲交流 & 12.15 & 12.2\\
    全国大学生家史大赛 & 1.31 & 12.2\\
    25年南大会学团报名 & 12.11 & 12.4\\
    女性劳动策展工坊 & 12.16 & 12.4\\
    防艾剧本杀 & 12.15 & 12.5\\
    金融消费者大赛 & 12.31 & 12.5\\
    药丸周边征稿 & 12.15 & 12.6\\
    南大演说家决赛 & 12.10 & 12.6\\
    四六级准考证打印 & 12.14 & 12.6\\
    花旗杯报名 & 1.3 & 12.6\\
    挑杯校园双选会 & 12.15 & 12.7\\
    南新读书会 & 12.11 & 12.7\\
    朋辈数模分享 & 12.15 & 12.7\\
    羊山公园环保市集 & 12.15 & 12.7\\
    C Through代码大赛 & 12.15 & 12.8\\
    数学公共课答疑报名 & 12.11 & 12.8\\
    教师实习经验分享会 & 12.14 & 12.9\\
    澳门回归对谈会 & 12.15 & 12.9\\
    午餐读书会 & 12.11 & 12.9\\
    健雄朋导分享会 & 12.11 & 12.9\\
    西安史学论坛征稿 & 3.20 & 12.9\\
    地海暖冬活动报名 & 12.11 & 12.9\\
    \hline
\end{longtable}
\unskip
\unpenalty
\unpenalty}\unvbox\colbbox
\end{multicols}
\hrule
\pagebreak
\begin{multicols}{2}

\section{讲座}
\begin{tabular}{|c|c|c|}
    \hline
    往期讲座 & 开展日期 & 刊载日期\\
    \hline\hline
    《专利查新与规避...》 & 12.19 & 10.3\\
    basics on ... & 12.11 & 12.2\\
    《在线时尚零售的...》 & 12.11 & 12.6\\
    《中国文化中的点心》& 12.11 & 12.8\\
    《比较逻辑与社会...》 & 12.11 & 12.8\\
    《读雷吉斯•德布雷的媒介学》 & 12.13 & 12.9\\
    《佛教量论因明学...》 & 12.11 & 12.9\\
    《英帝国及其抒情...》 & 12.11 & 12.9\\
    《waveguide...》 & 12.12 & 12.9\\
    《数学研究中的what...》 & 12.11 & 12.9\\
    《读雷吉斯德布雷...》 & 12.13 & 12.9\\
    \hline
\end{tabular}

1.佛教量论因明学的体系构造\\
主讲人:顺真(贵州大学哲学学院教授)\\
主持人:张建军(南京大学哲学学院教授)\\
时间:12月11日(周三)19:00-21:00\\
地点:哲学学院(薛光林楼)401报告厅\\
摘要:既然“空”“有”二宗已然建立,则又何以必然性地开出“量论”“因明”体系呢?本讲座直接明确“量论宗”“因明学”的学科性质,进而从“陈那的思想历程”、“因明与逻辑”等诸多论域出发,精要阐释构成“佛教量论因明学”体系的基底骨架及其相关的核心论题。\\

2.“通往印度的路”:英帝国及其抒情诗人\\
侨裕讲坛 第202期\\
讲座人:程巍 中国社会科学院外国文学研究所研究员\\
主持人:何宁 南京大学外国语学院教授\\
讲座地点:外国语学院侨裕楼303会议室\\
讲座时间:2024年12月11日(周三)16:15-18:15\\
摘要:来由法国人设计和开凿的苏伊士运河却最终不是一条“通往越南的路”,因为英国人很快就获得了这条沙漠运河的经营权,而英国当初有关这条“臭水沟”的酷评也随着英国的接手而在文学中逐渐浪漫化,变成一条“通往印度的路”。十九世纪英帝国的抒情诗将殖民帝国的大工程涂抹上一道浪漫色彩,召呼着殖民者们“肩负起自人的重担”。\\
(编辑:Wheelrunner)

3.物理学院学术报告会(第44期)\\
题 目:Waveguide Quantum Electrodynamics: manipulating the interaction of photons and atoms\\
报告人:吕新友 华中科技大学\\
时 间:2024年12月12日(周四)15:30\\
地 点:鼓楼校区唐仲英楼B501\\
讲座简介等见原文\url{https://mp.weixin.qq.com/s/zg84_GsH7GhaH083MAJyDg}\\

4.数学学院本科生论坛(教师系列第91讲——大一专场)\\
题目:数学研究中的What,Why,How\\
报告人:尤建功 南开大学\\
时间:12月11日(星期三)16:00-17:30\\
地点:戊己庚四楼北\\
腾讯会议:870-7007-3326\\
简介见原文\url{https://mp.weixin.qq.com/s/f8XYiXiYClKq54hy4nuviw}\\

5.指向月亮的那根“手指”:再说媒介学 ——读雷吉斯•德布雷的媒介学\\
主讲人:黄旦 浙江大学文科资深教授\\
主持人:孙玮 复旦大学新闻学院教授复旦大学信息与传播研究中心主任\\
时间:2024年12月13日(周五)上午10:00-12:00\\
地点:仙林校区新闻传播学院紫金楼215\\

6.“软件新技术讲坛” 学术报告\\
标题:大模型垂直领域适配技术探索与应用\\
时间:204年12月11日(星期三) 10:00\\
地点:计算机科学技术楼230室\\

\section{教师实习经验分享会}
活动时间:2024年12月14日14:30-16:30\\
活动地点:杜厦图书馆125教室\\
本次活动旨在帮助南研学子更好地了解教师职业发展的实际情况,为未来的职业规划提供指导和帮助。\\

\section{澳门回归25周年圆桌对谈会}
活动时间:12月15日(周日)14:00\\
活动地点:仙Ⅰ-112\\
参与方式:点击链接扫码报名\url{https://mp.weixin.qq.com/s/qyDKFgD4uudCKu9IcDGbXw}\\



\section{新生午餐读书会第七场}
活动时间:2024年12月11日(周三)中午12:15-13:40\\
地点:鼓楼校区南大出版社党建活动室\\
主讲老师:新闻传播学院 胡翼青\\
1.米歇尔、汉森主编:《媒介研究批评术语集》,南京大学出版社,2019年版。 \\
2.库尔德利:《现实的中介化建构》,刘泱育译,复旦大学出版社,2023 年版。\\
流程:两位同学发言各15分钟,提问讨论25分钟,老师评论导读30分钟\\
本次活动向参加读书会活动的全体老师同学提供午餐\\
抽签码见\url{https://mp.weixin.qq.com/s/E_slAiaVTsb1pyhtArIxSA}
(小编:草昌思)\\

\section{健雄书院朋导分享会}
主题:健雄书院学科认知分享会\\
活动时间:2024年12月11日下午16:10\\
活动地点:鼓楼校区教学楼101\\
活动对象:健雄书院全体同学\\
本期嘉宾:\\
谭荣熙(人工智能学院 2021 级本科生。曾担任开甲书院朋辈导师,曾发表人工智能重要会议期刊论文 2 篇,参与校企合作研究项目)\\
将和大家分享交流在人工智能/智能科学与技术专业方面的学习体验、科研经验、职业规划等\\
熊丘桓(软件学院 2024 级硕士研究生,现任健雄书院助理辅导员。曾在软件学院和健雄书院担任朋辈导师,曾入选腾讯犀牛鸟精英人才计划,参与校企合作科研项目)\\
将从软件工程专业的课程学习、学术研究和技术应用方面和大家交流\\
报名码见\url{https://mp.weixin.qq.com/s/m3jBzwAu8VvVrxFaWSuxQg}
(小编:草昌思)\\

\section{ 第十五届西安史学“新潮”论坛征稿启事}
第十五届西安史学“新潮”论坛拟定于2025年4月于西北大学举办。现面向全国热爱历史学的本科生征稿。
征稿截止日期:2025.3.20\\
参会事宜与征稿须知等详见原推文\url{https://mp.weixin.qq.com/s/fidTuP-5mHyA80W7sZs7Vw}
(小编:忘情水)

\section{拾光坞暖冬主题月系列活动}
1 冬日拾光 \\
每日打卡,分为分享、运动、阅读三种打卡形式,即日开始,累计完成14天即可获奖。\\
2 于冬致春\\
收集写给春天的话语和祝福,可以写给自己,可以写给别人,也可以单纯赞美暖冬、期待春日到来。工作人员会进行汇总,并选择一部分定制明信片,在12月末发放给大家。同时提供“代存”服务,在春分时刻将明信片or信件送到指定地点。投稿截止时间为12月16日。\\
3 研途送暖\\
12月末地海学院领导和老师们将会来到各实验室及研究生学习室,为同学们送上新年的问候与祝福。\\
若有想要探讨的话题,或是心中有期待的新年礼物,可在12月11日前扫码填写收集表。\\
活动群、奖品设置、投稿方式等见原文\url{https://mp.weixin.qq.com/s/81a-ddtZFIbwUR9InvRC1A}
\end{multicols} 
\end{document}