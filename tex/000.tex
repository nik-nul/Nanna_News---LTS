% HEAD BEGIN
\documentclass[letterpaper, 12pt]{article}
\newsavebox\colbbox
\usepackage{graphicx}
\usepackage{multicol}
\usepackage{anysize}
\usepackage{fontspec}
\usepackage[fontset=none]{ctex}
\usepackage{tabularx}
\usepackage{longtable}
\PassOptionsToPackage{hyphens}{url}
\usepackage[breaklinks=true, colorlinks=true]{hyperref}
\expandafter\def\expandafter\UrlBreaks\expandafter{\UrlBreaks\do\a\do\b\do\c\do\d\do\e\do\f\do\g\do\h\do\i\do\j\do\k\do\l\do\m\do\n\do\o\do\p\do\q\do\r\do\s\do\t\do\u\do\v\do\w\do\x\do\y\do\z\do\A\do\B\do\C\do\D\do\E\do\F\do\G\do\H\do\I\do\J\do\K\do\L\do\M\do\N\do\O\do\P\do\Q\do\R\do\S\do\T\do\U\do\V\do\W\do\X\do\Y\do\Z}
% \let\oldurl\url
% \renewcommand{\url}[1]{\begin{sloppypar}\oldurl{#1}\end{sloppypar}}
\setlength\columnsep{30pt}
\marginsize{30pt}{30pt}{10pt}{20pt}
\setmainfont{TeX Gyre Bonum}
\setCJKmainfont[BoldFont=Noto Serif CJK SC Bold, ItalicFont=FandolKai]{Source Han Sans SC}
\setlength{\parindent}{0cm}
% \setCJKmonofont{Noto Sans CJK SC}
\begin{document}
\begin{center}
    \Huge\textbf{南哪大专醒前消息}
\end{center}
\vspace{4mm}
\hrule
\renewcommand\tabularxcolumn[1]{m{#1}}
\begin{tabularx}{\textwidth}{>{\hsize.2\hsize}X>{\hsize.6\hsize}X>{\hsize.2\hsize}X}
    \begin{flushleft}
        2025.4.16\, No.222
    \end{flushleft}
    &
    \begin{center}
        \textit{“秉中持正、求新博闻。”}
    \end{center}
    &
    \begin{flushright}
        \textbf{南京市栖霞区}
    \end{flushright}
\end{tabularx}
\vspace{-3.5mm}
\hrule
\vspace{4mm}
% HEAD END
\centerline{\huge\textbf{活动预告}}
\begin{multicols}{2}
\section{订阅方式和加入编辑部}  
编辑部招聘人才,用爱发电,工作轻松,详情可联系QQ:1329527951 客服小千\\想订阅本消息或获取PDF版(便于查看超链接和往期),可加QQ群:\href{https://qm.qq.com/q/4HL41Nt3sQ}{466863272}.
\section{活动清单}
\setbox\colbbox\vbox{
\makeatletter\col@number\@ne
\begin{longtable}{|>{\centering\arraybackslash}m{.3\textwidth}|m{.06\textwidth}|m{.06\textwidth}|}
    \hline
    活动 & 开展时间 & 刊载时间\\
    \hline\hline
    南大版deepseek & / & 2.22\\
    悦读课程群 & / & 2.24\\
    eScience AI科研助手 & / & 3.11\\
    地科博物馆开放安排 & / & 3.22\\ 
    2025年分流和转专业政策通知 & / & 4.7\\
    乐跑 & 5.16 & 3.10\\
    本科生劳育实践 & 7.20 & 2.19\\
    银星杯论文赛 & 4.22 & 2.27\\
    高教社杯 & 4.25 & 3.5\\
    大文大理题目征集 & 期末 & 3.8\\
    5月免费上网 & ? & 3.9\\
    基础学科论坛 & 4.20 & 3.9\\
    外教社杯 & 5.27 & 3.12\\
    江苏创青春赛事 & 4.30 & 3.26\\
    浦口音乐跑 & 5.30 & 3.31\\
    程设大赛 & 4.26 & 4.2\\
    瑞声杯 & 4.20 & 4.4\\
    仙林校区志愿法律咨询 & / & 4.4\\
    青春活力大赛 & 5.17 & 4.7\\
    在校生自愿体检 & 6.20 & 4.8\\
    南大购买WPS & / & 4.8\\
    24级程设大赛 & 4.27 & 4.11\\
    法治情景剧策划大赛 & 4.23 & 4.11\\
    仙林猫鼠游戏 & 4.19 & 4.12\\
    EL程设大赛 & 4.27 & 4.13\\
    全国行研大赛 & 4.20 & 4.13\\
    南大网双公开赛报名 & 4.18 & 4.13\\
    中美中心2025年证书项目 & 5.24 & 4.14\\
    心里剧本创作大赛 & 4.20 & 4.14\\
    体育月舞蹈教学2 & 4.18 & 4.15\\
    春季学期创新训练计划结题考核通知 & 4.28 & 4.15\\
    竺可桢讲师团义卖 & 4.18 & 4.15\\
    心理中心开放日 & 4.19 & 4.16\\
    植物微景观活动 & 4.23 & 4.16\\
    “天池杯”AI创新大赛 & 4.28 & 4.17\\
    
    \hline
\end{longtable}
\unskip
\unpenalty
\unpenalty}\unvbox\colbbox
\end{multicols}
\begin{multicols}{2}
\pagebreak

\section{讲座}
\begin{tabular}{|>{\centering\arraybackslash}m{.3\textwidth}|m{.06\textwidth}|m{.06\textwidth}|}
    \hline
    讲座 & 开展时间 & 刊载时间\\
    \hline\hline
    社交媒体分享实践的语用学研究 & 4.18 & 4.9\\\hline
    Deepseek现象中的管理学 & 4.18 & 4.10\\\hline
    智能时代的中国式养老:理论与实践”学术研讨会 & 4.18-20 & 4.10\\\hline
    从感知到疗愈:人脑音乐加工机制 & 4.25 & 4.11\\\hline
    Accretion-generated rings:coplaner and polar structures & 4.18 & 4.11\\\hline
    Ionizing spotlight of Active Galactic Nucleus & 4.23 & 4.11\\\hline
    人智协同式内容创作方法初探 & 4.17 & 4.15\\\hline
    在记忆星丛中走向他人 & 4.18 & 4.15\\\hline
    量子模型中一类新的超扩散机制 & 4.17 & 4.15\\\hline
    软件挖掘:机遇与挑战 & 4.17 & 4.15\\\hline
    A Statistical Theory of Contrastive Pre-training and Multimodal Generative AI & 4.17 & 4.16\\\hline
    Can Artificial Intelligence Improve Gender Equality? Evidence from a Natural Experiment & 4.17 & 4.16\\\hline
    人工智能与大模型的应用方法 & 4.18 & 4.16\\\hline
    科技之巅:人机共智2030 & 4.22 & 4.16\\\hline
    软件发展与技术漫谈 & 4.29 & 4.16\\\hline
    DeepSeek: 从人工智能到大模型及应用思考 & 4.24 & 4.16\\\hline
    从语言到智能 ⸺ 大语言模型的奥秘与应用 & 5.6 & 4.16\\\hline
    据出土与传世文献几个语法现象谈谈《左传》语言的时代性 & 4.19 & 4.17\\\hline
    汉语{每天}的编码类型及演变 & 4.19 & 4.17\\\hline
    句法结构的历史重建:理论方法与研究实践 & 4.19 & 4.17\\\hline
    “相因生义”界定及新证 & 4.19 & 4.17\\\hline
    现代中国诸社会思潮的地域起源 & 4.21 & 4.17\\\hline
    智慧物流助力智能制造 & 4.23 & 4.17\\\hline
    Magic moire materials & 4.18 & 4.17\\\hline
    Extended module categories & 4.19 & 4.17\\\hline
    Silting interval reduction and 0-Auslander extriangulated categories & 4.19 & 4.17\\\hline
    媒介见证的黄昏? & 4.21 & 4.17\\\hline
    社保知识助你维护个人权益 & 4.20 & 4.17\\\hline
    
\end{tabular}
%讲座预告写在这。用subsection
\subsection{“记忆”系列讲座预告|李红涛:媒介见证的黄昏?}
讲座信息
\\主 题:媒介见证的黄昏?
\\时 间:2025年4月21日(周一)19:00-21:00
\\主讲人:李红涛
\\ 复旦大学信息与传播研究中心研究员
\\ 复旦大学新闻学院教授
\\与谈人:于京东
\\ 南京大学政府管理学院副教授
\\地 点:南京大学仙林校区 仙Ⅱ-112教室
\\详见:\url{https://mp.weixin.qq.com/s/jK1dnu2JCkzl1562n8fPRg}
\subsection{“职场新生必修课”第二期 | 社保知识助你维护个人权益!}
第二期活动聚焦社会保险,特邀行业专家深入解读社保政策,解答应届生在求职与就业过程中可能遇到的社保相关问题,消除因政策盲区导致的权益保障风险。
\\活动时间:4月20日(周日)14:00-15:00
\\活动地点:南京大学仙林校区计算机学院111报告厅
\\参与方式:扫描二维码加入活动群聊。也欢迎填写你最关心的社保问题,将汇总反馈至嘉宾,灵活调整分享内容。
\\详见:\url{https://mp.weixin.qq.com/s/BgkepInua0ogHtXt0heBzA}
\subsection{4月21日 “软件新技术讲坛” 学术报告 - 贾晓华 教授}
时间:2025年4月21日(星期一) 15:00
\\地点:计算机科学技术楼230室
\\主讲人:贾晓华, 教授 香港城市大学
\\详见:\url{https://mp.weixin.qq.com/s/5DPp7gWQFAIMOfGXz1k3ug}
\subsection{文院历史语言学专题四讲}
时间:2025年4月19日(周六)

地点:南京大学文学院活水轩

第一场:8:30-10:00

据出土与传世文献几个语法现象谈谈《左传》语言的时代性

主讲人:史文磊 浙江大学文学院教授

主持人:李梓铭 南京大学文学院本科生

与谈人:程少轩 南京大学文学院教授

第二场:10:00-11:30

汉语{每天}的编码类型及演变

主讲人:董正存 中国人民大学文学院教授

主持人:杨思齐 南京大学文学院本科生

与谈人:王玲 南京大学文学院教授

下午:14:00-17:00

第三场:14:00-15:30

句法结构的历史重建:理论方法与研究实践

主讲人:盛益民 复旦大学中文系教授

主持人:蒋正时 南京大学文学院本科生

与谈人:孙凯 南京大学文学院助理研究员

第四场:15:30-17:00

“相因生义”界定及新证

主讲人:宋亚云 北京大学中文系长聘副教授

主持人:闫憬彤 南京大学文学院本科生

与谈人:张福通 南京大学文学院副教授

\subsection{南雍群学·青年学者讲堂——第五讲(孟庆延,中国政法大学)}
讲座题目:万类霜天竞自由:现代中国诸社会思潮的地域起源
\\主讲人:孟庆延,中国政法大学社会学院教授,博士生导师
\\讲座时间:2025年4月21日 14:00-16:00
\\讲座地点:南京大学社会学院(河仁楼)合美堂401室
\\详见:\url{https://mp.weixin.qq.com/s/LfLpxePNJslE71yDlc9hxg}

\subsection{讲座预告丨智慧物流助力智能制造}
讲座主题:智慧物流助力智能制造
\\讲座时间:4月23日(周三)20:00开始
\\主讲嘉宾:徐正林江苏六维智能物流装备股份有限公司董事长、总经理,南京理工大学客座教授和硕士研究生导师
\\讲座地点:腾讯会议 线上讲座
\\报名请扫描二维码,报名成功报名后,将于讲座开始前发送会议地址
\\详见:\url{https://mp.weixin.qq.com/s/i6GjOLSTljD-ZomDrxYOPg}

\subsection{数院学术报告预览}
题目: Magic moire materials
\\报告人:Simon Becker(ETH Zurich)
\\时间: 2025年4月18日 10:00-11:00
\\地点:蒙民伟楼1105
\\
\\题目:Extended module categories
\\报告人:周宇 教授(北京师范大学)
\\时间: 2025年4月19日 15:30
\\地点:西大楼 108
\\
\\题目:Silting interval reduction and 0-Auslander extriangulated categories
\\报告人:朱彬 教授(清华大学)
\\时间: 2025年4月19日 16:30
\\地点:西大楼 108
\\详见:\url{https://mp.weixin.qq.com/s/dJuXwlP6LNBIuM8YicJPcg}


\section{“南学之声”权益问题反馈平台升级}
“南学之声”是由南京大学学生会公共权益部设计运营的权益问题反馈平台。从设施改进到活动反馈,从宿舍体验到课程建议,无论是您在校园生活中遇到的问题,还是您对我们的工作有更好的建议,我们都欢迎您通过本平台畅所欲言,我们将会尽快对您提出的问题进行核实、处理,并及时对处理情况进行跟进与反馈~此次新增了问题处理进度查询功能。
\\平台入口位于“南京大学学生会”微信公众号下方菜单栏“权益南大”板块,在此板块选择“南学之声”,即可自动跳转到反馈平台页面。
\\具体使用指南,
\\
\\
\\详见:\url{https://mp.weixin.qq.com/s/thJqQ9kzy4oyRksFvQecmA}
\section{2024-2025学年第二学期数学期中考试安排}
2024-2025学年第二学期数学期中考试时间为2025年4月19日(周六)
\\具体安排和注意事项见链接
\\详见:\url{https://jw.nju.edu.cn/85/59/c26263a755033/page.htm}

\section{“天池杯”AI创新大赛}
报名和组队要求
\\1.参赛对象:南京大学全体本科生、南京重点高中学生
\\2.组队规则:参赛团队自由组队,每队1-4人(含团队负责人),鼓励跨学段、跨学科组队,高中生可自行成团或加入本科生团队。
\\3.指导老师:每队可邀请1名中学、大学教师或阿里云资深技术导师担任指导,提供项目优化建议(指导教师不占用组队名额)
\\报名(4月14日-4月28日)
\\具体见链接
\\详见:\url{https://jw.nju.edu.cn/85/92/c26263a755090/page.htm}
\section{院级活动}
\begin{tabular}{|>{\centering\arraybackslash}m{.3\textwidth}|m{.06\textwidth}|m{.06\textwidth}|}
\hline
    活动 & 开展时间 & 刊载时间\\
    \hline\hline
    文院剧本创作研讨会 & 9.30 & 3.2\\
    物院征集课程指南 & 6.15 & 3.3\\
    地海征集春日影 & 6.15 & 3.14\\
    五院乒乓球赛 & 4.19 & 3.31\\
    法院党建征文 & 5.20 & 4.2\\
    地学乒赛 & 4.19 & 4.2\\
    软院征集 & 4.20 & 4.4\\
    地学趣运会 & 4.26 & 4.9\\
    四院音乐节 & 5.11 & 4.7\\
    商院征集 & 5.5 & 4.8\\
    毓秀羽球 & 4.20 & 4.8\\
    物院运动打卡 & 5.14 & 4.12\\
    地学定向越野 & 4.19 & 4.12\\
    物院开放日 & 4.19 & 4.16\\
    地海图书漂流 & 4.23 & 4.16\\
    法院趣运会 & 4.20 & 4.17\\
    史院求职分享 & 4.18 & /\\
    \hline
\end{tabular}
\subsection{法学院趣运会}
活动时间:2025年4月20日

活动地点:仙林校区炜华体育场/四组团体育场(具体地点后续在群内通知)

参与对象:南京大学法学院全体同学、行知书院全体同学

\subsection{}
\section{社团活动}
\begin{tabular}{|>{\centering\arraybackslash}m{.3\textwidth}|m{.06\textwidth}|m{.06\textwidth}|}
    \hline
    社团活动 & 开展时间 & 刊载时间\\
    \hline\hline
    天文台开放日 & / & 1.6\\
    重唱诗歌奖征稿 & 4.30 & 3.31\\
    拳击社体验 & 4.22 & 4.1\\
    轮滑社体验 & 4.22 & 4.1\\
    定向赛 & 4.20 & 4.1\\
    体育舞蹈教学 & 4.25 & 4.1\\
    吉他社歌手招募 & 4.20 & 4.4\\
    吉他社春日音 & 4.26 & 4.4\\
    天健捐衣 & 4.20 & 4.13\\
    毽球趣味赛 & 4.19 & 4.15\\
    流光影映 & 4.19 & 4.16\\
    汉服社摆摊 & 4.26 & 4.17\\
    车协科普 & 4.26 & 4.17\\
    招协招募 & 4.22 & 4.17\\
    摇联春日音 & 4.19 & 4.17\\
    心协团辅活动 & 4.23 & 4.17\\
    \hline
\end{tabular}
%这里是写社团活动的,社团活动就是由社团主办、主要针对社团内部人员的活动。不要把非社团活动写在这里。
\subsection{南韵汉服社十二周年庆祝活动}
南韵十二时辰 南韵汉服社十二周年庆祝活动
\\活动时间:2025.4.26 13:00-17:00
\\活动地点:仙林校区大学生活动中心入口以及一二楼连廊等空地
\\活动形式:十二时辰主题文化摊位游园会 体验式集章兑奖
\\详见:\url{https://mp.weixin.qq.com/s/zgQ9AYvwc6RH0SGaYe7g4Q}
\subsection{篮协明日赛程}
男篮院系杯小组赛
\\计科vs电子 12:30-13:30
\\物理vs匡院 20:30-21:30
\\女篮院系杯小组赛
\\政管vs外院 19:30-20:30
\\研究生院系杯小组赛
\\现工vs信管 18:30-19:30
\\地点:一组团篮球场
\\详见:\url{https://mp.weixin.qq.com/s/jArZM6_4a0JaE-vBKWED6Q}
\subsection{车协计成课堂南大站}
\\1.      科学的骑行方法与理念
\\2.      骑行训练与车辆选择的建议
\\3.      正确的运动饮食与休息恢复理念。
\\时间及地点
\\活动时间:4月26日 13:30-16:30
\\活动地点:南京大学仙林校区仙I-108
\\详见:\url{https://mp.weixin.qq.com/s/hto0F_i8vsLUnnxc_-yP5Q}
\subsection{成于传承丨招协新任中学理事招募}
中学理事以地区为单位(江苏省内以地级市为单位,江苏省外以省、直辖市、自治区为单位)进行招募,每中学1-2人,任期一年。任期结束后,将获得本科招办颁发的任职证明。
\\希望中学理事对招生宣传怀有热情,善于沟通、乐于协作,耐心细致、有责任心,能够高质量完成相关工作任务。优先考虑曾参与往届“南星梦想计划”的同学。优先考虑各年级本科生。
\\如有意向报名,请通过链接前往原推文下载报名表,并查看各地区招生宣传理事长的联系方式。请于2025年4月22日(周二)18:00前向各地区理事长提交报名表。后续不组织面试,将由各地区理事长根据报名表确定入围人选。
\\详见:\url{https://mp.weixin.qq.com/s/O98H6vzfQaoVrCPl0Nh4QQ}

\subsection{院系杯 | 决赛辩题选择}
请从下列候选辩题中勾选出
\\你最支持的决赛辩题(单选)。
\\截止时间:4月18日(周五) 22:00
\\语言是人们理解的桥梁/阻碍
\\故事的结局重要/不重要
\\见过了霍格沃兹的烟火,要/不要接受自己只是一个麻瓜
\\详见:\url{https://mp.weixin.qq.com/s/sl6srTllQWlgwNzHWEPYEg}
\subsection{摇联 spring party}
时间:4月19日18:30
\\地点:炜华体育场
\\节目单详见原文
\\详见:\url{https://mp.weixin.qq.com/s/rq4Abd1hO12EdQnmCkgETQ}

\subsection{5•25心理健康节 主题团辅活动 }
用OH卡开启自我探索之旅
\\活动时间:4月23日(周三)下午15:00
\\活动地点:仙Ⅱ-101
\\活动环节:
\\1.情绪摩天轮:
\\随机卡牌探索隐藏的自我投射
\\2.潜意识链接:
\\在回忆的流动中重新认识自己
\\3.时空之门:
\\构建专属心灵拼图的N种可能
\\详见:\url{https://mp.weixin.qq.com/s/7Dr5Mz6MhMIbq7MfDubELw}
\end{multicols}
\end{document}
