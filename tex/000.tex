% HEAD BEGIN
\documentclass[letterpaper, 12pt]{article}
\usepackage{graphicx}
\usepackage{multicol}
\usepackage{anysize}
\usepackage{fontspec}
\usepackage[fontset=none]{ctex}
\usepackage{tabularx}
\PassOptionsToPackage{hyphens}{url}
\usepackage[breaklinks=true, colorlinks=true]{hyperref}
\expandafter\def\expandafter\UrlBreaks\expandafter{\UrlBreaks\do\a\do\b\do\c\do\d\do\e\do\f\do\g\do\h\do\i\do\j\do\k\do\l\do\m\do\n\do\o\do\p\do\q\do\r\do\s\do\t\do\u\do\v\do\w\do\x\do\y\do\z\do\A\do\B\do\C\do\D\do\E\do\F\do\G\do\H\do\I\do\J\do\K\do\L\do\M\do\N\do\O\do\P\do\Q\do\R\do\S\do\T\do\U\do\V\do\W\do\X\do\Y\do\Z}
% \let\oldurl\url
% \renewcommand{\url}[1]{\begin{sloppypar}\oldurl{#1}\end{sloppypar}}
\setlength\columnsep{30pt}
\marginsize{30pt}{30pt}{10pt}{20pt}
\setmainfont{TeX Gyre Bonum}
\setCJKmainfont[BoldFont=Noto Serif CJK SC Bold, ItalicFont=FandolKai]{Noto Sans CJK SC}
\setlength{\parindent}{0cm}
% \setCJKmonofont{Noto Sans CJK SC}
\begin{document}
\begin{center}
    \Huge\textbf{南哪大专醒前消息}
\end{center}
\vspace{4mm}
\hrule
\renewcommand\tabularxcolumn[1]{m{#1}}
\begin{tabularx}{\textwidth}{>{\hsize.2\hsize}X>{\hsize.6\hsize}X>{\hsize.2\hsize}X}
    \begin{flushleft}
        2024.9.26\, No.72
    \end{flushleft}
    &
    \begin{center}
        \textit{“For man, when perfected, is the best of animals, \\but, when seperated from law and justice, \\he is the worst of all.”}
    \end{center}
    &
    \begin{flushright}
        \textbf{南京市栖霞区}
    \end{flushright}
\end{tabularx}
\vspace{-3.5mm}
\hrule
\vspace{4mm}
% HEAD END
\centerline{\huge\textbf{活动预告}}
\begin{multicols}{2}

\section{Deadline Ongoing}
\begin{tabular}{|c|c|c|}
    \hline
    消息(未见ddl的,不刊) & 截止日期 & 刊载日期\\
    \hline\hline
    仙林校史馆招募讲解员 & 10.30 & 9.12\\
    管道宣传志愿团队遴选 & 9.27 & 9.12\\
    国优计划报名 & 10.7 & 9.19\\
    本科生暑期课程评教 & 10.31 & 9.19\\
    网易雷火大赛 & 10.7 & 9.22\\
    部分课程增加名额 & 9.27 & 9.24\\
    大创训练计划申报 & 11.18 & 9.24\\
    历史学院宣传技能培训 & 9.28 & 9.24\\
    苏州校区音乐会 & 10.19 & 9.25\\
    外院国庆摄影征集 & 10.7 & 9.25\\
    历史学院新疆项目 & 9.30 & 9.25\\
    生涯力提升课报名 & 9.27 & 9.25\\
    Flicker周常影映 & 9.28 & 9.25\\
    II剧剧场讨论 & 9.27 & 9.25\\
    多阅志愿者招募 & 10.1 & 9.26\\
    留学生汉语朗诵比赛 & 9.29 & 9.26\\
    雨花成长计划课堂报名 & 10.3 & 9.26\\
    重唱诗社招新放映会 & 9.27 & 9.26\\
    音乐思政大课堂 & 9.29 & 9.26\\
    港澳台生中华文化大赛 & 10.9 & 9.26\\
    学子雁回工程 & 9.27 & 9.26\\
    \hline
\end{tabular}
\section{编辑部招聘人才}
编辑部招聘人才,用爱发电,工作轻松,详情可联系QQ:1329527951 客服小祥\\想订阅本消息或获取PDF版(便于查看超链接),可加QQ群:\href{https://qm.qq.com/q/FGX1VYCrGS}{962626571}.
\section{讲座}
\begin{tabular}{|c|c|c|}
    \hline
    往期讲座 & 开展日期 & 刊载日期\\
    \hline\hline
   《作为批评的文学史》 & 9.29 & 9.25\\
      \hline
\end{tabular}\\\\
\section{学子雁回工程}
9月27日(周五)上午,鼓楼区“学子雁回工程”进高校·南京大学站启动仪式暨青春嘉年华校园市集将在鼓楼校区大礼堂和苏浙运动场同时举办。\\
启动仪式:上午10点,校大礼堂将揭开鼓楼区“学子雁回工程”进高校·南京大学站序幕,同时还将举办第六届“创赢鼓楼”创新创业大赛颁奖活动。现场将邀请多位院士出席,并有创新创业大咖分享。\\
鼓楼区方面为同学们提供文创礼包,另外签到有盲盒小礼品。\\
校园市集:位于苏浙体育场,包括七大特色板块。\\
详见:\url{https://mp.weixin.qq.com/s/zXyMBGhH0E92rfs0l_Ny7w}
\section{志愿者招募}
南悦团队现通知“多阅阅读桥”乡村儿童远程阅读陪伴活动志愿者招募事宜。

多阅阅读桥是多阅从2021年开始的一个乡村儿童阅读推广项目。多阅公益将在我校招募95名志愿者,通过借助Classin软件在线直播授课的方式,为云南双柏县大庄中心小学的14个班级和查姆中心小学的33个班级提供为期8周的阅读陪伴服务。志愿者职责:

1、担任课程主讲,进行课前备课并讲授至少4次阅读课程(授课课件由多阅提供)

2、担任课程助教,记录课堂情况和学生表现,按时提交课堂反馈,协助课程开展

每完成一次阅读课程可获得3小时的志愿服务时长,参加培训等活动的时长另行计算。表现优异的志愿者会得到志愿者证书、活动纪念品等奖励。

报名详见:\url{https://mp.weixin.qq.com/s/IRDPi_Hb4bXPxCKv0D92wA}截止时间:10.1.
\section{留学生汉语演讲朗诵比赛}
2024年9月29日 下午4点鼓楼校区中美中心匡亚明报告厅(东楼一楼),开展来华留学生迎国庆汉语演讲朗诵比赛,邀请观众,答对问题就有机会赢取“精美”礼物(南大记忆文件夹)详见:\url{https://mp.weixin.qq.com/s/f7jLJCahz2yo_8ogUtj1cQ}
\section{新生接种乙肝疫苗和九价HPV疫苗的通知}
南大医院近期组织集体接种乙肝疫苗和九价HPV疫苗。集体接种时间为 9.27 至 9.29;乙肝检测结果和疫苗接种预约均需在“南京大学医院”微信公众号上完成。时间冲突者可另行预约。具体检测结果查询和疫苗接种预约流程,以及接种时间、注意事项等请见南京大学医院微信公众号的推送\url{https://mp.weixin.qq.com/s/aPg59YjkG4YTOoXAG7gqdw}。
\section{内地(大陆)高校港澳台学生中华文化知识大赛}
教育部港澳台事务办公室拟举办2024年内地 (大陆) 高校港澳台学生中华文化知识大赛。\\
参赛对象:在内地(大陆)高校学习的港澳台大学生和部分内地 (大陆) 大学生。 \\
比赛分三个阶段:校内报名、全国复赛和总决赛。\\
1. 校内报名:由学生自由组队后经南大台港澳交流协会筛选。根据筛选结果,每校推荐1支参加全国复赛的队伍(人数为3人,至少包含两名港澳台学生,以所持证件为准)。\\
2. 全国复赛:时间为2024年11月8日至19日,采用线上的参赛方式,形式为中华文化知识相关的答题。全国复赛按照所有参赛队分数由高至低排序,选取总分排名前20的队伍进入总决赛。\\
3. 总决赛:拟于2024年12月在线下举行,形式为中华文化知识相关的答题。决赛试题将不限于题库已有题目。\\
校内报名方式:建议组队参与,也可以个人名义报名。若为组队参与,每队人数要求为3人,至少包含两名港澳台学生。报名截止日期为10月9日,后续将安排面试遴选。\\
详情\url{https://mp.weixin.qq.com/s/D-N3vjMR2RCkwu8F6bdx-A}
\section{秋季学期“雨花斑斓成长计划”开课}
雨花斑斓成长计划”是南京大学党委学生工作部紧紧围绕立德树人这一根本任务,面向本科家庭经济困难学生开展能力培养提升的系列培训课程。\\
报名对象:南京大学2021-2024级本科生,优先考虑家庭经济困难学生。\\
报名要求:报名成功的学生需全程参与课程,完成全部课程将颁发结业证书、参与评选优秀学员。\\
报名方式:每生仅可参加1个班的课程学习,报名时可以选择1-3个平行志愿,录取以报名时间先后顺序进行排列,依照平行志愿的原则进行录取。\\
报名时间、地点:见各课程简介\\
截止时间:10月3日12:00\\
详情:\url{ https://mp.weixin.qq.com/s/D9qKATZP9Lpoii9HP2nkVw}
\section{庆祝建国75周年音乐思政大课堂启动}
为进一步弘扬爱国主义精神,增强广大师生的家国情怀,南京大学团委将举办“弦歌启征程,奋进谱新篇”南京大学庆祝中华人民共和国成立75周年音乐思政大课堂。届时将会由南京大学交响乐团带来16首经典曲目和师生代表推介曲目演出。\\
时间:2024年9月29日(周日)晚7:00\\
地点:南京大学仙林校区恩玲剧场\\
\section{重唱诗社招新放映会 | 《死亡诗社》}
9月27号晚七点在鼓楼校区新教103举行。
\end{multicols} 

\end{document}