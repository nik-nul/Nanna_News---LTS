% HEAD BEGIN
\documentclass[letterpaper, 12pt]{article}
\newsavebox\colbbox
\usepackage{graphicx}
\usepackage{multicol}
\usepackage{anysize}
\usepackage{fontspec}
\usepackage[fontset=none]{ctex}
\usepackage{tabularx}
\usepackage{longtable}
\PassOptionsToPackage{hyphens}{url}
\usepackage[breaklinks=true, colorlinks=true]{hyperref}
\expandafter\def\expandafter\UrlBreaks\expandafter{\UrlBreaks\do\a\do\b\do\c\do\d\do\e\do\f\do\g\do\h\do\i\do\j\do\k\do\l\do\m\do\n\do\o\do\p\do\q\do\r\do\s\do\t\do\u\do\v\do\w\do\x\do\y\do\z\do\A\do\B\do\C\do\D\do\E\do\F\do\G\do\H\do\I\do\J\do\K\do\L\do\M\do\N\do\O\do\P\do\Q\do\R\do\S\do\T\do\U\do\V\do\W\do\X\do\Y\do\Z}
% \let\oldurl\url
% \renewcommand{\url}[1]{\begin{sloppypar}\oldurl{#1}\end{sloppypar}}
\setlength\columnsep{30pt}
\marginsize{30pt}{30pt}{10pt}{20pt}
\setmainfont{TeX Gyre Bonum}
\setCJKmainfont[BoldFont=Noto Serif CJK SC Bold, ItalicFont=FandolKai]{Source Han Sans SC}
\setlength{\parindent}{0cm}
% \setCJKmonofont{Noto Sans CJK SC}
\begin{document}
\begin{center}
    \Huge\textbf{南哪大专醒前消息}
\end{center}
\vspace{4mm}
\hrule
\renewcommand\tabularxcolumn[1]{m{#1}}
\begin{tabularx}{\textwidth}{>{\hsize.2\hsize}X>{\hsize.6\hsize}X>{\hsize.2\hsize}X}
    \begin{flushleft}
        2025.3.9\, No.186
    \end{flushleft}
    &
    \begin{center}
        \textit{“秉中持正、求新博闻。”}
    \end{center}
    &
    \begin{flushright}
        \textbf{南京市栖霞区}
    \end{flushright}
\end{tabularx}
\vspace{-3.5mm}
\hrule
\vspace{4mm}
% HEAD END
\centerline{\huge\textbf{活动预告}}
\begin{multicols}{2}
    \section{订阅方式和加入编辑部}  
编辑部招聘人才,用爱发电,工作轻松,详情可联系QQ:1329527951 客服小千\\想订阅本消息或获取PDF版(便于查看超链接和往期),可加QQ群:\href{https://qm.qq.com/q/4HL41Nt3sQ}{466863272}.
\section{Deadline Ongoing}
\setbox\colbbox\vbox{
\makeatletter\col@number\@ne
\begin{longtable}{|c|c|c|}
    \hline
    消息(未见ddl的,不刊) & 截止日期 & 刊载日期\\
    \hline\hline
    南大版deepseek & / & 2.22\\
    天文台开放日 & / & 1.6\\
    悦读课程群 & / & 2.24\\
    乐跑即将开始 & 3.10 & 3.4\\
    原创剧本联合孵化报名 & 3.20 & 1.10\\
    本科生劳育实践 & 7.20 & 2.19\\
    医保零星报销 & 3.31 & 2.19\\
    银星杯论文赛 & 4.22 & 2.27\\
    商院影绘 & 3.16 & 3.2\\
    文院剧本创作研讨会 & 9.30 & 3.2\\
    毓秀摄影 & 3.14 & 3.2\\
    数院手工 & 3.10 & 3.2\\
    物院征集课程指南 & 6.15 & 3.3\\
    中国国际大学生创新大赛 & 3.16 & 3.4\\
    大挑志愿者招募 & 3.15 & 3.5\\
    高教社杯 & 4.25 & 3.5\\
    大创报名 & 3.23 & 3.6\\
    银星杯论文竞赛 & 4.22 & 3.6\\
    南辩院系杯 & 4.12 & 3.6\\
    心协剧本杀 & 3.16 & 3.6\\
    重修缴费 & 3.16 & 3.7\\
    智软征集春日影 & 3.15 & 3.7\\
    大文大理题目征集 & 学期末 & 3.8\\
    马兰花开剧组招募 & 3.15 & 3.8\\
    5月免费上网 & ? & 3.9\\
    书法交流活动 & 3.16 & 3.9\\
    粤协粤语课 & 3.15 & 3.9\\
    《星际穿越》观影会 & 3.15 & 3.9\\
    基础学科论坛 & 4.20 & 3.9\\
    \hline
\end{longtable}
\unskip
\unpenalty
\unpenalty}\unvbox\colbbox
\end{multicols}
\hrule
\pagebreak
\begin{multicols}{2}

\section{讲座}
\begin{tabular}{|>{\centering\arraybackslash}m{.3\textwidth}|m{.06\textwidth}|m{.06\textwidth}|}
    \hline
    讲座 & 开展时间 & 刊载时间\\
    \hline\hline
    春与死:格非《春尽江南》读书会 & 3.22 & 3.4\\\hline
    南京大学MBA科创训练营校友分享会 & 3.15 & 3.7\\\hline
    如何应对国际时尚供应链的挑战与机遇 & 3.14 & 3.7\\\hline
    陶行知对中国教育现代化问题的探索 & 3.24 & 3.7\\\hline
    流体力学中的几个数学问题 & 3.12 & 3.9\\\hline
    中文与联合国 & 3.13 & 3.9\\\hline
    从汉代的猫说起——关于图像、文献、考古材料的运用 & 3.10 & 3.9\\\hline
    性别与权力:性别研究的视角与方法论 & 3.11 & 3.9\\\hline
    人工智能中的数据优化策略 & 3.12 & 3.9\\\hline
    南新读书会 & 3.12 & 3.9\\\hline
    华为AI实习生招聘交流会 & 3.13 & 3.9\\\hline
\end{tabular}
1. “中文与联合国”主题讲座
讲座主题:中文与联合国\\
讲座时间:2025年3月13日19:30(周四)\\
讲座地点:南京大学仙林校区就业中心体验馆\\
报名链接:见推文\\
欢迎校内外同学来听讲座~
名额有限,额满表单自动停止。\\
推文:\url{https://mp.weixin.qq.com/s/uDeeYDk1um-8eid9Exebng}\\

2.从汉代的猫说起——关于图像、文献、考古材料的运用

主讲人:徐志君(南京大学艺术学院副研究员)

时间:3月10日 14:00-16:00

地点:南京大学浦口校区艺术楼B109会议室

内容摘要:马王堆汉墓漆器上的狸猫形象,是西汉前期的重要图像材料。将之置于整个汉代的图像材料语境中,我们可以发问:汉代的猫对于人的生活意味着什么?图像、文献、考古资料提供的信息有可以相互印证之处,也有相互矛盾的地方。如何看待这些不同类材料的关系?运用不同类材料来探究学术问题,谨慎处理不同类材料之间的信息差异,是我们在研究中需要学习的能力。\\

3.性别与权力:性别研究的视角与方法论

主讲人:黄淑贞(宾夕法尼亚州联邦大学传播系长聘副教授、美国亚利桑那州立大学传播学博士)

时间:3月11日 10:00起

地点:南京大学仙林校区紫金楼119教室

内容摘要:生理性别与社会性别的区别是什么?作为分析视角,它们的局限性在哪里?如何理解性别作为动词的含义?本次讲座将探讨性别这一概念在社会研究中的复杂性。我们将通过介绍与讨论立场理论(立场理论)、交叉性理论(交叉性)和性别操演理论(表演性)等,分析当前美国主流的性别研究视角,并探讨这些视角如何影响知识生产及对社会议题的观察和理解。\\

4.人工智能中的数据优化策略

主讲人:申富饶(南京大学人工智能学院教授、博士生导师)

时间:3月12日 19:30-21:00

参加方式:扫描下方海报二维码参与线上直播

内容摘要:在深度学习时代,数据、模型、算力是推动人工智能发展的三架马车,数据不仅是模型的燃料,更是决定AI性能的关键因素。报告将围绕数据在深度学习中的核心作用展开,探讨数据增强及数据压缩的价值及其挑战,包括如何在不依赖训练效果的情况下评估数据增强方法、如何平衡数据一致性与多样性,以及数据增强策略的优化与可解释性等关键科研问题。同时,将介绍剪枝、量化、知识蒸馏等数据压缩核心技术,探讨在计算资源受限的环境下如何实现高效AI。\\

5.华为AI实习生招聘交流会

2025年3月13日(星期四)14:00-16:00 南京大学鼓楼校区物理楼356室

面向对象:物理学院、数学学院全体在读学生(本硕博不限)
\section{南京大学第28届基础学科论坛征稿启动}
投稿阶段:2025年3月10日--2025年4月20日

评审阶段:2025年4月21日--2025年5月15日

结果公示:2025年5月16日--2025年5月20日

颁奖仪式:2025年5月25日

(1)本届论坛继续采用网上投稿模式。投稿网站scholar.nju.edu.cn及操作指南(见附录)。请关注匡亚明学院网站(kym.nju.edu.cn)、微信公众号“南京大学匡亚明学院”以及论坛QQ群(群号:1019951818)的相关信息。

(2)投稿时需填写所有作者信息。同篇论文不可多次投稿、串届投稿。多人合作的论文请以第一作者身份投稿,并在投稿页面填写其他作者信息。每人可以投递不超过2篇稿件,每篇文章请单独在网页上提交。

 (3)论文格式参见《中文论文格式说明及模版》或《英文论文格式说明及模版》。模板下载链接:https://box.nju.edu.cn/d/48d3eb23675c45a08bfc/。提取码:kymxy2025。未能按照模板投稿的论文将无法进入后续评审阶段。论文提交格式仅限word文档(后缀名.doc或.docx)或者pdf文件(后缀名.pdf)。
 
(5)论坛咨询邮箱:donghao@nju.edu.cn,论坛QQ群的群号1019951818。

详见:\url{https://mp.weixin.qq.com/s/5FwHYJJFvbzWIxX5bFtvew}
\section{做问卷调查,获免费上网}
参与 itsc 举办的小蓝鲸智能助手使用调研(\url{https://wenjuan.nju.edu.cn/vm/ryuKLKT.aspx}),可获 5 月份免费上网。详见\url{https://mp.weixin.qq.com/s/jHVnhO5o78elXdGM04HBlQ}

\section{“笔下风华,墨中雅趣”书法交流活动报名}
时间:2025.3.16周日下午\\
地点:鼓楼校区\\
活动嘉宾:南京大学艺术学院陈伦乓老师\\
陈伦乓老师将在活动现场亲自示范书法创作,从笔法、墨法到构图、意境,全方位、细致入微地为我们讲解书法创作的要点与技巧。\\
仅限10个名额,先到先得。\\
报名方式见:\url{https://mp.weixin.qq.com/s/2Bnhp2nbFiqmzPwBF8fsyA}

\section{粤语课 | 再品《花样年华》}
时间:3月15日14:00\\
地点:鼓楼新教304\\
活动详情见:\url{https://mp.weixin.qq.com/s/oCU3RXE3LNoVgHF-t-vpqQ}

\section{第四届【全国高校戏曲社团青音会】}
本次活动为全国高校戏曲类社团活动\\
仅限全国高校戏曲类社团报名参加\\
2025年3月22日晚上18:00截止报名\\
报名视频提交邮箱:944382848@qq.com\\
报名社团须提供表演视频以供筛选\\
2025年3月24日至3月25日展演彩排\\
2025年3月27日正式线上展演\\
活动详情见:\url{https://mp.weixin.qq.com/s/b3q0W23SGWgMKf1gYEoTLw}

\section{南大科幻奇幻协会x南大天协|《星际穿越》观影\&讨论会}
时间:2025.3.15 19:00-22:30\\
地点:费A-410\\
观影请入群:825075391


\section{3.10-3.12学术文化活动概览}
周一(3.10)\\
从汉代的猫说起——关于图像、文献、考古材料的运用\\
周二(3.11)\\
性别与权力:性别研究的视角与方法论\\
周三(3.12)\\
1.流体力学中的几个数学问题\\
2.人工智能中的数据优化策略\\
\url{https://mp.weixin.qq.com/s/rczVX60wazfO1XeG5E9NFA}


\section{南新读书会预告}
本周的南新读书会将于3月12日(周三)19:00在新闻传播学院311室举行,陈志聪助理研究员将分享刘海龙《宣传:观念、话语及其正当化》,文湘龙博士研究生将分享亚历山大·R.加洛韦《不可计算:漫长数字时代的游戏与政治》,欢迎全体师生参加。


\end{multicols} 

\end{document}