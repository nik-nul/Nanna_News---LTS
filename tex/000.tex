% HEAD BEGIN
\documentclass[letterpaper, 12pt]{article}
\newsavebox\colbbox
\usepackage{graphicx}
\usepackage{multicol}
\usepackage{anysize}
\usepackage{fontspec}
\usepackage[fontset=none]{ctex}
\usepackage{tabularx}
\usepackage{longtable}
\PassOptionsToPackage{hyphens}{url}
\usepackage[breaklinks=true, colorlinks=true]{hyperref}
\expandafter\def\expandafter\UrlBreaks\expandafter{\UrlBreaks\do\a\do\b\do\c\do\d\do\e\do\f\do\g\do\h\do\i\do\j\do\k\do\l\do\m\do\n\do\o\do\p\do\q\do\r\do\s\do\t\do\u\do\v\do\w\do\x\do\y\do\z\do\A\do\B\do\C\do\D\do\E\do\F\do\G\do\H\do\I\do\J\do\K\do\L\do\M\do\N\do\O\do\P\do\Q\do\R\do\S\do\T\do\U\do\V\do\W\do\X\do\Y\do\Z}
% \let\oldurl\url
% \renewcommand{\url}[1]{\begin{sloppypar}\oldurl{#1}\end{sloppypar}}
\setlength\columnsep{30pt}
\marginsize{30pt}{30pt}{10pt}{20pt}
\setmainfont{TeX Gyre Bonum}
\setCJKmainfont[BoldFont=Noto Serif CJK SC Bold, ItalicFont=FandolKai]{Source Han Sans SC}
\setlength{\parindent}{0cm}
% \setCJKmonofont{Noto Sans CJK SC}
\begin{document}
\begin{center}
    \Huge\textbf{南哪大专醒前消息}
\end{center}
\vspace{4mm}
\hrule
\renewcommand\tabularxcolumn[1]{m{#1}}
\begin{tabularx}{\textwidth}{>{\hsize.2\hsize}X>{\hsize.6\hsize}X>{\hsize.2\hsize}X}
    \begin{flushleft}
        2025.5.3\, No.235
    \end{flushleft}
    &
    \begin{center}
        \textit{“秉中持正、求新博闻。”}
    \end{center}
    &
    \begin{flushright}
        \textbf{南京市栖霞区}
    \end{flushright}
\end{tabularx}
\vspace{-3.5mm}
\hrule
\vspace{4mm}
% HEAD END
\centerline{\huge\textbf{活动预告}}
\begin{multicols}{2}
\section{订阅方式和加入编辑部}  
编辑部招聘人才,用爱发电,工作轻松,详情可联系QQ:1329527951 客服小千\\想订阅本消息或获取PDF版(便于查看超链接和往期),可加QQ群:\href{https://qm.qq.com/q/4HL41Nt3sQ}{466863272}.
\section{活动清单}
\setbox\colbbox\vbox{
\makeatletter\col@number\@ne
\begin{longtable}{|>{\centering\arraybackslash}m{.3\textwidth}|m{.06\textwidth}|m{.06\textwidth}|}
    \hline
    活动 & 开展时间 & 刊载时间\\
    \hline\hline
    南大版deepseek & / & 2.22\\
    悦读课程群 & / & 2.24\\
    eScience AI科研助手 & / & 3.11\\
    地科博物馆开放安排 & / & 3.22\\ 
    2025年分流和转专业政策通知 & / & 4.7\\
    2025年转专业志愿填报通知 & / & 4.24\\
    乐跑 & 5.16 & 3.10\\
    本科生劳育实践 & 7.20 & 2.19\\
    大文大理题目征集 & 期末 & 3.8\\
    5月免费上网 & ? & 3.9\\
    外教社杯 & 5.27 & 3.12\\
    浦口音乐跑 & 5.30 & 3.31\\
    仙林校区志愿法律咨询 & / & 4.4\\
    青春活力大赛 & 5.17 & 4.7\\
    在校生自愿体检 & 6.20 & 4.8\\
    南大购买WPS & / & 4.8\\
    中美中心2025年证书项目 & 5.24 & 4.14\\
    粤歌赛决赛 & 5.10 & 4.21\\
    汉字文化技能大赛 & 5.4 & 4.21\\ 
    校博岩画展 & 6.22 & 4.23\\
    CASHL“畅读”活动 & 5.23 & 4.24\\
    江苏高校凤凰读书节 & 6.15 & 4.24\\
    图书馆征集春日影 & 5.10 & 4.28\\
    汉字知识竞赛 & 5.4 & 4.28\\
    在校生体检 & 5.7 & 4.29\\
    无偿献血 & 5.9 & 4.29\\
    新生创意大赛 & 5.5 & 4.29\\
    数智应用大赛 & 5.10 & 5.3\\
    港澳台生征文 & 6.20 & 5.3\\
    南心支教 & 5.8 & 5.3\\
    非遗体验课 & 5.8 & 5.3\\
    \hline
\end{longtable}
\unskip
\unpenalty
\unpenalty}\unvbox\colbbox
\end{multicols}
\begin{multicols}{2}
\pagebreak

\section{讲座}
\begin{tabular}{|>{\centering\arraybackslash}m{.3\textwidth}|m{.06\textwidth}|m{.06\textwidth}|}
    \hline
    讲座 & 开展时间 & 刊载时间\\
    \hline\hline
    从语言到智能 ⸺ 大语言模型的奥秘与应用 & 5.6 & 4.16\\\hline
    儿童脑智发育与人口神经科学 & 5.14 & 4.30\\\hline
    Social Simulation with Large Language Model-based Agents & 5.8 & 4.29\\\hline
    Dark Side of AI Use: Firm-Level Evidence on Corporate Performance & 5.8 & 4.30\\\hline
    基于专利大数据的全球创新集群识别与全球创新网络分析 & 5.8 & 5.1\\\hline
    中国发明专利对全球专利引用研究 & 5.8 & 4.30\\\hline
    技术与产品关联视角下的区域产品创新 & 5.8 & 4.30\\\hline
    Short-selling Profitability, Stock Lending Fees, and Asset Pricing Anomalies & 5.8 & 4.30\\\hline
    BIM创新主题演化及技术成熟度分析 & 5.7 & 4.30\\\hline
\end{tabular}
\section{南京大学首届数智应用大赛} %  describer: instruconrevo
一、组织单位
\\主办单位:南京大学党委学生工作部
\\承办单位:南京大学信息管理学院
\\二、参赛对象
\\南京大学全体在校本科生、研究生。
\\(每支队伍人数为3 - 5人,设1名队长。团队成员可跨院系、跨专业)
\\三、比赛时间线
\\报名和提交作品PPT:即日起至2025年5月10日,投稿作品经专家评审后,择优进入现场决赛。
\\现场决赛:2025年5月中旬
\\四、比赛流程
\\(一)团队抢答赛
\\(二)团队汇报展示
\\五、奖项设置
\\一等奖1名:奖金3000元
\\二等奖2名:奖金2000元
\\三等奖3名:奖金1000元
\\六、报名方式
\\请各参赛团队在2025年5月10日24:00前将作品PPT和报名表(点击文末阅读原文获取)打包,以“队长学号 + 队长姓名 + 决赛报名”命名,提交至链接:https://box.nju.edu.cn/u/d/7ad72a7d8b8c49709d99/
\\同时,队长须加入“数智倍乘”冠军赛队长群,以及时获取决赛相关通知。
\\详见:\url{https://mp.weixin.qq.com/s/_F4rRnriSbqCxza9NEsS0g}

\section{ 每周实习速递(九)} %  describer: blob
1.平安证券
\\2.浙商证券研究所
\\3.毅达资本
\\4.江苏共创人造草坪股份有限公司
\\
\\
\\详见:\url{https://mp.weixin.qq.com/s/AXPzuzwkAcQmyCDK3wFHmg}

\section{“青春筑梦想 科技创辉煌” 港澳台学生主题征文活动} % 校级活动 describer: Hikari
01 征文要求
\\1 主题 
\\以“青春筑梦想 科技创辉煌”为主题,鼓励港澳台青年学子以文字形式记录面对新时代、新使命、新征程的所见、所闻、所想。
\\2 对象  
\\在内地(大陆)高校就读的港澳台本科生、研究生及进修生、交换生,以及有内地(大陆)高校学习经历的港澳台校友。
\\3 要求(节选)
\\3)作品题目自拟,体裁不限,字数在1000 至5000 字之间(诗歌不限字数)。
\\4)标题下注明作者所在学校、学院、专业、年级、生源地及作者姓名等信息。
\\5)格式要求:标题二号字,方正小标宋简体;正文内容三号字,仿宋 GB2312;一级标题三号字,黑体;二级标题三号字,楷体,加粗;三级标题三号字,仿宋 GB2312,加粗;行距28 磅。
\\02 投稿方式
\\征文截稿时间为2025年6月20日
\\提交材料包括:
\\1. 征文电子版(word格式)
\\2. “青春筑梦想 科技创辉煌”港澳台学⽣主题征⽂活动统计表(附件1)
\\3. “青春筑梦想 科技创辉煌”港澳台学⽣主题征⽂活动授权书(附件2)
\\请参与征文活动的同学在截稿日期前,将所有材料提交至邮箱twhkmacao@nju.edu.cn
\\参赛作品邮件名和打包文件请备注:学院+学生姓名+生源地+作品名称
\\详见:\url{https://mp.weixin.qq.com/s/4wWRVeszcn3LEmjMpS9y9w}

\section{“星座杯”排球娱乐赛} % 校级活动 describer: Ando
活动时间:5月17日-5月18日
\\活动地点:方肇周副馆
\\活动内容:根据报名人数,比赛为单or双循环娱乐排球赛,马拉松形式;每场比赛2小时不间断(暂定);前十分钟为赛前热身(尽量让每位同学都有充分的上场机会)。
\\分组规则\&特殊技能:本次以星象分队,场上不少于两个女生,每队都将被赋予“特定技能”,具体见推文。
\\本次活动为小型娱乐比赛,欢迎所有本科生、研究生、教职工、毕业校友参赛,谢绝外来人员组队。本次活动每人收取10元报名费,主要用于奖品购买和裁判工资。所有参赛队员比赛结束均可获得精美礼品一份\textasciitilde{}冠亚军有奖杯和奖牌。报名截止日期为5月11日晚24时。
\\详见:\url{https://mp.weixin.qq.com/s/WzZeI58resyiSgXWmSkqBg}

\section{2025年南心支教团实践志愿招募} % 校级活动 describer: Ando
2025年南心支教团志愿者招募将通过初筛和面试,选拔出8-10位优秀的志愿者,包括实地支教志愿者和线上工作志愿者。分工包括队长、课程组、宣传组、财务组、评估组等。
\\具体执行暂定7月15-29日,工作包括课程实施(每天上午9:00-11:00、下午15:00-17:00进行教学活动,并根据学生的实际需求提供课后辅导与作业辅导),社会实践,志愿者管理,整理反思等。
\\报名方式:点击文末“阅读原文”填写报名表,加入志愿者咨询群等待后续通知。报名截止时间为5月8日20:00。
\\详见:\url{https://mp.weixin.qq.com/s/UQ7EeykN1qBEtHeVzNRWSg}

\section{非遗面塑体验课} % 校级活动 describer: Ando
面塑,俗称面花、礼馍、花糕、捏面人,以糯米面为骨,以植物色彩为魂,经匠人“揉、搓、捏、掀、切、刻”等技法雕琢人物、生肖、花卉等。
\\时间:2025年5月8日(周四) 14:00-16:00
\\地点:南京大学仙林校区仙Ⅰ-318
\\面向对象:南京大学全体学生
\\活动内容:1. 面塑文化了解:邀请非遗传承人讲解面塑渊源、工艺及寓意,通过实物、影像和展板展示,设问答环节解疑。
\\2. 面塑实践创作:学基础手法塑造造型,融入地域文化雕琢细节,用色素或亮片装饰,老师现场指导并记录创作。
\\3. 作品分享交流:展示面塑作品,分享创作思路,总结技艺特点,共探非遗传承意义。
\\报名方式:扫描下方二维码填写问卷报名,名额有限,快来报名~
\\详见:\url{https://mp.weixin.qq.com/s/89M61-j9Jt0oPNDd8mPyaA}

\section{体重管理年 | 青年领航“法特莱克跑”} % 校级活动 describer: Jolly
活动流程:即日起-5月8日,线上报名+Keep教程学习;5月8日17:00仙林校区炜华体育场,线下活动
\\活动亮点:急救英雄刘其鑫担任形象大使等
\\报名:扫码加群报名,限30-40人,向全校同学开放报名,报名即送伴手礼
\\详见:\url{https://mp.weixin.qq.com/s/t9bio2Byxd1sDBiZ7-WN4w}
\section{院级活动}
\begin{tabular}{|>{\centering\arraybackslash}m{.3\textwidth}|m{.06\textwidth}|m{.06\textwidth}|}
\hline
    活动 & 开展时间 & 刊载时间\\
    \hline\hline
    文院剧本创作研讨会 & 9.30 & 3.2\\
    物院征集课程指南 & 6.15 & 3.3\\
    地海征集春日影 & 6.15 & 3.14\\
    法院党建征文 & 5.20 & 4.2\\
    四院音乐节 & 5.11 & 4.7\\
    商院征集 & 5.5 & 4.8\\
    物院运动打卡 & 5.14 & 4.12\\
    电院征集 & 5.11 & 4.22\\
    智院摄影 & 5.6 & 4.22\\
    商院征集 & 5.9 & 4.27\\
    物院研讨会 & 5.11 & 4.30\\
    健雄摄影 & 5.20 & 4.30\\
    工管学实杯 & 5.8 & 4.30\\
    开甲许愿 & 5.10 & 4.30\\
    社院征集 & 5.5 & 5.3\\
    数院文创 & 5.31 & 5.3\\
    电院求职 & 5.15 & 5.3\\
    \hline
\end{tabular}
\subsection{社会学院五一特辑线上征集} % 院级活动 describer: instruconrevo
来稿内容经整理后将发布于“南大社苑”公众号。参与投稿的同学可获得文创礼品一份,请大家在投稿时留下自己的联系方式,方便后续的奖品发放。
\\投稿须知:扫描下方二维码,选择对应的作品类型并进行上传。署名可填写笔名。为奖品发放顺利,请务必备注【姓名+学号+联系方式】。
\\截止时间:2025年5月5日24:00。
\\五一假期后,奖品的发放和领取将在线下开展。
\\详见:\url{https://mp.weixin.qq.com/s/hc8spR9EEdm_GW7NEufbxQ}

\subsection{数院特色文创设计大赛} % 院级活动 describer: instruconrevo
投稿内容:设计图+设计说明各一份。要求如下:
\\设计图:请使用PDF,图片等常见文件格式,直观展示设计构想。手绘、数字设计等方式均可。
\\设计说明:Word文档形式,重点解释此设计中融合的数院特色与设计思路,200字内。
\\注:请将两个文件打包为zip文件夹,并以“学号+姓名”命名。
\\截止日期:2025年5月31日
\\优秀奖将获得全套学院定制文创、南大记忆南雍志帆布包,并可任选 50 元内书籍作为奖励。
\\提交方式见原文
\\详见:\url{https://mp.weixin.qq.com/s/BD55SWejXW0Bvx1waXDjOg}

\subsection{电院求职经验分享会综合专场} % 院级活动 describer: instruconrevo
电子学院研究生会将于5月15日(周四)上午9:30举办“职涯点津”求职经验分享会综合专场。此次专场特邀在研究所、选调生、运营商、银行、电网等热门就业方向成功斩获多个优质offer的优秀学长学姐
\\报名链接:https://table.nju.edu.cn/dtable/forms/2aee2b4c-39d2-45f9-aab3-02bba2c5aeff/
\\报名截止时间:5月14日中午12:00
\\报名成功的同学扫描二维码加群,入群请备注姓名-专业,活动具体举办地点等细节后续视报名人数群内通知
\\详见:\url{https://mp.weixin.qq.com/s/gjbzcDTOL9Y1OIRbtUoPPA}

\section{社团活动}
\begin{tabular}{|>{\centering\arraybackslash}m{.3\textwidth}|m{.06\textwidth}|m{.06\textwidth}|}
    \hline
    社团活动 & 开展时间 & 刊载时间\\
    \hline\hline
    天文台开放日 & / & 1.6\\
    红会一块走 & 5.20 & 4.21\\
    集庆折子戏 & 5.7 & 4.22\\
    九歌大会 & 5.11 & 4.27\\
    红会图书角 & 5.7 & 4.29\\
    街舞社路演 & 5.9 & 4.29\\
    排协鼓楼杯报名 & 5.5 & 4.29\\
    紫藤钱币研学 & 5.11 & 4.30\\
    “ai荔枝”辩论会 & 5.7 & 5.3\\
    \hline
\end{tabular}
%这里是写社团活动的,社团活动就是由社团主办、主要针对社团内部人员的活动。不要把非社团活动写在这里。
\subsection{“ai荔枝”辩论会} % 社团活动 describer: nik_nul
时间:5.7(周三)晚19:30
\\地点:仙林校区 张心瑜剧场
\\辩题:在大学生活中发现自己不合群,需要/不需要改变
\\对阵:南京大学(正方)vs东南大学(反方)
\\详见:\url{https://mp.weixin.qq.com/s/6tnPQN0Y-848naCB97rQYA}

\subsection{《在菜场,在人间》共读分享会} % 院级活动 describer: Ando
时间:2025年5月11日(周日) 下午14:00
\\主讲人:陈   慧
\\与谈人:周   琪
\\地点:可一书店·仙林艺术中心(负二层 可一实验剧场)
\\梅园经典共读小组将在可一书店举办第八十三期沙龙,邀请作者陈慧和南京大学文学博士周琪一起,共读她的代表作《在菜场,在人间》。这同时也是小组上半年“当代文学里的工与农”系列专场的第三场活动,期待同各位师友线下相会,从菜市场故事中看到最鲜活的社会缩影,发现平凡生活的诗意,探讨生活的韧性。
\\
\\详见:\url{https://mp.weixin.qq.com/s/932ZbOsvTh9R4iyZiGdAPg}
\end{multicols}
\end{document}
