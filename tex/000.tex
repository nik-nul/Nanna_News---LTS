% HEAD BEGIN
\documentclass[letterpaper, 12pt]{article}
\newsavebox\colbbox
\usepackage{graphicx}
\usepackage{multicol}
\usepackage{anysize}
\usepackage{fontspec}
\usepackage[fontset=none]{ctex}
\usepackage{tabularx}
\usepackage{longtable}
\PassOptionsToPackage{hyphens}{url}
\usepackage[breaklinks=true, colorlinks=true]{hyperref}
\expandafter\def\expandafter\UrlBreaks\expandafter{\UrlBreaks\do\a\do\b\do\c\do\d\do\e\do\f\do\g\do\h\do\i\do\j\do\k\do\l\do\m\do\n\do\o\do\p\do\q\do\r\do\s\do\t\do\u\do\v\do\w\do\x\do\y\do\z\do\A\do\B\do\C\do\D\do\E\do\F\do\G\do\H\do\I\do\J\do\K\do\L\do\M\do\N\do\O\do\P\do\Q\do\R\do\S\do\T\do\U\do\V\do\W\do\X\do\Y\do\Z}
% \let\oldurl\url
% \renewcommand{\url}[1]{\begin{sloppypar}\oldurl{#1}\end{sloppypar}}
\setlength\columnsep{30pt}
\marginsize{30pt}{30pt}{10pt}{20pt}
\setmainfont{TeX Gyre Bonum}
\setCJKmainfont[BoldFont=Noto Serif CJK SC Bold, ItalicFont=FandolKai]{Source Han Sans SC}
\setlength{\parindent}{0cm}
% \setCJKmonofont{Noto Sans CJK SC}
\begin{document}
\begin{center}
    \Huge\textbf{南哪大专醒前消息}
\end{center}
\vspace{4mm}
\hrule
\renewcommand\tabularxcolumn[1]{m{#1}}
\begin{tabularx}{\textwidth}{>{\hsize.2\hsize}X>{\hsize.6\hsize}X>{\hsize.2\hsize}X}
    \begin{flushleft}
        2025.3.11\, No.188
    \end{flushleft}
    &
    \begin{center}
        \textit{“秉中持正、求新博闻。”}
    \end{center}
    &
    \begin{flushright}
        \textbf{南京市栖霞区}
    \end{flushright}
\end{tabularx}
\vspace{-3.5mm}
\hrule
\vspace{4mm}
% HEAD END
\centerline{\huge\textbf{活动预告}}
\begin{multicols}{2}

\section{活动清单}
\begin{tabular}{|>{\centering\arraybackslash}m{.3\textwidth}|m{.06\textwidth}|m{.06\textwidth}|}
    \hline
    活动 & 开展时间 & 刊载时间\\
    \hline\hline
    南大版deepseek & / & 2.22\\
    天文台开放日 & / & 1.6\\
    悦读课程群 & / & 2.24\\
    eScience AI科研助手 & / & 3.11\\
    乐跑 & 5.16 & 3.10\\
    原创剧本联合孵化报名 & 3.20 & 1.10\\
    本科生劳育实践 & 7.20 & 2.19\\
    医保零星报销 & 3.31 & 2.19\\
    银星杯论文赛 & 4.22 & 2.27\\
    中国国际大学生创新大赛 & 3.16 & 3.4\\
    大挑志愿者招募 & 3.15 & 3.5\\
    高教社杯 & 4.25 & 3.5\\
    大创报名 & 3.23 & 3.6\\
    银星杯论文竞赛 & 4.22 & 3.6\\
    南辩院系杯 & 4.12 & 3.6\\
    心协剧本杀 & 3.16 & 3.6\\
    重修缴费 & 3.16 & 3.7\\
    大文大理题目征集 & 期末 & 3.8\\
    马兰花开剧组招募 & 3.15 & 3.8\\
    5月免费上网 & ? & 3.9\\
    书法交流活动 & 3.16 & 3.9\\
    基础学科论坛 & 4.20 & 3.9\\
    仙林法律咨询招新 & 3.12 & 3.10\\
    食堂春菜 & 3.13 & 3.10\\
    香雪海游园会 & 3.16 & 3.11\\
    四六级 & 3.18 & 3.11\\
    \hline
\end{tabular}


\section{讲座}
\begin{tabular}{|>{\centering\arraybackslash}m{.3\textwidth}|m{.06\textwidth}|m{.06\textwidth}|}
    \hline
    讲座 & 开展时间 & 刊载时间\\
    \hline\hline
    春与死:格非《春尽江南》读书会 & 3.22 & 3.4\\\hline
    南京大学MBA科创训练营校友分享会 & 3.15 & 3.7\\\hline
    如何应对国际时尚供应链的挑战与机遇 & 3.14 & 3.7\\\hline
    陶行知对中国教育现代化问题的探索 & 3.24 & 3.7\\\hline
    流体力学中的几个数学问题 & 3.12 & 3.9\\\hline
    中文与联合国 & 3.13 & 3.9\\\hline
    性别与权力:性别研究的视角与方法论 & 3.11 & 3.9\\\hline
    人工智能中的数据优化策略 & 3.12 & 3.9\\\hline
    南新读书会 & 3.12 & 3.9\\\hline
    华为AI实习生招聘交流会 & 3.13 & 3.9\\\hline
    交换代数中的同调方法—Koszul复形 & 3.11 & 3.10\\\hline
    找回组织:雇佣决策的一个社会学分析 & 3.13 & 3.10\\\hline
    两会精神与经济形势分析 & 3.12 & 3.10\\\hline
    严格解在非厄米系统中的应用 & 3.13 & 3.10\\\hline
    系统公正的跨文化差异:松紧文化的视角 & 3.18 & 3.11\\\hline
    心理健康急救培训及其在中国的文化适应性研究 & 3.19 & 3.11\\\hline
\end{tabular}
1.系统公正的跨文化差异:松紧文化的视角
讲座时间:2025年3月18日(星期二)13:00-14:30\\
讲座地点:南京大学河仁楼(社会学院)合美堂401室\\
主讲人:李文岐,南京大学\\
与谈人:肖承丽 南京大学社会学院教授、博士生导师、副院长\\

2.吕淑荣:心理健康急救培训及其在中国的文化适应性研究
主持人:李鸣 南京大学社会学院教授\\
时间:2025年3月19日(周三)下午14:00\\
地点:社会学院401会议室\\

4.加州大学伯克利分校宣讲会\\
讲座时间:3月12日 12:30-13:30\\
参会方式:Zoom\\
Meeting ID: 993 3228 2158\\
Passcode: 189528\\
\section{订阅方式和加入编辑部}  
编辑部招聘人才,用爱发电,工作轻松,详情可联系QQ:1329527951 客服小千\\想订阅本消息或获取PDF版(便于查看超链接和往期),可加QQ群:\href{https://qm.qq.com/q/4HL41Nt3sQ}{466863272}.
\section{2025年上半年四六级考试相关通知汇总}
本科生院报名信息核对通知:\\
\url{https://jw.nju.edu.cn/68/2f/c26263a747567/page.htm}\\
全国大学英语四、六级考试口试(以下简称口试)考试时间为5月24—25日,24日开考英语口语四级,25日开考英语口语六级。

全国大学英语四、六级考试笔试(以下简称笔试)考试时间为6月14日,开考科目为英语、日语、德语、俄语的四级、六级和法语四级。

我省报名时间为3月18日12:00至3月24日17:00。各考点学校将在此时间段内自行选择报名系统开放时间及开考科目,请考生注意所在学校报名通知,并在规定时间登录CET全国网上报名系统(cet-bm.neea.edu.cn)完成资格审核、笔试及口试报名缴费。如口试和笔试兼报,应先报笔试再报口试。


\section{仙林校区香雪海春风雅集}
时间:3月16日13:00-17:00\\
古风志愿者招募:展示服章之美、收获丰厚的志愿时长回馈\\
活动详情与招募信息见:\url{https://mp.weixin.qq.com/s/saeK84nrk-SZIbWViGBF5A}



\section{2025秋香港大学交流项目}
交流名额:10人\\
交流时间:2025年秋季一学期\\
享受待遇:免香港大学学费,其余费用自理\\
申请时间及材料\\
一、校内遴选阶段:欲申请学生通过本科生院交换生系统报名,报名截止为3月16日;\\
二、校内遴选之后,由南京大学台港澳事务办公室组织入选同学准备相关材料,并向对方高校提名;\\
三、提名通过后,南京大学台港澳事务办公室将组织候选同学填报其他赴港申请材料。\\
英语要求\\
托福网考成绩达93分或以上;或雅思总分达6.5或以上。\\
注意:\\
法律学科英语要求为托福网考成绩达97分或以上;或雅思总分达7.0或以上,且单科不低于6.5。\\
奖学金评选\\
名称:冯氏奖学金\\
简介:由经纶慈善基金捐助,用以支持香港大学开展学生交流活动,包括代付学生住宿费用及相关生活津贴。经由我校本科生院遴选后,报送香港大学确认。\\
名额:3人\\
资助额度:16000元/人\\



\section{2025年春季学期第4至第6周新生学院学业辅导}
本学期的辅导答疑QQ群和涵盖科目均有增加,具体分组如下:
1、数学组QQ群,答疑科目有:微积分II(第一层次)、微积分II与线性代数(第二层次)、线性代数(第一层次)、微积分II(匡院第一层次)、高等代数、数学分析;\\
2、物理组QQ群,答疑科目有:大学物理、普通物理、热学、光学;\\
3、化学组QQ群,答疑科目有:大学化学、化学导论、有机化学;\\
4、程序设计组QQ群,答疑科目有:python程序设计、C程序设计(层次I)、程序设计基础、离散数学等。\\
详情见\url{https://mp.weixin.qq.com/s/tZSSLyrp5WUgywWznFiFuA}\\
\section{eScience中心定制AI科研助手}
eScience中心现上线了RAG增强AI平台,课题组可以使用已有工具链快速在eScience智算集群上搭建属于自己的AI助手。详见\url{https://mp.weixin.qq.com/s/x7Biqqxqu61hpA-WqAggeg}
\section{院级活动}
\begin{tabular}{|>{\centering\arraybackslash}m{.3\textwidth}|m{.06\textwidth}|m{.06\textwidth}|}
    \hline
    院级活动 & 开展时间 & 刊载时间\\
    \hline\hline
    商院影绘 & 3.16 & 3.2\\
    文院剧本创作研讨会 & 9.30 & 3.2\\
    毓秀摄影 & 3.14 & 3.2\\
    数院手工 & 3.10 & 3.2\\
    智软征集春日影 & 3.15 & 3.7\\
    物院征集课程指南 & 6.15 & 3.3\\
    计院实习分享会 & 3.14 & 3.10\\
    信地海环四院羽球赛 & 3.23 & 3.10\\
    电子学院学长指导 & 3.13 & 3.10\\
    电子学院淘宝简历面试指导 & 3.21 & 3.10\\
    电子学院腾讯简历面试指导 & 3.24 & 3.10\\
    外院互联网+预赛 & 3.16 & 3.11\\
    物院朋辈分享会 & 3.13 &  \\
    外院就业分享会 & 3.14 & 3.11\\
    \hline
\end{tabular}
\subsection{外国语学院2025年中国国际大学生创新大赛(“互联网+”大赛)项目预征集}
参赛人员:我校全日制在校生(包括本科生、研究生),或毕业5年以内的全日制学生(2020年之后的毕业生),且年龄不超过35岁(1990年3月1日之后出生)。\\
参赛领域:鼓励各种类型项目参赛。\\
作品评选:将评选出一等奖、二等奖、三等奖各若干个。按照《南京大学推免学分绩加分办法》,给予获得省级一等奖以上团队的同学相应的推免学分绩加分奖励。\\
材料报送:请有意向报名的项目于2025年3月16日之前填写大赛项目征集表(格式见附件)发送到邮箱wurong@nju.edu.cn。\\
原文:\url{https://mp.weixin.qq.com/s/7V_IH30Ix7_xTanPwxpcuQ}

\subsection{“语通寰宇 职创未来”就业创业经验分享会}\\
活动时间:3月14日 15:00 - 17:00\\
活动地点:南京大学仙林校区 仙I 206\\
报名方式:扫描二维码填写问卷即可报名\\
报名后请加入微信群并实名\\
详情见\url{https://mp.weixin.qq.com/s/oLJlxO7BdnJp6EaR_kllDw}\\
\section{社团活动}
\begin{tabular}{|>{\centering\arraybackslash}m{.3\textwidth}|m{.06\textwidth}|m{.06\textwidth}|}
    \hline
    社团活动 & 开展时间 & 刊载时间\\
    \hline\hline
    粤协粤语课 & 3.15 & 3.9\\
    《星际穿越》观影会 & 3.15 & 3.9\\
    NJUMUNC观察员招募 & 3.18 & 3.11\\
    \hline
\end{tabular}
\subsection{Passion街舞社 | Back to School Vol.2}
时间:2025.3.15星期六(13:30签到,14:30开始)\\
地点:东城汇CC公社\\
赛制:Freestyle 1on1\\
参赛人员:南京在校大学生\\
报名方式:扫码进群\\
见原文:\url{https://mp.weixin.qq.com/s/753wdsT5PG4i5Gnxevoq1A}\\
\subsection{NJUNMUNC2025观察员招募}
为使更多的同学有了解模联、走近模联的机会,大会决定开放20-25个观察员名额,观察员招募政策如下:\\
1.观察员招录采取滚动录取的方式,报名与缴费截止时间为3月18日22:00,录满即止,逾期不候;\\
2.申请者须为高中及以上学历的高等院校全日制在读学生、全日制高中在读学生;\\
3.请领队(南京大学校内代表无需填写领队信息)督促每位欲报名观察员的代表团成员在规定时间内填写好报名信息,每份报名表提交后报名信息将以回执的形式发送至代表团领队邮箱,请领队注意查收;\\
4.若需要退会,请通过邮件njumunmsc@163.com的形式联系大会组委会,观察员退会会费不予返还;\\
5.高中生报名观察员时需一并上传经指导单位签字盖章的《高中生代表团指导单位知情同意书》和《监护人知情同意书》(点击文末“阅读原文”获取)。\\
观察员费用:240元\\
详情见:\url{https://mp.weixin.qq.com/s/c_yZ4xQAFK7lQzPROzAWqw}
\end{multicols}
\end{document}

