% HEAD BEGIN
\documentclass[letterpaper, 12pt]{article}
\newsavebox\colbbox
\usepackage{graphicx}
\usepackage{multicol}
\usepackage{anysize}
\usepackage{fontspec}
\usepackage[fontset=none]{ctex}
\usepackage{tabularx}
\usepackage{longtable}
\PassOptionsToPackage{hyphens}{url}
\usepackage[breaklinks=true, colorlinks=true]{hyperref}
\expandafter\def\expandafter\UrlBreaks\expandafter{\UrlBreaks\do\a\do\b\do\c\do\d\do\e\do\f\do\g\do\h\do\i\do\j\do\k\do\l\do\m\do\n\do\o\do\p\do\q\do\r\do\s\do\t\do\u\do\v\do\w\do\x\do\y\do\z\do\A\do\B\do\C\do\D\do\E\do\F\do\G\do\H\do\I\do\J\do\K\do\L\do\M\do\N\do\O\do\P\do\Q\do\R\do\S\do\T\do\U\do\V\do\W\do\X\do\Y\do\Z}
% \let\oldurl\url
% \renewcommand{\url}[1]{\begin{sloppypar}\oldurl{#1}\end{sloppypar}}
\setlength\columnsep{30pt}
\marginsize{30pt}{30pt}{10pt}{20pt}
\setmainfont{TeX Gyre Bonum}
\setCJKmainfont[BoldFont=Noto Serif CJK SC Bold, ItalicFont=FandolKai]{Source Han Sans SC}
\setlength{\parindent}{0cm}
% \setCJKmonofont{Noto Sans CJK SC}
\begin{document}
\begin{center}
    \Huge\textbf{南哪大专醒前消息}
\end{center}
\vspace{4mm}
\hrule
\renewcommand\tabularxcolumn[1]{m{#1}}
\begin{tabularx}{\textwidth}{>{\hsize.2\hsize}X>{\hsize.6\hsize}X>{\hsize.2\hsize}X}
    \begin{flushleft}
        2025.3.15\, No.191
    \end{flushleft}
    &
    \begin{center}
        \textit{“秉中持正、求新博闻。”}
    \end{center}
    &
    \begin{flushright}
        \textbf{南京市栖霞区}
    \end{flushright}
\end{tabularx}
\vspace{-3.5mm}
\hrule
\vspace{4mm}
% HEAD END
\centerline{\huge\textbf{活动预告}}
\begin{multicols}{2}
\section{订阅方式和加入编辑部}  
编辑部招聘人才,用爱发电,工作轻松,详情可联系QQ:1329527951 客服小千\\想订阅本消息或获取PDF版(便于查看超链接和往期),可加QQ群:\href{https://qm.qq.com/q/4HL41Nt3sQ}{466863272}.
\section{活动清单}
\setbox\colbbox\vbox{
\makeatletter\col@number\@ne
\begin{longtable}{|>{\centering\arraybackslash}m{.3\textwidth}|m{.06\textwidth}|m{.06\textwidth}|}
    \hline
    活动 & 开展时间 & 刊载时间\\
    \hline\hline
    南大版deepseek & / & 2.22\\
    天文台开放日 & / & 1.6\\
    悦读课程群 & / & 2.24\\
    eScience AI科研助手 & / & 3.11\\
    乐跑 & 5.16 & 3.10\\
    原创剧本联合孵化报名 & 3.20 & 1.10\\
    本科生劳育实践 & 7.20 & 2.19\\
    医保零星报销 & 3.31 & 2.19\\
    银星杯论文赛 & 4.22 & 2.27\\
    中国国际大学生创新大赛 & 3.16 & 3.4\\
    高教社杯 & 4.25 & 3.5\\
    大创报名 & 3.23 & 3.6\\
    银星杯论文竞赛 & 4.22 & 3.6\\
    南辩院系杯 & 4.12 & 3.6\\
    心协剧本杀 & 3.16 & 3.6\\
    重修缴费 & 3.16 & 3.7\\
    大文大理题目征集 & 期末 & 3.8\\
    5月免费上网 & ? & 3.9\\
    书法交流活动 & 3.16 & 3.9\\
    基础学科论坛 & 4.20 & 3.9\\
    香雪海游园会 & 3.16 & 3.11\\
    四六级 & 3.18 & 3.11\\
    普通话测试 & 3.28 & 3.12\\
    外教社杯 & 5.27 & 3.12\\
    春季毕业双选会 & 3.16 & 3.14\\
    心理中心全媒体招新 & 3.25 & 3.14\\
    Π节活动 & 3.16 & 3.14\\
    教超优惠 & 3.17 & 3.15\\
    研会免费证件照拍摄 & 3.21 & 3.15\\
    权服侠征集表情包 & 3.16 & 3.15\\
    双南义诊 & 3.16 & 3.15\\
    \hline
\end{longtable}
\unskip
\unpenalty
\unpenalty}\unvbox\colbbox
\end{multicols}
\begin{multicols}{2}
\pagebreak

\section{讲座}
\begin{tabular}{|>{\centering\arraybackslash}m{.3\textwidth}|m{.06\textwidth}|m{.06\textwidth}|}
    \hline
    讲座 & 开展时间 & 刊载时间\\
    \hline\hline
    多圈层作用驱动的油气形成与富集理论 & 3.18 & 3.14\\\hline
    春与死:格非《春尽江南》读书会 & 3.22 & 3.4\\\hline
    陶行知对中国教育现代化问题的探索 & 3.24 & 3.7\\\hline
    系统公正的跨文化差异:松紧文化的视角 & 3.18 & 3.11\\\hline
    心理健康急救培训及其在中国的文化适应性研究 & 3.19 & 3.11\\\hline
    AI时代的创业、就业思考 & 3.16 & 3.15\\\hline
    文明交流互鉴视域下的中国传统文化及其特性 & 3.18 & 3.15\\\hline
    生成式人工智能的变革影响:策略、应用与创新 & 3.18 & 3.15\\\hline
    系统公正的跨文化差异:松紧文化的视角 & 3.18 & 3.15\\\hline
    从人工智能到大模型及应用思考 & 3.18 & 3.15\\\hline
    Sums of squares and Hopf invariant one & 3.19 & 3.15\\\hline
    在国际知名社会政策期刊发表论文的经验 & 3.19 & 3.15\\\hline
\end{tabular}
%讲座预告写在这
1.创变未来:AI时代的创业、就业思考\\
主讲人:袁岳(零点有数董事长、中国民营经济研究会副会长)、邬健敏(简鸣资本创始合伙人、天使投资人、南京大学创业导师)\\
时间:3月16日(周日)14:00-17:00\\
地点:鼓楼校区中美中心C101室\\
详见\url{https://mp.weixin.qq.com/s/4UiIyw-cBRfMfQO2-vYcHw} 

2.文明交流互鉴视域下的中国传统文化及其特性\\
主讲人:张亮(南京大学哲学学院院长、教授、博士生导师)\\
时间:3月18日(周二)14:00\\
地点:南京大学鼓楼校区海外教育学院101报告厅
详见\url{https://mp.weixin.qq.com/s/4UiIyw-cBRfMfQO2-vYcHw}

3.生成式人工智能的变革影响:策略、应用与创新\\
主讲人:professor ACM/IEEE fellow Zhu Han\\
时间:3月18日(周二)10:00\\
地点:腾讯会议516-780-704\\

4.系统公正的跨文化差异:松紧文化的视角\\
主讲人:李文岐(南京大学社会学院心理学系助理教授,北京师范大学心理学博士)\\
时间:3月18日(周二)13:00-14:30\\
地点:仙林校区河仁楼(社会学院)合美堂401室\\

5.DeepSeek:从人工智能到大模型及应用思考\\
主讲人:王崇骏(南京大学计算机学院教授,南京大学中德HPI研究院联合创始人,南京大学计算机应用研究所副所长)\\
时间:3月18日(周二)18:30\\
地点:鼓楼校区中美文化研究中心A106会议室\\

5.Sums of squares and Hopf invariant one\\
主讲人:林兆波(东南大学)\\
时间:3月19日(周三)16:00-17:30\\
地点:鼓楼校区戊己庚四楼北;腾讯会议399-1313-1750\\

6.心理健康急救培训及其在中国的文化适应性研究\\
主讲人:吕淑荣(中国国家疾病预防控制中心流行病学与生物统计硕士,墨尔本大学全球卫生学院哲学博士,墨尔本大学精神卫生中心高级研究员)\\
时间:3月19日(周三)14:00\\
地点:仙林校区河仁楼(社会学院)401会议室\\

7.在国际知名社会政策期刊发表论文的经验——兼论英国社会政策教育的发展现状\\
主讲人:周翠雯(英国诺丁汉大学社会学与社会政策学院政策学科带头人)\\
时间:3月19日(周三)19:00-21:00\\
地点:仙林校区河仁楼(社会学院)401室\\

%活动预告写在这。只对某院学生开放的活动和由社团举办、主要针对社团内部成员的活动不要写在这里。
\section{关于开展 “一二课堂融通”课程学分认定工作的通知}
本次学分认定针对2021、2022级本科生开展。\\
申请学分认定的本科生应于2025年5月30日前按照《“一二课堂融通”课程学分认定评价标准》在教务系统中的“学分认定”模块进行学分申请。本次学分认定针对2021、2022级本科生开展。\\
链接:\url{https://jw.nju.edu.cn/6c/06/c26263a748550/page.htm}\\
\section{“国际消费者权益日”线上线下系列活动}
活动时间:3月17日(周一)11:30-14:30\\
活动地点:仙林校区一组团教超、