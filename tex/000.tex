% HEAD BEGIN
\documentclass[letterpaper, 12pt]{article}
\newsavebox\colbbox
\usepackage{graphicx}
\usepackage{multicol}
\usepackage{anysize}
\usepackage{fontspec}
\usepackage[fontset=none]{ctex}
\usepackage{tabularx}
\usepackage{longtable}
\PassOptionsToPackage{hyphens}{url}
\usepackage[breaklinks=true, colorlinks=true]{hyperref}
\expandafter\def\expandafter\UrlBreaks\expandafter{\UrlBreaks\do\a\do\b\do\c\do\d\do\e\do\f\do\g\do\h\do\i\do\j\do\k\do\l\do\m\do\n\do\o\do\p\do\q\do\r\do\s\do\t\do\u\do\v\do\w\do\x\do\y\do\z\do\A\do\B\do\C\do\D\do\E\do\F\do\G\do\H\do\I\do\J\do\K\do\L\do\M\do\N\do\O\do\P\do\Q\do\R\do\S\do\T\do\U\do\V\do\W\do\X\do\Y\do\Z}
% \let\oldurl\url
% \renewcommand{\url}[1]{\begin{sloppypar}\oldurl{#1}\end{sloppypar}}
\setlength\columnsep{30pt}
\marginsize{30pt}{30pt}{10pt}{20pt}
\setmainfont{TeX Gyre Bonum}
\setCJKmainfont[BoldFont=Noto Serif CJK SC Bold, ItalicFont=FandolKai]{Noto Sans CJK SC}
\setlength{\parindent}{0cm}
% \setCJKmonofont{Noto Sans CJK SC}
\begin{document}
\begin{center}
    \Huge\textbf{南哪大专醒前消息}
\end{center}
\vspace{4mm}
\hrule
\renewcommand\tabularxcolumn[1]{m{#1}}
\begin{tabularx}{\textwidth}{>{\hsize.2\hsize}X>{\hsize.6\hsize}X>{\hsize.2\hsize}X}
    \begin{flushleft}
        2024.11.7\, No.111
    \end{flushleft}
    &
    \begin{center}
        \textit{“道器相济、兼有天下。”}
    \end{center}
    &
    \begin{flushright}
        \textbf{南京市栖霞区}
    \end{flushright}
\end{tabularx}
\vspace{-3.5mm}
\hrule
\vspace{4mm}
% HEAD END
\centerline{\huge\textbf{活动预告}}
\begin{multicols}{2}
    \section{订阅方式和加入编辑部}  
编辑部招聘人才,用爱发电,工作轻松,详情可联系QQ:1329527951 客服小祥\\想订阅本消息或获取PDF版(便于查看超链接和往期),可加QQ群:\href{https://qm.qq.com/q/VXIW7fgsEe}{849644979}.
\section{Deadline Ongoing}
\setbox\colbbox\vbox{
\makeatletter\col@number\@ne
\begin{longtable}{|c|c|c|}
    \hline
    消息(未见ddl的,不刊) & 截止日期 & 刊载日期\\
    \hline\hline
    紫藤学刊征稿 & 12.15 & 10.22\\
    校运会 & 11.8 & 10.21\\
    后革命鲁迅研究征文 & 11.10 & 10.8\\
    大创训练计划申报 & 11.18 & 9.24\\
    招生宣传创意征集大赛 & 11.18 & 10.21\\ 
    EBSCO数据库检索大赛 & 11.20 & 10.3\\
    文院征稿 & 11.20 & 10.20\\
    乐跑 & 12.6 & 10.12\\
    国际访学计划申报 & 11.22 & 10.22\\
    普通话测试网络报名 & 11.12 & 10.29\\
    南大演说家报名 & 11.9 & 10.30\\
    南大会征募会设 & 11.15 & 11.1\\
    心协十一月征稿 & 11.10 & 11.2\\
    秉文心理短视频 & 11.25 & 11.3\\
    扭泵音乐节 & 11.8 & 11.3\\
    法学主题参会 & 11.11 & 11.4\\
    心协香囊活动 & 11.10 & 11.4\\
    BRAVO草地音乐节 & 11.9 & 11.4\\
    高校联合徒步报名 & 11.10 & 11.5\\
    天文台车赛报名 & 11.12 & 11.5\\
    南大模联校内会报名 & 11.11 & 11.5\\
    公务员面试大赛报名 & 11.12 & 11.6\\
    炜华音乐会(地海院) & 11.10 & 11.6\\
    舒伯特钢琴转长 & 11.9 & 11.7\\
    粤歌会 & 11.9 & 11.7\\
    简历大赛 &11.17 & 11.7\\
    艺术疗愈工作坊 & 11.9 & 11.7\\
    非遗拓印活动 & 11.9 & 11.7\\
    \hline
\end{longtable}
\unskip
\unpenalty
\unpenalty}\unvbox\colbbox
\end{multicols}
\hrule
\pagebreak
\begin{multicols}{2}

\section{讲座}
\begin{tabular}{|c|c|c|}
    \hline
    往期讲座 & 开展日期 & 刊载日期\\
    \hline\hline
    《电池及电化学能...》 & 11.24 & 10.3\\
    《专利查新与规避...》 & 12.19 & 10.3\\
    图书馆系列讲座 & 12.3 & 10.20\\
    《卢卡奇1919与19...》 & 11.8 & 11.2\\
    《青年卢卡奇论马...》 & 11.8 & 11.2\\
    《比较文化研究与...》 & 11.8 & 11.3\\
    《卢卡奇遗产中的...》 & 11.10 & 11.4\\
    《史料场与问题域...》 & 11.8 & 11.5\\
    《从微观数据到宏...》& 11.11 & 11.5\\
    《健雄学科认知分享》 & 11.8 & 11.6\\
    《从马克思到当代》 & 11.8 & 11.7\\
    《卢卡奇的学术人生》 & 11.9 & 11.7\\
    《教室性别结构对...》 & 11.14 & 11.7\\
    《Learning from AI》 & 11.13 & 11.7\\
    \hline
\end{tabular}

1.“两个结合”大讲堂 | 仰海峰:从马克思到当代:一种知识地图学的分析\\
主讲人:仰海峰(北京大学马克思主义学院院长,北京大学哲学系教授)\\
时间:2024.11.08(周五)15:00\\
地点:仙II-306\\
简介:在以思想史与社会历史为坐标的知识地图中,一个思想家的思想及其效应可以得到较为合适的定位与描绘,马克思的哲学在当代的发展图景也可以这一知识地图中去探索。\\
讲座详情参见:\url{https://mp.weixin.qq.com/s/71yREKUiLpjFFPUEWRDsDQ}\\

2.展览预告|卢卡奇的学术人生\\
开展时间:2024年11月9日(周六)9:00\\
展期:2024年11月-12月\\
地点:南京大学哲学学院一楼\\
简介:卢卡奇已经像盐溶于水那样,融入并成为当代中国马克思主义学术研究传承的有机组成部分,吸引着一代代中国学人投身对卢卡奇的再研究、再认识。\\
展览详情参见:\url{https://mp.weixin.qq.com/s/gV-ZJ9KFABW_j-hde-icRw}\\

3、毓秀教育经济学青年学者讲座第9期\\
题目:教师性别结构对学生非认知特征的影响\\
主讲人:杨素红,中央财经大学政府管理学院副教授\\
主持人:康乐 特任副研究员\\
时间:2024.11.14(周四)13:30-15:30\\
讲座地点:南京大学仙林校区潘琦楼B座103会议室\\

4、詹姆斯·E·卡茨教授:Learning from AI\\
主题:  Learning from AI:Imagining AI’s Future, Facing AI’s Realities in Education\\
主讲人:詹姆斯·E·卡茨JAMES E. KATZ美国波士顿大学新兴媒体研究菲尔德名誉教授\\
主持人:卢盛舟 南京大学外国语学院副教授高研院第20期驻院学者\\
时:2024年11月13日(周三)16:00-17:30\\
地点: 鼓楼校区逸夫馆9楼高研院报告厅\\
备注:英语演讲\\
具体链接:\url{https://mp.weixin.qq.com/s/dG-xYgVWpo8kOXViMgU9rA}\\

\section{林泉钢琴社|舒伯特钢琴作品专场}
时间:11月9日(周六)19:30\\
地点:仙林校区敬文活动中心320\\
演奏曲目:舒伯特《c小调第19号钢琴奏鸣曲》(D.958),舒伯特《A大调第20号钢琴奏鸣曲》(D.959)\\
\url{https://mp.weixin.qq.com/s/gjeXLJKXyZxz341pqAZOcA}

\section{粤歌会}
时间:11月9日(星期六)18:30\\
地点:鼓楼校区issue咖啡厅旁\\
详见\url{https://mp.weixin.qq.com/s/BXFq4uUwbE81pKOrJP_-Eg}



\section{艺术疗愈实践工作坊}
时间:2024年11月9日(周六)19:00-20:30\\
地点:腾讯会议 745 980 120\\
会议密码:2024\\
主办单位:南京大学本科生院、南京大学艺术学院\\
此次工作坊将基于分享者的艺术疗愈实践经验,旨在探讨如何运用叙事、绘画、拼贴以及人工智能生成内容(AIGC)等多种艺术创作实践,以游戏化的方法进行跨媒介的自我探索、表达与疗愈,整合过去的生命经验,锚定当下的感知,并以图像思维开展对未来图景的描绘,从而处理个人成长的议题,并探索自我与社群之间的关系。
\section{史院简历制作大赛}
简历征集对象:南京大学全体本科及研究生同学\\
赛制:本次大赛将设立两个岗位,同学们可视自身情况投递,只能投递一次。两个岗位将分开评选。岗位A:客户经理,岗位B:公共关系经理\\
参赛方式:即日起至2024年11月17日24:00将参赛作品邮件发送至njulsxyyh@163.com。
作品要求为pdf格式,命名为“姓名+院系年级+岗位A/B+学号”。
作品要求:\\
1. 尺寸规格:A4纸,封面和排版格式不限。\\
2.作品内容:参赛作品的内容应当根据投递岗位针对性设计,重点体现与岗位的匹配度,要求简洁、实用。\\
3. 设计软件:Photoshop、illustrator、Microsoft Word等。\\
4. 注意:要求简历作品以自己为人物,不可写其他人,并且绝对禁止借此机会进行抨击、侮辱他人等行为。\\
详情见\url{https://mp.weixin.qq.com/s/32pTjJyDiAhwy5FhY4X9Vw}
\section{非遗拓印体验活动}
活动时间:2024年11月9日(周六)14:00-16:00

活动地点:历史学院大美楼231

活动对象:历史学院全体在读学生

活动人数:15人

报名方式:见\url{https://mp.weixin.qq.com/s/-HbpL4WsdYPTjXxwKlv5Eg}

\end{multicols} 

\end{document}