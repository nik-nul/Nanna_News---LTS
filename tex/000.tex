% HEAD BEGIN
\documentclass[letterpaper, 12pt]{article}
\newsavebox\colbbox
\usepackage{graphicx}
\usepackage{multicol}
\usepackage{anysize}
\usepackage{fontspec}
\usepackage[fontset=none]{ctex}
\usepackage{tabularx}
\usepackage{longtable}
\PassOptionsToPackage{hyphens}{url}
\usepackage[breaklinks=true, colorlinks=true]{hyperref}
\expandafter\def\expandafter\UrlBreaks\expandafter{\UrlBreaks\do\a\do\b\do\c\do\d\do\e\do\f\do\g\do\h\do\i\do\j\do\k\do\l\do\m\do\n\do\o\do\p\do\q\do\r\do\s\do\t\do\u\do\v\do\w\do\x\do\y\do\z\do\A\do\B\do\C\do\D\do\E\do\F\do\G\do\H\do\I\do\J\do\K\do\L\do\M\do\N\do\O\do\P\do\Q\do\R\do\S\do\T\do\U\do\V\do\W\do\X\do\Y\do\Z}
% \let\oldurl\url
% \renewcommand{\url}[1]{\begin{sloppypar}\oldurl{#1}\end{sloppypar}}
\setlength\columnsep{30pt}
\marginsize{30pt}{30pt}{10pt}{20pt}
\setmainfont{TeX Gyre Bonum}
\setCJKmainfont[BoldFont=Noto Serif CJK SC Bold, ItalicFont=FandolKai]{Source Han Sans SC}
\setlength{\parindent}{0cm}
% \setCJKmonofont{Noto Sans CJK SC}
\begin{document}
\begin{center}
    \Huge\textbf{南哪大专醒前消息}
\end{center}
\vspace{4mm}
\hrule
\renewcommand\tabularxcolumn[1]{m{#1}}
\begin{tabularx}{\textwidth}{>{\hsize.2\hsize}X>{\hsize.6\hsize}X>{\hsize.2\hsize}X}
    \begin{flushleft}
        2025.3.27\, No.203
    \end{flushleft}
    &
    \begin{center}
        \textit{“秉中持正、求新博闻。”}
    \end{center}
    &
    \begin{flushright}
        \textbf{南京市栖霞区}
    \end{flushright}
\end{tabularx}
\vspace{-3.5mm}
\hrule
\vspace{4mm}
% HEAD END
\centerline{\huge\textbf{活动预告}}
\begin{multicols}{2}
\section{订阅方式和加入编辑部}  
编辑部招聘人才,用爱发电,工作轻松,详情可联系QQ:1329527951 客服小千\\想订阅本消息或获取PDF版(便于查看超链接和往期),可加QQ群:\href{https://qm.qq.com/q/4HL41Nt3sQ}{466863272}.
\section{活动清单}
\setbox\colbbox\vbox{
\makeatletter\col@number\@ne
\begin{longtable}{|>{\centering\arraybackslash}m{.3\textwidth}|m{.06\textwidth}|m{.06\textwidth}|}
    \hline
    活动 & 开展时间 & 刊载时间\\
    \hline\hline
    南大版deepseek & / & 2.22\\
    悦读课程群 & / & 2.24\\
    eScience AI科研助手 & / & 3.11\\
    地科博物馆开放安排 & / & 3.22\\ 
    乐跑 & 5.16 & 3.10\\
    本科生劳育实践 & 7.20 & 2.19\\
    医保零星报销 & 3.31 & 2.19\\
    银星杯论文赛 & 4.22 & 2.27\\
    高教社杯 & 4.25 & 3.5\\
    南辩院系杯 & 4.12 & 3.6\\
    大文大理题目征集 & 期末 & 3.8\\
    5月免费上网 & ? & 3.9\\
    基础学科论坛 & 4.20 & 3.9\\
    普通话测试 & 4.11 & 3.25\\
    外教社杯 & 5.27 & 3.12\\
    Python比赛 & 4.6 & 3.16\\
    扎染志愿者招募 & 3.28 & 3.18\\
    本科生院征集大鸣大放 & 4.4 & 3.21\\
    两会知识竞赛 & 3.30 & 3.21\\
    纸鸢工作坊 & 4.3 & 3.22\\
    南大博篆刻体验课 & 4.2 & 3.23\\
    粤歌赛 & 4.12 & 3.24\\
    外词杯 & 3.31 & 3.25\\
    十大海选 & 3.28 & 3.26\\
    3D打印体验 & 3.28 & 3.26\\
    江苏创青春赛事 & 4.30 & 3.26\\
    石膏绘画活动 & 3.29 & 3.26\\
    悦读测试 & 4.6 & 3.27\\
    南大数学竞赛 & 4.15 & 3.27\\
    \hline
\end{longtable}
\unskip
\unpenalty
\unpenalty}\unvbox\colbbox
\end{multicols}
\begin{multicols}{2}
\pagebreak

\section{讲座}
\begin{tabular}{|>{\centering\arraybackslash}m{.3\textwidth}|m{.06\textwidth}|m{.06\textwidth}|}
    \hline
    讲座 & 开展时间 & 刊载时间\\
    \hline\hline
    数字化转型时代的科研创新 & 3.28 & 3.24\\\hline
    国家、资本与社会 & 3.28 & 3.24\\\hline
    关于赴法留学的二三事 & 3.28 & 3.25\\\hline
    Professional Judges or Government Agents? & 3.28 & 3.26\\\hline
    如何影响消费者动机与心理健康支持 & 4.1 & 3.26\\\hline
    Conversations in Finance & 3.28 & 3.26\\\hline
    “图绎文心:清代文艺思想探研”工作坊 & 3.29 & 3.27\\\hline
\end{tabular}
%讲座预告写在这。用subsection

\subsection{活动预告|“图绎文心:清代文艺思想探研”工作坊}
召集人:张昊苏(南开大学)
\\主持人:沙先一(南京师范大学)张宗友(南京大学)
\\与谈人:徐雁平(南京大学)
\\时间:2025年3月29日(周六)14:00开始
\\地点:南京大学仙林校区邵逸夫楼国际学院C308
\\工作坊议程:
\\杨洪升(南开大学)题目:缪荃孙、梁启超、王国维三札阐释与晚清学术演进——兼及缪荃孙易代之际的士人心态
\\马腾飞(嘉兴大学)题目:朱彝尊与清初诗坛联句活动考论
\\张昊苏(南开大学)题目:朱彝尊《烟雨归耕图》阐微
\\谢葆瑭(南京大学)题目:《竹垞小志》的编纂与嘉兴文化景观塑造——嘉庆初年阮元重建曝书亭述论
\\圆桌讨论与自由交流
\\详见:\url{https://mp.weixin.qq.com/s/cyrZ9Gj3TV9r3qy-UcCDpw}


\subsection{南大MBA | 本周校园讲座汇总(3.28-4.3)}
包括:
\\1. Conversations in Finance: A Journey through Thought to Publication
\\2. Professional Judges or Government Agents? A Study of Identity Conflicts of Chinese Patent Examiners
\\3. 从历史和文化视角看波兰人的民族性格
\\4. 国家、资本与社会
\\5. 探索银发经济 解码投资机遇
\\6. 五次全球化与中国的历史角色
\\7. At the Table with Power: Gender Diversity in Finance Committees and Foreign Debt Decisions of Indian Firms
\\8. 数智时代的拟人化设计:如何影响消费者动机与心理健康支持
\\9. 碳排放承诺的跨组织效应——基于国际供应链的证据
\\10. Doing and publishing high-quality research-personal experience and JMS perspective
\\详见:\url{https://mp.weixin.qq.com/s/wdFEcnhWYoNoif58O_DoYQ}

\subsection{学术文化活动概览}
1.周五(3.28)
\\Professional Judges or Government Agents? A Study of Identity Conflicts of Chinese Patent Examiners
\\2.周六(3.29)
\\“图绎文心:清代文艺思想探研”工作坊
\\3.周日(3.30)
\\关于早期生命演化的定量化研究
\\4.周一(3.31)
\\“拉奇娜”在中国:文学作为文化交流与对话的契机
\\5.国际访问学者系列讲座
\\(1) 周五(3.28)
\\Finance Research and Publication
\\(2) 周一(3.31)
\\Does Divergence of Opinions Make Better Minds? Evidence from Social Media
\\详见:\url{https://mp.weixin.qq.com/s/VfM4JbMZSQzulaVcwZSXXQ}


%此处写校级活动,请不要把讲座、院级活动和社团活动写在这里orz orz orz
\section{一分钱“暖心早餐”即将正式回归}
自3月31日起,一分钱“暖心早餐”正式回归!早餐仍面向本校全体全日制学生,每天限量供应550份。三种美味早餐助力早起的学子们开启活力每一天\textasciitilde{}
\\供应时间:2025年3月31日起(寒暑假除外),工作日早晨6:45-7:15(ps:以刷卡时间为准)。
\\供应地点:鼓楼校区(第一、第二食堂),仙林校区(第四、第五、第十、第十一食堂、教二),浦口校区(第十五食堂)。
\\详见:\url{https://mp.weixin.qq.com/s/-t16JiPuGAN37fz1N168WA}

\section{每周实习速递(五)}
1.米哈游
\\2.拼多多
\\3.Shopee
\\4.思必驰
\\
\\详见:\url{https://mp.weixin.qq.com/s/eqj2ds00tR3SlEK8ryhM0w}

\section{本学期悦读测试通知}
本学期悦读测试开放时间为:3月31日-4月6日。测试面向2021级及以前学生开放,请需要测试的同学提前复习(测试书目见附件)。
\\测试形式为在线网络测试,时长60分钟。题库网址为:\url{https://eztest.org/manager/student/6609/enroll/list/}。测试包括客观题和主观题两个部分,主观题为开放式简答题,客观题与该单元所有书目相关。学生报名时选择该单元某一书目,即表示主观题与所选书目相关。每个单元只可选择一本书目答题,仅有一次答题机会,请想要通过参加测试来认定悦读学分的同学谨慎选择。答题完成后,需等相关书目的悦读导师批阅后才能看到成绩(教务系统显示成绩)。60分及格,及格的成绩可以用来认定悦读学分。
\\详见:\url{https://jw.nju.edu.cn/75/d4/c26263a751060/page.psp}


\section{成就卓越的你|第八届“南京大学—郑钢” 未来领袖人才训练营招募!}
“南京大学-郑钢”未来领袖人才训练营,由南京大学杰出校友和名誉校董郑钢先生赞助,南京大学心理健康教育与研究中心与菁英人才培育中心联合举办,旨在培养科学培养领导力、与精英同行,全程参与并考核合格的同学,将获得由南京大学心理中心与菁英人才培育中心联合颁发的结业证书,考核后入选优秀营员的同学将有机会获得“郑钢菁英奖学金”的推荐名额!
\\详见:\url{https://mp.weixin.qq.com/s/rDHdhiJIRQ5j990tNmhrzg}

\section{关于举办第一届南京大学“歆·恺杯”数学竞赛的通知}
参赛对象:南京大学所有全日制在校本科生。竞赛分为数学专业类、非数学专业A类、非数学专业B类。数学类专业(代码为0701)、统计学类专业(代码为0712)的学生只能报考数学专业类,专业代码为07(理科)、08(工科)、02(商科)的学生只能报考数学专业类或非数学专业A类,其他学生报考类别不限。
\\本届竞赛拟定于2025年5月10日(星期六)上午9:00-11:30举办,如有变动另行通知。
\\竞赛地点:竞赛地点视各校区报名情况进行安排,另行通知。
\\竞赛内容:
\\1、非数学专业类(含A类和B类):考试内容为微积分、线性代数(所占总分的比例分别为80\%、20\%左右)。
\\2、数学专业类:考试内容为数学分析、高等代数、解析几何(所占总分的比例分别为50\%、35\%、15\%左右)。
\\报名方法:2025年4月15日前,填写报名表单\url{https://table.nju.edu.cn/dtable/forms/bf98c6e5-a5eb-4ac7-8103-4a71b1079d9d/},并加入竞赛QQ群(群号:1031370835)。
\\详见:\url{https://jw.nju.edu.cn/75/ed/c26263a751085/page.htm}
\section{志愿招募 | 美丽中国第十二期“解忧杂货店”书信交流活动}
美丽中国第十二期“解忧杂货店”项目正面向全国高校大学生招募乡村孩子们的“解忧人”,在一封又一封的书信中,与乡村孩子们进行一次心连心的交流,让孩子们在书信中分享成长中的小故事,得到一份来自远方大哥哥或大姐姐的陪伴,为他们的成长注入更多的信心和力量。
\\活动时间:2025年4月、5月
\\活动形式:每两周一次一对一的书信交流,共4次
\\服务对象:甘肃省平凉市静宁县甘沟镇甘沟中心小学100位四到六年级的小朋友
\\招募对象:南京大学鼓楼校区、仙林校区在校学生
\\招募人数:100人
\\志愿时长:每次书信交流认证2小时志愿时长,本次活动进行4次书信交流,共计8小时
\\活动流程、志愿者要求详见推文。请加入志愿者QQ群(1017621904),后续相关信息将在群里发布。
\\报名方式:在2025年3月29日17:00前填写推文内报名问卷,问卷填写情况将作为志愿者录取的重要依据。
\\详见:\url{https://mp.weixin.qq.com/s/KMBTSEUPPhYGYloMcTq_fA}
\section{院级活动}
\begin{tabular}{|>{\centering\arraybackslash}m{.3\textwidth}|m{.06\textwidth}|m{.06\textwidth}|}
\hline
    活动 & 开展时间 & 刊载时间\\
    \hline\hline
    文院剧本创作研讨会 & 9.30 & 3.2\\
    物院征集课程指南 & 6.15 & 3.3\\
    地海征集春日影 & 6.15 & 3.14\\
    AI院影色舞 & 3.29 & 3.19\\
    商院羽球 & 3.29 & 3.19\\
    物院访企 & 3.28 & 3.22\\
    法院主题餐会 & 3.28 & 3.23\\
    社院学术节 & 4.18 & 3.25\\
    生科栽培 & 3.30 & 3.25\\
    有训行知集体生日会 & 3.30 & 3.26\\
    地学研讨会 & 3.29 & 3.27\\
    电院趣运会 & 3.30 & 3.27\\
    \hline
\end{tabular}

\subsection{第二届“区域国别研究:跨学科跨领域探索的机会之窗”  学生科研创新项目征集公告}
为促进南大学生对跨学科跨领域知识的了解,南京大学区域国别研究院现面向全校各学院不同专业方向的同学,开展“区域国别研究:跨学科跨领域探索的机会之窗”学生科研创新项目征集工作。项目要求、成果形式、申报条件、评奖及资助、时间安排详见“南京大学区域国别研究院”公众号推文。
\\
\\详见:\url{https://mp.weixin.qq.com/s/IISY7sA8fagWSdBRx4sxzg}


\subsection{行歌破晓,知海启航 | 行知书院2024级主题晚会,等你来闪耀!}
晚会初定安排:5月中上旬
\\书院主题晚会主持人火热招募中!我们会根据报名信息进行第一轮初步筛选,通过者将进行第二轮现场选拔,选拔标准与流程届时公布。最终主持人确定人数4-5人。
\\报名方式:https://table.nju.edu.cn/dtable/forms/f4159bfd-4b17-4726-9742-1d2dd58946be/
\\点击链接报名晚会主持人
\\报名截止时间:2025年3月31日(周一)晚22:00
\\节目要求:演唱、舞蹈、乐器、短剧、创意表演等形式不限,鼓励结合中华优秀传统文化设计表演,内容积极向上,文明得体,充分展现行知学子的青春风采。
\\报名方式:https://table.nju.edu.cn/dtable/forms/2b6a6c0f-fb6f-4f83-9d82-4c4209b57971/
\\报名截止时间:2025年4月7日(周一)晚22:00
\\详见:\url{https://mp.weixin.qq.com/s/dnZ9qjOHdZOMDrdk99_ExQ}

\subsection{第十九届地学文化节系列活动之五院拔尖AI研讨会}
活动时间:3月29日(周六)下午15:00-17:00
\\活动地点:大气科学院楼D103
\\参会人员:地学五院同学
\\活动内容:
\\本次研讨会特邀地学五院优秀学生代表进行专题报告,重点探讨人工智能技术在大气科学、地质工程、地理信息、环境监测等领域的创新应用与发展前景。
\\在学术交流环节,与会师生将围绕"AI+地学"的学科交叉主题,结合各自研究方向和实践经验,就技术方法创新、学科融合路径等议题展开深入研讨。
\\活动特别设置自由讨论环节,鼓励参会者分享研究成果,交流学术见解,共同探索人工智能赋能地学研究的未来发展方向。
\\活动嘉宾:大气科学学院 张梓轩、环境学院 王虹杰、地理与海洋科学学院 刘云举、地球科学与工程学院 孔德戬
\\报名和QQ群见链接。
\\
\\详见:\url{https://mp.weixin.qq.com/s/banI0ZPfgWPAjVs5aA3gPA}

\subsection{南京大学电子科学与工程学院飞盘趣味运动会}
活动时间与地点
\\3月30上午九点,四组团足球场
\\活动内容
\\1.飞盘教学
\\2.趣味游戏
\\活动内容详情及报名方式详见原推
\\详见:\url{https://mp.weixin.qq.com/s/ROLBFnSzc2-5GjhL_2IEoA}

\subsection{“计风古韵”投票}
计算学院学生活动中编制的纺织品投票。
\\详见:\url{https://mp.weixin.qq.com/s/7eWRxJwQgeobbNBq0QbVpQ}

\section{社团活动}
\begin{tabular}{|>{\centering\arraybackslash}m{.3\textwidth}|m{.06\textwidth}|m{.06\textwidth}|}
    \hline
    社团活动 & 开展时间 & 刊载时间\\
    \hline\hline
    天文台开放日 & / & 1.6\\
    鸿新社捐书活动 & 3.30 & 3.17\\
    长歌行声演剧 & 3.29 & 3.19\\
    雁行南大x以伴云陪伴线上志愿者招募 & 3.30 & 3.24\\
    AMD Workshop & 3.28 & 3.26\\
    林泉CAC流行音乐会 & 3.30 & 3.26\\
    悲惨世界观影沙龙 & 3.29 & 3.26\\
    孤独症主题活动 & 3.30 & 3.27\\
    
    \hline
\end{tabular}
%这里是写社团活动的,社团活动就是由社团主办、主要针对社团内部人员的活动。不要把非社团活动写在这里。
\subsection{2025 Oliver Wyman Impact咨询案例大赛报名}
2025 Oliver Wyman Impact咨询案例大赛正式启动,报名将于3.30 23时59分截止。报名方式见微信推文
\\详见:\url{https://mp.weixin.qq.com/s/MPSjhMzdTofiSiseOHXU6g}


\subsection{南大篮球明日赛程}
男篮院系杯小组赛
\\材料 vs 匡院 
\\19:00-20:00
\\地点:一组团篮球场
\\详见:\url{https://mp.weixin.qq.com/s/YUfSzNB7VAr_CZHN0YuQ8A}


\subsection{志愿者招募 | 世界孤独症关注日主题活动}
活动主题:织梦星途,静待春芽绽放——第18个“世界孤独症关注日”特别主题活动
\\活动内容:春日游园会,趣味运动会
\\活动时间:2025年3月30日(周日)9:00-17:00
\\活动地点:南京市鼓楼区小市街道党群服务中心
\\工作内容:摄影、活动组织、陪伴孤独症儿童等等
\\报名请加qq群:705790615
\\南京大学“星空下的守望者”志愿团队欢迎你的加入!
\end{multicols}
\end{document}
