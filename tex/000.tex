% HEAD BEGIN
\documentclass[letterpaper, 12pt]{article}
\newsavebox\colbbox
\usepackage{graphicx}
\usepackage{multicol}
\usepackage{anysize}
\usepackage{fontspec}
\usepackage[fontset=none]{ctex}
\usepackage{tabularx}
\usepackage{longtable}
\PassOptionsToPackage{hyphens}{url}
\usepackage[breaklinks=true, colorlinks=true]{hyperref}
\expandafter\def\expandafter\UrlBreaks\expandafter{\UrlBreaks\do\a\do\b\do\c\do\d\do\e\do\f\do\g\do\h\do\i\do\j\do\k\do\l\do\m\do\n\do\o\do\p\do\q\do\r\do\s\do\t\do\u\do\v\do\w\do\x\do\y\do\z\do\A\do\B\do\C\do\D\do\E\do\F\do\G\do\H\do\I\do\J\do\K\do\L\do\M\do\N\do\O\do\P\do\Q\do\R\do\S\do\T\do\U\do\V\do\W\do\X\do\Y\do\Z}
% \let\oldurl\url
% \renewcommand{\url}[1]{\begin{sloppypar}\oldurl{#1}\end{sloppypar}}
\setlength\columnsep{30pt}
\marginsize{30pt}{30pt}{10pt}{20pt}
\setmainfont{TeX Gyre Bonum}
\setCJKmainfont[BoldFont=Noto Serif CJK SC Bold, ItalicFont=FandolKai]{Noto Sans CJK SC}
\setlength{\parindent}{0cm}
% \setCJKmonofont{Noto Sans CJK SC}
\begin{document}
\begin{center}
    \Huge\textbf{南哪大专醒前消息}
\end{center}
\vspace{4mm}
\hrule
\renewcommand\tabularxcolumn[1]{m{#1}}
\begin{tabularx}{\textwidth}{>{\hsize.2\hsize}X>{\hsize.6\hsize}X>{\hsize.2\hsize}X}
    \begin{flushleft}
        2024.1.4\, No.159
    \end{flushleft}
    &
    \begin{center}
        \textit{“秉中持正、求新博闻。”}
    \end{center}
    &
    \begin{flushright}
        \textbf{南京市栖霞区}
    \end{flushright}
\end{tabularx}
\vspace{-3.5mm}
\hrule
\vspace{4mm}
% HEAD END
\centerline{\huge\textbf{活动预告}}
\begin{multicols}{2}
    \section{订阅方式和加入编辑部}  
编辑部招聘人才,用爱发电,工作轻松,详情可联系QQ:1329527951 客服小祥\\想订阅本消息或获取PDF版(便于查看超链接和往期),可加QQ群:\href{https://qm.qq.com/q/VXIW7fgsEe}{849644979}.
\section{Deadline Ongoing}
\setbox\colbbox\vbox{
\makeatletter\col@number\@ne
\begin{longtable}{|c|c|c|}
    \hline
    消息(未见ddl的,不刊) & 截止日期 & 刊载日期\\
    \hline\hline
    安邦征稿 & 1.12 & 11.16\\
    仙林通宵自习室 & 1.12 & 11.26\\
    全国大学生家史大赛 & 1.31 & 12.2\\
    西安史学论坛征稿 & 3.20 & 12.9\\
    本科评教 & 1.12 & 12.13\\
    12306学生优惠票 & 2.12 & 12.13\\
    期末考试安排 & 1.12 & 12.17\\
    南大博物馆展览 & 6.16 & 12.17\\
    排超志愿者招募 & 1.16 & 12.25\\
    南星小红书创作 & 2.6 & 12.27\\
    挂职干部选拔 & 2.13 & 12.31\\
    ASC25报名 & 2.21 & 1.6\\
    食堂免费腊八粥 & 1.7 & 1.6\\
    天文台开放日 & / & 1.6\\
    建院公益调研活动 & 1.9 & 1.6\\
    生科迎春活动 & 1.7 & 1.6\\
    开甲
    \hline
\end{longtable}
\unskip
\unpenalty
\unpenalty}\unvbox\colbbox
\end{multicols}
\hrule
\pagebreak
\begin{multicols}{2}

\section{讲座}
\begin{tabularx}{0.5\textwidth}{|X|X|X|}
    \hline
    讲座 & 开展时间 & 刊载时间\\
    \hline\hline
    《移动计算与社会治理》 & 1.15 & 1.6\\\hline 
    《从发展到后发展的范式转换》 & 1.7 & 1.6\\\hline
    《江苏省青年物理学家论坛第76期》 & 1.7 & 1.6\\\hline
    《Recent Progresses on 2D Charge-Transferonic》 & 1.9 & 1.6\\\hline
    《适用于大规模问题的高阶非凸优化算法和线性求解器》 & 1.7 & 1.6\\\hline
\end{tabularx}

1.江苏省青年物理学家论坛第76期\\
报告人:张瑞 副教授 南京大学\\
报告时间:1月7日(周二)中午12点\\
报告地点:南京大学鼓楼校区唐仲英楼B501\\
含直播\\
\url{https://mp.weixin.qq.com/s/CJTQHGujj32kTJpdDTOK1Q}\\

2.从发展到后发展的范式转换:携手公众社会科学家共创繁荣\\
主讲人:何嫄、刘柳\\
讲座时间:1月7日 12:30-13:50\\
讲座地点:仙林校区国际学院C308\\

3.南大MBA科技与人文系列讲座\\
讲座题目:移动计算与社会治理\\
主讲嘉宾:吕欣 国防科技大学系统工程学院首席专家,教授,博士生导师,“对抗性复杂系统智能决策”创新研究群体负责人\\
讲座介绍:本报告将结合吕欣教授长期以来应用大数据技术在社会治理以及地震、台风、海啸、疫情等重大国内外灾害事件中开展应急救援的多项工作,讲述如何对数以百万计人类个体的行为在时空维度上进行分析和挖掘,进一步探索从个体到整体的群体行为作用机制和演化模式。\\
讲座时间:1月15日(星期三)20:00\\
讲座地点:腾讯会议 线上讲座(会议地址将于报名后讲座开始前发送)\\
报名二维码见原文\url{https://mp.weixin.qq.com/s/vvjkqwopAt12CC_WMpEeaQ}\\

5.物理学院学术报告会(第47期)\\
题 目:Recent Progresses on 2D Charge-Transferonics\\
报告人:韩拯,山西大学\\
时 间:2025年1月9日(周四)15:30\\
地 点:鼓楼校区唐仲英楼B501\\
摘要等见原文\url{https://mp.weixin.qq.com/s/bKIhfy37TlXX_XUS7D1cXw}\\

6.数学学院青年学者论坛\\
题目:适用于大规模问题的高阶非凸优化算法和线性求解器\\
主讲人:刘 洋 Yang Liu , University of Oxford\\
现场报告时间:北京时间2025年1月7日(周二)上午10:30-11:30\\
现场报告地点:西大楼108报告厅\\
腾讯会议:507-659-912\\
摘要等见原文\url{https://mp.weixin.qq.com/s/0KReB7_3kpwdLoOFrkoYSQ}\\


\section{ASC25(世界大学生超算竞赛)报名}
ASC25(世界大学生超算竞赛)初赛已经开启,参赛学生要求为本科生。ASC始于2012,与欧洲ISC和美国SC并称全球三大超算竞赛,初赛提交截止日期是2月21日,有意的同学请与姚舸老师联系。
竞赛内容详见\url{https://www.asc-events.net/StudentChallenge/ASC25/static/ASC25_Preliminary_Round_Announcement.pdf}


\section{腊八粥免费供应}
腊八节到来之际,后勤服务集团将于2025年1月7日(腊八节)在四校区同步供应免费腊八粥,为奋战在教学与科研一线的师生送去温暖祝福。\\\\
供应详情:\\
四校区学生食堂(除仙林校区学生第十二食堂):晚餐时段  免费供应处 \\
鼓楼、仙林校区教工餐厅:中、晚餐时段  免费供应处\\
西苑、南苑会议中心:中、晚餐时段(就餐时赠送)\\
南大国际会议中心:1.全日制餐厅:早餐;2.三楼:中、晚餐时段(就餐时赠送)\\
苏州学术交流中心:中、晚餐时段(就餐时赠送)\\\\
详见\url{https://mp.weixin.qq.com/s/YPkh7XU7fDYpGL0VgRipyQ}



\section{图书馆寒假通知}
1. 考试结束后,请同学们认领图书馆的私人物品。1月17日上午仙林和鼓楼图书馆将对二至五层所有空间的私人物品进行全面清理,寒假期间临时置物架暂停使用。\\
2. 假期还书日期统一延至2月28日\\
3. 跨校区图书委托申请截止时间为1月13号中午12点,本学期最后一次送书时间是1月17日。\\
全文:\url{https://mp.weixin.qq.com/s/08s1YmvpK5htny5EZdxgnQ}

\section{黑匣子|周末剧场《糟糕的熙德》}
THE WASTED WEDDING\\
2024培源艺术节•青年艺术人才培养及作品孵化单元孵化作品\\
时间:01.10 19:30\\
01.11 19:30\\
地点:南京大学 敬文学生活动中心 黑匣子剧场\\
活动详情见推文。\\
\url{https://mp.weixin.qq.com/s/0nO2synLae6zoyX44ec_MQ}

\section{天文台开放日}
时间:长期活动 每周六上午9:00-10:30\\
地点:南京大学仙林校区 左涤江天文台\\
面向人群:南京大学全体师生\\
报名方式及注意事项详见推文。\\
\url{https://mp.weixin.qq.com/s/Ynojk8WZGlHarVu8tSYCNA}

\section{2025寒假全国大学生返乡公益调研活动}
主题:“爱在乡土・情系三农”\\
活动时间:2025年1月13日至2025年2月16日期间,具体时间由参与同学自行安排。\\
调研地点:参与同学在家乡所在地或自行选择具有代表性的乡村地区\\
招募要求:熟悉当地乡村环境,可与村民开展有效沟通交流,家住乡镇或返乡探亲的同学优先\\
成果要求:调研客观全面、数据素材翔实的调研报告1份\\
报名时间:1月9日截止\\
报名方式:推文内扫码\\
活动详情见推文。\\
\url{https://mp.weixin.qq.com/s/96MfbZnvVgtb8K0XbfFsmw}
\section{人民日报|2025新年晚安短信计划}
2025新年晚安短信计划报名将于7日截止\\
报名方式:关注“人民日报”微信公众号,在人民日报微信对话框内输入“晚安”即可报名。\\
详情见推文:\url{https://mp.weixin.qq.com/s/zmHKUYtrGqgpCmXmazZBDw}
\section{生命科学学院|2025年迎春送福活动}
活动时间:1月7日 12:00-14:00\\
活动地点:生科院一楼大厅\\
活动详情见推文。\\
\url{https://mp.weixin.qq.com/s/Vngr3EnAf8Gw-aWbpGg1dA
}
\section{开甲新生早期科研训练项目第一期}
项目周期:1年\\
内容包括:参加立项申请、中期检查和结题答辩\\
对象:2024级学生,其中项目负责人须为开甲书院学生\\
参与方式:“导师命题立项”和“自主选题立项”两种方式完成立项。“导师命题项目”小组人数根据各项目招募情况确定,“自主选题立项”项目每个小组一般不超过3人。\\
1)导师命题项目:\\
共23项,内容及报名码详见\url{https://mp.weixin.qq.com/s/Tk7uxDTQh_1gBOJ32VmphA}\\
2)自主命题项目:\\
自主与新生导师或相关老师进行联系和沟通,确定研究方向。\\
2月17日前提交项目申请。\\
立项申请书提交前请交给指导教师审核并按照要求修改后再提交,书院会进行审核并反馈立项结果。\\
评优及奖励:计入敦行成绩单智育项目。所有立项将视项目质量与完成情况给予相应的资金支持;顺利结项后,根据结项审核结果颁发优秀/合格结项证书。
\section{扫码进群|王者荣耀观赛群}
2024年王者荣耀挑战者杯x高校电竞社观赛活动群\\
入群即可凭购票证明报名乘坐高校专属【观赛\\
接驳大巴】,学校-场馆往返接送,乘车还可参与史诗/传说皮肤券抽奖。\\
欲加群请将此链接转换为二维码后再设法用微信扫描\url{https://weixin.qq.com/g/AwYAAPZJHB9mYG-f-1G9T-RpYnxjEYu5FcGtHpLfik2ckUj6gQq8PqisZPeiAKHV}

\end{multicols} 
\end{document}