% HEAD BEGIN
\documentclass[letterpaper, 12pt]{article}
\newsavebox\colbbox
\usepackage{graphicx}
\usepackage{multicol}
\usepackage{anysize}
\usepackage{fontspec}
\usepackage[fontset=none]{ctex}
\usepackage{tabularx}
\usepackage{longtable}
\PassOptionsToPackage{hyphens}{url}
\usepackage[breaklinks=true, colorlinks=true]{hyperref}
\expandafter\def\expandafter\UrlBreaks\expandafter{\UrlBreaks\do\a\do\b\do\c\do\d\do\e\do\f\do\g\do\h\do\i\do\j\do\k\do\l\do\m\do\n\do\o\do\p\do\q\do\r\do\s\do\t\do\u\do\v\do\w\do\x\do\y\do\z\do\A\do\B\do\C\do\D\do\E\do\F\do\G\do\H\do\I\do\J\do\K\do\L\do\M\do\N\do\O\do\P\do\Q\do\R\do\S\do\T\do\U\do\V\do\W\do\X\do\Y\do\Z}
% \let\oldurl\url
% \renewcommand{\url}[1]{\begin{sloppypar}\oldurl{#1}\end{sloppypar}}
\setlength\columnsep{30pt}
\marginsize{30pt}{30pt}{10pt}{20pt}
\setmainfont{TeX Gyre Bonum}
\setCJKmainfont[BoldFont=Noto Serif CJK SC Bold, ItalicFont=FandolKai]{Noto Sans CJK SC}
\setlength{\parindent}{0cm}
% \setCJKmonofont{Noto Sans CJK SC}
\begin{document}
\begin{center}
    \Huge\textbf{南哪大专醒前消息}
\end{center}
\vspace{4mm}
\hrule
\renewcommand\tabularxcolumn[1]{m{#1}}
\begin{tabularx}{\textwidth}{>{\hsize.2\hsize}X>{\hsize.6\hsize}X>{\hsize.2\hsize}X}
    \begin{flushleft}
        2024.10.31\, No.104
    \end{flushleft}
    &
    \begin{center}
        \textit{“克明峻德。”}
    \end{center}
    &
    \begin{flushright}
        \textbf{南京市栖霞区}
    \end{flushright}
\end{tabularx}
\vspace{-3.5mm}
\hrule
\vspace{4mm}
% HEAD END
\centerline{\huge\textbf{活动预告}}
\begin{multicols}{2}
    \section{订阅方式和加入编辑部}  
编辑部招聘人才,用爱发电,工作轻松,详情可联系QQ:1329527951 客服小祥\\想订阅本消息或获取PDF版(便于查看超链接和往期),可加QQ群:\href{https://qm.qq.com/q/VXIW7fgsEe}{849644979}.
\section{Deadline Ongoing}
\setbox\colbbox\vbox{
\makeatletter\col@number\@ne
\begin{longtable}{|c|c|c|}
    \hline
    消息(未见ddl的,不刊) & 截止日期 & 刊载日期\\
    \hline\hline
    紫藤学刊征稿 & 12.15 & 10.22\\
    黑匣招募 & 11.1 & 10.19\\
    学位英语考试报名 & 11.3 & 10.17\\
    校运会 & 11.8 & 10.21\\
    后革命鲁迅研究征文 & 11.10 & 10.8\\
    大创训练计划申报 & 11.18 & 9.24\\
    招生宣传创意征集大赛 & 11.18 & 10.21\\ 
    EBSCO数据库检索大赛 & 11.20 & 10.3\\
    文院征稿 & 11.20 & 10.20\\
    乐跑 & 12.6 & 10.12\\
    国际访学计划申报 & 11.22 & 10.22\\
    百团大战 & 11.2 & 10.26\\
    仙林草地音乐节 & 11.3 & 10.27\\
    南大网球排位赛报名 & 11.1 & 10.28\\
    普通话测试网络报名 & 11.12 & 10.29\\
    全球学习交流展 & 11.4 & 10.29\\
    健雄捡秋活动 & 11.3 & 10.29\\
    南大演说家 & 11.9 & 10.30\\
    导游志愿者招募 & 11.3 & 10.30\\
    南大演说家报名 & 11.9 & 10.30\\
    对话苏童 & 11.2 & 10.30\\
    腾讯线下观影 & 11.3 & 10.30\\
    杜厦剧本杀 & 11.3 & 10.30\\
    秉文朋导分享会 & 11.3 & 10.31\\
    《美丽人生》观影会 & 11.1 & 10.31\\
    科创集市 & 11.2 & 10.31\\
    《乱世佳人》放映会 & 11.2 & 10.31\\
    物院飞盘活动 & 11.3 & 10.31\\
    心协乐跑 & 11.3 & 10.31\\
    \hline
\end{longtable}
\unskip
\unpenalty
\unpenalty}\unvbox\colbbox
\end{multicols}
\hrule
\pagebreak
\begin{multicols}{2}

\section{讲座}
\begin{tabular}{|c|c|c|}
    \hline
    往期讲座 & 开展日期 & 刊载日期\\
    \hline\hline
    《电池及电化学能...》 & 11.24 & 10.3\\
    《专利查新与规避...》 & 12.19 & 10.3\\
    图书馆系列讲座 & 12.3 & 10.20\\
    《志工人力资源的...》 & 11.4 & 10.23\\
    《华人社会工作的...》 & 11.4 & 10.23\\
    《从全球视角探讨...》 & 11.4 & 10.28\\
    《瑞典电力和氢能...》 & 11.7 & 10.29\\
    《志愿服务大讲堂...》 & 11.1 & 10.30\\
    《组织动员如何影...》 & 11.6 & 10.30\\
    《展览,游戏,还...》 & 11.1 & 10.30\\
    《信息与现代信息...》 & 11.6 & 10.31\\
    《Python绘图与建...》 & 11.1 & 10.31\\
    \hline
\end{tabular}

1.毓琇讲坛27期:从“三星堆文明”到通用人工智能AGI ——漫话信息和现代信息技术\\
主讲人:祝世宁 院士\\
时间:2024年11月06日 16:00\\
地点:南京大学鼓楼校区大礼堂\\
报名二维码见\url{https://mp.weixin.qq.com/s/OVZ4j9NgCsWuf0OIcAnfKA}

2.开甲惟学沙龙丨绘图与建模--Python 工具包应用\\
主讲人:李尚敖(2022级计算机学院本科生,曾获2022-2023学年南京大学优秀学生。现为江苏省人工智能学会南京大学支会会员、南京大学人工智能协会社长)\\
时间:2024年11月1日(周五)18:30\\
地点:南京大学鼓楼校区南园综合楼509\\
报名二维码\url{https://mp.weixin.qq.com/s/-06NHL8fthV8vg7cKAnhIw}
\section{《美丽人生》观影会预告}
时间:2024年11月1日 18:30-21:00

地点:南京大学鼓楼校区新教501

嘉宾:陆远,历史学学士、硕士,南京大学-日本名古屋大学联合培养社会学博士。

报名详见:\url{https://mp.weixin.qq.com/s/rWCbGdbPMMviqBc7yKjqTg}

QQ群:965609785

主办:南京大学学生会

承办:南大社院学生会、南大射雕社

本期观影会,由南京大学学生会主办,南京大学社会学院学生会、南京大学社会调查协会承办,以观看和评析电影《美丽人生》的形式,邀请老师和同学们共同欣赏影片,在温馨的氛围中赏人间百态、悟人生哲理,提升文艺审美水平、丰富课余生活、培养健全人格。
\section{2024科创集市}
时间:11月2日14:00\\
地点:南京大学鼓楼校区苏浙体育场\\
具体链接:\url{https://mp.weixin.qq.com/s/ZHK6B1H70TQUswv9nn9DOA}\\
今年,校团委学创部将以百团大战为契机,携手南京大学科创类社团,特别设置“科创集市”展台,推出“科创秘宝大挑战”活动,诚邀各位小南鲸认识科创、走近科创、爱上科创。在这里,你可以接触最新的科创动态、感受浓厚的科创氛围;在这里,你可以收获宝贵的“挑战杯”竞赛信息,也可以对明年南大承办“挑战杯”竞赛提出你的创意想法。



\section{篮协明日赛程}
日期:11月1日\\
地点:仙林校区一组团篮球场\\
男篮院系杯小组赛:\\
匡院vs政管 18:30-20:00\\
女篮院系杯小组赛:\\
外院vs软院 20:30-22:00\\
研究生男篮:\\
化学vs计科 16:00-17:00\\
详见\url{https://mp.weixin.qq.com/s/kp380UAePcjnYWlb7G3XUQ}

\section{百团大战倒计时两天(内含转发抽奖)}
南大社团发布\\
包含本次百团大战的活动规则、舞台演出和转发抽奖等讯息\\
详情请查看\url{https://mp.weixin.qq.com/s/n2JhHElZ35gbCtFzHPaPzA}\\
\section{秉文书院:朋辈导师分享会}
活动内容:邀请五位优秀的朋辈导师围绕学习方法、学术科创、团学工作、交流交换内容进行分享\\
活动时间:2024年11月3日下午3:00-5:00\\
活动地点:教学楼101\\
活动嘉宾:董文璐 马克思主义学院23级本科生——学习方法\\
曾曼睿 新闻传播学院22级本科生——学习方法\\
刘明瑞 哲学学院21级本科生——学术科创\\
刘双伊 外国语学院22级本科生——团学工作、交流交换\\
姚星宇 文学院22级本科生——学习方法\\
详见\url{https://mp.weixin.qq.com/s/TtkqRgdVuBGJdEKF_RfQEQ}

\section{心协乐跑活动}
主办单位:南京大学学生心理协会,南京大学心理健康教育与研究中心\\
活动时间:2024年11月3日至12月2日\\
南大全体在校学生均可参与\\
活动内容:总时长为30天,参与者需完成至少21天健康跑打卡和心理知识问答打卡
活动共设置单人跑步、双人跑步、四人跑步三个赛道,分开排名,设置不同奖品。对于大一新生设置“书院最高分”的个人奖。\\
详见:\url{https://mp.weixin.qq.com/s/2cKXb7r3OFmgHjbxJAhWiQ}\\
女生跑步打卡的时间额外增加7天,即女生跑步打卡在12月8号截止

\section{物理学院飞盘活动}
活动时间:2024年11月3日9:00-12:00\\
活动对象:物理学院24级研究生以及各位感兴趣的同学\\
活动地点:鼓楼校区苏浙运动场\\
活动环节:设置“教学时间”以及“萌芽赛道”、“角逐赛道”两个赛道。\\
报名方式:扫描推文中的二维码或者点击报名链接\url{https://table.nju.edu.cn/dtable/forms/b2bc51db-74e7-4ff1-9dea-2f9ff8956687/},填写提交报名表即可。QQ群:965610010。\\
详见\url{https://mp.weixin.qq.com/s/RQUd6ZHi2bcg45IRHTkujA}


\section{排协赛程预告}
新生杯 11月2日:\\
10:00-11:15 秉文书院队—行知书院队\\
11:30-12:45 安邦队—地虎开甲队\\
13:00-14:15	来都来了队—输赢都去开派队\\
14:30-15:45	炎龙开甲队—无论输赢学长都要请吃饭队\\
地点:鼓楼校区三号排球场\\
院系杯:\\
11.1 16:30 男排预赛 材料-数学\\
11.2 11:30 女排预赛 商院-地科\\
11.2 13:00男排预赛 地科-数学\\
11.2 14:30 男排预赛 电子-外院\\
11.1 19:30 女排预赛 商院-大气\\

\section{海岛放映 | 乱世佳人}
放映时间:11月2日(周六) 19:00\\
放映地点:\\
鼓楼校区/东大楼208\\
浦口校区/艺术楼B10
\section{走近名企第九弹}
南大SCDA发布\\
公司:江苏景枫投资控股集团有限公司\\
参访时间:2024年11月5日(周二)下午14:00\\
参访地点:景枫中心(江苏省南京市江宁区双龙大道1698号)\\
名额共三十五人\\
活动流程:\\
统一乘坐大巴前往,起始校区将根据学生报名情况决定。\\
14:00-14:40  领导致辞    14:40-15:30  企业介绍\\
15:40-16:20  场景体验    16:20-17:00  座谈交流\\
17:10  晚餐\\
报名方式:扫码报名。报名详情请查看\url{https://mp.weixin.qq.com/s/OGYpx2H600fwxEkskKeGlQ}\\
报名截止时间:2024年11月4日(周一)中午12:00\\
\end{multicols} 
\hrule
\vspace{4mm}
% APPENDIX BEGIN
\centerline{\huge\textbf{附录}}
\section{2024年秋季学期优质课程公开观摩信息(实时更新)}
具体链接:\url{https://jw.nju.edu.cn/00/04/c26263a720900/page.psp}\\

\end{document}