% HEAD BEGIN
\documentclass[letterpaper, 12pt]{article}
\newsavebox\colbbox
\usepackage{graphicx}
\usepackage{multicol}
\usepackage{anysize}
\usepackage{fontspec}
\usepackage[fontset=none]{ctex}
\usepackage{tabularx}
\usepackage{longtable}
\PassOptionsToPackage{hyphens}{url}
\usepackage[breaklinks=true, colorlinks=true]{hyperref}
\expandafter\def\expandafter\UrlBreaks\expandafter{\UrlBreaks\do\a\do\b\do\c\do\d\do\e\do\f\do\g\do\h\do\i\do\j\do\k\do\l\do\m\do\n\do\o\do\p\do\q\do\r\do\s\do\t\do\u\do\v\do\w\do\x\do\y\do\z\do\A\do\B\do\C\do\D\do\E\do\F\do\G\do\H\do\I\do\J\do\K\do\L\do\M\do\N\do\O\do\P\do\Q\do\R\do\S\do\T\do\U\do\V\do\W\do\X\do\Y\do\Z}
% \let\oldurl\url
% \renewcommand{\url}[1]{\begin{sloppypar}\oldurl{#1}\end{sloppypar}}
\setlength\columnsep{30pt}
\marginsize{30pt}{30pt}{10pt}{20pt}
\setmainfont{TeX Gyre Bonum}
\setCJKmainfont[BoldFont=Noto Serif CJK SC Bold, ItalicFont=FandolKai]{Noto Sans CJK SC}
\setlength{\parindent}{0cm}
% \setCJKmonofont{Noto Sans CJK SC}
\begin{document}
\begin{center}
    \Huge\textbf{南哪大专醒前消息}
\end{center}
\vspace{4mm}
\hrule
\renewcommand\tabularxcolumn[1]{m{#1}}
\begin{tabularx}{\textwidth}{>{\hsize.2\hsize}X>{\hsize.6\hsize}X>{\hsize.2\hsize}X}
    \begin{flushleft}
        2024.10.16\, No.89
    \end{flushleft}
    &
    \begin{center}
        \textit{“克明峻德。”}
    \end{center}
    &
    \begin{flushright}
        \textbf{南京市栖霞区}
    \end{flushright}
\end{tabularx}
\vspace{-3.5mm}
\hrule
\vspace{4mm}
% HEAD END
\centerline{\huge\textbf{活动预告}}
\begin{multicols}{2}
    \section{订阅方式和加入编辑部}  
编辑部招聘人才,用爱发电,工作轻松,详情可联系QQ:1329527951 客服小祥\\想订阅本消息或获取PDF版(便于查看超链接和往期),可加QQ群:\href{https://qm.qq.com/q/VXIW7fgsEe}{849644979}.
\section{Deadline Ongoing}
\setbox\colbbox\vbox{
\makeatletter\col@number\@ne
\begin{longtable}{|c|c|c|}
    \hline
    消息(未见ddl的,不刊) & 截止日期 & 刊载日期\\
    \hline\hline
    仙林校史馆招募讲解员 & 10.30 & 9.12\\
    本科生暑期课程评教 & 10.31 & 9.19\\
    大创训练计划申报 & 11.18 & 9.24\\
    苏州校区音乐会 & 10.19 & 9.25\\
    声谷创新基金 & 10.18 & 9.30\\
    大专戏曲知识竞赛 & 10.20 & 10.2\\
    EBSCO数据库检索大赛 & 11.20 & 10.3\\
    马院主题宣讲报名 & 10.25 & 10.5\\
    后革命鲁迅研究征文 & 11.10 & 10.8\\
    “南大新传”编辑部招新 & 10.20 & 10.10\\
    遵义精神宣讲团遴选 & 10.27 & 10.10\\
    心协十月征稿 & 10.20 & 10.11\\
    乐跑 & 12.8 & 10.12\\
    健雄书院院服设计赛 & 10.20 & 10.12\\
    计院迎新晚会征集节目 & 10.25 & 10.12\\
    毓秀素拓友谊赛 & 10.19 & 10.12\\
    行知院服设计赛 & 10.21 & 10.15\\
    安邦院服设计赛 & 10.20 & 10.15\\
    CTF竞赛宣讲 & 10.19 & 10.13\\
    行知趣味羽球赛 & 10.19 & 10.13\\
    计院趣味定向赛 & 10.20 & 10.13\\
    林泉钢琴社线上分享 & 10.21 & 10.13\\
    林泉音乐会 & 10.19 & 10.13\\
    普通话考试报名 & 10.28 & 10.14\\
    心协心理博客 & 10.20 & 10.14\\
    有训集体生日会 & 10.19 & 10.14\\
    安邦趣味运动会 & 10.19 & 10.14\\
    羽球书院杯报名 & 10.17 & 10.14\\
    青鸟剧场新戏招募 & 10.27 & 10.14\\
    NUBA招募 & 10.18 & 10.14\\
    新传团学联招新 & 10.18 & 10.15\\
    网易雷火宣讲 & 10.17 & 10.15\\
    法学院师生交流会 & 10.19 & 10.15\\
    理科志愿服务讲堂 & 10.18 & 10.15\\
    商院草地音乐会 & 10.20 & 10.15\\
    格庐数学朋导招募 & 10.20 & 10.16\\
    体测 & 10.27 & 10.16\\
    ‘满天星’调研大赛 & 10.22 & 10.16\\
    历史学院支教招募 & 10.17 & 10.16\\
    心协剧本杀招募 & 10.20 & 10.16\\
    街舞社公开课 & 10.19 & 10.16\\
    有训院服设计赛 & 10.22 & 10.16\\
    鸿蒙校园活动 & 10.19 & 10.16\\
    
    \hline
\end{longtable}
\unskip
\unpenalty
\unpenalty}\unvbox\colbbox
\end{multicols}
\hrule
\pagebreak
\begin{multicols}{2}

\section{讲座}
\begin{tabular}{|c|c|c|}
    \hline
    往期讲座 & 开展日期 & 刊载日期\\
    \hline\hline
    《聚合物的研发与...》 & 10.24 & 10.3\\
    《电池及电化学能...》 & 11.24 & 10.3\\
    《专利查新与规避...》 & 12.19 & 10.3\\
    《走进ESG暨案例分...》 & 10.17 & 10.12\\
    《大学生创新训练...》 & 10.18 & 10.15\\
    《走进本科生科研...》 & 10.18 & 10.15\\
    《第二现代性与儒...》& 10.21 &10.16\\
    《创业期税务筹划...》 & 10.20 & 10.16\\
    《生成式AI行业应...》 & 10.20 & 10.16\\
    《大国竞争与世界...》 & 10.18 & 10.16\\
    《中国法律形象西...》 & 10.23 & 10.16\\
    《与<自然>编辑对...》 & 10.30 & 10.16\\
    \hline
\end{tabular}

1.孙本文社会学论坛(第288)期
\\题目:第二现代性与儒学的现代化
\\主讲人:韩相震
\\讲座语言:英语\\讨论交流环节:中英韩皆可,现场配有多语种翻译
\\时间:10月21日(周一)下午15:00-16:30
\\地点:仙林社会学院河仁楼324\\

2.《创业期税务筹划与热点问题解读》\\
授课导师:李杰,美国注册管理会计师、中国税务师、建行大学华东学院讲师\\
时间:2024.10.20(周日)9:00-11:50\\
地点:众创空间422教室\\
内容简介:基于最新税收法规和国内目前实际情况,解读创业期需重点关注的个税、企业所得税、增值税相关税收政策,并在此基础上介绍创业期税务管理、筹划的要点与注意事项。\\

3、《生成式AI行业应用及创新的讲座》\\
主讲人:缪玉峰,微软(亚洲)互联网工程院工业元宇宙应用中心负责人,微软亚太研发集团首席产品总监。\\
时间:2024.10.20(周日)13:20-17:00\\
地点:众创空间422教室\\
内容简介:深入探讨生成式AI的行业应用及其创新实践,旨在为企业和技术爱好者提供前沿的洞察和实践经验\\

4、斯密论坛 第518期\\
主题:大国竞争与世界秩序重构\\
报告人:鞠建东,清华大学五道口金融学院讲席教授\\
主持人:郑江淮,南京大学经济学院产业经济学系\\
时间:2024年10月18日(周五)15:00\\
地点:安中楼312 腾讯会议900-755-556\\
内容简介:围绕“国际(中美)贸易摩擦与全球经济治理体系重构”的主题展开,系统分析国际贸易与金融争端,以及世界秩序的演变。\\

5.中国法律形象西语世界变迁与美国的中国法研究(The Transformation of China's Legal Image in Western World and the Study of Chinese Law in the United States)\\
主讲人:董晓波,南京师范大学教授、博士生导师江苏法学会比较法学研究会副会长\\
主持人:孙雯,中美文化研究中心副主任,南京大学法学院教授、博士生导师\\
时间:2024.10.23(周三)16:30开始\\
地点:中美文化研究中心A106会议室\\
注:本讲座使用中文\\
\url{https://mp.weixin.qq.com/s/1EZLzd2p1Bw2UpbZGMak2A}\\

6.加强您的综述论文:与《自然》编辑对话\\
讲座时间: 10月30日晚上 22:00-23:00\\
         11月5日下午 17:00-18:00\\
讲者:《自然综述: 材料》主编Giulia  Pacchioni和《自然综述:电气工程》主编Olga Bubnova\\
报名方式:扫描二维码提前报名,名额有限先到先得!\\
链接:\url{https://mp.weixin.qq.com/s/52o1uUERKVpR97ZmGnMUxQ}\\
注:两场的讲座语言皆为英语\\
\section{格庐数学辅导活动}
Math Glue数学学业辅导活动现面向全校招募人士,帮助有需求的同学激发学习兴趣、改善学习方法、掌握课程重难点,为学生大学四年的学涯发展赋能。\\
招募岗位:朋辈领学员、运营志愿者\\
截止时间为10月20日\\
岗位具体内容、要求和报名方式详见\url{https://mp.weixin.qq.com/s/ETMqL6ZJD27xd0CZbLEZZQ}

\section{体质健康测试通知}
本学期体质健康测试将于 10.21 至 11.30 举行,全体在籍本科生均需参加测试。在仙林校区的本科生以及在鼓楼校区无体育课的学生需自主预约测试,具体安排较为复杂,请参见体育部通知的 \href{https://box.nju.edu.cn/f/b5f214ead16b46729838/}{pdf 文档} (\url{https://box.nju.edu.cn/f/b5f214ead16b46729838/}).

\section{首届南京大学“满天星”调研报告大赛正式启动}
参与对象:南京大学本科生,不设专业及年级限制,以个人或团队形式参赛均可\\
大赛主题:“学习贯彻党的二十届三中全会精神”\\
比赛要求:参赛者需围绕大赛主题,认真学习领会党的二十届三中全会精神,从会议精神中汲取宝贵思想源泉,开展相关调研课题研究,形成具有一定学术水平与现实意义的研究报告\\
赛事流程:10.15-10.24 大赛动员和报名;11.19 24:00 参赛作品提交截止(更具体时间节点请见推文)\\
报名方式:有意向报名本届赛事的同学,请在2024年10月22日前以个人或组队的方式填写报名信息,其中,团队参赛由队长负责填写报名信息。报名链接:\url{https://table.nju.edu.cn/dtable/forms/04b2c0ad-8d51-48b6-b536-2248b521bc03/},并加入赛事官方交流群(904058615)\\
奖项设置:大赛为作品评选设立特等奖、一等奖、二等奖、三等奖和优秀奖五种奖项,并为获奖团队或个人颁发对应等级的“成功励志满天星实践奖学金”。遴选10篇左右优秀报告汇编入满天星优秀调研报告集。\\
注意事项:此次大赛将作为南京大学商学院项目制课程社会实践方向结项考核免答辩条件之一,并作为第十九届“挑战杯”全国大学生课外学术科技作品竞赛商学院预选赛\\
注:调研报告提交要求、参赛方式、评审方式与标准等详见推文\url{https://mp.weixin.qq.com/s/FSTcAvXia37xtz6WBvZMiA}\\

\section{线上乡村支教项目志愿者招募}
历史学院青协山海支教项目团队现招募志愿者,服务3-12年级的乡村学生,服务时长每周平均90min,不低于60min,共开展10周(至期末周前)。详情请关注南大青协推文\url{https://mp.weixin.qq.com/s/Mx9TY8xvM-CTT_xnU5fzYw}
\section{原创线下心理剧本杀招募}
南京大学心理健康与研究中心主办、迷失蓝鲸倾力打造、校学生心理学会支持的《于蓝鲸中迷失》剧本杀活动正式招募。\\
招募对象:南大全体师生\\
时间:10月20日 上午9:30-12:30(2场),下午14:30-17:30(1场)\\
地点:仙林校区敬文活动中心9层\\
详见:\url{https://mp.weixin.qq.com/s/octflm8dQ9RBRZI6MJ6PUg}
\section{鸿蒙生态学堂·校园行}
地点:华为南京研究所\\
时间:10月19日\\
活动简介:HarmonyOS作为新一代面向万物互联的智能终端操作系统,带来了简洁流畅安全可靠的全场景交互体验。为了帮助校园开发者进一步地探索鸿蒙,鸿蒙生态学堂·校园行以鸿蒙技术为引领,以人才培养为关键,邀请行业专家与同学们零距离交流。\\
活动议程:\\
10:30-11:30 探索鸿蒙生态\\
11:30-12:00 研究所参观\\
13:00-16:30 分会场1 体验官活动华为云空间应用体验\\
13:00-16:30 分会场2 Codelabs实战现场编程,技术闯关\\
主办方:华为技术有限公司\\
报名二维码见图2\\
注意:\\
1、报名结果最终以官方邮件/短信通知为准\\2、报名编程挑战的同学需提前注册华为开发者联盟账号且实名认证通过\\

\section{线下观赛 | 英雄联盟s14全球总决赛八强赛观赛活动}
10月18/19号在和园unique酒吧进行。\\
详情见原文:\url{https://mp.weixin.qq.com/s/FUB0aSCbHLdVn4T52iW1tQ}

\section{Passion街舞社公开课}
舞种:Jazz\\
时间:10月19日(周六)14:00 - 15:30\\
地点:鼓楼校区南青格庐舞蹈房\\
面向全校同学开放,详见\url{https://mp.weixin.qq.com/s/p8FtGCTXMyoV9QG3jDBVJw}

\section{有训书院:院服设计大赛期待你的作品!}
1.征集对象:有训书院2024级全体本科新生\\
2.投稿时间:即日起至2024年10月22日24:00\\
3.作品要求:(1) 体现出有训书院的精神,希望能巧妙使用有训书院的院训、有训红、有训鼎等特色的文化标志。同时,也可加入自己的创意\\
(2)利用画图或PS等软件设计并提交完整的有训书院院服设计图,设计图需至少包含正反两面的视图。允许并鼓励使用国产AI大模型与生成式人工智能应用软件辅助设计\\
3) 设计作品需另附200字左右的作品简介,包括但不限于作品创作灵感与元素、蕴含寓意与使用的设计工具等。作品简介最后须附个人信息,包括姓名、学号、联系电话\\
(4)作品须由个人原创设计,如有侵权嫌疑,将取消参赛评选资格,并由参赛者承担侵权责任\\
4.奖励:一等奖(1名):奖金300元+设计成品院服一件+证书一份(选中)\\
二等奖(1名):奖金200元+设计成品院服一件\\
三等奖(1名):奖金100元+设计成品院服一件优秀奖:设计成品院服一件\\
注:按照要求提交完整作品或作品新颖有创意即可以美育录入五育系统,计入“敦行成绩单”\\
提交要求、链接见\url{https://mp.weixin.qq.com/s/tTt0hXDJu2Gx0IJNIHouOA}

\end{multicols} 

\hrule
\vspace{4mm}
\centerline{\huge\textbf{参考消息}}
\begin{multicols}{2}
\section{美国中学“教育”疯狂的社会达尔文主义实践(二)}
不平均等级制代表人物如恩斯特·海克尔,主张“原初的不平等由于适应而强化,随着时间分化而越来越加强”“阶级社会地位取决于脑容量”“由于遗传禀赋的差别,人的进化并非同等速率,社会不平等系进化水平不同所致,实属自然”“社会主义违反生科原理、大众民主带来人类退化”;总而言之就是贫苦穷人有罪、剥削压迫有理。\\

美国普通中学赤裸裸地推行按分论次的不平均等级制。一些尚掩饰其反动实质的学校暗中推行不平均等级制的思想观念,毫不掩盖其反动实质的学校直接将不平均等级制体现在制度上。有人会说,美国普通中学坚持人人天资相近,只要努力都能成功,和不平均等级制岂不是根本相反的吗?其实,这也是完全欺骗性的说辞。天生各人有不同的做题和考试能力,有不善做题不善考试者、不善做题善考试者、善做题不善考试者、善做题善考试者,更多不能列举。平等是建立在承认不同的做题考试能力上的,对各人人格地位、法律地位平等的承认,现在美国普通中学却用做题考试能力的假平等,掩盖按分论次的真不平等,还要将其上升为人的根本原则。一些美国普通中学在物质上开展等级制,例如高分学生享有更多物质供应优先权和人身自由,低分学生被限制物质供应、被限制人身自由;一些美国中学在精神上开展等级制,例如高分学生享有更多人格尊严和荣誉,低分学生被剥夺人格尊严和被排除任何荣誉。欧洲种族主义学者从不平均等级制推出经济决定论,认为穷人承受种种灾难是因为其进化水平低、智力低下、懒惰寄生,因而有罪;美国普通中学认为低分学生承受种种灾难是因为其做题考试不努力、懒惰寄生,因而有罪。\\

优胜劣汰选择主张以通过遗传选择和特定禀赋为界定的种族为政治哲学的主体,代表人物是法国种族主义学者普拉热,他对十九世纪末二十世纪初流行一时的“颅相学”有重大贡献。普拉热和一批种族主义学者通过研究头颅的构造,指出雅利安人高贵、其他人种低下。还发展出一套竞争性冲突性的自然观社会观,如达尔文主张战争有利于促进进化,又反对开展慈善活动,因为慈善是“抑制优胜劣汰”“对生物进化的亵渎”;斯宾塞主张“放任政策”,即放任穷人挨饿至死政策;索姆纳主张不能用“poor”“weak”等词来形容下等阶层,因为这些词没有反映其寄生性和懒惰性。种种观点简直倒反天罡,总而言之就是侵略屠杀无罪、奴隶制度有理;富人享乐无罪、穷人饿死有理。\\

优胜劣汰选择是美国普通中学最典型的特征。有人会说,难道公开考试、择优录取,不是被历史证明科学合理的选拔制度吗?但是美国普通中学及其全国考试并不是一种旨在公开考试、择优录取的选拔制度,而是一种残酷奴役和镇压学生的机制。考试并不需要什么自然科学和社会科学知识,也不需要什么理性和感性能力,而是要求学生服从并精通考试标准和技巧。在这一过程中,文化和道德修养本来没有什么大差别的学生被打上种种人造的标签并被要求强制进化,不能进化的就淘汰为社会下等阶层,接受宿命决定论和不平均等级制的镇压。这哪是什么公开考试、择优录取,这是一种残酷野蛮的人为优胜劣汰的选择。\\

综上所述,社会达尔文主义的最根本色彩就是反动色彩,自从其产生以后,就逐渐有人批评,至今更是已经扫进历史的垃圾堆,就连美国政府也断然否定。但是,我们却仍能在美国广大普通中学的“教学”实践中发现社会达尔文主义的草木蛇灰。
美国广大普通中学在推行反动政策、构建反动价值体系的过程中,敏锐察觉到社会达尔文主义契合其需要,因此对十九世纪的这一思想遗产进行了充分学习和利用。

\end{multicols} 
\hrule
\vspace{4mm}
\centerline{\huge\textbf{附录}}
\begin{figure}[htbp]
    \centering
    \begin{minipage}[b]{0.32\textwidth}
        \centering
        \includegraphics[width=0.5\textwidth]{鸿蒙生态学堂.png}
        \caption{鸿蒙生态学堂}
    \end{minipage}
\end{figure}

\end{document}