% HEAD BEGIN
\documentclass[letterpaper, 12pt]{article}
\newsavebox\colbbox
\usepackage{graphicx}
\usepackage{multicol}
\usepackage{anysize}
\usepackage{fontspec}
\usepackage[fontset=none]{ctex}
\usepackage{tabularx}
\usepackage{longtable}
\PassOptionsToPackage{hyphens}{url}
\usepackage[breaklinks=true, colorlinks=true]{hyperref}
\expandafter\def\expandafter\UrlBreaks\expandafter{\UrlBreaks\do\a\do\b\do\c\do\d\do\e\do\f\do\g\do\h\do\i\do\j\do\k\do\l\do\m\do\n\do\o\do\p\do\q\do\r\do\s\do\t\do\u\do\v\do\w\do\x\do\y\do\z\do\A\do\B\do\C\do\D\do\E\do\F\do\G\do\H\do\I\do\J\do\K\do\L\do\M\do\N\do\O\do\P\do\Q\do\R\do\S\do\T\do\U\do\V\do\W\do\X\do\Y\do\Z}
% \let\oldurl\url
% \renewcommand{\url}[1]{\begin{sloppypar}\oldurl{#1}\end{sloppypar}}
\setlength\columnsep{30pt}
\marginsize{30pt}{30pt}{10pt}{20pt}
\setmainfont{TeX Gyre Bonum}
\setCJKmainfont[BoldFont=Noto Serif CJK SC Bold, ItalicFont=FandolKai]{Noto Sans CJK SC}
\setlength{\parindent}{0cm}
% \setCJKmonofont{Noto Sans CJK SC}
\begin{document}
\begin{center}
    \Huge\textbf{南哪大专醒前消息}
\end{center}
\vspace{4mm}
\hrule
\renewcommand\tabularxcolumn[1]{m{#1}}
\begin{tabularx}{\textwidth}{>{\hsize.2\hsize}X>{\hsize.6\hsize}X>{\hsize.2\hsize}X}
    \begin{flushleft}
        2024.10.23\, No.96
    \end{flushleft}
    &
    \begin{center}
        \textit{“Your dog is a true philosopher” \\Plato (427-347BC)}
    \end{center}
    &
    \begin{flushright}
        \textbf{南京市栖霞区}
    \end{flushright}
\end{tabularx}
\vspace{-3.5mm}
\hrule
\vspace{4mm}
% HEAD END
\centerline{\huge\textbf{活动预告}}
\begin{multicols}{2}
    \section{订阅方式和加入编辑部}  
编辑部招聘人才,用爱发电,工作轻松,详情可联系QQ:1329527951 客服小祥\\想订阅本消息或获取PDF版(便于查看超链接和往期),可加QQ群:\href{https://qm.qq.com/q/VXIW7fgsEe}{849644979}.
\section{Deadline Ongoing}
\setbox\colbbox\vbox{
\makeatletter\col@number\@ne
\begin{longtable}{|c|c|c|}
    \hline
    消息(未见ddl的,不刊) & 截止日期 & 刊载日期\\
    \hline\hline
    历史学院羽毛球赛 & 10.26 & 10.22\\
    紫藤学刊征稿 & 12.15 & 10.22\\
    急救培训活动报名 & 10.24 & 10.17\\
    计院迎新晚会征集节目 & 10.25 & 10.12\\
    马院主题宣讲报名 & 10.25 & 10.5\\
    织围巾志愿者招募 & 10.26 & 10.20\\
    心协DIY活动 & 10.26 & 10.20\\
    心协流光影院 & 10.26 & 10.17\\
    校园今日说法大赛 & 10.26 & 10.17\\
    遵义精神宣讲团遴选 & 10.27 & 10.10\\
    青鸟剧场新戏招募 & 10.27 & 10.14\\
    体测 & 10.27 & 10.16\\
    鼓楼音乐跑 & 10.27 & 10.20\\
    普通话考试报名 & 10.28 & 10.14\\
    仙林校史馆招募讲解员 & 10.30 & 9.12\\
    本科生暑期课程评教 & 10.31 & 9.19\\
    黑匣招募 & 11.1 & 10.19\\
    学位英语考试报名 & 11.3 & 10.17\\
    校运会 & 11.8 & 10.21\\
    后革命鲁迅研究征文 & 11.10 & 10.8\\
    大创训练计划申报 & 11.18 & 9.24\\
    招生宣传创意征集大赛 & 11.18 & 10.21\\ 
    EBSCO数据库检索大赛 & 11.20 & 10.3\\
    文院征稿 & 11.20 & 10.20\\
    乐跑 & 12.8 & 10.12\\
    大创课题成员招募 & 10.24 & 10.22\\
    国际访学计划申报 & 11.22 & 10.22\\
    校园涂鸦快闪 & 10.24 & 10.22\\
    毓琇书院宿舍评比 & 10.31 & 10.22\\
    物院原子弹爆炸活动 & 10.24 & 10.22\\
    流光影院 & 10.26 & 10.23\\
    
    \hline
\end{longtable}
\unskip
\unpenalty
\unpenalty}\unvbox\colbbox
\end{multicols}
\hrule
\pagebreak
\begin{multicols}{2}

\section{讲座}
\begin{tabular}{|c|c|c|}
    \hline
    往期讲座 & 开展日期 & 刊载日期\\
    \hline\hline
    《聚合物的研发与...》 & 10.24 & 10.3\\
    《电池及电化学能...》 & 11.24 & 10.3\\
    《专利查新与规避...》 & 12.19 & 10.3\\
    《与<自然>编辑对...》 & 10.30 & 10.16\\
    《语言能力与前近...》 & 10.25 & 10.18\\
    《国家图书馆的古...》 & 10.24 & 10.18\\
    图书馆系列讲座 & 12.3 & 10.20\\
    《大众视角与历史...》 & 10.25 & 10.21\\
    《近代中国女性史...》 & 10.24 & 10.22\\
    《量子非互易性》 & 10.24 & 10.22\\
    《志工人力资源的...》 & 11.4 & 10.23\\
    《华人社会工作的...》 & 11.4 & 10.23\\
    《困在历史中的卢...》 & 10.29 & 10.23\\
    《Animal Liberation...》 & 10.25 & 10.23\\
    《元明时代的中心...》 & 10.26 & 10.23\\
    《唐代青藏高原主...》 & 10.25 & 10.23\\
    《The Token-Effort...》 & 10.25 & 10.23\\
    《从诺奖看AI在科...》 & 10.26 & 10.23\\
    《Complexity of...》 & 10.24 & 10.23\\
    《Non-convergence ...》 & 10.24 & 10.23\\
    《基础Python……》 & 10.24 & 10.23\\
    \hline
\end{tabular}

1.社区资源开发与运用——志工人力资源的整合运用\\
主讲人:曾华源 台湾东海大学社会工作系教授\\
主持人:陈友华 南京大学社会学院教授、南京大学河仁社会慈善学院院长\\
时间:11月4日(周一)14:00-16:00\\
地点:社会学院(河仁楼)401室\\

2.华人社会工作的专业伦理实践——困境与出路\\
主讲人:白倩如 台湾彰化师范大学辅导与咨商学系 副教授\\
主持人:陈友华 南京大学社会学院教授、南京大学河仁社会慈善学院院长\\
时间:11月4日(周一)16:00-18:00\\
地点:社会学院(河仁楼)401室\\

3.伟大的坠落——困在历史中的卢卡奇\\
主讲人:张异宾(南京大学文科资深教授)\\
时间:10月29日18:30\\
地点:哲学学院(薛光林楼)401室\\

4.Animal Liberation Now\\
主讲人:Peter Singer(著名哲学家)\\
与谈人:陈真(南京师范大学哲学系教授)\\
主持人:胡星铭(南京大学哲学学院教授)\\
时间:10月25日15:00\\
地点:哲学学院(薛光林楼)401室\\
摘要:\url{https://mp.weixin.qq.com/s/EAAz6mAbnMiPJjTdZpUALA}\\

5.元明时代的“中心”与“外缘”:第一届南京大学民族与边疆研究中心青年工作坊\\
主办:南京大学民族与边疆研究中心\\
时间:2024年10月26~27日\\
地点:南京大学历史学院216室\\

6.唐代青藏高原主要部族的位置与境域
主题:唐代青藏高原主要部族的位置与境域\\
时间:10月25日 15:00\\
地点:南京大学历史学院216室\\
主讲人:周宏伟,陕西师范大学西北历史环境与经济社会发展研究院教授\\
主持人:胡箫白,南京大学历史学院副教授\\
与谈人:孙鹏浩,南京大学历史学院副教授\\

7.南京大学管理学院青年学者论坛(第687期)\\
主题:The Token-Effort Effect: Trivial Redemption Effort Increases Price Promotion Effectiveness\\
主讲人:张夼劼,南洋理工大学商学院市场营销系副教授\\
主持人:初星宇,南京大学商学院营销与电子商务系副教授\\
时间:2024年10月25日(周五)14:00\\
地点:鼓楼校区安中楼301\\

8.格物致理 | 物理与智能的碰撞:从诺奖看AI在科学中的崛起
时间:10月26日9点\\
地点:鼓楼校区逸夫馆1-407\\
主讲人:南京大学物理学院 卢毅教授\\




9.数学学院学术报告1\\
题目: Complexity of Inexact Cubic-regularized Primal-dual Methods for Finding Second-order Stationary Points\\
报告人: 王晓 研究员 (鹏城国家实验室)\\
时间: 10月24日 下午 15:30 - 16:30\\
地点: 西大楼 210\\
\url{https://mp.weixin.qq.com/s/Dld8dStfLNHuap7e7_dXsw}\\


10.数学学院学术报告2\\
题目: Non-convergence Analysis of Randomized Direct Search\\
报告人: 张在坤 博士 (香港理工大学)\\
时间: 10月24日 下午 16:30 - 17:30\\
地点:西大楼210\\
\url{https://mp.weixin.qq.com/s/Dld8dStfLNHuap7e7_dXsw}\\

11.开甲书院惟学沙龙:从基础开始——Python语法及Shell探究\\
主讲人:钟亚晨,南京大学计算机学院2024级硕士研究生,2020级开甲书院学生,2024年春《Python基础》助教\\
时间:2024年10月24日(周四)18:30\\
地点:南京大学鼓楼校区南青格庐多功能教室\\

\section{环院2024专场招聘会}
南京大学2024绿创未来·环境行业专场招聘会将于10月25日9:00-17:00在南京大学仙林校区环境学院举行。
\\
本次专场招聘会汇聚企事业单位25家,其中包括多家环境领域领军企业,招聘岗位涉及环境、能源、电气、化工、大气、给排水、机械、材料等十余类方向、近300个优质岗位。\\
当天下午13:20-14:00将举办生涯规划分享会,14:00-16:00部分企业还将进行现场宣讲和面试(具体时间、地点以相关企业现场通知为准)。\\
详见\url{https://mp.weixin.qq.com/s/En9H-3bIGsZ2hpzpqKwPeQ}\\
\section{浦口图书馆“书游中国”活动}
地点:浦口图书馆二楼平台\\
时间:10月25日 周五 中午12:30\\
主要内容:到场自由组队,在限定时间内找到领取的书籍中涉及的主要地名,按用时长短排名,有机会获得精美好礼\\
详情\url{https://mp.weixin.qq.com/s/ehyqOP24uc0GIuZM_7btgg}\\
转发该推文集赞可至活动现场领取纪念品\\


\section{流光影院 | 《海蒂和爷爷》}
本学期流光影院共有三场,第一场电影《海蒂和爷爷》即将开场\\
时间:2024.10.66 19:00\\
地点:南京大学仙林校区炜华体育场(若天气不佳则地点变更为仙林敬文大学生活动中心9楼,地点变更与否请加入推文中的活动群等待通知)\\
主办:南京大学学生心理协会\\
转发推文\url{https://mp.weixin.qq.com/s/NdMUAxYgMlgYyhsmNf-bRQ}并集赞可在活动现场兑换专属票根\\
\section{走近名企|第八弹——小米}
参访时间\\
2024年10月29日(周二)下午14:00\\
参访地点\\
小米南京科技园(南京市建邺区河西中央商务区)\\
参访人数\\
40人\\
活动流程\\
统一乘坐大巴前往,起始校区将根据学生报名情况决定\\
14:00-14:10  入园\\
14:20-14:40  小米之家参观\\
14:40-16:00  公司介绍\&小米人才培养合作\\
16:00-16:10  小米园区参访\\
16:10        返程\\
详见\url{https://mp.weixin.qq.com/s/etRwVkRXYTNhovRy4RfC6g}\\
\section{物院院服设计投票}
共有五种设计款式。详见原文:\url{https://mp.weixin.qq.com/s/W9aRWp__mNmhBBLjAE4nQg}\\
投票链接(包含图片实例):\url{https://tp.wjx.top/vm/ODIgtDY.aspx#}

\section{秉文、有训|文理融通4.0}
活动简介:你是否渴望突破文理界限,找到跨专业的好友,在学术生活上交流协作、有所进步呢?这里,既有数学大佬带你拿捏微积分,又有文史达人携你纵览古今;有一串串简洁有趣的python代码,也有一个个生机勃勃的英语单词……\\
活动时间:2024.10.28—2024.12.8\\
活动地点\\
1)小组结伴活动:\\
 线下:鼓楼校区的空闲教室或图书馆等线上:番茄自习室、召开腾讯会议进行学业辅导与交流讨论等\\
2)集体活动:\\
鼓楼校区教室举办集体学业辅导活动,或生活团建等\\
活动流程:\\
活动宣发后,请有意向参与的同学填写报名链接,并进入QQ交流群,彼此沟通了解,自由组成2-6人为单位包含秉文和有训两个书院同学的小组。完成结伴后,每个小组可根据关键词开展各种"文理融通"活动。同时,有训书院和秉文书院也将联合举办集体活动。\\
(会定期收集同学的活动照片、感想或者vlog等,活动频率较高、内容较丰富的小组将获得精美奖品哦~)\\
报名表和QQ群见\url{https://mp.weixin.qq.com/s/rFXSWeOV3S2kGY8UnjNL9w}

\section{物理学院朋辈讲堂(期中复习)}
原文:\url{https://mp.weixin.qq.com/s/DeKKkxt1p4kTDELPVmWk9w}\\
【电磁学】\\
时间:10月25日(周五)晚19:00-20:30\\
地点:仙Ⅰ-112\\
主讲人:陈冠铭\\
【近代物理】\\
时间:10月26日(周六)晚19:00-21:00\\
地点:仙Ⅰ-112\\
主讲人:魏鹏程\\


\end{multicols} 
\end{document}