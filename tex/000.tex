% HEAD BEGIN
\documentclass[letterpaper, 12pt]{article}
\newsavebox\colbbox
\usepackage{graphicx}
\usepackage{multicol}
\usepackage{anysize}
\usepackage{fontspec}
\usepackage[fontset=none]{ctex}
\usepackage{tabularx}
\usepackage{longtable}
\PassOptionsToPackage{hyphens}{url}
\usepackage[breaklinks=true, colorlinks=true]{hyperref}
\expandafter\def\expandafter\UrlBreaks\expandafter{\UrlBreaks\do\a\do\b\do\c\do\d\do\e\do\f\do\g\do\h\do\i\do\j\do\k\do\l\do\m\do\n\do\o\do\p\do\q\do\r\do\s\do\t\do\u\do\v\do\w\do\x\do\y\do\z\do\A\do\B\do\C\do\D\do\E\do\F\do\G\do\H\do\I\do\J\do\K\do\L\do\M\do\N\do\O\do\P\do\Q\do\R\do\S\do\T\do\U\do\V\do\W\do\X\do\Y\do\Z}
% \let\oldurl\url
% \renewcommand{\url}[1]{\begin{sloppypar}\oldurl{#1}\end{sloppypar}}
\setlength\columnsep{30pt}
\marginsize{30pt}{30pt}{10pt}{20pt}
\setmainfont{TeX Gyre Bonum}
\setCJKmainfont[BoldFont=Noto Serif CJK SC Bold, ItalicFont=FandolKai]{Noto Sans CJK SC}
\setlength{\parindent}{0cm}
% \setCJKmonofont{Noto Sans CJK SC}
\begin{document}
\begin{center}
    \Huge\textbf{南哪大专醒前消息}
\end{center}
\vspace{4mm}
\hrule
\renewcommand\tabularxcolumn[1]{m{#1}}
\begin{tabularx}{\textwidth}{>{\hsize.2\hsize}X>{\hsize.6\hsize}X>{\hsize.2\hsize}X}
    \begin{flushleft}
        2025.2.25\, No.175
    \end{flushleft}
    &
    \begin{center}
        \textit{“秉中持正、求新博闻。”}
    \end{center}
    &
    \begin{flushright}
        \textbf{南京市栖霞区}
    \end{flushright}
\end{tabularx}
\vspace{-3.5mm}
\hrule
\vspace{4mm}
% HEAD END
\centerline{\huge\textbf{活动预告}}
\begin{multicols}{2}
    \section{订阅方式和加入编辑部}  
编辑部招聘人才,用爱发电,工作轻松,详情可联系QQ:1329527951 客服小祥\\想订阅本消息或获取PDF版(便于查看超链接和往期),可加QQ群:\href{https://qm.qq.com/q/VXIW7fgsEe}{849644979}.
\section{Deadline Ongoing}
\setbox\colbbox\vbox{
\makeatletter\col@number\@ne
\begin{longtable}{|c|c|c|}
    \hline
    消息(未见ddl的,不刊) & 截止日期 & 刊载日期\\
    \hline\hline
    南大版deepseek & / & 2.22\\
    天文台开放日 & / & 1.6\\
    悦读课程群 & / & 2.24\\
    原创剧本联合孵化报名 & 3.20 & 1.10\\
    njumun代表报名 & 3.2 & 1.16\\
    课程补退选 & 3.2 & 2.19\\
    南大育教新媒体招新 & 2.27 & 2.19\\
    本科生劳育实践 & 7.20 & 2.19\\
    医保零星报销 & 3.31 & 2.19\\
    第二届大学生阅读分享活动 & 3.7 & 2.21\\
    心理中心助理招新 & 2.28 & 2.20\\
    招办全媒体招新 & 3.5 & 2.20\\
    交响乐团招新 & 3.7 & 2.20\\
    歌魅剧务招募 & 2.26 & 2.21\\
    萌马音乐工作室招新 & 2.28 & 2.22\\
    秉文书院早晚自习 & 3.3 & 2.23\\
    图协招新 & 2.28 & 2.23\\
    “核真录”招新 & 3.2 & 2.24\\
    菁菁南数招募讲师 & 3.9 & 2.24\\
    四六级查分 & 2.27 & 2.24\\
    新火星观影会 & 3.2 & 2.25\\
    瓦观赛 & 2.27 & 2.25\\
    车协骑行 & 3.1 & 2.25\\
    
    \hline
\end{longtable}
\unskip
\unpenalty
\unpenalty}\unvbox\colbbox
\end{multicols}
\hrule
\pagebreak
\begin{multicols}{2}

\section{讲座}
\begin{tabular}{|>{\centering\arraybackslash}m{.3\textwidth}|m{.06\textwidth}|m{.06\textwidth}|}
    \hline
    讲座 & 开展时间 & 刊载时间\\
    \hline\hline
    人机协同背景下高等外语教育的守正创新 & 2.27 & 2.17\\\hline
    大陆的起源 & 3.4 & 2.17\\\hline
    南京“世界文学之都”的前世今生 & 2.27 & 2.20\\\hline
    香港大学经管学院硕士课程高校专场线下宣讲会 & 2.27 & 2.20\\\hline
    2025香港大学暑期课程宣讲会 & 2.26 & 2.21\\\hline
    在哲学与艺术之间 & 2.26 & 2.23\\\hline
    Nonlinear magneto-optical effect in 2D magnets & 2.27&2.24\\\hline
    社会分层视野下的教育失配研究:博士论文的写作、拓展与反思 & 2.26 & 2.24\\\hline
    MIMOS: the asynchronous paradigm and tools for safety-critical software design and updates & 2.27 & 2.24\\\hline
    长江文明系列讲座第一期 & 2.27 & 2.24\\\hline
    椭圆,单摆与算数-几何平均 & 2.26 & 2.25\\\hline
    身体的重写本 & 3.7 & 2.25\\\hline
    探秘华为开发者大赛体系 & 2.28 & 2.25\\\hline
    党的创建和伟大建党精神 & 2.28 & 2.25\\\hline
\end{tabular}

1.数学学院本科生论坛(教师系列第93讲——大一专场)\\
题目: 椭圆,单摆与算数—几何平均\\
报告人:石亚龙\\
时间:02月26日(星期三) 16:00-17:30\\
地点:戊己庚四楼北\\
腾讯会议:399-1313-1750\\
摘要: 我们将从梅加强老师《数学分析》第二章的一道习题出发,介绍椭圆积分的故事。\\
\url{https://mp.weixin.qq.com/s/yZpmcWm-YN1lBswe8-0JGg}\\
\\
2.黑匣子|讲座:身体的重写本\\
讲座时间:3月7日 15:00\\
讲座地点:黑匣子剧场(敬文大学生活动中心3层)\\
人数:70人\\
(开放参加,成功提交报名信息即报名成功)\\
详情见:\url{https://mp.weixin.qq.com/s/Bntx5BFzww5iX7Iletlbdw}(推文后半段)\\

3.青年学术科创日:探秘开发者大赛体系\\
活动时间:2月28日(本周五)下午14:30—16:00\\
活动地点:南京大学仙林校区 南青报告厅\\
链接:\url{https://mp.weixin.qq.com/s/ncOS3yUoBs_YQMucFG4ZtQ}\\
\section{【转发】网络通识课《认识地球》和《自然灾害与人》开课通知}
开课日期2025年2月24日,结课日期2025年4月20日\\
链接:\url{https://jw.nju.edu.cn/5e/69/c26263a745065/page.htm}\\

4.党的创建和伟大建党精神

2.26 10:00 仙II 211

丁晓强,华东师范大学特聘教授\\
\section{心协|南苏招新}
南京大学学生心理协会面向苏州校区招新,可收获海量心理学知识和丰富志愿时长等超多福利\\
报名链接:\url{https://table.nju.edu.cn/dtable/forms/8fffc0b6-bb8b-4547-a840-6396f919825c/}\\
详情:\url{https://mp.weixin.qq.com/s/1st2I2sE-fvqaNhYMeDhIw}\\


\section{新火星观影会|动漫电影《千年女优》}
时间:3月2日18:30(本周日) 地点:鼓楼校区费A410\\
观影群:907939564\\
《千年女优》为今敏导演作品,讲述了影星藤原千代子终其一生跨越时空,追逐一名于20世纪30年代被她救援的被捕反战画家的故事。影片以蒙太奇手法展开,展现了贯穿千代子半生的抗争,旨在探讨“追逐”本身的意义并反思日本的现代性神话。感兴趣的同学可前来观影,观影后另有可选择参与的讨论环节。\\


\section{南京大学医学院春季游园会}
活动时间:2025年3月1日 14:00-17:00   活动地点:鼓楼校区东南楼前\\
同学们凭借专属的“医趣币”可以游玩欢乐套福气、绘扇漆香浓、巧思解灯谜、惊喜盲盒乐、香韵囊艺制、福运抽签乐等六项活动。“医趣币”获取攻略:转发预热推送至朋友圈,每集5个赞可得1枚医趣币,每人最多可得15枚。详见公众号“南大医 NJUMed”推文。另:本次参加游园活动将加3h学科实践。\\
链接:\url{https://mp.weixin.qq.com/s/A4C7IyUHoDBDMHJYkyeD_A}



\section{观赛预告|无畏契约曼谷大师赛}
时间:2月27日(周四)18:00\\
地点:鼓楼校区南青格庐多功能厅\\
QQ群见:\url{https://mp.weixin.qq.com/s/f9x0ON3r2RlJiKM1M_Jn8Q}

\section{南大车协骑行}
活动时间:2025.3.1\\
集合地点:仙林校区早八点在仙林校区1栋活动室集合出发,鼓楼校区八点四十五在汉口路集合出发。十点前在牛首山金陵小城集合点汇合。\\
骑行路线:从集合点出发,途经大唐金香草谷、黄龙岘,最终抵达石塘竹海景区,自鼓楼出发往返约100公里,仙林出发约130公里。在石塘景区吃午饭、休整后带队返回。\\
报名方式:进入南大车协qq群或微信群报名。见原文:\url{https://mp.weixin.qq.com/s/YIi3X2tSsvFLgDtNX3F6nA}


\section{黑匣工作坊|“Palimpsest|行间”工作坊招募}
工作坊时间:3月8日-9日 周六-周日 全天\\
地点:黑匣子剧场\\
人数:12人\\
(本次工作坊采取报名筛选制,入选后工作人员会与您联络,戏剧影视艺术系师生优先)\\
详情见:\url{https://mp.weixin.qq.com/s/Bntx5BFzww5iX7Iletlbdw}


\section{春季南京大学排球院系杯章程}
本次院系杯仅接受队伍报名,各队队长将队员名单整理并于3月1日前发至组织部邮箱ndpxzzb@163.com,邮件名称请备注“院系杯+队伍名称”,截止日期前均可调整名单。\\
详细说明见:\url{https://mp.weixin.qq.com/s/4M1L2Z86TvnCcpTzZ9Tgxw}\\
本赛季新队员注册、外援及留队说明见:\url{https://mp.weixin.qq.com/s/VP2iSnuz0BwbtMtoIQfDhg}

\section{饕餮大餐 | 2.25-2.27(周二~周四)学术文化活动概览}
本期“饕餮大餐·学术文化概览”为你汇总了2月25日到2月27日(周二至周四)的学术讲座。\\
详情见:\url{https://mp.weixin.qq.com/s/7k809esFYyg39oACLy4tJw}

\end{multicols} 
\hrule
\vspace{4mm}
\centerline{\huge\textbf{参考消息}}
\begin{multicols}{2}
\section{南哪消息同学小文连载板块}
因收到小说投稿一篇,南哪消息现在开辟了同学小文连载板块。如想评论,可以发至邮箱:1329527951@qq.com,第二天会刊在此处。如想投稿渠道相同。
\section{多位在五食就餐的同学病倒}
多位同学反映,昨晚(2月24日)出现急性肠胃炎症状,且前往校医院后发现有大量同学因肠胃炎前来输液,其中不少同学此前在五食就餐,有食用烤鸭饭的同学呕吐不止。虽然目前还不确定食堂的饭菜与肠胃炎的关系,但出于慎重考虑建议大家在官方排除食品卫生问题可能性前避免前往。
\section{《等待,遗忘》(3)}
金映樺\\
\newCJKfontfamily\fan{FandolFang}\fan
回家之后立刻被娘娘拉住询问这天的情况,期间被迫接受了审视的目光、不屑的冷哼、恨铁不成钢的微笑。好吧,好吧,总之最后百般央求讨要到了她的微信号。我兴奋地捧着手机打开微信按下加号搜索好友想要立即发送好友申请,但是突然被什么不知名的东西绊住了,犹豫片刻将手机扔到一遍。好纠结。我不禁回想起不久前我脱口而出的并不好笑的笑话,会不会太唐突,她会不会觉得我脑子不大好使(这绝对是偏见)。我反复揣摩当时她的反应,却怎么也分不清那是会心的笑颜还是出于基本礼仪的客气疏离。\\

要不要加呢,要不要加呢,现在就加吗,还是之后?我试图让自己冷静下来,于是打开了PPT决定复习一下(我一直这么好学)。\\

“五蕴者,谓色蕴、受蕴、想蕴、行蕴、识蕴也。”这门课中我一直不是很懂佛教的这些概念,玄而又玄,不过“玄”似乎也不适用于佛教。但是这句话我有印象,因为娘娘经常抄的经中有,“观自在菩萨,行深般若波罗蜜多时,照见‘五蕴’皆空。”识是能知的心,它带动其他的心念,执著我、爱他所执著的我,这是有情众生。我突然顿悟了。我应该现在就去加林望舒的微信,不执着的话最后什么也没有,就像,就像...我下意识地咬住嘴唇,直到口腔中弥散开腥甜的味道,我才发现因为太用力,我开始流血。我不想再像过去那样失去一样失去她,就算我们现在只有一面之缘。我知道我必须做出什么,才能带来改变。这或许也是我没有所谓慧根的原因。我不明白为什么要无所挂念,才能真正解脱因果的束缚。我或许这辈子都无法明白。我只是那个为了命运的改变而行善的小和尚,为了心安而祈祷的信徒。我无法真正绝望。私欲、欲望、望舒。\\

我终于发送了好友请求,“你好,我是余淮。”等了一会,她没有通过。我陷入了焦虑之中,无意间抬眼发现已经是深夜了。我松了一口气,将自己扔在床上,却无法入睡。后来我陷入了半梦半醒的游离状态,回到我的高中时光。林望舒不知为何成了我的高中同学,我们仿佛很熟悉,又好像很陌生。
我梦见下课铃响后,她回头跟我搭话,她像往常一样把手撑在我的桌边,我抬头看她,是高中时的她没错了。感觉她努力组织语言,我却听不清她在说什么。我想握住她的手却怎么也够不到。直到泪水滑至嘴角,我才发现那是我在哭。\\

我梦见高三的那个冬天,听课的人寥寥无几,窗外下雪了,我的座位恰巧在窗边,白色让我无法呼吸,我不知道应该去哪,路又在什么地方。她把卷子传过来的瞬间,我得以喘息。周六我们一起补课,人生就是大大的痛苦和小小的快乐。我突然发现在高中的任何一个地点我都能回忆起和她在这具体的场景。楼梯间隐蔽的告白,考完试她无法抑制流个不停的眼泪,食堂里浓重的油烟无味的饭菜,吃完午饭回班级路上大片的阳光,十元一杯总因喝得太急而将上颚烫掉层皮的热可可。心悸的气味扑鼻而来几乎无法抵挡,以至于我惊觉是否将绝大部分的我都留在了那里,只有心微微不同了?我理应对那段记忆深恶痛绝,但总像内里苦涩但包层甜美糖衣的毒药令人想一而再再而三地舔舐。或许因为真的得到过什么,得到过悸动、美好和共享虚幻未来的满足。\\

我猛地从床上坐起。梦境太真实,以至于我有些分不清这是现实还是仍在梦中。这一切都未曾发生过。我的高中生活平凡无趣,按部就班,有人为我安排好了未来的路,我从未感到迷茫。\\

我拿起手机,发现有一条新消息。\\

“你好,余淮。”\\

我终于抑制不住地开始大哭,执著贪爱的妄想心,我永远也无法抵达离喜妙乐地了。\\

\end{multicols} 

\end{document}
