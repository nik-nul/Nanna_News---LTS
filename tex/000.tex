% HEAD BEGIN
\documentclass[letterpaper, 12pt]{article}
\newsavebox\colbbox
\usepackage{graphicx}
\usepackage{multicol}
\usepackage{anysize}
\usepackage{fontspec}
\usepackage[fontset=none]{ctex}
\usepackage{tabularx}
\usepackage{longtable}
\PassOptionsToPackage{hyphens}{url}
\usepackage[breaklinks=true, colorlinks=true]{hyperref}
\expandafter\def\expandafter\UrlBreaks\expandafter{\UrlBreaks\do\a\do\b\do\c\do\d\do\e\do\f\do\g\do\h\do\i\do\j\do\k\do\l\do\m\do\n\do\o\do\p\do\q\do\r\do\s\do\t\do\u\do\v\do\w\do\x\do\y\do\z\do\A\do\B\do\C\do\D\do\E\do\F\do\G\do\H\do\I\do\J\do\K\do\L\do\M\do\N\do\O\do\P\do\Q\do\R\do\S\do\T\do\U\do\V\do\W\do\X\do\Y\do\Z}
% \let\oldurl\url
% \renewcommand{\url}[1]{\begin{sloppypar}\oldurl{#1}\end{sloppypar}}
\setlength\columnsep{30pt}
\marginsize{30pt}{30pt}{10pt}{20pt}
\setmainfont{TeX Gyre Bonum}
\setCJKmainfont[BoldFont=Noto Serif CJK SC Bold, ItalicFont=FandolKai]{Noto Sans CJK SC}
\setlength{\parindent}{0cm}
% \setCJKmonofont{Noto Sans CJK SC}
\begin{document}
\begin{center}
    \Huge\textbf{南哪大专醒前消息}
\end{center}
\vspace{4mm}
\hrule
\renewcommand\tabularxcolumn[1]{m{#1}}
\begin{tabularx}{\textwidth}{>{\hsize.2\hsize}X>{\hsize.6\hsize}X>{\hsize.2\hsize}X}
    \begin{flushleft}
        2024.10.24\, No.97
    \end{flushleft}
    &
    \begin{center}
        \textit{“克明峻德。”}
    \end{center}
    &
    \begin{flushright}
        \textbf{南京市栖霞区}
    \end{flushright}
\end{tabularx}
\vspace{-3.5mm}
\hrule
\vspace{4mm}
% HEAD END
\centerline{\huge\textbf{活动预告}}
\begin{multicols}{2}
    \section{订阅方式和加入编辑部}  
编辑部招聘人才,用爱发电,工作轻松,详情可联系QQ:1329527951 客服小祥\\想订阅本消息或获取PDF版(便于查看超链接和往期),可加QQ群:\href{https://qm.qq.com/q/VXIW7fgsEe}{849644979}.
\section{Deadline Ongoing}
\setbox\colbbox\vbox{
\makeatletter\col@number\@ne
\begin{longtable}{|c|c|c|}
    \hline
    消息(未见ddl的,不刊) & 截止日期 & 刊载日期\\
    \hline\hline
    毓琇书院期中辅导 & 10.26、27 & 10.24\\
    历史学院羽毛球赛 & 10.26 & 10.22\\
    紫藤学刊征稿 & 12.15 & 10.22\\
    急救培训活动报名 & 10.24 & 10.17\\
    计院迎新晚会征集节目 & 10.25 & 10.12\\
    马院主题宣讲报名 & 10.25 & 10.5\\
    织围巾志愿者招募 & 10.26 & 10.20\\
    心协DIY活动 & 10.26 & 10.20\\
    心协流光影院 & 10.26 & 10.17\\
    校园今日说法大赛 & 10.26 & 10.17\\
    遵义精神宣讲团遴选 & 10.27 & 10.10\\
    青鸟剧场新戏招募 & 10.27 & 10.14\\
    体测 & 10.27 & 10.16\\
    鼓楼音乐跑 & 10.27 & 10.20\\
    普通话考试报名 & 10.28 & 10.14\\
    仙林校史馆招募讲解员 & 10.30 & 9.12\\
    本科生暑期课程评教 & 10.31 & 9.19\\
    黑匣招募 & 11.1 & 10.19\\
    学位英语考试报名 & 11.3 & 10.17\\
    校运会 & 11.8 & 10.21\\
    后革命鲁迅研究征文 & 11.10 & 10.8\\
    大创训练计划申报 & 11.18 & 9.24\\
    招生宣传创意征集大赛 & 11.18 & 10.21\\ 
    EBSCO数据库检索大赛 & 11.20 & 10.3\\
    文院征稿 & 11.20 & 10.20\\
    乐跑 & 12.8 & 10.12\\
    大创课题成员招募 & 10.24 & 10.22\\
    国际访学计划申报 & 11.22 & 10.22\\
    校园涂鸦快闪 & 10.24 & 10.22\\
    毓琇书院宿舍评比 & 10.31 & 10.22\\
    物院原子弹爆炸活动 & 10.24 & 10.22\\
    流光影院 & 10.26 & 10.23\\
    南大献血周 & 10.31 & 10.24\\
    南京马拉松志愿者招募 & 10.26 & 10.24\\
    信件盲盒活动 & 10.26 & 10.24\\
    
    \hline
\end{longtable}
\unskip
\unpenalty
\unpenalty}\unvbox\colbbox
\end{multicols}
\hrule
\pagebreak
\begin{multicols}{2}

\section{讲座}
\begin{tabular}{|c|c|c|}
    \hline
    往期讲座 & 开展日期 & 刊载日期\\
    \hline\hline
    《聚合物的研发与...》 & 10.24 & 10.3\\
    《电池及电化学能...》 & 11.24 & 10.3\\
    《专利查新与规避...》 & 12.19 & 10.3\\
    《与<自然>编辑对...》 & 10.30 & 10.16\\
    《语言能力与前近...》 & 10.25 & 10.18\\
    《国家图书馆的古...》 & 10.24 & 10.18\\
    图书馆系列讲座 & 12.3 & 10.20\\
    《大众视角与历史...》 & 10.25 & 10.21\\
    《近代中国女性史...》 & 10.24 & 10.22\\
    《量子非互易性》 & 10.24 & 10.22\\
    《志工人力资源的...》 & 11.4 & 10.23\\
    《华人社会工作的...》 & 11.4 & 10.23\\
    《困在历史中的卢...》 & 10.29 & 10.23\\
    《Animal Liberation...》 & 10.25 & 10.23\\
    《元明时代的中心...》 & 10.26 & 10.23\\
    《唐代青藏高原主...》 & 10.25 & 10.23\\
    《The Token-Effort...》 & 10.25 & 10.23\\
    《Complexity of...》 & 10.24 & 10.23\\
    《Non-convergence ...》 & 10.24 & 10.23\\
    《基础Python……》 & 10.24 & 10.23\\
    《行知学堂……》 & 10.27 & 10.24\\
    《从诺奖看AI在科...》 & 10.26 & 10.24\\

    
    \hline
\end{tabular}

1.2024诺奖解读系列报告会——“物理与智能的碰撞:从诺奖看AI在科学中的崛起”\\
讲座时间:10.26(周六)上午9点\\
讲座地点:鼓楼校区 逸夫馆1-407\\
主讲人:卢毅,现任南京大学物理学院教授。2010年于北京大学获得物理学学士学位,2017年于德国马克思普朗克固体研究所获得博士学位,随后在德国海德堡大学理论物理所开展博士后研究工作,并于2021年加入南京大学。研究领域涵盖凝聚态理论与量子多体计算,重点关注强关联电子体系的新奇物相探索,以及基于张量网络和神经网络等手段的关联计算方法开发。

2.行知学堂:下一代信任网络:技术革新、数据流通与系统安全\\
时间:10月27日(周日)下午14:00\\
地点:鼓楼校区新教学楼207\\
主讲人:高承实(安徽栈谷科技有限公司董事长)——分讲座:区块链的发展和未来\\
杨征(北京大学长沙计算与数字经济研究院、科技开发部副部长、研究员)——分讲座:数据要素市场化配置探析\\
孙猛(北京大学数学科学学院信息与计算科学系副主任,教授、博士生导师)——分讲座:胜却人间无数——当区块链与形式化验证相逢\\
主持人:颜嘉麒(南京大学信息管理学院信息管理科学系主任,教授、博士生导师)\\
讲座报名二维码见\url{https://mp.weixin.qq.com/s/2CxULx2OR1p2bWb9JAp6qQ}

\section{江苏省高校智盟卡推出}
申办江苏省高校数字图书馆联盟(JALIS)推出的高校智盟卡,可前往联盟高校图书馆借阅对外开放的图书,目前已经有东南大学、南京师范大学等多所学校加入。\\
具体申请指南和服务内容请关注南京大学图书馆推文\url{https://mp.weixin.qq.com/s/_DKYPPnYqSWEMRflTa0ihw}\\
\section{玄武区产业链党建“青柠助理”岗位招募}
玄武团区委开展第二期玄武区产业链党建“青柠助理”选派行动,计划招募选派优秀硕(博)士研究生,担任玄武区产业链党建党务助理、产业实习助理。\\
招募条件、见习时间、岗位职责、管理保障、岗位需求、录用程序等请点击链接\url{https://mp.weixin.qq.com/s/hN8YupOfzb3qdHj1TaGbww}查看。\\
\section{南京大学献血周即将到来}
活动安排:\\
鼓楼校区:10月31日 9:00-16:00 逸夫馆报告厅\\
仙林校区:10月31日-11月1日 9:00-16:00 敬文活动中心多功能厅\\
\section{2024南京马拉松志愿者招募启动}
2024南京马拉松将于11月17日上午7:30鸣枪起跑。\\
志愿者招募信息:
活动地点:南京马拉松赛道9公里补给点(近中山南路与白下路十字路口)。\\
志愿招募对象:115名南京大学在读学生(鼓楼校区优先)。\\
培训时间:2024年11月初,具体时间待定,非特殊情况必须参加。\\
报名链接:\url{https://table.nju.edu.cn/dtable/forms/6c68c833-ec64-488b-8c20-7fffa3b377df/}\\
具体事项请阅读原文\url{https://mp.weixin.qq.com/s/um1feivhvwd-SYcci-Nu6Q}\\
报名截止时间:10月26日12:00\\
\section{排协院系杯赛程预告}
男排小组赛:地科-人文\\
时间:10月25日20:30\\
地点:方肇周副馆\\
\section{心理学小圆桌}
时间:10月26日13:30\\
地点:仙林校区敬文大学生活动中心9楼\\
讨论书目:《我们内心的冲突》\\
讨论话题:\\
1、内心冲突的来源\\
2、冲突与人际关系和人格的相互影响\\
3、内心冲突外化的表现\\
4、内心理想化形象的功能\\
报名方式:详见推文\url{https://mp.weixin.qq.com/s/MF58axnYQsOteFBz2r-Qog}
\section{信件盲盒第一期}
活动流程:本次书信采取盲盒的方式,报名参加的同学领取信封后,可以自己选择信纸,选择感兴趣的小主题,进行投递,一至两周后将截止收信,打乱后在次日再随机发给报名同学。同学们可以进行回信,在下一次主题收信时与新写的信件一起投递,一共投递两封信。\\
第一期主题:“薄荷时刻”,“时空隧道”,“不说永远,在每个瞬间”,“世界疏离而繁杂,谢谢你赋予我关于爱的想象”。\\
参与方式:有意愿参加的同学在10月25日-10月26日仙林大学生活动中心或鼓楼南青格庐填写报名表单,登记笔名,加入微信群,领取信封。\\
微信群聊详见推文\url{https://mp.weixin.qq.com/s/ksyoOgnn1Fxp9xljTjUo2Q}
\section{毓琇书院:“朋辈薪火”学业讲师团开课啦!}
“周末课堂”计划\\
开设的科目概览:\\
1.微积分一层次、二层次:常见题型讲解,经验方法分享,知识点梳理串联\\
2.C语言:基础语法夯实,编程思路开拓,思路拓展\\
3.Python:学习基础语法和数据结构,握编程逻辑\\
4.普通物理学:探索力、热、光、电等自然现象背后的物理原理,以及物理原理背后的数学方法\\
5.电路分析:了解电子元件性质,分析理解基本电路\\
6.大学化学:走进微观世界,探索物质变化的奥秘\\
7.英语:提升英语听说读写能力,讲解四六级、网测等应试题型与技巧\\
时间安排:\\
10月26日(周六):14:00-16:00:C程序设计(一层次)(授课讲师:唐玺骋)常见考点回顾+编程实战\\
14:00-16:00:Python(授课讲师:袁翊硕)基础语法和数据结构+编程逻辑\\
14:00-16:00:普通物理(授课讲师:于小舒)核心概念总结+解题技巧\\
16:00-18:00:电路分析(授课讲师: 袁翊硕)关键公式讲解+实例分析\\
10月27日(周日):14:00-16:00:微积分I(一层次)(授课讲师:柯昌纬)真题回顾与讲解+知识点梳理\\
16:00-18:00:微积分I(二层次)(授课讲师:熊徐琢)真题回顾与讲解+知识点梳理\\
(有真题回顾哦!)\\
各课程具体地点请关注书院官方通知,我们期待着每一位热爱学习的你的加入!\\

\section{篮球明日赛程}
周五赛程(10.25)\\
男篮院系杯小组赛\\
工程vs文院 12:30-14:00\\
海外vs计科 19:00-20:30\\
地点:一组团篮球场\\
研究生男篮院系杯小组赛\\
建城vs生科 12:30-13:30\\
地点:一组团篮球场

\end{multicols}
\hrule
\vspace{4mm}
\pagebreak
\centerline{\huge\textbf{参考消息}}
\begin{multicols}{2}
\section{“Free Hug”人士现身教一西南门}
本消息编辑部成员发现,今日18:00左右,有一名女生持写有“Free Hug”字样的标牌站立教一西南门。本报特别邀请了两位读者发表感想,同时也邀请各位读者发表感想。\\

特约评论员  孔 丘\\

自猎得麒麟,迩来二千多年。今有FreeHug现世,或惑于伪经,谓之不合礼法,此说巨谬。礼乐所以约束者,自己也。施诸他人,去礼而谬之远矣。少正卯尝欲为,乃诛之。嗟嗟!道之不行者,我知之矣。修齐治平之事,尚不能为,而诘他人,谓我为冠带之民,灵秀所踵,孔孟遗教,克明峻德。何也?摇手触禁,万马齐喑,而谓我所遗教,三代以来,未之闻也。我谓天命之谓性,率性之谓道,修道之谓教,故君子常常顾諟天之明命,立而不倚,强哉矫。道者费而隐。所以为礼为乐者,君子之无所适而不自得也,未尝一日顾諟他人言行而诘之以为去道。今后若有如是者,非我徒子徒孙矣。\\

特约评论员  Paul of Tarsus\\

Our Lord in his flesh used to say that\\
Do not judge others\\
	So that you do not be judged.\\
And I used to say to the church of Rome, condemning and warning them of judging others, for they who judge others, will they not be judged, on the day of judgement? And judgement according to law is going to vanish. And they who teach others, do they teach themselves? So to judge is not my wish. Whereas should I give a comment on NJUQUIDNUNC, I will say that I see no transgression against the gospel, of which I am the apostle. Jesus uesd to give free hugs, in Jerusalem, in Galilee and in all Judah. Whoever he hugs was healed of affliction. So much for the comment, and see you next time.

\end{multicols} 
\end{document}