% HEAD BEGIN
\documentclass[letterpaper, 12pt]{article}
\newsavebox\colbbox
\usepackage{graphicx}
\usepackage{multicol}
\usepackage{anysize}
\usepackage{fontspec}
\usepackage[fontset=none]{ctex}
\usepackage{tabularx}
\usepackage{longtable}
\PassOptionsToPackage{hyphens}{url}
\usepackage[breaklinks=true, colorlinks=true]{hyperref}
\expandafter\def\expandafter\UrlBreaks\expandafter{\UrlBreaks\do\a\do\b\do\c\do\d\do\e\do\f\do\g\do\h\do\i\do\j\do\k\do\l\do\m\do\n\do\o\do\p\do\q\do\r\do\s\do\t\do\u\do\v\do\w\do\x\do\y\do\z\do\A\do\B\do\C\do\D\do\E\do\F\do\G\do\H\do\I\do\J\do\K\do\L\do\M\do\N\do\O\do\P\do\Q\do\R\do\S\do\T\do\U\do\V\do\W\do\X\do\Y\do\Z}
% \let\oldurl\url
% \renewcommand{\url}[1]{\begin{sloppypar}\oldurl{#1}\end{sloppypar}}
\setlength\columnsep{30pt}
\marginsize{30pt}{30pt}{10pt}{20pt}
\setmainfont{TeX Gyre Bonum}
\setCJKmainfont[BoldFont=Noto Serif CJK SC Bold, ItalicFont=FandolKai]{Noto Sans CJK SC}
\setlength{\parindent}{0cm}
% \setCJKmonofont{Noto Sans CJK SC}
\begin{document}
\begin{center}
    \Huge\textbf{南哪大专醒前消息}
\end{center}
\vspace{4mm}
\hrule
\renewcommand\tabularxcolumn[1]{m{#1}}
\begin{tabularx}{\textwidth}{>{\hsize.2\hsize}X>{\hsize.6\hsize}X>{\hsize.2\hsize}X}
    \begin{flushleft}
        2024.11.1\, No.105
    \end{flushleft}
    &
    \begin{center}
        \textit{“既有令名、复求寿考、可兼得乎。”}
    \end{center}
    &
    \begin{flushright}
        \textbf{南京市栖霞区}
    \end{flushright}
\end{tabularx}
\vspace{-3.5mm}
\hrule
\vspace{4mm}
% HEAD END
\centerline{\huge\textbf{活动预告}}
\begin{multicols}{2}
    \section{订阅方式和加入编辑部}  
编辑部招聘人才,用爱发电,工作轻松,详情可联系QQ:1329527951 客服小祥\\想订阅本消息或获取PDF版(便于查看超链接和往期),可加QQ群:\href{https://qm.qq.com/q/VXIW7fgsEe}{849644979}.
\section{Deadline Ongoing}
\setbox\colbbox\vbox{
\makeatletter\col@number\@ne
\begin{longtable}{|c|c|c|}
    \hline
    消息(未见ddl的,不刊) & 截止日期 & 刊载日期\\
    \hline\hline
    紫藤学刊征稿 & 12.15 & 10.22\\
    学位英语考试报名 & 11.3 & 10.17\\
    校运会 & 11.8 & 10.21\\
    后革命鲁迅研究征文 & 11.10 & 10.8\\
    大创训练计划申报 & 11.18 & 9.24\\
    招生宣传创意征集大赛 & 11.18 & 10.21\\ 
    EBSCO数据库检索大赛 & 11.20 & 10.3\\
    文院征稿 & 11.20 & 10.20\\
    乐跑 & 12.6 & 10.12\\
    国际访学计划申报 & 11.22 & 10.22\\
    百团大战 & 11.2 & 10.26\\
    仙林草地音乐节 & 11.3 & 10.27\\
    普通话测试网络报名 & 11.12 & 10.29\\
    全球学习交流展 & 11.4 & 10.29\\
    健雄捡秋活动 & 11.3 & 10.29\\
    南大演说家 & 11.9 & 10.30\\
    导游志愿者招募 & 11.3 & 10.30\\
    南大演说家报名 & 11.9 & 10.30\\
    对话苏童 & 11.2 & 10.30\\
    腾讯线下观影 & 11.3 & 10.30\\
    杜厦剧本杀 & 11.3 & 10.30\\
    秉文朋导分享会 & 11.3 & 10.31\\
    科创集市 & 11.2 & 10.31\\
    《乱世佳人》放映会 & 11.2 & 10.31\\
    物院飞盘活动 & 11.3 & 10.31\\
    心协乐跑 & 11.3 & 10.31\\
    读书午餐会报名 & 11.6 & 11.1\\
    南大会征募会设 & 11.15 & 11.1\\
    新生午餐会报名 & 11.4 & 11.1\\
    \hline
\end{longtable}
\unskip
\unpenalty
\unpenalty}\unvbox\colbbox
\end{multicols}
\hrule
\pagebreak
\begin{multicols}{2}

\section{讲座}
\begin{tabular}{|c|c|c|}
    \hline
    往期讲座 & 开展日期 & 刊载日期\\
    \hline\hline
    《电池及电化学能...》 & 11.24 & 10.3\\
    《专利查新与规避...》 & 12.19 & 10.3\\
    图书馆系列讲座 & 12.3 & 10.20\\
    《志工人力资源的...》 & 11.4 & 10.23\\
    《华人社会工作的...》 & 11.4 & 10.23\\
    《从全球视角探讨...》 & 11.4 & 10.28\\
    《瑞典电力和氢能...》 & 11.7 & 10.29\\
    《组织动员如何影...》 & 11.6 & 10.30\\
    《信息与现代信息...》 & 11.6 & 10.31\\
    
    \hline
\end{tabular}

1.地院||“面向新质生产力的关键矿产资源研究”学术创新论坛\\               
时间:11月2日 9:00-16:30\\
地点:朱共山楼125;205\\  

2.管理科学论坛\\
题目:如何成功运用质性研究方法\\
主讲人:Giorgio Locatelli 米兰理工大学\\
时间:11月8日 14:00-15:30\\
地点:协鑫楼108\\
详情见\url{https://mp.weixin.qq.com/s/MIGPKOZTICSLU3vAvcj9YA}\\

3.数学学院本科生论坛(教师系列第86讲)\\
题目: 关于partly smoothness研究介绍\\
报告人:陶敏\\
时间:11月06日(星期三)16:00-17:30\\
地点:戊己庚四楼北\\
腾讯会议:870-7007-3326\\
摘要:  本讲座将深入浅出介绍一类非凸稀疏优化问题,从问题的研究背景、算法设计、收敛性分析、以及如何挖掘算法的变分性质、进而设计更高效的算法等角度展开介绍。\\



\section{2025年南大会征募会设}
南京大学学生模拟联合国协会现通知会设征募事宜,目前待定2个中文常规委员会和1个英文常规委员会。\\
申请截止时间:2024年11月15日23:59\\
详见\url{https://mp.weixin.qq.com/s/1LUXhyToYfG-lQCNmDe6Pw}

\section{SIECA Welcome Party}
南大学生国际交流与合作协会(NJUSIECA)发布\\
2024年度迎新party公告\\
活动时间:2024年11月3日\\
活动地点:敬文学生活动中心多功能厅\\
参加活动或可领取礼品\\
详情请查看\url{https://mp.weixin.qq.com/s/_ecAjTKV5utiZSmNUxnUJA}
\section{2024南大演说家}
南京大学党委学生工作部主办,商学院、行知书院、苏州校区党群团工作办公室承办“立报国强国大志向,做挺膺担当奋斗者”2024年南大演说家主题演讲比赛。\\
比赛形式:设置“科技自强”与“文化自信”主题,分为仙林赛区、鼓楼赛区、苏州赛区三个赛区,决赛在仙林校区统一举办。三个赛区独立评审,采取相同的赛制与评分标准。\\
报名对象:南京大学全体全日制在校生\\
报名方式:选手可通过院系推荐(每个院系、书院推荐2名)或个人报名的方式参与,并通过线上方式提交稿件及个人演讲视频。报名视频截止时间为2024年11月9日24:00\\
详情见:\url{https://mp.weixin.qq.com/s/LdcGQP_1ZopIiH_WVQ3Q1Q}\\
报名链接:\url{https://table.nju.edu.cn/dtable/forms/41e3e14a-3ded-403d-980b-b61328f1026f/}
\section{午餐读书会旁听报名}
活动时间:2024年11月6日(周三)中午12:15-13:40\\
地点:鼓楼校区南大出版社党建活动室\\
流程:两位同学发言各15分钟,提问讨论25分钟,老师评论导读30分钟(本次活动向参加读书会活动的全体老师同学提供午餐)\\
阅读书目:

1.福柯《规训与惩罚》\\
2.西蒙东《技术物的存在模式》\\
指导老师:蓝江 哲学学院\\
说明:本活动只面向南大校内师生开放,未报名者不得参与,活动当日需签到。
20个名额抽签制,旁听报名见\url{https://mp.weixin.qq.com/s/YerEVagjOpjI60gCMxCKlA}
\section{“重写剧场史·南京”中国当代表演艺术文献开放展}
展出开放时间\\
2024年10月28日至12月28日,每周一至周五,10:00-17:00(16:30停止入场)\\
展出地点\\
南京大学星云楼一楼展厅\\
观展预约方式详见\url{https://mp.weixin.qq.com/s/izZ5trO78E8ceMHe1gKN-Q}\\

\section{高研院“新生午餐会” 第43场}
题目:在江南水师-陆师学堂与鲁迅相遇\\
谈话人:沈卫威 南京大学文学院教授\\
主持人:郭芳洁 艺术学硕士 南京大学高研院学术秘书\\
时间:2024年11月6日12:20-13:20\\
地点:鼓楼校区逸夫馆9楼高研院报告厅\\
抽签开始时间:11月4日中午12:30\\
抽签结束时间:11月5日中午12:30\\
人数限额为30人,先到先得。\\
具体链接:\url{https://mp.weixin.qq.com/s/TJrlDWZVemJuuJfa_fSarA}\\

\end{multicols} 


\end{document}