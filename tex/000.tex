% HEAD BEGIN
\documentclass[letterpaper, 12pt]{article}
\newsavebox\colbbox
\usepackage{graphicx}
\usepackage{multicol}
\usepackage{anysize}
\usepackage{fontspec}
\usepackage[fontset=none]{ctex}
\usepackage{tabularx}
\usepackage{longtable}
\PassOptionsToPackage{hyphens}{url}
\usepackage[breaklinks=true, colorlinks=true]{hyperref}
\expandafter\def\expandafter\UrlBreaks\expandafter{\UrlBreaks\do\a\do\b\do\c\do\d\do\e\do\f\do\g\do\h\do\i\do\j\do\k\do\l\do\m\do\n\do\o\do\p\do\q\do\r\do\s\do\t\do\u\do\v\do\w\do\x\do\y\do\z\do\A\do\B\do\C\do\D\do\E\do\F\do\G\do\H\do\I\do\J\do\K\do\L\do\M\do\N\do\O\do\P\do\Q\do\R\do\S\do\T\do\U\do\V\do\W\do\X\do\Y\do\Z}
% \let\oldurl\url
% \renewcommand{\url}[1]{\begin{sloppypar}\oldurl{#1}\end{sloppypar}}
\setlength\columnsep{30pt}
\marginsize{30pt}{30pt}{10pt}{20pt}
\setmainfont{TeX Gyre Bonum}
\setCJKmainfont[BoldFont=Noto Serif CJK SC Bold, ItalicFont=FandolKai]{Noto Sans CJK SC}
\setlength{\parindent}{0cm}
% \setCJKmonofont{Noto Sans CJK SC}
\begin{document}
\begin{center}
    \Huge\textbf{南哪大专醒前消息}
\end{center}
\vspace{4mm}
\hrule
\renewcommand\tabularxcolumn[1]{m{#1}}
\begin{tabularx}{\textwidth}{>{\hsize.2\hsize}X>{\hsize.6\hsize}X>{\hsize.2\hsize}X}
    \begin{flushleft}
        2024.2.14\, No.167
    \end{flushleft}
    &
    \begin{center}
        \textit{“秉中持正、求新博闻。”}
    \end{center}
    &
    \begin{flushright}
        \textbf{南京市栖霞区}
    \end{flushright}
\end{tabularx}
\vspace{-3.5mm}
\hrule
\vspace{4mm}
% HEAD END
\centerline{\huge\textbf{活动预告}}
\begin{multicols}{2}
    \section{订阅方式和加入编辑部}  
编辑部招聘人才,用爱发电,工作轻松,详情可联系QQ:1329527951 客服小祥\\想订阅本消息或获取PDF版(便于查看超链接和往期),可加QQ群:\href{https://qm.qq.com/q/VXIW7fgsEe}{849644979}.
\section{Deadline Ongoing}
\setbox\colbbox\vbox{
\makeatletter\col@number\@ne
\begin{longtable}{|c|c|c|}
    \hline
    消息(未见ddl的,不刊) & 截止日期 & 刊载日期\\
    \hline\hline
    南大博物馆展览 & 6.16 & 12.17\\
    ASC25报名 & 2.21 & 1.6\\
    天文台开放日 & / & 1.6\\
    开甲书院科研作坊 & 2.17 & 1.6\\
    原创剧本联合孵化报名 & 3.20 & 1.10\\
    阅读分享活动征稿 & 3.7 & 1.10\\
    njumun代表报名 & 3.2 & 1.16\\
    大气院寒假打卡 & 2.16 & 1.20\\
    毓秀文创 & 2.20 & 2.6\\
    计院摄影 & 2.16 & 2.6\\
    数院捐书 & 2.16 & 2.6\\
    生科论文沙龙 & 2.22 & 2.6\\
    地海训练营 & 2.21 & 2.14\\
    健雄摄影征集 & 2.15 & 2.14\\
    返校注册 & 2.23 & 2.14\\
    课程补退选 & 2.17 & 2.14\\
    
    \hline
\end{longtable}
\unskip
\unpenalty
\unpenalty}\unvbox\colbbox
\end{multicols}
\hrule
\pagebreak
\begin{multicols}{2}

\section{讲座}
\begin{tabularx}{0.5\textwidth}{|X|X|X|}
    \hline
    讲座 & 开展时间 & 刊载时间\\
    \hline\hline
基于ART的面向对象软件测试方法探讨 & 2.15 & 2.14\\\hline
適應隨機勝隨機 & 2.16 & 2.14\\\hline
\end{tabularx}

1.基于ART的面向对象软件测试方法探讨\\
陈锦富 教授 江苏大学计算机学院\\
时间:2月15日(星期六)9:30\\
地点:计算机科学技术楼112室\\
讲座摘要及主讲人简介见\url{https://mp.weixin.qq.com/s/rDOrWxivAVucSzFR7j91CQ}\\

2.適應隨機勝隨機\\
陈宗岳 教授 Swinburne University of Technology\\
时间:2月16日(星期日)10:30\\
地点:计算机科学技术楼112室\\
讲座摘要及主讲人简介见\url{https://mp.weixin.qq.com/s/zQe0dxQpEBUpsCokcKipkg}\\
\section{2025年春季学期本科教务事项通知}


- 返校后办理报到注册(2月17日-23日)

- 课程补退选时间为2月17日13:30  - 课程开课第三周的周一上午7:00(除体育课外。请登录“本科选课平台”(https://xk.nju.edu.cn/)。补选采用“直选式”,提交成功后即选中这门课程。 开课3-8周仍然可以退选,但是成绩单上会有退课记录(开课第二周的周日24:00之后,退课操作都会有记录)。 

详见:\url{https://jw.nju.edu.cn/56/ad/c26263a743085/page.htm}\\
\section{南京大学公共体育课程学习须知(2025年2月)}
链接:\url{https://jw.nju.edu.cn/56/54/c26263a742996/page.htm}\\

\section{2025年春季学期补考报名通知}
报名时间:本学期补考申请通道于2月14日13:30开通,公共课(数学、英语、计算机、思政)补考申请截止时间为2月19日12:00;其他课程的补考申请截止时间为2月21日17:00。\\
链接:\url{https://jw.nju.edu.cn/56/58/c26263a743000/page.htm}\\

\section{【2024级本科生】2025年春季学期美育核心课选课通知}
补选方式:2月17日下午13:30起,登录选课平台→“公共”→“美育”模块进行补选\\
链接:\url{https://jw.nju.edu.cn/56/17/c26263a742935/page.htm}\\

\section{“地”职赋能训练营系列活动重磅启幕}
面向对象:地理与海洋科学学院2021、2022级本科生、2023、2024级硕士研究生以及各年级博士研究生(定向委培除外)。\\
报名截止时间:2025年2月21日\\
活动时间:2025年3月-2025年5月\\
报名方式:学生自主报名,填写《报名汇总表》(附件1)加入活动群聊,于2025年2月21日前将报名汇总表发送至njugeo\_job@163.com,邮件和文件均命名为“姓名+训练营报名”。\\
详见\url{https://mp.weixin.qq.com/s/hYSqgMUgQPF87ETQwbvczg}\\
\section{在校学生暴露后预防免费通道正式开放}
为有效降低艾滋病感染风险,即日起南京市疾病预防控制中心联合建邺区都安全门诊部,建立在校学生艾滋病病毒暴露后阻断免费通道,第一时间提供专业和及时的艾滋病病毒暴露后预防服务\\
详情见\url{https://mp.weixin.qq.com/s/Kj8otQtCQCpR2OsXgLX42g}\\
\section{青年科光 |《新一代集成光子技术》课程补报名}
【课程名额】10人\\
【选课要求】对集成光子学的前沿研究和产业有浓厚兴趣\\
【选拔形式】提交一份300-500字左右的选课陈述,简述对集成光子领域的认识,以及自己的相关知识储备。\\
【课程周次】第4-11周(周六上午9:30-11:30)\\
有意选修的同学,补报名详见\url{https://mp.weixin.qq.com/s/BTWFDBEe4hhF\_oOUt6kPtw}!\\
本次报名截止时间为:2月17日24点\\
\section{健雄书院 “新春气象” 摄影征集}
征集主题:各地过年的风俗人情和美食装饰\\
征集对象:全体健雄书院老师及同学\\
征集时间:即日起至 2025 年 2 月 15 日\\
要求:拍摄题材各地过年的风俗人情和美食装饰,拍摄时间为 2025 年寒假,提交作品大小应控制在 10MB 以内,提交照片数量上限为 3 张,每张照片必须附上命名,并添加相应的文字介绍。\\
投稿方式:在\url{https://box.nju.edu.cn/d/3710203840dc467e92b1/}中上传\\
奖励和其他事宜详见原文\url{https://mp.weixin.qq.com/s/ib2AY8WsKKTFM2Bq6CFoIg}


\end{multicols} 

\end{document}