% HEAD BEGIN
\documentclass[letterpaper, 12pt]{article}
\newsavebox\colbbox
\usepackage{graphicx}
\usepackage{multicol}
\usepackage{anysize}
\usepackage{fontspec}
\usepackage[fontset=none]{ctex}
\usepackage{tabularx}
\usepackage{longtable}
\PassOptionsToPackage{hyphens}{url}
\usepackage[breaklinks=true, colorlinks=true]{hyperref}
\expandafter\def\expandafter\UrlBreaks\expandafter{\UrlBreaks\do\a\do\b\do\c\do\d\do\e\do\f\do\g\do\h\do\i\do\j\do\k\do\l\do\m\do\n\do\o\do\p\do\q\do\r\do\s\do\t\do\u\do\v\do\w\do\x\do\y\do\z\do\A\do\B\do\C\do\D\do\E\do\F\do\G\do\H\do\I\do\J\do\K\do\L\do\M\do\N\do\O\do\P\do\Q\do\R\do\S\do\T\do\U\do\V\do\W\do\X\do\Y\do\Z}
% \let\oldurl\url
% \renewcommand{\url}[1]{\begin{sloppypar}\oldurl{#1}\end{sloppypar}}
\setlength\columnsep{30pt}
\marginsize{30pt}{30pt}{10pt}{20pt}
\setmainfont{TeX Gyre Bonum}
\setCJKmainfont[BoldFont=Noto Serif CJK SC Bold, ItalicFont=FandolKai]{Noto Sans CJK SC}
\setlength{\parindent}{0cm}
% \setCJKmonofont{Noto Sans CJK SC}
\begin{document}
\begin{center}
    \Huge\textbf{南哪大专醒前消息}
\end{center}
\vspace{4mm}
\hrule
\renewcommand\tabularxcolumn[1]{m{#1}}
\begin{tabularx}{\textwidth}{>{\hsize.2\hsize}X>{\hsize.6\hsize}X>{\hsize.2\hsize}X}
    \begin{flushleft}
        2024.12.27\, No.156
    \end{flushleft}
    &
    \begin{center}
        \textit{“秉中持正、求新博闻。”}
    \end{center}
    &
    \begin{flushright}
        \textbf{南京市栖霞区}
    \end{flushright}
\end{tabularx}
\vspace{-3.5mm}
\hrule
\vspace{4mm}
% HEAD END
\centerline{\huge\textbf{活动预告}}
\begin{multicols}{2}
    \section{订阅方式和加入编辑部}  
编辑部招聘人才,用爱发电,工作轻松,详情可联系QQ:1329527951 客服小祥\\想订阅本消息或获取PDF版(便于查看超链接和往期),可加QQ群:\href{https://qm.qq.com/q/VXIW7fgsEe}{849644979}.
\section{Deadline Ongoing}
\setbox\colbbox\vbox{
\makeatletter\col@number\@ne
\begin{longtable}{|c|c|c|}
    \hline
    消息(未见ddl的,不刊) & 截止日期 & 刊载日期\\
    \hline\hline
    安邦征稿 & 1.12 & 11.16\\
    仙林通宵自习室 & 1.12 & 11.26\\
    全国大学生家史大赛 & 1.31 & 12.2\\
    金融消费者大赛 & 12.31 & 12.5\\
    花旗杯报名 & 1.3 & 12.6\\
    西安史学论坛征稿 & 3.20 & 12.9\\
    本科评教 & 1.12 & 12.13\\
    排协雪球杯 & 12.28 & 12.13\\
    12306学生优惠票 & 2.12 & 12.13\\
    期末考试安排 & 1.12 & 12.17\\
    南大博物馆展览 & 6.16 & 12.17\\
    南悦支教团队长报名 & 12.29 & 12.19\\
    可一读书 & 12.28 & 12.22\\
    NEC观影会 & 12.28 & 12.24\\
    雄狮少年影映 & 12.28 & 12.25\\
    排超志愿者招募 & 1.16 & 12.25\\
    昆曲社曲会 & 12.29 & 12.26\\
    跨年电子音乐快闪 & 12.31 & 12.26\\
    南星小红书创作 & 2.6 & 12.27\\
    形象哲学工作坊 & 12.29 & 12.27\\
    
    \hline
\end{longtable}
\unskip
\unpenalty
\unpenalty}\unvbox\colbbox
\end{multicols}
\hrule
\pagebreak
\begin{multicols}{2}

\section{讲座}
\begin{tabularx}{0.5\textwidth}{|X|X|X|}
    \hline
    讲座 & 开展时间 & 刊载时间\\
    \hline\hline
无 & / & / \\\hline

\end{tabularx}





\section{南星十载·小红书创作燃计划活动}
参与方式:1月6日起,至2月6日24:00,以个人小红书账号为发布阵地,包括但不限于图文笔记、视频笔记等,关注“南大招生小蓝鲸”官方小红书账号,使用指定话题标签“2025南星梦想计划”发布相关笔记并@小蓝鲸官号,即视作参与活动。\\
招办全媒体中心将对参评笔记进行审核打分。经过评分后,招办全媒体中心将结合实际情况及综合分数,认定至多30篇笔记为“优秀笔记”,获评“优秀笔记”的南星团队及创作个体将获得活动专项奖励。包括:1.文创奖品:优秀笔记创作者(个人)将获得南大精美文创奖品。2.获得优秀笔记认定的南星团队将在最终评优答辩中获得额外分数:一篇优秀笔记获得1分,两篇优秀笔记获得1.5分,三篇及以上优秀笔记获得2分。\\
活动细节见原文:\url{https://mp.weixin.qq.com/s/XfR7qmWQQxPqvAEPEtLnOQ}

\section{排协雪球杯}
请通过以下链接加入微信群:\url{https://mp.weixin.qq.com/s/EuMQcLvk54PnWQCY6jgSPw}

\section{南大昆曲研习社新年曲会}
时间:2024年12月29日(周日)晚7:00\\
地点:仙林校区文学院二楼活水轩报告厅\\
无需报名和票据,到场参加即可。以曲会友,欢迎广大师生到场。\\
详情:\url{https://mp.weixin.qq.com/s/x4_1-II5DFPpy6IzSNq-zQ}\\


\section{篮协赛程}
明日(12.28):\\
本科生男篮总决赛\\
软院vs化生 \\
19:30-21:30\\
地点:四组团体育馆


\section{“形象哲学”青年学术工作坊}
1.工作坊时间:2024年12月29日 9:00-17:00\\
2. 工作坊地点:南京大学艺术学院 东大楼106\\
详情见\url{https://mp.weixin.qq.com/s/wMEpmb2d0NxDxoT8d6AENA}
\section{ “早期科研能力启蒙计划”项目}
项目对象及人数:南京大学行知书院2024级新生10人左右\\
开展形式:(1)本学期结束前,将以线上线下相结合形式组织调研启动会,明确假期任务和分工等准备工作。\\
(2)2025年春季学期开始,每隔2-3周的周三下午前往青岛路社区实地调研,共计调研8次左右。\\
(3)同时,召开调研讨论交流会,就调研方法、调研发现等进行交流讨论。最终形成一份符合学术规范的调研报告。\\
\url{https://mp.weixin.qq.com/s/vvr614LCJNxUiZJg_maT3Q}\\
\section{可一跨年演讲}
时间:2024年12月31日(周二) 下午15:30\\
主讲人:杜骏飞\\
题目:让教育回到常识\\
地点:可一书店·仙林艺术中心 负二层 可一实验剧场\\
当日还有其他活动,详情以及报名见原文:\url{https://mp.weixin.qq.com/s/J4OaMpBt26l1qTI_e8au9w}

\end{multicols} 
\end{document}