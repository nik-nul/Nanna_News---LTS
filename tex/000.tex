% HEAD BEGIN
\documentclass[letterpaper, 12pt]{article}
\newsavebox\colbbox
\usepackage{graphicx}
\usepackage{multicol}
\usepackage{anysize}
\usepackage{fontspec}
\usepackage[fontset=none]{ctex}
\usepackage{tabularx}
\usepackage{longtable}
\PassOptionsToPackage{hyphens}{url}
\usepackage[breaklinks=true, colorlinks=true]{hyperref}
\expandafter\def\expandafter\UrlBreaks\expandafter{\UrlBreaks\do\a\do\b\do\c\do\d\do\e\do\f\do\g\do\h\do\i\do\j\do\k\do\l\do\m\do\n\do\o\do\p\do\q\do\r\do\s\do\t\do\u\do\v\do\w\do\x\do\y\do\z\do\A\do\B\do\C\do\D\do\E\do\F\do\G\do\H\do\I\do\J\do\K\do\L\do\M\do\N\do\O\do\P\do\Q\do\R\do\S\do\T\do\U\do\V\do\W\do\X\do\Y\do\Z}
% \let\oldurl\url
% \renewcommand{\url}[1]{\begin{sloppypar}\oldurl{#1}\end{sloppypar}}
\setlength\columnsep{30pt}
\marginsize{30pt}{30pt}{10pt}{20pt}
\setmainfont{TeX Gyre Bonum}
\setCJKmainfont[BoldFont=Noto Serif CJK SC Bold, ItalicFont=FandolKai]{Noto Sans CJK SC}
\setlength{\parindent}{0cm}
% \setCJKmonofont{Noto Sans CJK SC}
\begin{document}
\begin{center}
    \Huge\textbf{南哪大专醒前消息}
\end{center}
\vspace{4mm}
\hrule
\renewcommand\tabularxcolumn[1]{m{#1}}
\begin{tabularx}{\textwidth}{>{\hsize.2\hsize}X>{\hsize.6\hsize}X>{\hsize.2\hsize}X}
    \begin{flushleft}
        2024.12.11\, No.141
    \end{flushleft}
    &
    \begin{center}
        \textit{“天下熙熙、皆为绩来、天下攘攘、皆为绩往。”}
    \end{center}
    &
    \begin{flushright}
        \textbf{南京市栖霞区}
    \end{flushright}
\end{tabularx}
\vspace{-3.5mm}
\hrule
\vspace{4mm}
% HEAD END
\centerline{\huge\textbf{活动预告}}
\begin{multicols}{2}
    \section{订阅方式和加入编辑部}  
编辑部招聘人才,用爱发电,工作轻松,详情可联系QQ:1329527951 客服小祥\\想订阅本消息或获取PDF版(便于查看超链接和往期),可加QQ群:\href{https://qm.qq.com/q/VXIW7fgsEe}{849644979}.
\section{Deadline Ongoing}
\setbox\colbbox\vbox{
\makeatletter\col@number\@ne
\begin{longtable}{|c|c|c|}
    \hline
    消息(未见ddl的,不刊) & 截止日期 & 刊载日期\\
    \hline\hline
    紫藤学刊征稿 & 12.15 & 10.22\\
    安邦征稿 & 1.12 & 11.16\\
    猫鼠大战 & 12.15 & 11.22\\
    创意物理实验竞赛 & 12.21 & 11.15\\
    仙林通宵自习室 & 1.12 & 11.26\\
    防艾同伴教育 & 12.15 & 11.29\\
    南京高校戏曲交流 & 12.15 & 12.2\\
    全国大学生家史大赛 & 1.31 & 12.2\\
    女性劳动策展工坊 & 12.16 & 12.4\\
    防艾剧本杀 & 12.15 & 12.5\\
    金融消费者大赛 & 12.31 & 12.5\\
    药丸周边征稿 & 12.15 & 12.6\\
    四六级准考证打印 & 12.14 & 12.6\\
    花旗杯报名 & 1.3 & 12.6\\
    挑杯校园双选会 & 12.15 & 12.7\\
    朋辈数模分享 & 12.15 & 12.7\\
    羊山公园环保市集 & 12.15 & 12.7\\
    C Through代码大赛 & 12.15 & 12.8\\
    教师实习经验分享会 & 12.14 & 12.9\\
    澳门回归对谈会 & 12.15 & 12.9\\
    西安史学论坛征稿 & 3.20 & 12.9\\
    歌魅放映 & 12.14 & 12.10\\
    南苏新年晚会 & 12.15 & 12.10\\
    遇难同胞纪念活动 & 12.13 & 12.10\\
    民族文化分享会 & 12.12. & 12.10\\
    鼓楼旧书市集 & 12.14 & 12.10\\
    重唱英文评诗会 & 12.21 & 12.10\\
    叶嘉莹纪念征稿 & 12.25 & 12.10\\
    新新节闭幕晚会 & 12.15 & 12.11\\
    新火星观影会 & 12.15 & 12.11\\
    中医中药高校行 & 12.15 & 12.11\\
    冷波音乐节 & 12.15 & 12.11\\
    粤语课堂 & 12.22 & 12.11\\
    南说喜剧 & 12.15 & 12.11\\
    莫里斯作品展 & 12.15 & 12.11\\
    抗战博物馆参观 & 12.13 & 12.11\\
    新传保研分享会 & 12.13 & 12.11\\
    \hline
\end{longtable}
\unskip
\unpenalty
\unpenalty}\unvbox\colbbox
\end{multicols}
\hrule
\pagebreak
\begin{multicols}{2}

\section{讲座}
\begin{tabular}{|c|c|c|}
    \hline
    往期讲座 & 开展日期 & 刊载日期\\
    \hline\hline
    《现代化进程中的...》 & 12.17 & 12.11\\
    《英格兰的有偿护...》 & 12.13 & 12.11\\
    《诗到语言为止》& 12.14 & 12.11\\
    《全球视野中的加...》& 12.13 & 12.10\\
    《专利查新与规避...》 & 12.19 & 10.3\\
    《waveguide...》 & 12.12 & 12.9\\
    《读雷吉斯德布雷...》 & 12.13 & 12.9\\
    《GPU coroutines...》 & 12.12 & 12.10\\
    《前向推理的一阶...》 & 12.18 & 12.10\\
    《Probability...》 & 12.13 & 12.11\\
    《雨果的中国变形记》 & 12.14 & 12.11\\
    \hline
\end{tabular}

1.郑钢社会学人讲座第三十一期\\
讲座题目:现代化进程中的中国居民地位不一致问题研究\\
主讲人:龚顺 中国社会科学院社会学研究所副研究员\\
时间:12月17日(周二)14:00-16:00\\
地点:仙林社会学院河仁楼401室\\
讲座摘要:研究将围绕居民地位不一致的影响因素及其社会后果方面展开讨论。影响因素方面主要关注了住房金融化河中国市场化转型的制度背景;社会后果方面则尝试分析地位不一致对解释中国居民“消费不足”问题及社会阶层认同的作用。\\
2.孙本文社会学论坛第291期\\
讲座题目:英格兰的有偿照护工作者组织化:优先事项、障碍与进展?\\
主讲人:Liam Foster 英国谢菲尔德大学社会政策教授、CIRCLE联合主任及教育主管\\
时间:12月13日(周五)14:00-16:00\\
地点:腾讯会议:972-527-7232\\
讲座摘要:基于一些相关研究,讲座将介绍“照护中心”的工作及其相关研究,并提及“照护”在关于英国性别与养老金规划的进一步研究中的作用。\\


3.Probability Limit Theory: From Classical to Self-normalized\\
报告人:邹启满 教授 (南方科技大学)\\
时间:2024年12月13日 10:30\\
地点:鼓楼校区西大楼308\\
讲座摘要等见\url{https://mp.weixin.qq.com/s/lwQjx4VF1uzV1V18m8s3sA}\\

4.诗到语言为止\\
主讲:叶兆言 当代作家\\
主持:张光芒 南京大学中国新文学研究中心执行主任、教授\\
时间:12月14日(星期六)10:00-12:00\\
地点:文学院活水轩\\

5.可一人文讲座丨雨果的中国“变形记”\\
可一书店 发布\\
主讲人:吴天楚\\
时间:2024年12月14日(周六)下午14:30\\
地址:江苏省南京市栖霞区杉湖东路9号——可一书店·仙林艺术中心负二楼——可一书店·可一实验剧场\\
到场参与活动或可得赠书\\
详情\url{https://mp.weixin.qq.com/s/yJNik1Uz-tu-jvGl5QNCGg}
\section{“新新节”闭幕晚会倒计时}
南京大学第三届“新新节”闭幕晚会倒计时5天\\
演出时间:12月15日 19:00\\
演出地点:大礼堂\\
派票信息注意各书院通知\\
兑奖规则及活动详情:\url{https://mp.weixin.qq.com/s/UD2T0iBzHVOxU0Z63CsPew}


\section{新火星观影会}
影片:《出租车司机》(2017 韩国)\\
时间:12月15日(本周日)18:30\\
地点:鼓楼校区费A410(暂定)\\
观影群:907939564\\
剧情简介:本片讲述了出租车司机金万燮和记者彼得在1980年光州事件中的经历\\
观影后另有可选择参与的讨论环节

\section{南京大学2024年“大学生国家安全教育”考试提醒}
在线学习阶段结束时间:12月15日23点59分\\
考试开始时间:12月16日8:00\\
考试结束时间:12月22日23:59\\
具体链接:\url{https://jw.nju.edu.cn/21/2d/c26263a729389/page.psp}\\
\section{实验室环游记 | 12月17日移动站、12月19日华为站,双线以待,等你参观!}
报名截止时间:12月12日18:00\\
具体链接:\url{https://mp.weixin.qq.com/s/X-3VkWBEfSRgKJowp_SZlg}

\section{中医中药高校行}
活动时间:2024年12月15日14:30-17:00\\
活动地点:仙林校区 仙1-106\\
义诊将包括如下项目:\\
1. 养生互动与中医宣讲 2. 健康咨询 3. 推拿按摩 4. 中医内科 5. 中医妇科 6. 耳穴埋籽 7. 中医药茶 8. 血压测量\\
可扫码加群了解详情,详见\url{https://mp.weixin.qq.com/s/Ih6YgIZXuu5_jXiVicB6dw}

\section{关于2024-2025学年高等教育人工智能通识课学习的通知}
课程安排:本期人工智能通识课共 12 课时,自2024年12月6日起,每周一、三、五下午14:00-14:50。\\
同学们完成学习后,请于2025年1月12日前填写课程评价反馈表。\\
具体链接:\url{https://jw.nju.edu.cn/21/d0/c26263a729552/page.htm}\\

\section{冷波音乐节}
ROCK REPUBLIC南大摇滚联盟 发布\\
时间:12月15日17:00\\
地点:黑匣子剧场\\
详情\url{https://mp.weixin.qq.com/s/D4VxvQpEcHKiYXeF-DOH2A}\\

\section{粤知多少 |系定係定喺\&粤语课堂}
时间:2024年12月22日 15:00\\
地点:南京大学仙林校区(具体地点请关注QQ群后续通知)\\
粤语科普、活动地点详见推文。\\
原文\url{https://mp.weixin.qq.com/s/Q6OkAS839sfeRPBfYAG91Q}\\
(消息编辑:金映樺)

\section{排超 | 南京广电猫猫vs福建平潭}
时间:2024年12月12日 周四 19:30\\
地点:南京大学方肇周体育馆\\
*转发推文至朋友圈集赞十个,可参与抽奖\\
开奖时间:2024年12月12日 周四19:00\\
详见原文\url{https://mp.weixin.qq.com/s/y5QRpaTje8Jcutj8YEyKog}\\
(消息编辑:金映樺)

\section{南说喜剧 第五次开放麦}
时间:12月15日(周日)晚 19:00\\
地点:敬文学生活动中心(大活)九楼\\
报名及详情:\url{https://mp.weixin.qq.com/s/jJ5gWL2tNwL6-dkK9hPhAw}\\

\section{数学学院师生拔河赛}
比赛时间:12月26日下午14:00-16:00\\
比赛地点:鼓楼校区苏浙体育场\\
参赛对象:数学学院全体师生\\
鼓励大家按导师课题组组队报名(也可以考虑自行组队以及个人报名后续分配队伍等报名方式)。\\
拔河比赛一队7人,4男3女。(按照报名情况酌情调整)\\
奖品包括充电宝、抱枕等。\\
报名方式和比赛流程等见原文\url{https://mp.weixin.qq.com/s/DTMQEGYBt_ID4o7fOIR_sw}\\


\section{“政管杯”公务员面试模拟大赛决赛}
南京大学首届“政管杯”公务员面试模拟大赛收官在即,有志愿参加公务员招录的朋友们,如果你想要了解结构化面试流程,如果你对准备面试顾虑重重,抓住最后机会,“政管杯”公务员面试模拟大赛决赛观众报名进入倒计时啦!\\
观赛福利:\\
前100名到场的观众可领取考公教材!
现场答题还有奖品,快来参与互动吧!\\
报名方式及详情:\url{https://mp.weixin.qq.com/s/cA_bnhK8TwP-Wrc8I4D5iA}\\

\section{师说 | 解码焦虑}
嘉宾:罗小男老师\\
南京大学心理健康教育与研究中心办公室主任、专职教师,南京大学学生心理协会指导老师\\
时间:12月14日(周六) 16:00-18:00\\
地点:南京大学仙林校区敬文学生活动中心活动房209\\
流程:嘉宾分享、现场对谈、自由交流、合影留念\\
报名二维码\&QQ群见原文:\url{https://mp.weixin.qq.com/s/5Abu8F1lcIAiDCGv3EaeZQ}\\
(消息编辑:Wheelrunner)

\section{SRTP讲座整理}
12.12 周四:\\
1.Waveguide Quantum Electrodynamics: manipulating theinteraction of photons and atoms\\
12.13 周五:\\
2.追寻铁器时代欧亚游牧民族的踪迹:对中亚公元前一千年墓葬中马牙釉质的同位素分析\\
详见\url{https://mp.weixin.qq.com/s/-KM2D-v39_O5cEOV8y5LKw}

\section{莫奔时刻——Maurice Benayoun作品回顾展}
开幕仪式:2024年12月15日(星期日)\\
时间:2024年12月15日—2025年1月13日\\
地点:南京市汉口路22号南京大学鼓楼校区东大楼展厅\\

\section{南大高研院“古文字学前沿理论讲疏”专题研讨班}
主题:古文字学前沿理论讲疏\\
主讲人:蒋文 复旦大学出土文献与古文字研究中心副研究员 南京大学高研院2024年度访问学者\\
主持人:程少轩 南京大学文学院教授南京大学高研院特聘教授\\
时间:【第一讲】2024年12月13日(周五)上午10:00-11:30\\
【第二讲】2024年12月20日(周五)上午10:00-11:30\\
【第三讲】2024年12月20日(周五)下午14:00-15:30\\
地点:南京大学仙林校区文学院433室\\

\section{“铭记历史,珍爱和平”南京民间抗日战争博物馆参观活动}
时间:2024年12月13日 14:30\\
地点:南京市雨花台区安德门大街48号\\
报名详见\url{https://mp.weixin.qq.com/s/jbl4hYXbPzHB1PK-Awe8cA}\\

\section{新媒体中心非遗工艺系列第一期}
活动内容:动手制作羊毛戳戳钥匙扣挂件\\
活动时间2024/12/14(周六)14:00 \\
活动地点 南京大学鼓楼校区艺术学院东大楼208教室\\ 
参与人数:20人\\
报名方式\url{https://mp.weixin.qq.com/s/jmD3aAx1BIswKW8iSCVuug}(消息编辑:西野明日风)\\

\section{新传保研分享会}
为回答新传院广大学子对保研的种种问题,12月13日(本周五)中午12:20-13:50,新传团学联将在新传院215室举办“研”途璀璨:保研经验分享会,将有四位学长学姐前来分享。详阅:\url{https://mp.weixin.qq.com/s/AdhGR4OV37bdas4fDHMpxQ}。(消息编辑/小喇叭)

\section{预告|安邦暖冬“巧手筑梦空间”}
内容:制作扭扭棒手工、神秘材料装饰教室等\\
活动时间:2024年12月14日 15:00-17:00\\
活动地点:鼓楼校区 新教208\\
报名详见\url{https://mp.weixin.qq.com/s/QWiddpo6bzRFLT8d1vxnlw}(消息小编:草昌思)

\end{multicols} 

\end{document}