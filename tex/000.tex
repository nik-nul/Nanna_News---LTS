% HEAD BEGIN
\documentclass[letterpaper, 12pt]{article}
\newsavebox\colbbox
\usepackage{graphicx}
\usepackage{multicol}
\usepackage{anysize}
\usepackage{fontspec}
\usepackage[fontset=none]{ctex}
\usepackage{tabularx}
\usepackage{longtable}
\PassOptionsToPackage{hyphens}{url}
\usepackage[breaklinks=true, colorlinks=true]{hyperref}
\expandafter\def\expandafter\UrlBreaks\expandafter{\UrlBreaks\do\a\do\b\do\c\do\d\do\e\do\f\do\g\do\h\do\i\do\j\do\k\do\l\do\m\do\n\do\o\do\p\do\q\do\r\do\s\do\t\do\u\do\v\do\w\do\x\do\y\do\z\do\A\do\B\do\C\do\D\do\E\do\F\do\G\do\H\do\I\do\J\do\K\do\L\do\M\do\N\do\O\do\P\do\Q\do\R\do\S\do\T\do\U\do\V\do\W\do\X\do\Y\do\Z}
% \let\oldurl\url
% \renewcommand{\url}[1]{\begin{sloppypar}\oldurl{#1}\end{sloppypar}}
\setlength\columnsep{30pt}
\marginsize{30pt}{30pt}{10pt}{20pt}
\setmainfont{TeX Gyre Bonum}
\setCJKmainfont[BoldFont=Noto Serif CJK SC Bold, ItalicFont=FandolKai]{Noto Sans CJK SC}
\setlength{\parindent}{0cm}
% \setCJKmonofont{Noto Sans CJK SC}
\begin{document}
\begin{center}
    \Huge\textbf{南哪大专醒前消息}
\end{center}
\vspace{4mm}
\hrule
\renewcommand\tabularxcolumn[1]{m{#1}}
\begin{tabularx}{\textwidth}{>{\hsize.2\hsize}X>{\hsize.6\hsize}X>{\hsize.2\hsize}X}
    \begin{flushleft}
        2025.2.17\, No.168
    \end{flushleft}
    &
    \begin{center}
        \textit{“秉中持正、求新博闻。”}
    \end{center}
    &
    \begin{flushright}
        \textbf{南京市栖霞区}
    \end{flushright}
\end{tabularx}
\vspace{-3.5mm}
\hrule
\vspace{4mm}
% HEAD END
\centerline{\huge\textbf{活动预告}}
\begin{multicols}{2}
    \section{订阅方式和加入编辑部}  
编辑部招聘人才,用爱发电,工作轻松,详情可联系QQ:1329527951\\想订阅本消息或获取PDF版(便于查看超链接和往期),可加QQ群:\href{https://qm.qq.com/q/VXIW7fgsEe}{849644979}.
\section{Deadline Ongoing}
\setbox\colbbox\vbox{
\makeatletter\col@number\@ne
\begin{longtable}{|c|c|c|}
    \hline
    消息(未见ddl的,不刊) & 截止日期 & 刊载日期\\
    \hline\hline
    南大博物馆展览 & 6.16 & 12.17\\
    ASC25报名 & 2.21 & 1.6\\
    天文台开放日 & / & 1.6\\
    原创剧本联合孵化报名 & 3.20 & 1.10\\
    阅读分享活动征稿 & 3.7 & 1.10\\
    njumun代表报名 & 3.2 & 1.16\\
    毓秀文创 & 2.20 & 2.6\\
    生科论文沙龙 & 2.22 & 2.6\\
    地海训练营 & 2.21 & 2.14\\
    健雄摄影征集 & 2.15 & 2.14\\
    返校注册 & 2.23 & 2.14\\
    课程补退选 & 2.17 & 2.14\\
    DIY课程报名 & 2.22 & 2.17\\
    \hline
\end{longtable}
\unskip
\unpenalty
\unpenalty}\unvbox\colbbox
\end{multicols}
\hrule
\pagebreak
\begin{multicols}{2}

\section{讲座}
\begin{tabularx}{0.5\textwidth}{|X|X|X|}
    \hline
    讲座 & 开展时间 & 刊载时间\\
    \hline\hline
Unconventional magnetism & 2.20 & 2.17\\\hline
人工微结构声学材料 & 2.20 & 2.17\\\hline
人机协同背景下高等外语教育的守正创新 & 2.27 & 2.17\\\hline
大陆的起源 & 3.4 & 2.17\\\hline
单杏花先进事迹宣讲会 & 2.20 & 2.17\\\hline
维尔茨堡大学交流项目宣讲 & 2.19 & 2.17\\\hline
\end{tabularx}

1.物理学院学术报告会(第49期)\\
题 目:Unconventional magnetism\\
报告人:Congjun Wu(吴从军),Department of Physics, Westlake University\\
时 间:2025年2月20日(周四)15:30\\
地 点:鼓楼校区唐仲英楼B501\\
直播链接:\url{https://www.koushare.com/live/details/40699}\\
讲座摘要见原文\url{https://mp.weixin.qq.com/s/j7LNOt1xfUDPypvqv2BrLA}\\

2.人工微结构声学材料:智能制造新质生产力塑造未来产业发展新动能

主讲人:卢明辉,南京大学现代工程与应用科学学院材料科学与工程系教授。

2月20日周四 19:30-21:00 南大“敦学堂”移动端\\
详见:\url{https://mp.weixin.qq.com/s/odAZCNbP1Cc_RCeQRmXumQ}\\

3.人机协同背景下高等外语教育的守正创新

张俊翔,南京大学外国语学院教授。

2月27日(周四)晚19:30-21:00 南京大学“暾学堂”移动端\\
详见:\url{https://mp.weixin.qq.com/s/odAZCNbP1Cc_RCeQRmXumQ}\\

4.大陆的起源

3月4日(周二)晚19:30-21:00 南京大学“暾学堂”移动端

葛荣峰,南京大学地球科学与工程学院教授。

详见:\url{https://mp.weixin.qq.com/s/odAZCNbP1Cc_RCeQRmXumQ}\\

5.“时代楷模”单杏花同志先进事迹报告会

2月20日(周四)15:30-16:30  仙林校区图书馆报告厅 

报告团成员:路云军(铁科院集团公司党委副书记)、王明哲(铁科院集团公司电子所所长)、许媛媛(人民铁道报业公司记者)、武晋飞(铁科院集团公司电子所副研究员)、单杏花(“时代楷模”、铁科院集团公司首席研究员)

简介:向全校师生讲述单杏花先进事迹。

详见:\url{https://mp.weixin.qq.com/s/lgk4QGViPGAcdBmpC0Zr_A}\\

6.德国维尔茨堡大学学生交流项目宣讲

2月19日 16:00-17:00 仙林校区行政北楼809会议室

宣讲嘉宾:Prof. Dr. Hans Fehr(Current Research:Economics of Demographic Transition、Tax and Pension Reforms in Dynamic CGE Models、Computational Economics。)
\section{DIY研读研究课程报名}
1.本学期开设13门DIY研读研究课程\\
2.各课程大纲:\url{https://box.nju.edu.cn/d/3ac740ffb75849a28e61/}\\
3.报名方式:通过“灯下漫谈”公众号推送中的二维码,或南大表格链接:\url{https://table.nju.edu.cn/dtable/forms/1201d55d-5291-4987-8059-e4e3cd8fcdb8/},填写电子报名表进行报名。\\
4.报名截止日期:2025年2月22日(周六)24:00\\
5. 每位同学最多报名三门DIY课程。\\
课程详情及注意事项见推文:\url{https://mp.weixin.qq.com/s/V766M64BzGDrNdqqlhAKMQ}

\section{NJUNMUNC 2025 | 报名进行时}
2025年南京大学全国模拟联合国大会代表报名和学测提交截止时间均为北京时间3月2日22:00。\\
原文:\url{https://mp.weixin.qq.com/s/v5ejwxdb3VDDaMERA0PMYQ}

\section{关于教学立方平台资料的备份提醒}
(此通知原文意在面向老师)\\
因学校不再续购教学立方的技术支持,各位老师存在平台上的资料,请务必下载备份。

\section{南京大学密码管理服务上线}
南京大学密码管理服务(\url{https://pass.nju.edu.cn})即日起正式上线。具体功能和使用方法可参见网站中的教程或微信文章\url{https://mp.weixin.qq.com/s/FSPlG1kMA1UWeu1bX3X_pA}
\section{eScience小助手测试版上线}
eScience小助手测试版即日起上线,可询问有关eScience服务使用的问题,或作为大语言模型询问其他一般性的问题。可在\url{https://appcenter.bigmodel.cn/console/appcenter_v2/chat?share_code=Dr_lokanmpc86cxIVdEQd}进行测试。详见\url{https://mp.weixin.qq.com/s/XZqbNNRqaGjHipMo7rCDEQ}
\end{multicols} 

\end{document}