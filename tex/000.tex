% HEAD BEGIN
\documentclass[letterpaper, 12pt]{article}
\newsavebox\colbbox
\usepackage{graphicx}
\usepackage{multicol}
\usepackage{anysize}
\usepackage{fontspec}
\usepackage[fontset=none]{ctex}
\usepackage{tabularx}
\usepackage{longtable}
\PassOptionsToPackage{hyphens}{url}
\usepackage[breaklinks=true, colorlinks=true]{hyperref}
\expandafter\def\expandafter\UrlBreaks\expandafter{\UrlBreaks\do\a\do\b\do\c\do\d\do\e\do\f\do\g\do\h\do\i\do\j\do\k\do\l\do\m\do\n\do\o\do\p\do\q\do\r\do\s\do\t\do\u\do\v\do\w\do\x\do\y\do\z\do\A\do\B\do\C\do\D\do\E\do\F\do\G\do\H\do\I\do\J\do\K\do\L\do\M\do\N\do\O\do\P\do\Q\do\R\do\S\do\T\do\U\do\V\do\W\do\X\do\Y\do\Z}
% \let\oldurl\url
% \renewcommand{\url}[1]{\begin{sloppypar}\oldurl{#1}\end{sloppypar}}
\setlength\columnsep{30pt}
\marginsize{30pt}{30pt}{10pt}{20pt}
\setmainfont{TeX Gyre Bonum}
\setCJKmainfont[BoldFont=Noto Serif CJK SC Bold, ItalicFont=FandolKai]{Noto Sans CJK SC}
\setlength{\parindent}{0cm}
% \setCJKmonofont{Noto Sans CJK SC}
\begin{document}
\begin{center}
    \Huge\textbf{南哪大专醒前消息}
\end{center}
\vspace{4mm}
\hrule
\renewcommand\tabularxcolumn[1]{m{#1}}
\begin{tabularx}{\textwidth}{>{\hsize.2\hsize}X>{\hsize.6\hsize}X>{\hsize.2\hsize}X}
    \begin{flushleft}
        2024.10.21\, No.94
    \end{flushleft}
    &
    \begin{center}
        \textit{“克明峻德。”}
    \end{center}
    &
    \begin{flushright}
        \textbf{南京市栖霞区}
    \end{flushright}
\end{tabularx}
\vspace{-3.5mm}
\hrule
\vspace{4mm}
% HEAD END
\centerline{\huge\textbf{活动预告}}
\begin{multicols}{2}
    \section{订阅方式和加入编辑部}  
编辑部招聘人才,用爱发电,工作轻松,详情可联系QQ:1329527951 客服小祥\\想订阅本消息或获取PDF版(便于查看超链接和往期),可加QQ群:\href{https://qm.qq.com/q/VXIW7fgsEe}{849644979}.
\section{Deadline Ongoing}
\setbox\colbbox\vbox{
\makeatletter\col@number\@ne
\begin{longtable}{|c|c|c|}
    \hline
    消息(未见ddl的,不刊) & 截止日期 & 刊载日期\\
    \hline\hline
     ‘满天星’调研大赛 & 10.22 & 10.16\\
    有训院服设计赛 & 10.22 & 10.16\\
    南新读书会 & 10.23 & 10.20\\
    歌魅放映会 & 10.23 & 10.20\\
    急救培训活动报名 & 10.24 & 10.17\\
    计院迎新晚会征集节目 & 10.25 & 10.12\\
    马院主题宣讲报名 & 10.25 & 10.5\\
    织围巾志愿者招募 & 10.26 & 10.20\\
    心协DIY活动 & 10.26 & 10.20\\
    心协流光影院 & 10.26 & 10.17\\
    校园今日说法大赛 & 10.26 & 10.17\\
    遵义精神宣讲团遴选 & 10.27 & 10.10\\
    青鸟剧场新戏招募 & 10.27 & 10.14\\
    体测 & 10.27 & 10.16\\
    鼓楼音乐跑 & 10.27 & 10.20\\
    普通话考试报名 & 10.28 & 10.14\\
    仙林校史馆招募讲解员 & 10.30 & 9.12\\
    本科生暑期课程评教 & 10.31 & 9.19\\
    黑匣招募 & 11.1 & 10.19\\
    学位英语考试报名 & 11.3 & 10.17\\
    校运会 & 11.8 & 10.21\\
    后革命鲁迅研究征文 & 11.10 & 10.8\\
    大创训练计划申报 & 11.18 & 9.24\\
    招生宣传创意征集大赛 & 11.18 & 10.21\\ 
    EBSCO数据库检索大赛 & 11.20 & 10.3\\
    文院征稿 & 11.20 & 10.20\\
    乐跑 & 12.8 & 10.12\\
    
    \hline
\end{longtable}
\unskip
\unpenalty
\unpenalty}\unvbox\colbbox
\end{multicols}
\hrule
\pagebreak
\begin{multicols}{2}

\section{讲座}
\begin{tabular}{|c|c|c|}
    \hline
    往期讲座 & 开展日期 & 刊载日期\\
    \hline\hline
    《聚合物的研发与...》 & 10.24 & 10.3\\
    《电池及电化学能...》 & 11.24 & 10.3\\
    《专利查新与规避...》 & 12.19 & 10.3\\
    《中国法律形象西...》 & 10.23 & 10.16\\
    《与<自然>编辑对...》 & 10.30 & 10.16\\
    《博士论文的选题...》 & 10.22 & 10.17\\
    《语言能力与前近...》 & 10.25 & 10.18\\
    《物理信息神经网...》 & 10.23 & 10.18\\
    《国家图书馆的古...》 & 10.24 & 10.18\\
    《HKMW与当代德国...》 & 10.23 & 10.20\\
    图书馆系列讲座 & 12.3 & 10.20\\
    《中国电影有声转..》 & 10.22 & 10.20\\
    《则天文字在日本...》 & 10.23 & 10.20\\
    《大众视角与历史...》 & 10.25 & 10.21\\
    《中国电影史中的...》 & 10.23 & 10.21\\
    \hline
\end{tabular}



1.从自我确立到自我祛魅:中国电影史中的自反性研究\\
时间:10.23 14:30-16:30\\
地点:仙林校区文学院432\\
讲座人:罗婷\\
主持人:杨佳凝\\
内容提要:自反性是一种艺术作品自揭其本身虚构性的手法。本次讲座主要探讨1920 到1980 年代的中国电影史中自反性(self-reflexivity)的不同形式与功能演变。从早期电影中的自反性强化电影作“造梦机器”的吸引力,到八十年代,不同程度的自反性手法不仅出现在“探索片”中,也出现在此后的“娱乐片”里,呼应着并未远去的早期电影。\\
本次活动为2024-2025学年第一学期南京大学DIY研读研究课程“奇怪的电影”的拓展讲座。\\

2.大众视角与历史真相\\
主讲人:马俊亚 南京大学历史学院教授\\
主持人:胡翼青 南京大学新闻传播学院教授 南京大学高研院副院长\\
时间: 2024年10月25日(周五)16:00-17:30\\
地点: 南京大学鼓楼校区逸夫馆9楼高研院报告厅\\

\section{校运会}
时间:2024年11月8日至9日\\
地点:南京大学仙林校区第二运动场(四组团体育场)\\
项目:详见\url{https://mp.weixin.qq.com/s/bioUoy_zIRhd5cgW6vdKJw}\\
报名截止日期:详见各院具体通知。\\

\section{马理论论坛、征文合辑}
马克思主义学院公众号新近整理了一批马理论学科的期刊、会议征文,截止日期自本月至次年4月不等,不予详载,详阅:\url{https://mp.weixin.qq.com/s/G_KMEqh9Xpb8Riu6uBneiQ}。
\section{“点亮南星”南京大学2024年招生宣传创意征集大赛}
“点亮南星”——招生宣传创意征集大赛由本科招生办公室主办、招生宣传志愿者协会承办。\\
大赛设置“讲好南大故事”主赛道、“办学实力新跨越”专项赛道。作品共分“综合资料”“影音图像”“视觉设计”“绘画写作”“智能开发”五个组别。赛程分为自主报名、线上初审、网络展播、现场决赛四个阶段。赛果将为2025年“南星梦想计划”提供原创素材。\\
所有参赛同学都将获得荣誉证书和成套南大主题文创,并有机会赢得丰厚奖金,原创设计还有望落地应用。\\
参赛对象:南京大学全体在读全日制本科生、研究生。个人参加或组队参加均可,组队应不多于3人。\\
投稿内容:包含参赛报名表、作品文件。截止时间为2024年11月18日24:00,投稿链接为
\url{https://table.nju.edu.cn/dtable/forms/be80f26b-abdc-4549-9db0-c417443d39ac/}。之后11月19-24日依次包括线上初审阶段、网络展播阶段和现场决赛阶段
。\\
奖项设置:在符合要求的作品中遴选约40%三等奖、30%二等奖、20%一等奖,颁发荣誉证书和相应等级的南大文创套装。优选10件作品为特等奖,获取最高1000元奖金。\\
比赛详情见:\url{https://mp.weixin.qq.com/s/BmeWKf5ZY39JQXw7hKro1w}
\end{multicols} 
\end{document}