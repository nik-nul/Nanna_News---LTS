% HEAD BEGIN
\documentclass[letterpaper, 12pt]{article}
\newsavebox\colbbox
\usepackage{graphicx}
\usepackage{multicol}
\usepackage{anysize}
\usepackage{fontspec}
\usepackage[fontset=none]{ctex}
\usepackage{tabularx}
\usepackage{longtable}
\PassOptionsToPackage{hyphens}{url}
\usepackage[breaklinks=true, colorlinks=true]{hyperref}
\expandafter\def\expandafter\UrlBreaks\expandafter{\UrlBreaks\do\a\do\b\do\c\do\d\do\e\do\f\do\g\do\h\do\i\do\j\do\k\do\l\do\m\do\n\do\o\do\p\do\q\do\r\do\s\do\t\do\u\do\v\do\w\do\x\do\y\do\z\do\A\do\B\do\C\do\D\do\E\do\F\do\G\do\H\do\I\do\J\do\K\do\L\do\M\do\N\do\O\do\P\do\Q\do\R\do\S\do\T\do\U\do\V\do\W\do\X\do\Y\do\Z}
% \let\oldurl\url
% \renewcommand{\url}[1]{\begin{sloppypar}\oldurl{#1}\end{sloppypar}}
\setlength\columnsep{30pt}
\marginsize{30pt}{30pt}{10pt}{20pt}
\setmainfont{TeX Gyre Bonum}
\setCJKmainfont[BoldFont=Noto Serif CJK SC Bold, ItalicFont=FandolKai]{Source Han Sans SC}
\setlength{\parindent}{0cm}
% \setCJKmonofont{Noto Sans CJK SC}
\begin{document}
\begin{center}
    \Huge\textbf{南哪大专醒前消息}
\end{center}
\vspace{4mm}
\hrule
\renewcommand\tabularxcolumn[1]{m{#1}}
\begin{tabularx}{\textwidth}{>{\hsize.2\hsize}X>{\hsize.6\hsize}X>{\hsize.2\hsize}X}
    \begin{flushleft}
        2025.4.1\, No.207
    \end{flushleft}
    &
    \begin{center}
        \textit{“秉中持正、求新博闻。”}
    \end{center}
    &
    \begin{flushright}
        \textbf{南京市栖霞区}
    \end{flushright}
\end{tabularx}
\vspace{-3.5mm}
\hrule
\vspace{4mm}
% HEAD END
\centerline{\huge\textbf{活动预告}}
\begin{multicols}{2}
\section{订阅方式和加入编辑部}  
编辑部招聘人才,用爱发电,工作轻松,详情可联系QQ:1329527951 客服小千\\想订阅本消息或获取PDF版(便于查看超链接和往期),可加QQ群:\href{https://qm.qq.com/q/4HL41Nt3sQ}{466863272}.
\section{活动清单}
\setbox\colbbox\vbox{
\makeatletter\col@number\@ne
\begin{longtable}{|>{\centering\arraybackslash}m{.3\textwidth}|m{.06\textwidth}|m{.06\textwidth}|}
    \hline
    活动 & 开展时间 & 刊载时间\\
    \hline\hline
    南大版deepseek & / & 2.22\\
    悦读课程群 & / & 2.24\\
    eScience AI科研助手 & / & 3.11\\
    地科博物馆开放安排 & / & 3.22\\ 
    乐跑 & 5.16 & 3.10\\
    本科生劳育实践 & 7.20 & 2.19\\
    银星杯论文赛 & 4.22 & 2.27\\
    高教社杯 & 4.25 & 3.5\\
    南辩院系杯 & 4.12 & 3.6\\
    大文大理题目征集 & 期末 & 3.8\\
    5月免费上网 & ? & 3.9\\
    基础学科论坛 & 4.20 & 3.9\\
    普通话测试 & 4.11 & 3.25\\
    外教社杯 & 5.27 & 3.12\\
    Python比赛 & 4.6 & 3.16\\
    本科生院征集大鸣大放 & 4.4 & 3.21\\
    纸鸢工作坊 & 4.3 & 3.22\\
    南大博篆刻体验课 & 4.2 & 3.23\\
    粤歌赛 & 4.12 & 3.24\\
    江苏创青春赛事 & 4.30 & 3.26\\
    悦读测试 & 4.6 & 3.27\\
    南大数学竞赛 & 4.15 & 3.27\\
    AI素养大赛 & 4.15 & 3.31\\
    浦口音乐跑 & 5.30 & 3.31\\
    红会暑期项目招募 & 4.12 & 4.1\\
    \hline
\end{longtable}
\unskip
\unpenalty
\unpenalty}\unvbox\colbbox
\end{multicols}
\begin{multicols}{2}
\pagebreak

\section{讲座}
\begin{tabular}{|>{\centering\arraybackslash}m{.3\textwidth}|m{.06\textwidth}|m{.06\textwidth}|}
    \hline
    讲座 & 开展时间 & 刊载时间\\
    \hline\hline
    AI: The Destruction of the Imagination? & 4.2 & 3.28\\\hline
    日军侵华时期“万人坑”遗址的记忆与忘却 & 4.2 & 3.31\\\hline
    论运气的模态分析 & 4.2 & 3.31\\\hline
    仁本论:孔子的行为科学 & 4.2 & 3.31\\\hline
    秦汉玺印人名考析 & 4.9 & 3.31\\\hline
\end{tabular}
%讲座预告写在这。用subsection


%此处写校级活动,请不要把讲座、院级活动和社团活动写在这里orz orz orz
\section{南京大学博物馆清明放假通知}
清明节假日期间,4月4日(周五)闭馆。4月5日(周六)、4月6日(周日)一楼展厅正常开放。
\\
\\详见:\url{https://mp.weixin.qq.com/s/bM0cHNAxHW5c2zxrLY9qtw}

\section{南京大学体育月}
“鲸跃南雍,竞逐风华”体育月将于4月正式启幕,活动抢先一览:
\\PART.01:体育月趣味竞赛活动
\\“鲸彩飞旋”,飞盘大赛四校区联动激战;“鲸心弈海”,南京大学第二届五子棋大赛和你“弈”起点燃这场头脑风暴;“南雍寻踪”,满校春色等待定向越野的你前来一睹芳彩。
\\PART.02:体育月学生社团展示教学活动
\\箭矢无影,弓箭协会将带你领略传统弓技的魅力;排球跃动,排球协会将与你一起“垫”燃青春;还有网球、太极拳、足球、轮滑、游泳、拳击、飞镖等各大体育社团,将携手为我们带来一场燃动全身心的体育盛宴,其中弓箭、游泳等项目的现场活动,参与即有机会获得场馆内深度体验的专属机会。
\\图片
\\PART.03:体育月舞蹈教学活动
\\由体育部舞蹈教学专业老师带来的零基础友好教学活动,不仅可以在律动中舞动青春,更有与在艺术体育领域大显身手的国际健将们邂逅的机会。
\\各活动时间地点的初步安排详见推文,具体活动安排详见后续每期推送。
\\详见:\url{https://mp.weixin.qq.com/s/m7uSTqgzh6b_FVEWO-pjEQ}




\section{南大红会“博爱青春”暑期项目}
招募人数及分类
\\暂定按照南京、苏州、宿迁三组统一进行招募,报名时需确认意向
\\暂定每组招募9\textasciitilde{}10人
\\课程组:8\textasciitilde{}9人,每人承担至少两门课程教学任务
\\宣传组:1人,参与宣传拍摄、视频剪辑等宣传工作‍‍‍‍
\\报名时间
\\即日起至4月12日24点
\\报名方法
\\符合条件的同学请填写附件1《2025年南京大学红十字会“博爱青春”暑期项目成员报名表》,于4月12日24点前发送至校红会公邮nju\_redcross@163.com,邮件主题为“博爱青春成员报名+姓名+学号”,文件名为“校区+姓名+学号”。\\
详见:\url{https://mp.weixin.qq.com/s/xppWFsPOeNWOP4jTJzhWAw}


\section{院级活动}
\begin{tabular}{|>{\centering\arraybackslash}m{.3\textwidth}|m{.06\textwidth}|m{.06\textwidth}|}
\hline
    活动 & 开展时间 & 刊载时间\\
    \hline\hline
    文院剧本创作研讨会 & 9.30 & 3.2\\
    物院征集课程指南 & 6.15 & 3.3\\
    地海征集春日影 & 6.15 & 3.14\\
    社院学术节 & 4.18 & 3.25\\
    五院运动会 & 4.13 & 3.31\\
    电子南师春日交流 & 4.12 & 3.31\\
    五院乒乓球赛 & 4.19 & 3.31\\
    建城影展征集 & 4.16 & 3.31\\
    
    \hline
\end{tabular}

\section{社团活动}
\begin{tabular}{|>{\centering\arraybackslash}m{.3\textwidth}|m{.06\textwidth}|m{.06\textwidth}|}
    \hline
    社团活动 & 开展时间 & 刊载时间\\
    \hline\hline
    天文台开放日 & / & 1.6\\
    重唱诗歌奖征稿 & 4.30 & 3.31\\
    印社讲座 & 4.9 & 4.1\\
    弓箭社体验 & 4.8 & 4.1\\
    飞镖社体验 & 4.8 & 4.1\\
    排协网协体验 & 4.10 & 4.1\\
    杨协体验 & 4.12 & 4.1\\
    足协体验 & 4.15 & 4.1\\
    轮滑社体验 & 4.17 & 4.1\\
    拳击社体验 & 4.22 & 4.1\\
    轮滑社体验 & 4.22 & 4.1\\
    飞盘大赛 & 4.12-13 & 4.1\\
    五子棋大赛 & 4.13 & 4.1\\
    定向赛 & 4.20 & 4.1\\
    体育舞蹈教学 & 4.25 & 4.1\\
    \hline
\end{tabular}
%这里是写社团活动的,社团活动就是由社团主办、主要针对社团内部人员的活动。不要把非社团活动写在这里。
\subsection{NJUNMUNC2025丨社交晚会节目招募}
基本信息
\\晚会时间
\\2025年4月19日19:30-20:30
\\晚会地点
\\新纪元大酒店二楼嘉和厅
\\招募对象
\\全体学团、代表、观察员、组委
\\
\\节目要求
\\1.类型
\\节目类型不限,歌曲、舞蹈、乐器、相声……
\\统统都可以!尽情发挥、创意无界,期待你的“破圈”表演!
\\2.时长
\\节目时长建议3-5分钟(若节目有特殊需要,可在报名时注明)


\\详见:\url{https://mp.weixin.qq.com/s/wvXJJ-LGgB-5oI4GR2Zmiw}
\subsection{南初印社:秦汉玺印人名考析讲座预告}
主题 秦汉玺印人名考析
\\主讲 魏宜辉 南京大学文学院副教授
\\主持 李林晋 南京大学文学院2021级本科生
\\时间 2025年4月9日(周三)19:00‐21:00
\\地点 南京大学文学院活水轩
\\主办 南京大学文学院 “大美汉字”普及推广中心
\\承办 南京大学文学院学生会 南京大学学生南初印社 南京大学出土文献研究中心
\\详见:\url{https://mp.weixin.qq.com/s/JyqCQ2CMBe9cD5_A4pS8-w}
\end{multicols}
\end{document}
