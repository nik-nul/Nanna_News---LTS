% HEAD BEGIN
\documentclass[letterpaper, 12pt]{article}
\usepackage{graphicx}
\usepackage{multicol}
\usepackage{anysize}
\usepackage{fontspec}
\usepackage[fontset=none]{ctex}
\usepackage{tabularx}
\PassOptionsToPackage{hyphens}{url}
\usepackage[breaklinks=true, colorlinks=true]{hyperref}
\expandafter\def\expandafter\UrlBreaks\expandafter{\UrlBreaks\do\a\do\b\do\c\do\d\do\e\do\f\do\g\do\h\do\i\do\j\do\k\do\l\do\m\do\n\do\o\do\p\do\q\do\r\do\s\do\t\do\u\do\v\do\w\do\x\do\y\do\z\do\A\do\B\do\C\do\D\do\E\do\F\do\G\do\H\do\I\do\J\do\K\do\L\do\M\do\N\do\O\do\P\do\Q\do\R\do\S\do\T\do\U\do\V\do\W\do\X\do\Y\do\Z}
% \let\oldurl\url
% \renewcommand{\url}[1]{\begin{sloppypar}\oldurl{#1}\end{sloppypar}}
\setlength\columnsep{30pt}
\marginsize{30pt}{30pt}{10pt}{20pt}
\setmainfont{TeX Gyre Bonum}
\setCJKmainfont[BoldFont=Noto Serif CJK SC Bold, ItalicFont=FandolKai]{Noto Sans CJK SC}
\setlength{\parindent}{0cm}
% \setCJKmonofont{Noto Sans CJK SC}
\begin{document}
\begin{center}
    \Huge\textbf{南哪大专醒前消息}
\end{center}
\vspace{4mm}
\hrule
\renewcommand\tabularxcolumn[1]{m{#1}}
\begin{tabularx}{\textwidth}{>{\hsize.2\hsize}X>{\hsize.6\hsize}X>{\hsize.2\hsize}X}
    \begin{flushleft}
        2024.10.9\, No.83
    \end{flushleft}
    &
    \begin{center}
        \textit{“克明峻德。”}
    \end{center}
    &
    \begin{flushright}
        \textbf{南京市栖霞区}
    \end{flushright}
\end{tabularx}
\vspace{-3.5mm}
\hrule
\vspace{4mm}
% HEAD END
\centerline{\huge\textbf{活动预告}}
\begin{multicols}{2}
\section{Deadline Ongoing}
\begin{tabular}{|c|c|c|}
    \hline
    消息(未见ddl的,不刊) & 截止日期 & 刊载日期\\
    \hline\hline
    仙林校史馆招募讲解员 & 10.30 & 9.12\\
    本科生暑期课程评教 & 10.31 & 9.19\\
    大创训练计划申报 & 11.18 & 9.24\\
    苏州校区音乐会 & 10.19 & 9.25\\
    心理中心征稿 & 10.10 & 9.28\\
    周末剧场 & 10.10 & 9.28\\
    第十九届大挑 & 10.15 & 9.30\\
    声谷创新基金 & 10.18 & 9.30\\
    午餐读书会 & 10.10 & 9.30\\
    鹰角校招宣讲 & 10.15 & 10.2\\
    大专戏曲知识竞赛 & 10.20 & 10.2\\
    EBSCO数据库检索大赛 & 11.20 & 10.3\\
    炜华音乐跑 & 12.8 & 10.4\\
    台港澳交流协会招新 & 10.10 & 10.5\\
    马院主题宣讲报名 & 10.25 & 10.5\\
    NJU MAJOR & 10.13 & 10.8\\
    本科毕业生图像补采 & 10.10 & 10.8\\
    新生午餐会报名 & 10.10 & 10.8\\
    世界精神卫生日活动 & 10.10 & 10.8\\
    后革命鲁迅研究征文 & 11.10 & 10.8\\
    街舞社开放活动 & 10.11 & 10.8\\
    心协黑胶唱片活动 & 10.13 & 10.8\\
    黑匣子对谈招募 & 10.11 & 10.8\\
    鼓楼草地音乐节前瞻 & 10.11 & 10.9\\
    鼓楼草地音乐节 & 10.13 & 10.9\\
    重唱诗社匿名评诗会 & 10.13 & 10.9\\
    \hline
\end{tabular}
\section{订阅方式和加入编辑部}
编辑部招聘人才,用爱发电,工作轻松,详情可联系QQ:1329527951 客服小祥\\想订阅本消息或获取PDF版(便于查看超链接),可加QQ群:\href{https://qm.qq.com/q/FGX1VYCrGS}{962626571}.
\section{讲座}
\begin{tabular}{|c|c|c|}
    \hline
    往期讲座 & 开展日期 & 刊载日期\\
    \hline\hline
    《聚合物的研发与...》 & 10.24 & 10.3\\
    《电池及电化学能...》 & 11.24 & 10.3\\
    《专利查新与规避...》 & 12.19 & 10.3\\
    物院学术报告会 & 10.10 & 10.8\\
    《恋爱是门技术活》 & 10.14 & 10.8\\
    《宋代佛教书籍史》 & 10.11 & 10.8\\
    《阿赫迈底亚教派...》 & 10.11 & 10.8\\
    《对于人工智能时...》 & 10.16 & 10.9\\
    《职普比大体相当...》 & 10.11 & 10.9\\
    《异域的反思与开...》 & 10.11 & 10.9\\
    《中国古代文学中...》 & 10.12 & 10.9\\
      \hline
\end{tabular}\\\\
1. 对于人工智能时代的自然智能的哲学反思\\
主讲人:Maurizio E. V. Ferraris(意大利都灵大学哲学系教授,师从德里达与瓦蒂莫)\\
时间:10月16日9:00\\
地点:哲学学院(薛光林楼)402室\\

2. 职普比“大体相当”:问题与改进\\
题目:行知论坛第242期 职普比“大体相当”:问题与改进\\
主讲人:沈有禄教授\\
主持人:黄斌教授\\
时间:2024.10.11(周五)10:00\\
地点:仙林校区潘琦楼B座103室\\
\url{https://mp.weixin.qq.com/s/cUkWKPGNg9AUSWOlAsysMw}\\

3.异域的反思与开拓--晚近海外陶渊明研究刍议\\
题目:南京大学文学院建院110周年系列讲座 异域的反思与开拓--晚近海外陶渊明研究刍议\\
主讲人:张月(澳门大学中文系副教授、博士生导师)\\
主持人:卞东波(南京大学文学院教授)\\
时间:2024.10.11(点5:00-17:00)\\
地点:高研院(仙林校区国际学院C308/即逸C308)\\
\url{https://mp.weixin.qq.com/s/uLhQnz5UBl0zurwpiPSvaw}\\

4.“文章奥府:中国古代文学中的经典与历史” 暨苏港澳古代文学青年学者论坛\\
时间:2024年 10月12日—10月13日\\
地点:南京大学仙林校区文学院活水轩\\
论坛的报告安排详见\url{https://mp.weixin.qq.com/s/Re31mDv34WOWiUK3QWFarw}
\section{“中华民族共同体概论”课程选课通知}
选课规则:本学期“中华民族共同体概论”课程在仙林校区开展线下授课,选课方式为“直选式”,先到先得,开课两周内可进行补退选。\\
选课时间:10月9日12:00至10月27日24:00(含开课两周内的补退选)\\
上课安排:第7-17周的周三5-6节(具体教学安排详见课程教学周历)\\
具体链接:\url{https://jw.nju.edu.cn/f4/fb/c26263a718075/page.psp}
\section{南京大学2025年硕士研究生招生章程}
南京大学2025年硕士研究生招生章程公布,详见\url{https://mp.weixin.qq.com/s/IVbsuEMwndqg3YgYhb9dVg}
\section{草地音乐节前瞻}
2024小蓝鲸草地音乐节即将来临。在鼓楼、仙林校区同步设有展台,准备了精美的宣传物料,欢迎大家前来了解更多的小蓝鲸草地音乐节的详细讯息。\\
时间:10月11号 12:00-13:00\\
地点:鼓楼校区南园广场、仙林校区四五六食堂前平台\\
草地音乐节开始时间、地点:10月13日 19:00; 鼓楼校区苏浙体育场
\section{奋进音乐跑荧光小活动}
在奋进音乐跑开跑前,学生会为大家特别准备了荧光游戏活动。同学们可以将五彩荧光棒粘贴在身上,在黑夜中装扮成荧光火柴人造型,拍摄充满趣味的照片。\\
活动时间:10月9日-10月12日 21:30-22:10\\
荧光棒领取地点:仙林校区炜华体育场北侧入口展台
\section{2024年地学初体验野外科考活动报名通知} 
“地学初体验”活动旨在以自然为课堂,通过专业老师的带队讲解和高年级助教的悉心指导,让大一的新同学了解地学领域的知识。注:今年表现优秀的同学会有奖学金奖励。
活动时间:10月26-27日(两天一夜,在外住宿)。
活动面向对象:对地球科学与工程学院相关专业感兴趣,以及已经是地球科学与工程学院(地学大类,地质学拔尖班,行星班,生物演化与环境国际班)的2024级本科生。
活动费用:食宿行费用均由南大地科院承担。
报名详见\url{https://mp.weixin.qq.com/s/EIRvqgZwaXT0xKn0FqUMhg}
\section{第八届“青春影像”青少年微电影短视频征集}
校团委现通知征稿事宜,\\
活动单元:微电影征集单元、短视频征集单元、微短剧剧本征集单元以及#有梦有方向# 校园歌手征集展示活动。\\
征集对象:全国高校大学生及毕业3年内青年创作者。\\
获奖作品可获得官方证书、奖金等奖励。\\
详见:\url{https://mp.weixin.qq.com/s/zS373tlnNcr77E2tbEigZw}
\section{重唱诗社 | 匿名评诗会}
“或可熬过一时之危”匿名评诗会\\
时间:2024年10月13日(周日)14:00\\
地点:文学院437(南京大学仙林校区)\\
投稿邮箱:aichongchang@163.com(备注匿名评诗会,限投一首)\\
截稿日期:2024年10月12日(周六)12:00\\
\section{文软联袂·情满重阳——茱萸香囊DIY活动}
这是一场跨越学科界限的雅集。我们将亲手缠丝,将吴茱萸、野菊花、山苍子等中草药填入囊中,体会传统手工的古雅韵味。同时,活动融入现代科技元素,大家有机会以代码为线,编织出独具匠心的“赛博香囊”,一览科技与人文融合的现代之美。\\
活动当日更设置了“智趣问答”“意趣竞猜”与“逢九过”等游戏,在活动的尾声,还有“共绘重阳登高图”与“馨香传情·敬师雅集”等温馨环节.\\
活动时间 10月13日14:00-16:00\\
活动地点 仙1-308\\
活动人数 50人(面向文学院及软件学院的全体同学)\\
报名方式详见\url{https://mp.weixin.qq.com/s/WowiVgrfWFGP4j1NCcEkCw}

\end{multicols} 
\end{document}