% HEAD BEGIN
\documentclass[letterpaper, 12pt]{article}
\newsavebox\colbbox
\usepackage{graphicx}
\usepackage{multicol}
\usepackage{anysize}
\usepackage{fontspec}
\usepackage[fontset=none]{ctex}
\usepackage{tabularx}
\usepackage{longtable}
\PassOptionsToPackage{hyphens}{url}
\usepackage[breaklinks=true, colorlinks=true]{hyperref}
\expandafter\def\expandafter\UrlBreaks\expandafter{\UrlBreaks\do\a\do\b\do\c\do\d\do\e\do\f\do\g\do\h\do\i\do\j\do\k\do\l\do\m\do\n\do\o\do\p\do\q\do\r\do\s\do\t\do\u\do\v\do\w\do\x\do\y\do\z\do\A\do\B\do\C\do\D\do\E\do\F\do\G\do\H\do\I\do\J\do\K\do\L\do\M\do\N\do\O\do\P\do\Q\do\R\do\S\do\T\do\U\do\V\do\W\do\X\do\Y\do\Z}
% \let\oldurl\url
% \renewcommand{\url}[1]{\begin{sloppypar}\oldurl{#1}\end{sloppypar}}
\setlength\columnsep{30pt}
\marginsize{30pt}{30pt}{10pt}{20pt}
\setmainfont{TeX Gyre Bonum}
\setCJKmainfont[BoldFont=Noto Serif CJK SC Bold, ItalicFont=FandolKai]{Source Han Sans SC}
\setlength{\parindent}{0cm}
% \setCJKmonofont{Noto Sans CJK SC}
\begin{document}
\begin{center}
    \Huge\textbf{南哪大专醒前消息}
\end{center}
\vspace{4mm}
\hrule
\renewcommand\tabularxcolumn[1]{m{#1}}
\begin{tabularx}{\textwidth}{>{\hsize.2\hsize}X>{\hsize.6\hsize}X>{\hsize.2\hsize}X}
    \begin{flushleft}
        2025.4.3\, No.209
    \end{flushleft}
    &
    \begin{center}
        \textit{“秉中持正、求新博闻。”}
    \end{center}
    &
    \begin{flushright}
        \textbf{南京市栖霞区}
    \end{flushright}
\end{tabularx}
\vspace{-3.5mm}
\hrule
\vspace{4mm}
% HEAD END
\centerline{\huge\textbf{活动预告}}
\begin{multicols}{2}
\section{订阅方式和加入编辑部}  
编辑部招聘人才,用爱发电,工作轻松,详情可联系QQ:1329527951 客服小千\\想订阅本消息或获取PDF版(便于查看超链接和往期),可加QQ群:\href{https://qm.qq.com/q/4HL41Nt3sQ}{466863272}.
\section{活动清单}
\setbox\colbbox\vbox{
\makeatletter\col@number\@ne
\begin{longtable}{|>{\centering\arraybackslash}m{.3\textwidth}|m{.06\textwidth}|m{.06\textwidth}|}
    \hline
    活动 & 开展时间 & 刊载时间\\
    \hline\hline
    南大版deepseek & / & 2.22\\
    悦读课程群 & / & 2.24\\
    eScience AI科研助手 & / & 3.11\\
    地科博物馆开放安排 & / & 3.22\\ 
    乐跑 & 5.16 & 3.10\\
    本科生劳育实践 & 7.20 & 2.19\\
    银星杯论文赛 & 4.22 & 2.27\\
    高教社杯 & 4.25 & 3.5\\
    南辩院系杯 & 4.12 & 3.6\\
    大文大理题目征集 & 期末 & 3.8\\
    5月免费上网 & ? & 3.9\\
    基础学科论坛 & 4.20 & 3.9\\
    普通话测试 & 4.11 & 3.25\\
    外教社杯 & 5.27 & 3.12\\
    Python比赛 & 4.6 & 3.16\\
    粤歌赛 & 4.12 & 3.24\\
    江苏创青春赛事 & 4.30 & 3.26\\
    悦读测试 & 4.6 & 3.27\\
    南大数学竞赛 & 4.15 & 3.27\\
    AI素养大赛 & 4.15 & 3.31\\
    浦口音乐跑 & 5.30 & 3.31\\
    红会暑期项目招募 & 4.12 & 4.1\\
    程设大赛 & 4.26 & 4.2\\
    主持人大赛报名 & 4.10 & 4.4\\
    春影摄影大赛 & 4.13 & 4.4\\
    奇绩创业宣讲课 & 4.11 & 4.4\\
    \hline
\end{longtable}
\unskip
\unpenalty
\unpenalty}\unvbox\colbbox
\end{multicols}
\begin{multicols}{2}
\pagebreak

\section{讲座}
\begin{tabular}{|>{\centering\arraybackslash}m{.3\textwidth}|m{.06\textwidth}|m{.06\textwidth}|}
    \hline
    讲座 & 开展时间 & 刊载时间\\
    \hline\hline
    秦汉玺印人名考析 & 4.9 & 3.31\\\hline
    先人后事 破局之道 & 4.11 & 3.3\\\hline
\end{tabular}
%讲座预告写在这。用subsection
\subsection{CSAI 卓越科学家大讲堂 }
Regularization, Heuristics, and Strategy: A Long Journey Towards Understanding a Few Fundamental yet Fuzzy Concepts in Computing
\\讲座嘉宾:滕尚华 教授
\\时间:4月8日(星期二)9:00
\\地点:计算机科学技术楼111室
\\详见:\url{https://mp.weixin.qq.com/s/zkSlqUZAGW5TTBOZKaTE6g}

\subsection{Consumer Awareness, Noisy Certification, and Corporate Social Responsibility under Asymmetric Information}
报告人:肖光教授 香港理工大学
\\主持人:肖条军教授 南京大学
\\报告时间:2025年4月9日(周三)下午14:00-15:15
\\报告地点:协鑫楼108
\\详见:\url{https://mp.weixin.qq.com/s/hr8iaVxyrs2-NKC00aOuvg}


\subsection{Assortment Optimization Under History-Dependent Effects}
报告人:郑欢教授 上海交通大学
\\主持人:肖条军教授 南京大学
\\报告时间:2025年4月11日(周五)下午14:00-15:30
\\报告地点:协鑫楼108
\\
\\详见:\url{https://mp.weixin.qq.com/s/Z9iBMf-jlearET7JlfsuTA}


%此处写校级活动,请不要把讲座、院级活动和社团活动写在这里orz orz orz
\section{招募 | 世贸组织2026年青年专业人员}
组织:世界贸易组织
\\语言:流利使用英语,掌握法语或者西班牙语是一个优势
\\月薪:4000瑞士法郎
\\详细招募信息(英文原文)见链接。
\\申请截止日期:2025年4月22日
\\详见:\url{https://mp.weixin.qq.com/s/tPKFKwYt_C9epznprTj1Sg}

\section{志愿者招募 | 旧物焕新生,童心护地球}
活动概述:诚邀20位热心志愿者,一起为环保助力,陪伴特殊儿童(自闭症)度过一段充满意义的时光。
\\活动内容:两个阶段
\\1.在线下活动前,志愿者们收集废弃的可回收材料,发挥创意将其制作成环保教具。
\\2.在课堂上,通过教具向小朋友们讲解生态保护知识,开展问答互动,给答对的小朋友发奖品。课堂后半段,志愿者将协助孩子们在环保袋上创作环保主题绘画。
\\活动时间:
\\1. 校内准备:2025年4月14日 - 4月20日,收集材料制作教具。
\\2. 校外活动:2025年5月12日下午,开展线下环保课堂。
\\报名截止日期——2025年4月13日
\\二维码和QQ群详见链接。
\\详见:\url{https://mp.weixin.qq.com/s/N6c2WP-B6qcEjn0y7ICaKQ}

\section{ “华为杯”第八届研究生创“芯”大赛}
参赛对象包括:
\\1. 中国大陆、港澳台地区在读研究生;(硕士/博士/留学生)
\\2. 已获得研究生录取资格的大四本科生;(须提供录取证明)
\\3. 国外高校在读研究生。
\\组队方式:
\\1. 每支队伍2-3名学生,可配1-2名指导教师;
\\2. 每位指导教师最多指导5支队伍;
\\3. 每位学生仅可加入1支队伍。
\\报名时间安排
\\报名启动时间:2025年3月25日
\\报名截止时间:2025年6月15日
\\资格审核及作品提交截止时间:2025年6月20日
\\决赛时间:2025年7月27日—31日(拟)
\\决赛地点:南京大学苏州校区
\\参赛方式与作品要求
\\参赛队可选择“自主命题”或“企业命题”:自主命题需提交语音讲解PPT及技术附件,内容涵盖应用背景、创新设计、系统功能演示等,播放时长不超过8分钟;企业命题需参考官网发布的题目及要求,提交针对性的解决方案;
\\注:参赛作品涉及领域包括模拟电路、AI芯片、射频设计、器件制造等十余个方向,详见官网分类说明。
\\报名及提交渠道
\\大赛官网:https://cpipc.acge.org.cn/cw/hp/10
\\请参赛同学提前完成团队组建、命题选择、材料准备等工作。报名及作品提交均通过大赛官网完成。
\\详见:\url{https://mp.weixin.qq.com/s/DrQGy2SGdBAOgY-Z7iOx1g}

\section{【报名启动】南京大学“瑞声杯”建模挑战赛}
参赛对象:南京大学全体在读本科生、硕士生、博士生,以不超过3人组队形式参赛(可以单人参赛、鼓励跨学院组队)。
\\赛程规划
\\报名阶段:即日起至4月20日
\\赛题发布:4月23日
\\建模攻坚:4月23日至5月7日
\\决赛(公开答辩)、颁奖典礼:5月20日左右
\\赛题主题
\\本次赛事选题聚焦两大算法核心问题:
\\(1)大规模矩阵的特征值高效稳定求解
\\(2)振动声学仿真中涉及的声学仿真结果评估
\\详见:\url{https://mp.weixin.qq.com/s/snHHXpwhiCSHwhMHST3Vug}

\section{2025年江苏大学生志愿服务乡村振兴计划报名}
招募对象:面向2025年江苏省普通高等学校应届毕业生或在读研究生,到岗前获得毕业证书和学位证书。中共党员(含中共预备党员)、优秀团学干部、有志愿服务经历的优先录用。已被录取为研究生的应届高校毕业生参加“志愿服务乡村振兴计划”的,学校应为其保留学籍。
\\工作内容:按照公开招募、自愿报名、组织审核、考核面试、体检培训等程序,集中选拔不少于1000名志愿者,续签120名左右志愿者,赴徐州、连云港、淮安、盐城、宿迁等设区市基层单位开展为期一年的志愿服务。服务内容主要包括:基础教育、农业科技、医疗卫生、法律服务、青年工作、社会治理等。
\\招募流程、政策支持详见推文。校内QQ咨询群:729263400(群内实名制)
\\详见:\url{https://mp.weixin.qq.com/s/IsEaUkAkTN9quErKTv2Rlw}


\section{仙林校区志愿法律咨询项目正式启动}
法律咨询志愿者由法学院团委统一选拔,志愿者在仙林校区图书馆127室值班,负责接待来访的咨询者,提供基础法律咨询服务。
\\值班时间为每周一至周五9:30-11:30、14:00-18:00(节假日休息)。如您有咨询需求,可在工作时间前往一站式法治育人社区。如有合同、起诉状等材料,可一并携带。
\\本项目暂时仅有线下咨询渠道。图书馆一楼入口位于图书馆东侧(藜照湖旁),127室位于进门左侧。其他信息详见推文。
\\详见:\url{https://mp.weixin.qq.com/s/BtRJqy8Oo0NAF0IGSPcYOQ}

\section{院级活动}
\begin{tabular}{|>{\centering\arraybackslash}m{.3\textwidth}|m{.06\textwidth}|m{.06\textwidth}|}
\hline
    活动 & 开展时间 & 刊载时间\\
    \hline\hline
    文院剧本创作研讨会 & 9.30 & 3.2\\
    物院征集课程指南 & 6.15 & 3.3\\
    地海征集春日影 & 6.15 & 3.14\\
    社院学术节 & 4.18 & 3.25\\
    五院运动会 & 4.13 & 3.31\\
    电子南师春日交流 & 4.12 & 3.31\\
    五院乒乓球赛 & 4.19 & 3.31\\
    建城影展征集 & 4.16 & 3.31\\
    法院党建征文 & 5.20 & 4.2\\
    地学乒赛 & 4.19 & 4.2\\
    匡计社商联谊 & 4.13 & 4.2\\
    
    \hline
\end{tabular}

\subsection{阳光南数 | 数海竞技,羽动几何—数学学院羽毛球赛}
时间:2025年4月12日(周六)14:00-18:00
\\比赛地点:南京大学鼓楼校区体育馆
\\参赛对象:数学学院全体师生
\\项目设置:男单、女单
\\报名问卷链接:https://www.wjx.cn/vm/YrMZSZx.aspx\# 
\\比赛规则、奖项设置见原文。
\\详见:\url{https://mp.weixin.qq.com/s/3K-lintx54wbUZnQyGz6Og}

\subsection{计科清明桌游会}
2025年4月5日,计科院楼233。活动自由分组,每位同学根据自己喜欢的桌游进行游玩,下午1点开始,可以随时离开,最晚不超过晚7点。
\\详见:\url{https://mp.weixin.qq.com/s/wrChSnYf5dxYTOXiT87qvA}

\subsection{软院桌游会}
【清明小憩·桌间逢春】
\\春意渐浓,思绪纷繁,恰是清明寄情时。为助力同学们在传统节气中舒缓身心,软件学院学生会特别组织策划了一场清明桌游会,让同学们在策略协作间感受春日乐趣,在欢声笑语释放压力,既得春日雅趣之闲适,亦获同窗共进之默契,以清明澄澈之心境拥抱盎然春意。
\\1.🎈快乐游戏🎈你画我猜等小游戏,在交流中认识新朋友
\\2.🎈经典桌游🎈UNO、三国杀、狼人杀,畅享乐趣
\\3.🎈互动桌游🎈不要做挑战、谁是卧底,欢笑中释放压力
\\4.🎈传统棋类🎈象棋、围棋、五子棋,智慧与策略的较量
\\学习之余,不妨给自己一个放松的机会,一起度过一段轻松愉快的时光吧!
\\
\\时间:【4月6日(本周日)15:00-20:00】
\\地点:费B501
\\参与方式:加入群聊465023305👇🏻


\subsection{创想无界 智绘未来 | 软件学院形象IP设计征集活动}
作品征集时间:2025年4月3日至2025年4月20日
\\评审时间:2025年4月21日至2025年4月30日
\\参加对象:软件学院全体师生及海内外院友
\\详情见推文。
\\详见:\url{https://mp.weixin.qq.com/s/XFrlifylKT0Nqv5ym3ao1w}

\section{社团活动}
\begin{tabular}{|>{\centering\arraybackslash}m{.3\textwidth}|m{.06\textwidth}|m{.06\textwidth}|}
    \hline
    社团活动 & 开展时间 & 刊载时间\\
    \hline\hline
    天文台开放日 & / & 1.6\\
    重唱诗歌奖征稿 & 4.30 & 3.31\\
    印社讲座 & 4.9 & 4.1\\
    弓箭社体验 & 4.8 & 4.1\\
    飞镖社体验 & 4.8 & 4.1\\
    排协网协体验 & 4.10 & 4.1\\
    杨协体验 & 4.12 & 4.1\\
    足协体验 & 4.15 & 4.1\\
    轮滑社体验 & 4.17 & 4.1\\
    拳击社体验 & 4.22 & 4.1\\
    轮滑社体验 & 4.22 & 4.1\\
    飞盘大赛 & 4.13 & 4.1\\
    五子棋大赛 & 4.13 & 4.1\\
    定向赛 & 4.20 & 4.1\\
    体育舞蹈教学 & 4.25 & 4.1\\
    吉他社歌手招募 & 4.20 & 4.4\\
    吉他社春日音 & 4.26 & 4.4\\
    国学社寄明信片 & 4.14 & 4.4\\
    \hline
\end{tabular}
%这里是写社团活动的,社团活动就是由社团主办、主要针对社团内部人员的活动。不要把非社团活动写在这里。
\subsection{男篮院系杯小组赛Ⅰ明日赛程}
数学 vs 海外
\\时间:4月5日15:00 - 16:00
\\地点:一组团篮球场
\\详见:\url{https://mp.weixin.qq.com/s/M2tgp0UDLsUnxgDjpWP_yg}

\end{multicols}
\end{document}
