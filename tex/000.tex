% HEAD BEGIN
\documentclass[letterpaper, 12pt]{article}
\newsavebox\colbbox
\usepackage{graphicx}
\usepackage{multicol}
\usepackage{anysize}
\usepackage{fontspec}
\usepackage[fontset=none]{ctex}
\usepackage{tabularx}
\usepackage{longtable}
\PassOptionsToPackage{hyphens}{url}
\usepackage[breaklinks=true, colorlinks=true]{hyperref}
\expandafter\def\expandafter\UrlBreaks\expandafter{\UrlBreaks\do\a\do\b\do\c\do\d\do\e\do\f\do\g\do\h\do\i\do\j\do\k\do\l\do\m\do\n\do\o\do\p\do\q\do\r\do\s\do\t\do\u\do\v\do\w\do\x\do\y\do\z\do\A\do\B\do\C\do\D\do\E\do\F\do\G\do\H\do\I\do\J\do\K\do\L\do\M\do\N\do\O\do\P\do\Q\do\R\do\S\do\T\do\U\do\V\do\W\do\X\do\Y\do\Z}
% \let\oldurl\url
% \renewcommand{\url}[1]{\begin{sloppypar}\oldurl{#1}\end{sloppypar}}
\setlength\columnsep{30pt}
\marginsize{30pt}{30pt}{10pt}{20pt}
\setmainfont{TeX Gyre Bonum}
\setCJKmainfont[BoldFont=Noto Serif CJK SC Bold, ItalicFont=FandolKai]{Source Han Sans SC}
\setlength{\parindent}{0cm}
% \setCJKmonofont{Noto Sans CJK SC}
\begin{document}
\begin{center}
    \Huge\textbf{南哪大专醒前消息}
\end{center}
\vspace{4mm}
\hrule
\renewcommand\tabularxcolumn[1]{m{#1}}
\begin{tabularx}{\textwidth}{>{\hsize.2\hsize}X>{\hsize.6\hsize}X>{\hsize.2\hsize}X}
    \begin{flushleft}
        2025.4.12\, No.218
    \end{flushleft}
    &
    \begin{center}
        \textit{“秉中持正、求新博闻。”}
    \end{center}
    &
    \begin{flushright}
        \textbf{南京市栖霞区}
    \end{flushright}
\end{tabularx}
\vspace{-3.5mm}
\hrule
\vspace{4mm}
% HEAD END
\centerline{\huge\textbf{活动预告}}
\begin{multicols}{2}
\section{订阅方式和加入编辑部}  
编辑部招聘人才,用爱发电,工作轻松,详情可联系QQ:1329527951 客服小千\\想订阅本消息或获取PDF版(便于查看超链接和往期),可加QQ群:\href{https://qm.qq.com/q/4HL41Nt3sQ}{466863272}.
\section{活动清单}
\setbox\colbbox\vbox{
\makeatletter\col@number\@ne
\begin{longtable}{|>{\centering\arraybackslash}m{.3\textwidth}|m{.06\textwidth}|m{.06\textwidth}|}
    \hline
    活动 & 开展时间 & 刊载时间\\
    \hline\hline
    南大版deepseek & / & 2.22\\
    悦读课程群 & / & 2.24\\
    eScience AI科研助手 & / & 3.11\\
    地科博物馆开放安排 & / & 3.22\\ 
    2025年分流和转专业政策通知 & / & 4.7\\
    乐跑 & 5.16 & 3.10\\
    本科生劳育实践 & 7.20 & 2.19\\
    银星杯论文赛 & 4.22 & 2.27\\
    高教社杯 & 4.25 & 3.5\\
    大文大理题目征集 & 期末 & 3.8\\
    5月免费上网 & ? & 3.9\\
    基础学科论坛 & 4.20 & 3.9\\
    外教社杯 & 5.27 & 3.12\\
    江苏创青春赛事 & 4.30 & 3.26\\
    南大数学竞赛 & 4.15 & 3.27\\
    AI素养大赛 & 4.15 & 3.31\\
    浦口音乐跑 & 5.30 & 3.31\\
    红会暑期项目招募 & 4.12 & 4.1\\
    程设大赛 & 4.26 & 4.2\\
    春影摄影大赛 & 4.13 & 4.4\\
    瑞声杯 & 4.20 & 4.4\\
    仙林校区志愿法律咨询 & / & 4.4\\
    天健志愿者招募 & 4.13 & 4.4\\
    外新社征集春日影 & 4.13 & 4.5\\
    青春活力大赛 & 5.17 & 4.7\\
    在校生自愿体检 & 6.20 & 4.8\\
    南大定向赛报名 & 4.13 & 4.8\\
    数智应用大赛 & 4.16 & 4.9\\
    南大购买WPS & / & 4.8\\
    手链DIY & 4.13 & 4.10\\
    24级程设大赛 & 4.27 & 4.11\\
    新生午餐会报名 & 4.13 & 4.11\\
    南书房支教招新 & 4.15 & 4.11\\
    法治情景剧策划大赛 & 4.23 & 4.11\\
    仙林猫鼠游戏 & 4.19 & 4.12\\
    \hline
\end{longtable}
\unskip
\unpenalty
\unpenalty}\unvbox\colbbox
\end{multicols}
\begin{multicols}{2}
\pagebreak

\section{讲座}
\begin{tabular}{|>{\centering\arraybackslash}m{.3\textwidth}|m{.06\textwidth}|m{.06\textwidth}|}
    \hline
    讲座 & 开展时间 & 刊载时间\\
    \hline\hline
    黑色素瘤防治科普讲座 & 4.13 & 4.8\\\hline
    社交媒体分享实践的语用学研究 & 4.18 & 4.9\\\hline
    基于归纳程序合成的算法自动应用 & 4.15 & 4.10\\\hline
    面向推荐大模型的参数存储系统研究 & 4.15 & 4.10\\\hline
    REUSE IN SWITZERLAND-2 PILOT PROHJECTS BY BAUBÜRO IN SITU & 4.14 & 4.10\\\hline
    特朗普2.0与中美日关系 & 4.16 & 4.10\\\hline
    Deepseek现象中的管理学 & 4.18 & 4.10\\\hline
    智能时代的中国式养老:理论与实践”学术研讨会 & 4.18-20 & 4.10\\\hline
    从感知到疗愈:人脑音乐加工机制 & 4.25 & 4.11\\\hline
    Russian observatories for extragalactic research & 4.14 & 4.11\\\hline
    Accretion-generated rings:coplaner and polar structures & 4.18 & 4.11\\\hline
    Ionizing spotlight of Active Galactic Nucleus & 4.23 & 4.11\\\hline
    用法律知识为职场保驾护航 & 4.16 & 4.12\\\hline
\end{tabular}
%讲座预告写在这。用subsection
\subsection{“职场新生必修课”第一讲,用法律知识为职场保驾护航}
本期活动特别邀请法学院周长征副教授、校友赵可阳律师开展法律知识科普讲座,系统化解答求职过程中可能遇到的法律问题,强化毕业生对三方协议法律效力、劳动权益保障、租赁合同审查等核心内容的认知,降低因信息不对称导致的职场风险,助力南大学子在职业发展道路上实现从“学法”到“知法” “用法”的能力跃升。
\\活动时间:4月16日(周三)19:00
\\活动地点:线下:南京大学浦口校区 浦Ⅱ-101;线上:(腾讯会议)943-459-004
\\参与方式:扫码获取活动群聊二维码,后续所有活动的详细信息将在群内发布\textasciitilde{}
\\详见:\url{https://mp.weixin.qq.com/s/bTEnanz-T6IDUvpI1_-arA}
\section{猫鼠游戏}
4月19日,南京大学仙林校区,“智趣猫鼠,灵动南雍”进化版猫鼠游戏这场智慧与速度的较量等你来战!参与者将化身 "老鼠"、"猫咪"或 "猎猫人",通过策略躲藏、团队协作、技能进化,展开一场智慧与体能的较量!
\\活动时间:2025年4月19日
\\第一场:14:00-15:30(13:45集合)
\\第二场:15:30-17:00(15:15集合)
\\活动地点:南京大学仙林校区
\\参与人数:每场限65人,两场共130席,先到先得!
\\报名方式:扫码填写问卷即视为报名。报名成功后,工作人员会与您联系。
\\活动亮点、游戏规则详见推文。
\\
\\详见:\url{https://mp.weixin.qq.com/s/a2F0traL6k5nnpkikIJrsg}

\section{关于4月13日仙林校区校园交通管控的通知}
“2025南京仙林半程马拉松”定于2025年4月13日(本周日)上午7:30在羊山公园举行,其中迷你跑参赛人数2000人,终点设置在南京大学仙林校区第一运动场(炜华运动场)。为保障活动顺利举行,比赛期间仙林校区部分校门和校内道路将采取临时交通管控措施,具体通知如下:
\\1.交通管控校门及道路:西门南口-问渠路南侧道路-远东大道南段道路-第一运动场(炜华运动场)。东门正常通行,南门正常通行,西门北口只出不进。
\\交通管控时间:
\\6:30--9:20 禁止机动车辆驶入赛道
\\7:00--9:30 禁止非机动车辆及行人穿越赛道
\\9:30-- 禁止机动车辆驶入赛道
\\3.在符合校内交通管控的基础上,机动车可从东门、南门进、出校园;从西门北口驶出校园。非机动车和行人可从东门、南门进、出口进出校园,西门北口出校园。
\\请大家提前规划出行路线和时间,配合现场工作人员的指挥和疏导,注意安全。上述措施可能给师生员工工作、学习及生活带来不便,感谢理解和支持
\\详见:\url{https://mp.weixin.qq.com/s/FgzE_heIK5Fbb3h2QIOmsQ}
\section{院级活动}
\begin{tabular}{|>{\centering\arraybackslash}m{.3\textwidth}|m{.06\textwidth}|m{.06\textwidth}|}
\hline
    活动 & 开展时间 & 刊载时间\\
    \hline\hline
    文院剧本创作研讨会 & 9.30 & 3.2\\
    物院征集课程指南 & 6.15 & 3.3\\
    地海征集春日影 & 6.15 & 3.14\\
    社院学术节 & 4.18 & 3.25\\
    五院运动会 & 4.13 & 3.31\\
    五院乒乓球赛 & 4.19 & 3.31\\
    建城影展征集 & 4.16 & 3.31\\
    法院党建征文 & 5.20 & 4.2\\
    地学乒赛 & 4.19 & 4.2\\
    匡计社商联谊 & 4.13 & 4.2\\
    软院征集 & 4.20 & 4.4\\
    地学趣运会 & 4.26 & 4.9\\
    四院音乐节 & 5.11 & 4.7\\
    商院征集 & 5.5 & 4.8\\
    毓秀征集 & 4.13 & 4.8\\
    毓秀羽球 & 4.20 & 4.8\\
    大气设计 & 4.18 & 4.8\\
    文院诗歌 & 4.18 & 4.8\\
    化院摄影 & 4.15 & 4.9\\
    毓秀宿舍 & 4.16 & 4.10\\
    社院访企 & 4.16 & 4.11\\
    信管诗会 & 4.14 & 4.12\\
    物院运动打卡 & 5.14 & 4.12\\
    大气留学分享会 & 4.15 & 4.12\\
    地学定向越野 & 4.19 & 4.12\\
    \hline
\end{tabular}

\subsection{信管AI诗会}
投稿要求:(1)本次诗会分为原创组和AI组两个赛道,每个赛道内设新诗和旧体诗两个类别。(2)诗歌主题:围绕“青春、担当、创新”主题创作,鼓励内容与作者所就读的专业相结合。(3)诗歌体裁:新诗的作品长度需控制在两行至八行(含)之间;旧体诗则限定在60字以内,须遵循传统的格律和押韵规则。原创赛道禁止使用AI工具,严禁代笔。AI赛道需使用国内大模型,投稿时需注明生成指令和AI工具。
\\投稿时间:4月14日 18:00之前
\\投稿方式、评选方式、奖项设置详见推文。
\\详见:\url{https://mp.weixin.qq.com/s/cXK2w7fu4wXOOAQffdroxQ}

\subsection{挥洒汗水,不负春日 | 物理学院2025春季运动打卡活动马上开启!}
活动时间:2025年4月15日至2025年5月14日(共30天)
\\活动对象:物理学院全体师生
\\打卡项目:跑步、游泳、球类运动、健身器械、徒步等
\\打卡方式:加入我们的活动群,在群内发布的打卡链接进行打卡,要求填写姓名、运动项目及上传相关图片证明。
\\打卡要求:
\\1. 跑步:一天内里程数大于1.5公里的APP截图,配速要求小于8min/km,需要附上步数信息(建议使用运动APP或手环进行记录)
\\2. 游泳:游泳馆票据照片/泳道自拍,也可上传运动手环的记录信息哦!
\\3. 球类运动和健身器械等:运动后的本人照片!(举个铁也要记得微笑呀)
\\4. 徒步:单次5公里以上的运动路线APP截图,林荫道or城市步道都欢迎!
\\5. 运动请在当日24点前打卡(过期不可补卡),让新鲜汗水及时存档!图片证明需附带时间信息,具体要求请进群查看。每日打卡成功可解锁1积分(最多得1分),每15天工作人员统计一次打卡情况,并及时公布在运动打卡群中。5月17日前(最后一个打卡日后2天内)公布总排名及获奖情况。本次活动的解释权归物理学院研究生会所有。
\\详见:\url{https://mp.weixin.qq.com/s/g7JWxLpDz-ByVM58GUJOWw}

\subsection{欢迎报名丨电子学院携手新东方留学咨询持续进行中}
电子学院公益新东方留学咨询,有意请进入原文加群
\\详见:\url{https://mp.weixin.qq.com/s/bN4d1ooPptFrhCRkZ0827w}

\subsection{榜样“留”声,“研”途起航丨留学申请专题分享会}
时间地点
\\时间:4月15日 12点30
\\地点: 大气科学学院院楼D103
\\分享会主讲人及分享内容等信息详见原推https://mp.weixin.qq.com/s/zZ1\_QB4pRTEvmZC\_pySV7g
\\详见:\url{https://mp.weixin.qq.com/s/zZ1_QB4pRTEvmZC_pySV7g}


\subsection{“探索地学密码,拼出科学之魂”丨地学文化节“GeoSeek” 趣味定向越野}
【活动时间】
\\4 月 19 日(周六)8:30-12:00
\\【活动地点】
\\仙林校区(10 个打卡点分布详见活动当天发放地图)
\\【参与对象】
\\地学院系同学,组队参赛(每队 3 人,至少包含 1 位外院系同学)
\\【活动目标】
\\让身体与自然共运动,用协作体验团队精神,通过趣味活动理解学科交叉融合
\\报名方式及奖励见原推https://mp.weixin.qq.com/s/ahHjOGrLFocKrsVFXCKqNg
\\详见:\url{https://mp.weixin.qq.com/s/ahHjOGrLFocKrsVFXCKqNg}

\section{社团活动}
\begin{tabular}{|>{\centering\arraybackslash}m{.3\textwidth}|m{.06\textwidth}|m{.06\textwidth}|}
    \hline
    社团活动 & 开展时间 & 刊载时间\\
    \hline\hline
    天文台开放日 & / & 1.6\\
    重唱诗歌奖征稿 & 4.30 & 3.31\\
    足协体验 & 4.15 & 4.1\\
    轮滑社体验 & 4.17 & 4.1\\
    拳击社体验 & 4.22 & 4.1\\
    轮滑社体验 & 4.22 & 4.1\\
    飞盘大赛 & 4.13 & 4.1\\
    五子棋大赛 & 4.13 & 4.1\\
    定向赛 & 4.20 & 4.1\\
    体育舞蹈教学 & 4.25 & 4.1\\
    吉他社歌手招募 & 4.20 & 4.4\\
    吉他社春日音 & 4.26 & 4.4\\
    国学社寄明信片 & 4.14 & 4.4\\
    五子棋大赛 & 4.13 & 4.8\\
    歌魅放映 & 4.13 & 4.8\\
    新火星放映 & 4.13 & 4.10\\
    心协小圆桌 & 4.13 & 4.10\\
    \hline
\end{tabular}
%这里是写社团活动的,社团活动就是由社团主办、主要针对社团内部人员的活动。不要把非社团活动写在这里。
\subsection{院系杯辩论半决赛赛事预告}
赛事预告
\\1.辩题:人活一辈子/几瞬间
\\时间:14:00-16:00
\\地点:地科报告厅
\\对阵双方:化生辩论队 VS 电子辩论队
\\南京信息工程大学辩论队 沈煜恒
\\南京师范大学辩论队 高毛苏阳
\\南京大学辩论队副队长 范杨阳
\\南京大学辩论队副队长 赵家栋
\\南京大学辩论队成员 于同舟
\\2.辩题:缺爱/不缺爱的人更懂得如何爱
\\时间:16:00-18:00
\\地点:地科报告厅
\\对阵双方:工试辩论队VS 人文辩论队
\\评委:
\\南京师范大学辩论队 高毛苏阳
\\南京林业大学辩论队 王昇洋
\\南京大学辩论队队长 孟政屹
\\南京大学辩论队副队长 赵家栋
\\南京大学辩论队成员 黄靖雯
\\详见:\url{https://mp.weixin.qq.com/s/37YPrNEETO_UdtgRbveZYQ}


\end{multicols}
\end{document}
