% HEAD BEGIN
\documentclass[letterpaper, 12pt]{article}
\newsavebox\colbbox
\usepackage{graphicx}
\usepackage{multicol}
\usepackage{anysize}
\usepackage{fontspec}
\usepackage[fontset=none]{ctex}
\usepackage{tabularx}
\usepackage{longtable}
\PassOptionsToPackage{hyphens}{url}
\usepackage[breaklinks=true, colorlinks=true]{hyperref}
\expandafter\def\expandafter\UrlBreaks\expandafter{\UrlBreaks\do\a\do\b\do\c\do\d\do\e\do\f\do\g\do\h\do\i\do\j\do\k\do\l\do\m\do\n\do\o\do\p\do\q\do\r\do\s\do\t\do\u\do\v\do\w\do\x\do\y\do\z\do\A\do\B\do\C\do\D\do\E\do\F\do\G\do\H\do\I\do\J\do\K\do\L\do\M\do\N\do\O\do\P\do\Q\do\R\do\S\do\T\do\U\do\V\do\W\do\X\do\Y\do\Z}
% \let\oldurl\url
% \renewcommand{\url}[1]{\begin{sloppypar}\oldurl{#1}\end{sloppypar}}
\setlength\columnsep{30pt}
\marginsize{30pt}{30pt}{10pt}{20pt}
\setmainfont{TeX Gyre Bonum}
\setCJKmainfont[BoldFont=Noto Serif CJK SC Bold, ItalicFont=FandolKai]{Noto Sans CJK SC}
\setlength{\parindent}{0cm}
% \setCJKmonofont{Noto Sans CJK SC}
\begin{document}
\begin{center}
    \Huge\textbf{南哪大专醒前消息}
\end{center}
\vspace{4mm}
\hrule
\renewcommand\tabularxcolumn[1]{m{#1}}
\begin{tabularx}{\textwidth}{>{\hsize.2\hsize}X>{\hsize.6\hsize}X>{\hsize.2\hsize}X}
    \begin{flushleft}
        2025.1.9\, No.160
    \end{flushleft}
    &
    \begin{center}
        \textit{“秉中持正、求新博闻。”}
    \end{center}
    &
    \begin{flushright}
        \textbf{南京市栖霞区}
    \end{flushright}
\end{tabularx}
\vspace{-3.5mm}
\hrule
\vspace{4mm}
% HEAD END
\centerline{\huge\textbf{活动预告}}
\begin{multicols}{2}
    \section{订阅方式和加入编辑部}  
编辑部招聘人才,用爱发电,工作轻松,详情可联系QQ:1329527951 客服小祥\\想订阅本消息或获取PDF版(便于查看超链接和往期),可加QQ群:\href{https://qm.qq.com/q/VXIW7fgsEe}{849644979}.
\section{Deadline Ongoing}
\setbox\colbbox\vbox{
\makeatletter\col@number\@ne
\begin{longtable}{|c|c|c|}
    \hline
    消息(未见ddl的,不刊) & 截止日期 & 刊载日期\\
    \hline\hline
    安邦征稿 & 1.12 & 11.16\\
    仙林通宵自习室 & 1.12 & 11.26\\
    全国大学生家史大赛 & 1.31 & 12.2\\
    西安史学论坛征稿 & 3.20 & 12.9\\
    本科评教 & 1.12 & 12.13\\
    12306学生优惠票 & 2.12 & 12.13\\
    期末考试安排 & 1.12 & 12.17\\
    南大博物馆展览 & 6.16 & 12.17\\
    排超志愿者招募 & 1.16 & 12.25\\
    南星小红书创作 & 2.6 & 12.27\\
    挂职干部选拔 & 2.13 & 12.31\\
    ASC25报名 & 2.21 & 1.6\\
    天文台开放日 & / & 1.6\\
    开甲书院科研作坊 & 2.17 & 1.6\\
    PL读书会 & 2.12 & 1.9\\
    寒假社会实践立项 & 1.13 & 1.9\\
    春季选课 & 1.20 & 1.9\\
    \hline
\end{longtable}
\unskip
\unpenalty
\unpenalty}\unvbox\colbbox
\end{multicols}
\hrule
\pagebreak
\begin{multicols}{2}

\section{讲座}
\begin{tabularx}{0.5\textwidth}{|X|X|X|}
    \hline
    讲座 & 开展时间 & 刊载时间\\
    \hline\hline
    《移动计算与社会治理》 & 1.15 & 1.6\\\hline 

\end{tabularx}
\section{2025年春季学期本科课程选课通知}
1.春季学期课程选课时间:1月20日10:00-2月7日23:00开通选课,抽签结果将于2月14日发布。\\
 2.2025年1月17日公布2024-2025学年第二学期课表。\\
 3.春季学期的本科课程测评正在进行,请同学们在1月12日前完成测评,测评完成后才能进行选课。\\
选课范围、注意事项详见原文。\\

\section{关于组织开展2025年寒假学生社会实践活动的通知}
2025年寒假社会实践通知已经发布在南大青年官网(\url{https://tuanwei.nju.edu.cn/36/3d/c24691a734781/page.htm}),有意向参加寒假社会实践的同学可加入25寒假社会实践咨询群(QQ群号:1012412263)。

\section{PL读书会}
从25年春季学期开始,PLaX研究组(\url{https://plax-lab.github.io/})将举办面向大二、大三本科生的PL读书会,欢迎对PL理论感兴趣的同学报名。具体计划如下:

计划读的第一本书:Robert Harper's PFPL (Practical Foundations of Programming Languages)

读书形式:每周线下聚会(1至2小时),同学们轮流讲解书中章节,自由讨论书中内容

举办条件:报名不少于5人

报名要求:二、三年级本科生,在仙林,对PL理论感兴趣

退出机制:除了有讲解任务的同学外,其他同学可以随时自由退出

报名方式:实名加入QQ群1006577416

报名截止日期:2月12日

本读书会无学分,也无任何世俗奖励。

有疑问请联系梁红瑾老师,邮箱:

\href{mailto:hongjin@nju.edu.cn}{hongjin@nju.edu.cn}
\section{Lean4 工作坊}
北京国际数学研究中心 AI4MATH 课题组的徐天一老师将于 1月14日至1月17日每日上午9:00-12:00于鼓楼校区西大楼308举行 AI4Math 工作坊

内容包括 Lean 简介,类型论入门和进阶,以及 AI for formalized math 等话题

报名将于1月10日12:00截止。报名链接:\url{https://wenjuan.nju.edu.cn/vm/wWck4v1.aspx}

\url{http://jw.nju.edu.cn/3a/5b/c26263a735835/page.htm}
\section{排超赛程预告 | 1月12日排超季后赛}
南京广电猫猫VS广东台山和力\\
时间:1月12日19:30\\
地点:南京大学方肇周体育馆\\

\end{multicols} 

\end{document}