% HEAD BEGIN
\documentclass[letterpaper, 12pt]{article}
\newsavebox\colbbox
\usepackage{graphicx}
\usepackage{multicol}
\usepackage{anysize}
\usepackage{fontspec}
\usepackage[fontset=none]{ctex}
\usepackage{tabularx}
\usepackage{longtable}
\PassOptionsToPackage{hyphens}{url}
\usepackage[breaklinks=true, colorlinks=true]{hyperref}
\expandafter\def\expandafter\UrlBreaks\expandafter{\UrlBreaks\do\a\do\b\do\c\do\d\do\e\do\f\do\g\do\h\do\i\do\j\do\k\do\l\do\m\do\n\do\o\do\p\do\q\do\r\do\s\do\t\do\u\do\v\do\w\do\x\do\y\do\z\do\A\do\B\do\C\do\D\do\E\do\F\do\G\do\H\do\I\do\J\do\K\do\L\do\M\do\N\do\O\do\P\do\Q\do\R\do\S\do\T\do\U\do\V\do\W\do\X\do\Y\do\Z}
% \let\oldurl\url
% \renewcommand{\url}[1]{\begin{sloppypar}\oldurl{#1}\end{sloppypar}}
\setlength\columnsep{30pt}
\marginsize{30pt}{30pt}{10pt}{20pt}
\setmainfont{TeX Gyre Bonum}
\setCJKmainfont[BoldFont=Noto Serif CJK SC Bold, ItalicFont=FandolKai]{Source Han Sans SC}
\setlength{\parindent}{0cm}
% \setCJKmonofont{Noto Sans CJK SC}
\begin{document}
\begin{center}
    \Huge\textbf{南哪大专醒前消息}
\end{center}
\vspace{4mm}
\hrule
\renewcommand\tabularxcolumn[1]{m{#1}}
\begin{tabularx}{\textwidth}{>{\hsize.2\hsize}X>{\hsize.6\hsize}X>{\hsize.2\hsize}X}
    \begin{flushleft}
        2025.3.24\, No.200
    \end{flushleft}
    &
    \begin{center}
        \textit{“秉中持正、求新博闻。”}
    \end{center}
    &
    \begin{flushright}
        \textbf{南京市栖霞区}
    \end{flushright}
\end{tabularx}
\vspace{-3.5mm}
\hrule
\vspace{4mm}
% HEAD END
\centerline{\huge\textbf{活动预告}}
\begin{multicols}{2}
\section{订阅方式和加入编辑部}  
编辑部招聘人才,用爱发电,工作轻松,详情可联系QQ:1329527951 客服小千\\想订阅本消息或获取PDF版(便于查看超链接和往期),可加QQ群:\href{https://qm.qq.com/q/4HL41Nt3sQ}{466863272}.
\section{活动清单}
\setbox\colbbox\vbox{
\makeatletter\col@number\@ne
\begin{longtable}{|>{\centering\arraybackslash}m{.3\textwidth}|m{.06\textwidth}|m{.06\textwidth}|}
    \hline
    活动 & 开展时间 & 刊载时间\\
    \hline\hline
    南大版deepseek & / & 2.22\\
    悦读课程群 & / & 2.24\\
    eScience AI科研助手 & / & 3.11\\
    地科博物馆开放安排 & / & 3.22\\ 
    乐跑 & 5.16 & 3.10\\
    本科生劳育实践 & 7.20 & 2.19\\
    医保零星报销 & 3.31 & 2.19\\
    银星杯论文赛 & 4.22 & 2.27\\
    高教社杯 & 4.25 & 3.5\\
    南辩院系杯 & 4.12 & 3.6\\
    大文大理题目征集 & 期末 & 3.8\\
    5月免费上网 & ? & 3.9\\
    基础学科论坛 & 4.20 & 3.9\\
    普通话测试 & 3.28 & 3.12\\
    外教社杯 & 5.27 & 3.12\\
    心理中心全媒体招新 & 3.25 & 3.14\\
    Python比赛 & 4.6 & 3.16\\
    扎染志愿者招募 & 3.28 & 3.18\\
    中美中心开放日 & 3.26 & 3.19\\
    本科生院征集大鸣大放 & 4.4 & 3.21\\
    两会知识竞赛 & 3.30 & 3.21\\
    纸鸢工作坊 & 4.3 & 3.22\\
    南大博篆刻体验课 & 4.2 & 3.23\\
    粤歌赛 & 4.12 & 3.24\\
    博士博后招聘会 & 3.26 & 3.24\\
    \hline
\end{longtable}
\unskip
\unpenalty
\unpenalty}\unvbox\colbbox
\end{multicols}
\begin{multicols}{2}
\pagebreak

\section{讲座}
\begin{tabular}{|>{\centering\arraybackslash}m{.3\textwidth}|m{.06\textwidth}|m{.06\textwidth}|}
    \hline
    讲座 & 开展时间 & 刊载时间\\
    \hline\hline
    Vinaya Revival on Baohua Mountain in Ming–Qing China & 3.25 & 3.18 \\\hline
    What Can Ecological Spatiotemporal Indicators Tell Us about the Resilience to Economic Crisis & 3.25 & 3.20\\\hline
    Eigenvector Spatial Filtering in Areal and Origin-Destination Data & 3.25 & 3.20\\\hline
    智能时代下文科何为 & 3.25 & 3.21\\\hline
    基层治理现代化语境下的居民社区责任 & 3.25 & 3.21\\\hline
    分数量子霍尔态的相变以及围绕任意子的密度波的特征 & 3.27 & 3.21\\\hline
    A Geometric Perspective on the Compressible Euler... & 3.26 & 3.22\\\hline
    铜氧化物超导体中配对密度波态和强电子 & 3.25 & 3.22\\\hline
    无导数优化简介 & 3.26 & 3.23\\\hline
    南新读书会 & 3.26 & 3.23\\\hline
    “研途星光·朋辈领航”国奖对话会 & 3.25 & 3.23\\\hline
    镍氧化物压力下的高温超导电性研究 & 3.27 & 3.24\\\hline
    数字化转型时代的科研创新 & 3.28 & 3.24\\\hline
    作为过程的资本与苦难批判 & 3.25 & 3.24\\\hline
    抽象劳动与货币价值论 & 3.26 & 3.24\\\hline
    国家、资本与社会 & 3.28 & 3.24\\\hline
    大语言模型的逻辑基础与研究应用 & 3.27 & 3.24\\\hline
    用数学的眼光观察世界 & 3.25 & 3.24\\\hline
\end{tabular}
%讲座预告写在这。用subsection
\subsection{镍氧化物压力下的高温超导电性研究}
报告人:王猛,中山大学
\\时间:2025年3月27日(周四)15:30
\\地点:鼓楼校区唐仲英楼B501
\\详见:\url{https://mp.weixin.qq.com/s/fPaD4rvCDG3J49QvGLq7Nw}

\subsection{CSAI 卓越科学家大讲堂}
主题:数字化转型时代的科研创新
\\报告人: 梅宏 北京大学
\\时间:3月28日(星期五)14:00
\\地点:计算机科学技术楼111室
\\摘要:报告以“数字经济时代正在开启”为切入,指出数字化转型已成为不可逆的时代趋势,人类社会已经站在信息社会的门口,即将迈向数字文明。进而在回顾梳理科学研究范式变迁的基础上,探讨了在社会经济全面转型大背景下,如何拥抱科研范式的新变迁,开展大数据和人工智能赋能的科研创新。
\\详见:\url{https://mp.weixin.qq.com/s/upRa2_9y-4TsyumsQCTHxA}


\subsection{讲座预告 | 南京大学马克思主义学院国际学者讲座(第18-20期)}
第一场:作为过程的资本与苦难批判
\\时间:3月25日18:30
\\地点:薛光林楼402
\\第二场:抽象劳动与货币价值论
\\时间:3月26日18:30
\\地点:薛光林楼402
\\第三场:国家、资本与社会
\\时间:3月28日18:30
\\地点:薛光林楼402
\\详见:\url{https://mp.weixin.qq.com/s/Zz0kEtzxvqeAjEC-nUDy8w}

\subsection{DeepSeek时代的自我赋能:大语言模型的逻辑基础与研究应用}
主讲人:马志浩
\\时间:2025年3月27日(周四)19:30-21:00
\\本讲座仅可线上参与。讲座介绍和线上参与方法见公众号推文
\\详见:\url{https://mp.weixin.qq.com/s/V8oGCHY5EMv4JAfVBrWQZA}

\subsection{学术专栏 | 用数学的眼光观察世界}
从冬奥会火炬台谈起,讲阿基米德密铺。从2025年中央台春节晚会的题头片谈起,讲对称,包括从运动的观点看对称,正四面体的对称群。
\\主讲人:林亚南
\\地点:鼓楼校区教-201
\\时间:2025年3月25日下午14:00
\\详见:\url{https://mp.weixin.qq.com/s/zYb5ushSvpxI9WunfJl8jQ}
%校级活动写在这。用section
\section{第十三届粤语歌唱大赛初赛}
时间:2025.4.12星期六 下午2:30
\\地点:大活多功能厅(仙林校区)
\\对象:南京大学全体师生
\\报名方式:请加QQ群 818124640 并填写报名问卷
\\欢迎南大在校学生届时观赛
\\详见:\url{https://mp.weixin.qq.com/s/jKr7U4uHOnQSvRKeV_V4aA}

\section{学术英语服务中心第五期·学术英语线上学习课程}
学术英语服务中心第五期即将开启。本期我们将继续安排9周服务,包括两场讲座(每场一小时)和七场一对一咨询(每场两小时)。
\\为了更好地服务于我校四校区的联动,我们将继续采用线上服务的方式:预约成功后,线上会议参与方式将会在服务开始前一周发送至预约人员邮箱。
\\同时推出“学术英语写作”线上课程:课程选用 “智慧树平台”开设,课程号:K2729537,课程内容涵盖了学术写作与批判性写作的标准和技巧、写作及文献参考的工具使用等,欢迎有需求的同学前往观看学习。
\\本期服务的具体主题及时间,详见推文,请大家按需预约,欢迎点击文章最下方小程序加入。
\\详见:\url{https://mp.weixin.qq.com/s/hFzcJF7pShhFAqlgR5dVyA}


\section{2025年春季博士、博士后人才全国巡回招聘会南大站}
举办时间:2025年3月26日(周三)14:00-17:30
\\举办地点:南京大学 · 仙林校区方肇州体育馆武术馆
\\主办单位:高校人才网、广州高才信息科技有限公司
\\参会单位:全国各人社厅/局、高校、科研院所、医院、企事业单位等有高层次人才引进需求的单位(持续新增中,名单见后)。
\\参会人才:面向南京及周边地区的海内外应往届博士毕业生、具有博士学位的社会在职人才、博士后出站人员等。
\\报名方式等信息见微信推文
\\详见:\url{https://mp.weixin.qq.com/s/gfxYv-MuU-EyHFL2D_9qag}

\section{第十九届“挑战杯”竞赛志愿者团队系列宣传物料征集中}
征集时间:即日起至2025年4月20日止(以邮件送达时间为准)。
\\投稿方式:填写问卷,请填写《著作权确认书》,与作品共同打包命名为“姓名-学号-作品类别-志愿者物料征集”。
\\具体要求:见链接
\\详见:\url{https://mp.weixin.qq.com/s/UQ2N7emPjnv9ensALMlWSg}

\section{华为2026届实习生招聘校园宣讲会(南京大学)}
时间:3月26日(星期三)18:30-20:00
\\地点:仙林校区-图书馆报告厅
\\报名方式:进入原文扫码报名 \url{https://mp.weixin.qq.com/s/lrk70UPR2UT60JVDLqTB-w}
\\详见:\url{https://mp.weixin.qq.com/s/lrk70UPR2UT60JVDLqTB-w}
%此处写校级活动,请不要把讲座、院级活动和社团活动写在这里orz orz orz

\section{院级活动}
\begin{tabular}{|>{\centering\arraybackslash}m{.3\textwidth}|m{.06\textwidth}|m{.06\textwidth}|}
\hline
    活动 & 开展时间 & 刊载时间\\
    \hline\hline
    文院剧本创作研讨会 & 9.30 & 3.2\\
    物院征集课程指南 & 6.15 & 3.3\\
    地海征集春日影 & 6.15 & 3.14\\
    AI院影色舞 & 3.29 & 3.19\\
    商院羽球 & 3.29 & 3.19\\
    史院就业 & 3.25 & 3.21\\
    美团NJUAI专场空宣会 & 3.25 & 3.21\\
    物院访企 & 3.28 & 3.22\\
    法院主题餐会 & 3.28 & 3.23\\
    大气求职 & 3.27 & 3.24\\
    
    \hline
\end{tabular}
%这里是写院级活动的,院级活动就是只限某院学生参加的活动,和由某院某部门主办、主要针对某院学生的活动。不要把对全校学生开放的活动写在这里。
\subsection{第十九届地学文化节系列活动之“气运宏图,地造‘职’梦”系列活动}
大气科学学院\&南赫学院专场求职经验分享会
\\时间:3月27日18:30
\\地点:大气科学学院院楼D103
\\生涯咨询室
\\3月26日和4月2日,开启你的职业探索之旅
\\生涯导师一对一咨询(第四期)将于3月23日正式开放预约,欢迎大家报名
\\详见:\url{https://mp.weixin.qq.com/s/FNGlXGDfCdEnxCGl8AAJ0g}
\section{社团活动}
\begin{tabular}{|>{\centering\arraybackslash}m{.3\textwidth}|m{.06\textwidth}|m{.06\textwidth}|}
    \hline
    社团活动 & 开展时间 & 刊载时间\\
    \hline\hline
    天文台开放日 & / & 1.6\\
    鸿新社捐书活动 & 3.30 & 3.17\\
    长歌行声演剧 & 3.29 & 3.19\\
    雁行南大x以伴云陪伴线上志愿者招募 & 3.30 & 3.24\\
    \hline
\end{tabular}
%这里是写社团活动的,社团活动就是由社团主办、主要针对社团内部人员的活动。不要把非社团活动写在这里。

\subsection{二剧帮推 | “牛首山·南京大学生青春戏剧季”征集启募}
竞演主题
\\以「戏剧“话”金陵」为主题,征集作品,题材不限,鼓励多元化创作
\\参赛要求
\\参赛作品必须是舞台戏剧作品,形式不限(如话剧、音乐剧、戏曲、肢体作品、影像互动等),但须符合下列条件:
\\01演出时长为20分钟至30分钟之间
\\02参赛者可使用由组委会提供的简单桌椅,但不得自行制作或携带其他大型道具
\\03每部参赛剧目在比赛之前进行的准备工作,包含装台、排练及彩排,舞台使用须严格依照组委会的安排进行
\\04剧目全体人员均需为全日制高校在校生(本科、硕士、博士不限),高校教师与校外人士不得以主创或演员身份参与竞演,仅可以指导教师身份呈现
\\日程安排
\\01线上征集:
\\征集时间:即日起至2025年4月15日13:00
\\征集方式:线上投递《作品征集报名表》《参赛承诺书》、剧本及完整演出视频(视频请压缩至1GB大小以内,填写网盘链接),至组委会邮箱:xjhjl2025@163.com
\\初选结果公布:2025年4月17日公布入围复赛的12部作品
\\02校园复赛:
\\4月20日(周日)10:00-18:00
\\复赛(一)【南京艺术学院·黄瓜园剧场】
\\4月26日(周六)10:00-18:00
\\复赛(二)【南京传媒学院·钟鼓剧场】
\\校园复赛分为南京艺术学院、南京传媒学院两大赛区,复赛前一日校内进行彩排,各赛区由复赛评审团现场评审6部作品,每个赛区评审结束后先由复赛评审团根据打分选取赛区内第一名进入决赛,两大赛区全部结束后再由复审评审团推选第3部作品进入牛首山展演与决赛环节,其余未进入决赛的9部复赛入围剧目获得“星火奖”
\\03牛首山展演:
\\5月10日、5月11日、5月17日14:00-14:30/15:00-15:30
\\入围决赛的3部作品在南京牛首山文化旅游区分别选一天展演2场,并开放现场游客投票评选出一组“最具人气奖”
\\04牛首山决赛:
\\5月25日(周日)13:30-16:00【南京牛首山文化旅游区·多功能厅】
\\前一日参赛剧组技术合成及彩排
\\05
\\颁奖典礼:
\\   5月25日(周日)17:00-18:00【南京牛首山文化旅游区·多功能厅】
\\详见:\url{https://mp.weixin.qq.com/s/z_-dCnMoS5r0KyGZMdKErA}

\subsection{熊熊快报 | 第六周社团活动}
南大社团公众号的新栏目“熊熊快报”!每周一,小熊记者都会在本栏目为大家预告本周内的社团活动。下面就和小熊一起浏览第六周的社团活动吧!
\\详见:\url{https://mp.weixin.qq.com/s/bN2ICtcBqo5xi-kLjkjI6Q}
\end{multicols}

\end{document}
