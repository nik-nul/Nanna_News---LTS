% HEAD BEGIN
\documentclass[letterpaper, 12pt]{article}
\newsavebox\colbbox
\usepackage{graphicx}
\usepackage{multicol}
\usepackage{anysize}
\usepackage{fontspec}
\usepackage[fontset=none]{ctex}
\usepackage{tabularx}
\usepackage{longtable}
\PassOptionsToPackage{hyphens}{url}
\usepackage[breaklinks=true, colorlinks=true]{hyperref}
\expandafter\def\expandafter\UrlBreaks\expandafter{\UrlBreaks\do\a\do\b\do\c\do\d\do\e\do\f\do\g\do\h\do\i\do\j\do\k\do\l\do\m\do\n\do\o\do\p\do\q\do\r\do\s\do\t\do\u\do\v\do\w\do\x\do\y\do\z\do\A\do\B\do\C\do\D\do\E\do\F\do\G\do\H\do\I\do\J\do\K\do\L\do\M\do\N\do\O\do\P\do\Q\do\R\do\S\do\T\do\U\do\V\do\W\do\X\do\Y\do\Z}
% \let\oldurl\url
% \renewcommand{\url}[1]{\begin{sloppypar}\oldurl{#1}\end{sloppypar}}
\setlength\columnsep{30pt}
\marginsize{30pt}{30pt}{10pt}{20pt}
\setmainfont{TeX Gyre Bonum}
\setCJKmainfont[BoldFont=Noto Serif CJK SC Bold, ItalicFont=FandolKai]{Source Han Sans SC}
\setlength{\parindent}{0cm}
% \setCJKmonofont{Noto Sans CJK SC}
\begin{document}
\begin{center}
    \Huge\textbf{南哪大专醒前消息}
\end{center}
\vspace{4mm}
\hrule
\renewcommand\tabularxcolumn[1]{m{#1}}
\begin{tabularx}{\textwidth}{>{\hsize.2\hsize}X>{\hsize.6\hsize}X>{\hsize.2\hsize}X}
    \begin{flushleft}
        2025.5.10\, No.241
    \end{flushleft}
    &
    \begin{center}
        \textit{“秉中持正、求新博闻。”}
    \end{center}
    &
    \begin{flushright}
        \textbf{南京市栖霞区}
    \end{flushright}
\end{tabularx}
\vspace{-3.5mm}
\hrule
\vspace{4mm}
% HEAD END
\centerline{\huge\textbf{活动预告}}
\begin{multicols}{2}
\section{订阅方式和加入编辑部}  
编辑部招聘人才,用爱发电,工作轻松,详情可联系QQ:1329527951 客服小千\\想订阅本消息或获取PDF版(便于查看超链接和往期),可加QQ群:\href{https://qm.qq.com/q/4HL41Nt3sQ}{466863272}.
\section{活动清单}
\setbox\colbbox\vbox{
\makeatletter\col@number\@ne
\begin{longtable}{|>{\centering\arraybackslash}m{.3\textwidth}|m{.06\textwidth}|m{.06\textwidth}|}
    \hline
    活动 & 开展时间 & 刊载时间\\
    \hline\hline
    南大版deepseek & / & 2.22\\
    悦读课程群 & / & 2.24\\
    eScience AI科研助手 & / & 3.11\\
    地科博物馆开放安排 & / & 3.22\\ 
    2025年分流和转专业政策通知 & / & 4.7\\
    2025年转专业志愿填报通知 & / & 4.24\\
    仙林校区志愿法律咨询 & / & 4.4\\
    南大购买WPS & / & 4.8\\
    关于毛概习概实践教学的要求 & / & 5.6\\
    乐跑 & 5.16 & 3.10\\
    本科生劳育实践 & 7.20 & 2.19\\
    外教社杯 & 5.27 & 3.12\\
    浦口音乐跑 & 5.30 & 3.31\\
    青春活力大赛 & 5.17 & 4.7\\
    在校生自愿体检 & 6.20 & 4.8\\
    中美中心2025年证书项目 & 5.24 & 4.14\\
    校博岩画展 & 6.22 & 4.23\\
    CASHL“畅读”活动 & 5.23 & 4.24\\
    江苏高校凤凰读书节 & 6.15 & 4.24\\
    港澳台生征文 & 6.20 & 5.3\\
    挑杯征集 & 5.25 & 5.8\\
    国际大学生创新大赛 & 5.18 & 5.9\\
    金箔体验课 & 5.13 & 5.10\\
    计算理论之美报名 & 6.14 & 5.11\\
    星光集市摊主招募 & 5.24 & 5.11\\
    \hline
\end{longtable}
\unskip
\unpenalty
\unpenalty}\unvbox\colbbox
\end{multicols}
\begin{multicols}{2}
\pagebreak

\section{讲座}
\begin{tabular}{|>{\centering\arraybackslash}m{.3\textwidth}|m{.06\textwidth}|m{.06\textwidth}|}
    \hline
    讲座 & 开展时间 & 刊载时间\\
    \hline\hline
    儿童脑智发育与人口神经科学 & 5.14 & 4.30\\\hline
    2026巴黎高科国际工程师项目宣讲会 & 5.15 & 5.6\\\hline
    中国北方地区清洁取暖的政策演变与可持续发展路径研究 & 5.14 & 5.8\\\hline
    Error propagation in Data-Driven Multi-Period Inventory System & 5.13 & 5.9\\\hline
    Publishing in Top Tier-Journals: An Asian Scholar’s Perspective & 5.13 & 5.9\\\hline
    技术与人类思考的交汇 & 5.14 & 5.9\\\hline
    中央银行数字货币与银行 & 5.16 & 5.9\\\hline
    一些女性创业者相关的课题简介 & 5.16 & 5.9\\\hline
    Platform Competition and App Development & 5.16 & 5.9\\\hline
    On infinite series with summands involving binomial coefficients & 5.14 & 5.9\\\hline
    德国留学宣讲会 & 5.13 & 5.10\\\hline
    “含英咀华“新生午餐读书会 & 5.14 & 5.10\\\hline
    下一代材质渲染:从物理精确性到神经表达的融合探索 & 5.14 & 5.11\\\hline
    谭铁牛:“拥抱人工智能,争做时代新人” & 5.15 & 5.11\\\hline
    人工智能与哲学谘商 & 51.4 & 5.11\\\hline
    儿童脑智发育与人口神经科学 & 5.14 & 5.11\\\hline
    Freedom in Nature & 5.16 & 5.11\\\hline
    \end{tabular}
\subsection{下一代材质渲染:从物理精确性到神经表达的融合探索} % 讲座 describer: nik_nul
时间:2025年5月14日(星期三) 15:00
\\地点:计算机科学技术楼229室
\\主讲人:朱君秋, 研究员 山东大学
\\详见:\url{https://mp.weixin.qq.com/s/qblBNjbInpXaQmw3Ds0CVQ}

\subsection{“拥抱人工智能,争做时代新人”} % 讲座 describer: Ando
主讲人:谭铁牛
\\时间:5月15日(周四)19:30-21:00
\\微信扫码观看直播
\\详见:\url{https://mp.weixin.qq.com/s/aKfROf4Wezczax7g76_R8w}

\subsection{5.12-5.16(周一\textasciitilde{}周五)讲座概览} % 讲座 describer: zty
周一(5.12)
\\Error propagation in Data-Driven Multi-Period Inventory System
\\周二(5.13)
\\Publishing in Top Tier-Journals: An Asian Scholar’s Perspective
\\周三(5.14)
\\1.人工智能与哲学谘商:技术与人类思考的交汇
\\2.儿童脑智发育与人口神经科学
\\3.On infinite series with summands involving binomial coefficients
\\周五(5.16)
\\1.Platform Competition and App Development
\\2.Central Bank Digital Currency and Banking: Literature Review and New Questions
\\3.一些女性创业者相关的课题简介
\\详见原文
\\详见:\url{https://mp.weixin.qq.com/s/y7Mj1PQpmWyrxgZSjAKDCw}

\subsection{讲座预告 | Marius Mjaaland:Freedom in Nature} % 讲座 describer: charlors
时间:2025年5月16日(周五)上午9点
\\地点:仙林校区哲学学院218
\\主讲人:Marius Mjaaland(挪威奥斯陆大学教授)
\\
\\详见:\url{https://mp.weixin.qq.com/s/QiHVkOfeJO_BxV65Xxz4pg}


    
\section{计算理论之美 (Summer 2025)} % 校级活动 describer: nik_nul
时间:2025年6月30日至7月3日
\\地点:计算机科学与技术楼 111报告厅
\\联系邮箱:nju\_tcs@163.com
\\联系人:黄棱潇:huanglingxiao@nju.edu.cn
\\班吟:175227530@qq.com
\\注册报名流程:由于前几届报名人数过多,本届将会采用问卷初筛形式进行报名。请同学们先填写以下调查问卷:
\\https://table.nju.edu.cn/external-apps/1d0c9688-b1e7-4d58-a21f-c5fc37910f5f
\\通过初筛后的同学将提前两周收到包含报名链接的邮件。 南大同学仅需填写问卷,讲习班期间直接到会场听报告即可。
\\费用:本科生包住宿,非本科生不包含住宿。餐费自理
\\报名截止日期: 2025.6.14
\\6月29日布置作业, 7月2日晚上12点截止提交。成绩优秀的同学可推荐参加南京大学计算机学院本科生开放日保研面试,并在录取后可凭借此次暑期讲习班申请两个研究生学分。
\\作业提交方式:发送电子版到邮箱nju\_tcs@163.com
\\详见:\url{https://tcs.nju.edu.cn/wiki/index.php?title=%E8%AE%A1%E7%AE%97%E7%90%86%E8%AE%BA%E4%B9%8B%E7%BE%8E_(Summer_2025)}

\section{校庆社团巡礼月晚会暨星光集市摊主招募} % 校级活动 describer: Ando
时间:2025年5月24日
\\地点:南京大学鼓楼校区苏浙体育场
\\晚会开设60个各具特色的摊位,宛如一条流淌着青春灵感的星河——“小蓝鲸”的创意、社团的奇思、学院(书院)的匠心、校企联动的巧艺,还有本次特别设立的“挑战杯”摊位等待解密,这些摊位交融成流动的盛宴。
\\星光集市预设20个左右个人摊位,现面向南京大学全体同学征集摊主,活动方将为各位摊主免费提供场地和摊位硬件等,请各位同学自行准备商品,自负盈亏。欢迎各位摊主售卖手工艺品等既能展现摊主个人特色又能突显NJUer全面发展的可爱潮酷商品。
\\报名途径:摊位报名时间从即日起至5月13日(下周二)24:00 ,有意报名个人摊位的同学请扫描二维码填写申请表,心动不如行动,感兴趣的各位快来报名吧!
\\详见:\url{https://mp.weixin.qq.com/s/sXcbQHlMoaYKiDFHaUZhDg}

\section{南有嘉鱼丨5月学科竞赛信息速递} % 校级活动 describer: charlors
一、第七届·2025 MindSpore量子计算黑客松全国大赛
\\报名时间:2025年5月20日 18:00前
\\参赛要求:
\\1.对量子计算感兴趣,具备基本Python编程能力的所有开发者;
\\2.选手可自行组队,每个团队最多3人报名,指导老师1人(指导老师无需报名);
\\3.每位选手仅可加入一个团队,可以跨专业、跨校、跨地域组队。
\\二、2025中国高校计算机大赛——智能交互创新赛
\\报名时间:2025年5月18日截止
\\报名要求:
\\1.面向全球高校学生,专业不限、年级不限;参赛队员须为高等学校在册、在校生;
\\2.每支参赛队伍至多由1名队长、3名队员组成,不允许跨校组队;
\\3.每支参赛队伍须有一名参赛队所属高校正式教师担任指导老师;
\\
\\详见:\url{https://mp.weixin.qq.com/s/l-WrfRoxH-YuZw_oDk8xNg}

\section{团体招募|唤醒内在的超级英雄——自我赋能团体} % 校级活动 describer: charlors
《唤醒内在的超级英雄》心理团辅招募
\\活动时间:5月17日、5月24日、5月31日晚 18:00
\\活动地点:南大仙林校区
\\招募人数:南大全体学生 10-12人(为了关注到团体内每个人)
\\带领者:韩奇桐,南大心理系在读研究生,心理中心见习咨询师,心理中心热线咨询师
\\详见:\url{https://mp.weixin.qq.com/s/uLsyPAPV0g-pi-szscs_RA}

\section{团体招募 | 我与“我”的相遇——自我成长之旅} % 校级活动 describer: charlors
南大心理中心特别策划——“自我觉察、接纳与成长”团体心理辅导,
\\招募人数:15人,限南京大学学生
\\团体时间:5月14日、5月17日、5月18日下午14:00-15:30
\\报名见原文
\\
\\详见:\url{https://mp.weixin.qq.com/s/-k1VddNa7rIH9yD3U_n6UA}

\section{院级活动}
\begin{tabular}{|>{\centering\arraybackslash}m{.3\textwidth}|m{.06\textwidth}|m{.06\textwidth}|}
\hline
    活动 & 开展时间 & 刊载时间\\
    \hline\hline
    文院剧本创作研讨会 & 9.30 & 3.2\\
    物院征集课程指南 & 6.15 & 3.3\\
    地海征集春日影 & 6.15 & 3.14\\
    法院党建征文 & 5.20 & 4.2\\
    物院运动打卡 & 5.14 & 4.12\\
    健雄摄影 & 5.20 & 4.30\\
    数院文创 & 5.31 & 5.3\\
    电院求职 & 5.15 & 5.3\\
    法院征集 & 5.31 & 5.6\\
    法专生沙龙 & 5.10 & 5.6\\
    商院竞赛 & 5.14 & 5.9\\
    地海求职 & 5.13 & 5.9\\
    开甲行知假面舞会 & 5.16 & 5.10\\
    秉文安邦联谊 & 5.18 & 5.10\\
    秉文手工 & 5.17 & 5.10\\
    计院参观 & 5.15 & 5.11\\
    新传读书 & 5.14 & 5.11\\
    \hline
\end{tabular}
\subsection{计算机学院OPPO南京办公室参观} % 院级活动 describer: nik_nul
活动时间:2025年5月15日(周四)14:30-17:00
\\参与对象:南京大学计算机学院学生 
\\参观流程
\\14:30-15:00 OPPO公司介绍
\\15:00-16:00 OPPO AI技术交流分享
\\16:00-16:45 校友交流
\\16:45-17:00 办公室环境参观
\\详见:\url{https://mp.weixin.qq.com/s/hKsOaR3LVJL5-7Q_G3MqOw}

\subsection{南新读书会} % 院级活动 describer: nik_nul
时间:2025年5月14日(周三)19:00
\\地点:南京大学新闻传播学院311室
\\《总体屏幕:从电影到智能手机》 分享人:刘敏心 2024级硕士研究生
\\《时间与他者》 分享人:闫炜炜 2024级硕士研究生
\\《Introduction à l'Anthropologie 》 分享人:董翟 南京大学21级哲学系本科生
\\详见:\url{https://mp.weixin.qq.com/s/yJdStoKbj8IMWYZGnD8n4g}

\section{社团活动}
\begin{tabular}{|>{\centering\arraybackslash}m{.3\textwidth}|m{.06\textwidth}|m{.06\textwidth}|}
    \hline
    社团活动 & 开展时间 & 刊载时间\\
    \hline\hline
    天文台开放日 & / & 1.6\\
    红会一块走 & 5.20 & 4.21\\
    歌魅演出 & 5.23 & 5.9\\
    passion演出 & 5.16 & 5.9\\
    拉丁舞社周年庆 & 5.19 & 5.10\\
    \hline
\end{tabular}
%这里是写社团活动的,社团活动就是由社团主办、主要针对社团内部人员的活动。不要把非社团活动写在这里。

\end{multicols}
\end{document}
