% HEAD BEGIN
\documentclass[letterpaper, 12pt]{article}
\newsavebox\colbbox
\usepackage{graphicx}
\usepackage{multicol}
\usepackage{anysize}
\usepackage{fontspec}
\usepackage[fontset=none]{ctex}
\usepackage{tabularx}
\usepackage{longtable}
\PassOptionsToPackage{hyphens}{url}
\usepackage[breaklinks=true, colorlinks=true]{hyperref}
\expandafter\def\expandafter\UrlBreaks\expandafter{\UrlBreaks\do\a\do\b\do\c\do\d\do\e\do\f\do\g\do\h\do\i\do\j\do\k\do\l\do\m\do\n\do\o\do\p\do\q\do\r\do\s\do\t\do\u\do\v\do\w\do\x\do\y\do\z\do\A\do\B\do\C\do\D\do\E\do\F\do\G\do\H\do\I\do\J\do\K\do\L\do\M\do\N\do\O\do\P\do\Q\do\R\do\S\do\T\do\U\do\V\do\W\do\X\do\Y\do\Z}
% \let\oldurl\url
% \renewcommand{\url}[1]{\begin{sloppypar}\oldurl{#1}\end{sloppypar}}
\setlength\columnsep{30pt}
\marginsize{30pt}{30pt}{10pt}{20pt}
\setmainfont{TeX Gyre Bonum}
\setCJKmainfont[BoldFont=Noto Serif CJK SC Bold, ItalicFont=FandolKai]{Noto Sans CJK SC}
\setlength{\parindent}{0cm}
% \setCJKmonofont{Noto Sans CJK SC}
\begin{document}
\begin{center}
    \Huge\textbf{南哪大专醒前消息}
\end{center}
\vspace{4mm}
\hrule
\renewcommand\tabularxcolumn[1]{m{#1}}
\begin{tabularx}{\textwidth}{>{\hsize.2\hsize}X>{\hsize.6\hsize}X>{\hsize.2\hsize}X}
    \begin{flushleft}
        2024.12.16\, No.146
    \end{flushleft}
    &
    \begin{center}
        \textit{“秉中持正、求新博闻。”}
    \end{center}
    &
    \begin{flushright}
        \textbf{南京市栖霞区}
    \end{flushright}
\end{tabularx}
\vspace{-3.5mm}
\hrule
\vspace{4mm}
% HEAD END
\centerline{\huge\textbf{活动预告}}
\begin{multicols}{2}
    \section{订阅方式和加入编辑部}  
编辑部招聘人才,用爱发电,工作轻松,详情可联系QQ:1329527951 客服小祥\\想订阅本消息或获取PDF版(便于查看超链接和往期),可加QQ群:\href{https://qm.qq.com/q/VXIW7fgsEe}{849644979}.
\section{Deadline Ongoing}
\setbox\colbbox\vbox{
\makeatletter\col@number\@ne
\begin{longtable}{|c|c|c|}
    \hline
    消息(未见ddl的,不刊) & 截止日期 & 刊载日期\\
    \hline\hline
    安邦征稿 & 1.12 & 11.16\\
    创意物理实验竞赛 & 12.21 & 11.15\\
    仙林通宵自习室 & 1.12 & 11.26\\
    全国大学生家史大赛 & 1.31 & 12.2\\
    金融消费者大赛 & 12.31 & 12.5\\
    花旗杯报名 & 1.3 & 12.6\\
    西安史学论坛征稿 & 3.20 & 12.9\\
    重唱英文评诗会 & 12.21 & 12.10\\
    叶嘉莹纪念征稿 & 12.25 & 12.10\\
    粤语课堂 & 12.22 & 12.11\\
    五院迎新晚会 & 12.21 & 12.12\\
    普通话测试报名 & 12.24 & 12.12\\
    本科评教 & 1.12 & 12.13\\
    排协雪球杯 & 12.28 & 12.13\\
    心协暖冬歌会 & 12.21 & 12.13\\
    12306学生优惠票 & 2.12 & 12.13\\
    高研院午餐会 & 12.18 & 12.14\\
    AI 朋辈就业分享 & 12.18 & 12.14\\
    校园迷你马拉松报名 & 12.20 & 12.14\\
    南北大联合读书会 & 12.19 & 12.15\\
    新传迎新晚会 & 12.21 & 12.15\\
    南新读书会 & 12.18 & 12.15\\
    外语社团联谊活动 & 12.21 & 12.15\\
    歌魅音乐会 & 12.22 & 12.15\\
    希音编程竞赛 & 12.21 & 12.15\\
    药石杯生化歌赛 & 12.22 & 12.15\\
    交响乐团室内乐 & 12.20 & 12.15\\
    南大新年音乐会 & 12.19 & 12.16\\
    flicker影映 & 12.21 & 12.16\\
    \hline
\end{longtable}
\unskip
\unpenalty
\unpenalty}\unvbox\colbbox
\end{multicols}
\hrule
\pagebreak
\begin{multicols}{2}

\section{讲座}
\begin{tabular}{|c|c|c|}
    \hline
    往期讲座 & 开展日期 & 刊载日期\\
    \hline\hline
《中国马克思主义...》& 12.17 & 12.16\\
《杨万里对苏轼诗...》 & 12.27 & 12.14\\
《现代化进程中的...》 & 12.17 & 12.11\\
《专利查新与规避...》 & 12.19 & 10.3\\
《前向推理的一阶...》 & 12.18 & 12.10\\
《非平衡任意子边...》 & 12.17 & 12.13\\
《Microlocal...》 & 12.17 & 12.15\\
《计算复杂性下界...》 & 12.19 & 12.16\\
《二维半导体中的...》 & 12.19 & 12.16\\
《On global...》 & 12.18 & 12.16\\
《Designing...》 & 12.19 & 12.16\\
《Mechanism...》 & 12.17 & 12.16\\
物院学术交流会 & 12.21 & 12.16\\
    \hline
\end{tabular}

1.计算复杂性下界的反推数学Reverse mathematics of complexity lower bounds\\
报告人:李嘉图 麻省理工学院博士生\\
时间:12月19日(星期四)10:00\\
腾讯会议:415154825\\

2.物理学院学术报告会(第45期)\\
题目:二维半导体中的光-物质相互作用\\
报告人:谭平恒 中国科学院半导体研究所研究员\\
时间:12月19日(周四)15:30\\
地点:鼓楼校区唐仲英楼B501\\

3.On global dynamics of 3-D irrotational compressible fluids\\
报告人:汪倩 牛津大学教授\\
时间:12月18日(周三)16:30\\
地点:鼓楼校区西大楼308\\

4.Designing for Digital Fabrication Strategies for Adoption in Construction Industry\\
报告人:陈倩 助理教授\\
主持人:刘琰 助理教授\\
时间:12月19日(周四)10:30-12:00\\
地点:工程管理学院北楼101\\

5.Mechanism Design for Exchange Market\\
报告人:程郁琨 教授\\
主持人:占杨 助理教授\\
时间:12月17日(周二)10:30-12:30\\
地点:协鑫楼206\\

4.中国马克思主义科技观的理论逻辑与中国式现代化科技强国的实践逻辑\\
主讲人:陈凡 (东北大学马克思主义学院教授)\\
时间:2024年12月17日(星期二) 14:00\\
地点:哲学学院(薛光林楼)212室\\

5.物理学院午间学术交流会第九期\\
时间:12月21日(周六)11:30-13:00\\
地点:新教405\\
报告人1:李添悦 王漱明教授课题组2019级硕博研究生\\
报告主题:Versatile Metaphotonic Spin-Orbit Interactions(超构光子学中的多元化自旋-轨道相互作用研究)\\
报告人2:何黎伟 李建新教授课题组2019级硕博研究生\\
报告主题:Fractional magnetization plateau and spinon quantum spin Hall state in kagome antiferromagnets(笼目反铁磁体中的分数磁化平台和自旋子量子自旋霍尔态)\\
报告人3:严文翔 丁剑平教授课题组2021级博士研究生\\
报告主题: Generation, Control, and Applications of Iso-Propagation Vortex Beams(涡旋光场等传输特性的调控及应用)\\
报名方式及连接请见推送:\url{https://mp.weixin.qq.com/s/H5r_8EamEm8asKQkw8-Twg/}\\


\section{新年音乐会}
演出时间:12月19日(周四)19:00\\
演出地点:仙林校区恩玲剧场\\
领票方式:\\
仙林校区:\\
12月17日12:00 - 13:30\\
仙林校区敬文学生活动中心一楼大厅\\
其他校区:\\
扫描二维码领票,二维码在\url{https://mp.weixin.qq.com/s/ZgMVcAMQmmGOlar3yf_B9g}\\
注意事项:\\
南京大学师生可凭校园卡或学生证领票,\\
一人一证一票,不可代领。\\
数量有限,领完为止,先到先得!\\
线上领票的同学请于12月19日17:45前\\
在恩玲剧场凭填写截图领票,\\
一人一票,不可代填。\\

\section{“暖冬相伴,物耀星光”2024年度物理学院师生联欢晚会}
时间:2024年12月21日19:00\\
地点:南京大学(鼓楼校区)科技馆一楼报告厅\\
以下为领票攻略\\