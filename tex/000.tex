% HEAD BEGIN
\documentclass[letterpaper, 12pt]{article}
\usepackage{graphicx}
\usepackage{multicol}
\usepackage{anysize}
\usepackage{fontspec}
\usepackage[fontset=none]{ctex}
\usepackage{tabularx}
\PassOptionsToPackage{hyphens}{url}
\usepackage[breaklinks=true, colorlinks=true]{hyperref}
\expandafter\def\expandafter\UrlBreaks\expandafter{\UrlBreaks\do\a\do\b\do\c\do\d\do\e\do\f\do\g\do\h\do\i\do\j\do\k\do\l\do\m\do\n\do\o\do\p\do\q\do\r\do\s\do\t\do\u\do\v\do\w\do\x\do\y\do\z\do\A\do\B\do\C\do\D\do\E\do\F\do\G\do\H\do\I\do\J\do\K\do\L\do\M\do\N\do\O\do\P\do\Q\do\R\do\S\do\T\do\U\do\V\do\W\do\X\do\Y\do\Z}
% \let\oldurl\url
% \renewcommand{\url}[1]{\begin{sloppypar}\oldurl{#1}\end{sloppypar}}
\setlength\columnsep{30pt}
\marginsize{30pt}{30pt}{10pt}{20pt}
\setmainfont{TeX Gyre Bonum}
\setCJKmainfont[BoldFont=Noto Serif CJK SC Bold, ItalicFont=FandolKai]{Noto Sans CJK SC}
\setlength{\parindent}{0cm}
% \setCJKmonofont{Noto Sans CJK SC}
\begin{document}
\begin{center}
    \Huge\textbf{南哪大专醒前消息}
\end{center}
\vspace{4mm}
\hrule
\renewcommand\tabularxcolumn[1]{m{#1}}
\begin{tabularx}{\textwidth}{>{\hsize.2\hsize}X>{\hsize.6\hsize}X>{\hsize.2\hsize}X}
    \begin{flushleft}
        2024.9.28\, No.73
    \end{flushleft}
    &
    \begin{center}
        \textit{“克明峻德。”}
    \end{center}
    &
    \begin{flushright}
        \textbf{南京市栖霞区}
    \end{flushright}
\end{tabularx}
\vspace{-3.5mm}
\hrule
\vspace{4mm}
% HEAD END
\centerline{\huge\textbf{活动预告}}
\begin{multicols}{2}
\section{编辑部招聘人才}
编辑部招聘人才,用爱发电,工作轻松,详情可联系QQ:1329527951 客服小祥\\想订阅本消息或获取PDF版(便于查看超链接),可加QQ群:\href{https://qm.qq.com/q/FGX1VYCrGS}{962626571}.
\section{Deadline Ongoing}
\begin{tabular}{|c|c|c|}
    \hline
    消息(未见ddl的,不刊) & 截止日期 & 刊载日期\\
    \hline\hline
    仙林校史馆招募讲解员 & 10.30 & 9.12\\
    国优计划报名 & 10.7 & 9.19\\
    本科生暑期课程评教 & 10.31 & 9.19\\
    网易雷火大赛 & 10.7 & 9.22\\
    大创训练计划申报 & 11.18 & 9.24\\
    苏州校区音乐会 & 10.19 & 9.25\\
    外院国庆摄影征集 & 10.7 & 9.25\\
    历史学院新疆项目 & 9.30 & 9.25\\
    多阅志愿者招募 & 10.1 & 9.26\\
    留学生汉语朗诵比赛 & 9.29 & 9.26\\
    雨花成长计划课堂报名 & 10.3 & 9.26\\
    音乐思政大课堂 & 9.29 & 9.26\\
    港澳台生中华文化大赛 & 10.9 & 9.26\\
    心理中心征稿 & 10.10 & 9.28\\
    周末剧场 & 10.10 & 9.28\\
    秉文书院国庆歌会 & 9.29 & 9.28\\
    历史学院国庆活动 & 10.7 & 9.28\\
    \hline
\end{tabular}

\section{讲座}
\begin{tabular}{|c|c|c|}
    \hline
    往期讲座 & 开展日期 & 刊载日期\\
    \hline\hline
   《作为批评的文学史》 & 9.29 & 9.25\\
   《透过情感基础设...》 & 9.30 & 9.28\\
      \hline
\end{tabular}\\\\
1.透过情感基础设施思考青年友好城市与社区建设

主讲人:香港大学城市规划及设计系主任Lady Edith Kotewall

主持人:南京大学建筑与城市规划学院陈浩 副教授

时间:2024年9月30日(周一)上午 9:15

地点:南京大学鼓楼校区  建良楼304室
\section{关于2024东京大学南京大学联合学习项目通知}
具体链接:\url{https://mp.weixin.qq.com/s/ormFXX6GlmOTc9PHnm8zuQ}

\section{周末剧场}
“青年不止是当下的年轻人,也是人的过去和未来。《青年史》是关于人怎样成长起来的历史”——赵川
演出时间

10 月 10 日(周四) 19:30

10 月 11 日(周五) 19:30





演出地点:南京大学仙林校区敬文学生活动中心三楼黑匣子剧场

演出时长:约 90 分钟

报名详见:\url{https://mp.weixin.qq.com/s/nFip0Idxv2PFbcwVieIvOA}
\section{心理中心新学期征稿}
征稿要求:讲述自己与勇敢迈出的那一步的真实故事,可以分享期间的经历、思考、咨询或治疗体验、受到的帮助和支持、印象深刻的一两件事等……

征稿日期截至2024年10月10日。

来稿要求:

1.原创,真实。

2.100—500字左右皆可

3.文章下方附上名字(不必真实),一个自己的头像(不必真实),简短的自我介绍。

投稿见:\url{https://mp.weixin.qq.com/s/LgmTXVlQhR1aIrbmtY7OPA}


\section{历史学院国庆系列活动}
活动一:颂祖国,寄深情\\
同学们可以在线填写电子表单或者录制短视频,表达对祖国的热爱、赞美和美好祝福,也可以选取饱含爱国情结的历史摘句。\\活动时间:即日起至2024年9月30日18点。参与方式:请将作品上传至下方链接,命名方式为学号+姓名+参与活动名称。\\\url{https://box.nju.edu.cn/u/d/dda7a18c24ff4a19aa75/}
活动二:扬国旗,显崇高\\
敬意国旗,是国家尊严与荣誉的象征。同学们可以上传与国旗的合影(或者与国庆有关的照片),表达对祖国深厚且崇高的敬意。活动时间:即日起至2024年10月7日24点。参与方式:请将作品上传至下方链接,命名方式为学号+姓名+参与活动名称。\url{https://box.nju.edu.cn/u/d/e453675f51f948e78462/}
活动三:踏山河,献青春\\
礼赞国庆假期是探索祖国大好河山的绝佳时机。我们征集以“我用脚步丈量河山”为主题的vlog、摄影作品、短视频。同时鼓励同学们附上大约100字左右的解释说明。活动时间:即日起至2024年10月7日24点。参与方式:请将作品上传至下方链接,命名方式为学号+姓名+参与活动名称。\url{https://box.nju.edu.cn/u/d/feb1452956444152bc13}
活动四,忆往昔,述国庆\\
收集同学们自己或者家人与国庆有关的故事。活动时间:即日起至2024年10月7日24点。参与方式:请将作品上传至下方链接,命名方式为学号+姓名+参与活动名称。\url{https://box.nju.edu.cn/u/d/cbdd4ab9dff04f8dad90/}
\\每项活动将评选出若干一二三等奖,赠送学院的精美文创产品。
\section{秉文书院国庆主题歌会}
秉文书院将于9月29日下午16:00至18:00,于南京大学科技馆报告厅举办“青春逐梦 奋进当歌”主题歌会。
\\活动时间:9月29日(周日)
\\活动地点:鼓楼校区科技馆报告厅
\\活动对象:南京大学秉文书院全体师生
\end{multicols} 


\end{document}