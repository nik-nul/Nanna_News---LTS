% HEAD BEGIN
\documentclass[letterpaper, 12pt]{article}
\newsavebox\colbbox
\usepackage{graphicx}
\usepackage{multicol}
\usepackage{anysize}
\usepackage{fontspec}
\usepackage[fontset=none]{ctex}
\usepackage{tabularx}
\usepackage{longtable}
\PassOptionsToPackage{hyphens}{url}
\usepackage[breaklinks=true, colorlinks=true]{hyperref}
\expandafter\def\expandafter\UrlBreaks\expandafter{\UrlBreaks\do\a\do\b\do\c\do\d\do\e\do\f\do\g\do\h\do\i\do\j\do\k\do\l\do\m\do\n\do\o\do\p\do\q\do\r\do\s\do\t\do\u\do\v\do\w\do\x\do\y\do\z\do\A\do\B\do\C\do\D\do\E\do\F\do\G\do\H\do\I\do\J\do\K\do\L\do\M\do\N\do\O\do\P\do\Q\do\R\do\S\do\T\do\U\do\V\do\W\do\X\do\Y\do\Z}
% \let\oldurl\url
% \renewcommand{\url}[1]{\begin{sloppypar}\oldurl{#1}\end{sloppypar}}
\setlength\columnsep{30pt}
\marginsize{30pt}{30pt}{10pt}{20pt}
\setmainfont{TeX Gyre Bonum}
\setCJKmainfont[BoldFont=Noto Serif CJK SC Bold, ItalicFont=FandolKai]{Source Han Sans SC}
\setlength{\parindent}{0cm}
% \setCJKmonofont{Noto Sans CJK SC}
\begin{document}
\begin{center}
    \Huge\textbf{南哪大专醒前消息}
\end{center}
\vspace{4mm}
\hrule
\renewcommand\tabularxcolumn[1]{m{#1}}
\begin{tabularx}{\textwidth}{>{\hsize.2\hsize}X>{\hsize.6\hsize}X>{\hsize.2\hsize}X}
    \begin{flushleft}
        2025.3.30\, No.206
    \end{flushleft}
    &
    \begin{center}
        \textit{“秉中持正、求新博闻。”}
    \end{center}
    &
    \begin{flushright}
        \textbf{南京市栖霞区}
    \end{flushright}
\end{tabularx}
\vspace{-3.5mm}
\hrule
\vspace{4mm}
% HEAD END
\centerline{\huge\textbf{活动预告}}
\begin{multicols}{2}
\section{订阅方式和加入编辑部}  
编辑部招聘人才,用爱发电,工作轻松,详情可联系QQ:1329527951 客服小千\\想订阅本消息或获取PDF版(便于查看超链接和往期),可加QQ群:\href{https://qm.qq.com/q/4HL41Nt3sQ}{466863272}.
\section{活动清单}
\setbox\colbbox\vbox{
\makeatletter\col@number\@ne
\begin{longtable}{|>{\centering\arraybackslash}m{.3\textwidth}|m{.06\textwidth}|m{.06\textwidth}|}
    \hline
    活动 & 开展时间 & 刊载时间\\
    \hline\hline
    南大版deepseek & / & 2.22\\
    悦读课程群 & / & 2.24\\
    eScience AI科研助手 & / & 3.11\\
    地科博物馆开放安排 & / & 3.22\\ 
    乐跑 & 5.16 & 3.10\\
    本科生劳育实践 & 7.20 & 2.19\\
    银星杯论文赛 & 4.22 & 2.27\\
    高教社杯 & 4.25 & 3.5\\
    南辩院系杯 & 4.12 & 3.6\\
    大文大理题目征集 & 期末 & 3.8\\
    5月免费上网 & ? & 3.9\\
    基础学科论坛 & 4.20 & 3.9\\
    普通话测试 & 4.11 & 3.25\\
    外教社杯 & 5.27 & 3.12\\
    Python比赛 & 4.6 & 3.16\\
    本科生院征集大鸣大放 & 4.4 & 3.21\\
    纸鸢工作坊 & 4.3 & 3.22\\
    南大博篆刻体验课 & 4.2 & 3.23\\
    粤歌赛 & 4.12 & 3.24\\
    江苏创青春赛事 & 4.30 & 3.26\\
    悦读测试 & 4.6 & 3.27\\
    南大数学竞赛 & 4.15 & 3.27\\
    AI素养大赛 & 4.15 & 3.31\\
    浦口音乐跑 & 5.30 & 3.31\\
    \hline
\end{longtable}
\unskip
\unpenalty
\unpenalty}\unvbox\colbbox
\end{multicols}
\begin{multicols}{2}
\pagebreak

\section{讲座}
\begin{tabular}{|>{\centering\arraybackslash}m{.3\textwidth}|m{.06\textwidth}|m{.06\textwidth}|}
    \hline
    讲座 & 开展时间 & 刊载时间\\
    \hline\hline
    如何影响消费者动机与心理健康支持 & 4.1 & 3.26\\\hline
    移民与流动研究的时间、情感和日常转向 & 4.1 & 3.28\\\hline
    AI: The Destruction of the Imagination? & 4.2 & 3.28\\\hline
    日军侵华时期“万人坑”遗址的记忆与忘却 & 4.2 & 3.31\\\hline
    论运气的模态分析 & 4.2 & 3.31\\\hline
    仁本论:孔子的行为科学 & 4.2 & 3.31\\\hline
    秦汉玺印人名考析 & 4.9 & 3.31\\\hline
\end{tabular}
%讲座预告写在这。用subsection
\subsection{仙林学术午餐会 王楠:日军侵华时期“万人坑”遗址的记忆与忘却}
发言人:王楠 高研院第21期驻院学者、马克思主义学院副教授
\\主持人:胡翼青 高研院副院长、新闻传播学院教授
\\与谈人:高研院第21期驻院学者 高研院2025年度访问学者 “悦读书社”同学
\\时间:2025年4月2日(周三)12:20开始
\\地点:仙林校区邵逸夫楼国际学院C308
\\欢迎感兴趣的老师和同学前来旁听,午餐请自理。
\\详见:\url{https://mp.weixin.qq.com/s/9xcZfEiipF4OExK0MtM7Fg}

\subsection{青年哲学家工作坊 | 何朝安:论运气的模态分析}
报告人:何朝安(东华大学)
\\时间:2025年4月2日周三下午2:30
\\地址:南京大学哲学院218哲学教研室
\\详见:\url{https://mp.weixin.qq.com/s/xDRpPjj2IdMHfiArK6rYBA}

\subsection{仁本论:孔子的行为科学}
报告人:俞宁 教授
\\主持人:杨学伟 教授
\\时间:4月2日(周三)16:00-18:00
\\地点:协鑫楼108
\\详见:\url{https://mp.weixin.qq.com/s/wLfSBiCYOEv99JuEDz9ezg}
\subsection{讲座预告 | 魏宜辉:秦汉玺印人名考析}
主题:秦汉玺印人名考析
\\主讲:魏宜辉 南京大学文学院副教授
\\主持:李林晋 南京大学文学院2021级本科生
\\时间:2025年4月9日(周三)19:00‐21:00
\\地点:南京大学文学院活水轩
\\详见:\url{https://mp.weixin.qq.com/s/acKRN6E3eTr2ge65F4msHw}
\section{饕餮大餐 | 4.1-4.3学术文化活动概览}
周二(4.1)
\\1.暖北极-冷欧亚模态的变化及其气候环境效应
\\2.“感传算一体化”赋能行业新质生产力
\\3.移民与流动研究的时间、情感和日常转向
\\4.数智时代的拟人化设计:如何影响消费者动机与心理健康支持
\\周三(4.2)
\\1.AI: The Destruction of the Imagination?
\\2.碳排放承诺的跨组织效应——基于国际供应链的证据
\\3.Doing and publishing high-quality research-personal experience and JMS perspective
\\4.大地的伤痕——日军侵华时期“万人坑”遗址的记忆与忘却
\\周四(4.3)
\\1.From Earnings Calls to Market Falls: The Predictive Power of Tone Surprises on Future Crash Risk
\\2.微纳世界的音乐之声–用原子尺度机电器件探索纳米世界
\\(其中部分活动,本消息亦有详载)
\\详见:\url{https://mp.weixin.qq.com/s/_oCdXHTAXwKhZCfVAFkTtQ}


\subsection{用原子尺度机电器件探索纳米世界}
题   目:微纳世界的音乐之声–用原子尺度机电器件探索纳米世界
\\报告人:王曾晖,电子科技大学
\\时   间:2025年4月3日(周四)14:30
\\地   点:鼓楼校区唐仲英楼B501
\\详见:\url{https://mp.weixin.qq.com/s/Sh-xB2a12HDKG1bjGf4VBg}


\subsection{南新读书会|下周预告}
本周南新读书会将于4月2日19:00在新闻传播学院311室举办,23硕刘静将分享威廉·弗卢塞尔《姿态:一种现象学实践》,23硕韩良弼将分享琼·罗宾逊《经济哲学》,欢迎全体师生参与。
\\详见:\url{https://mp.weixin.qq.com/s/lqY_pOnQFEWz6QbSxzFZqA}
%此处写校级活动,请不要把讲座、院级活动和社团活动写在这里orz orz orz
\section{温情关怀 | 南京大学开展云南地震受灾学生摸排和临时困难补助申请}
本科生:可登录“南京大学网上办事服务大厅——困难补助”模块(2024-2025学年)提交专项困难补助申请。
\\研究生:可登录“南京大学网上办事服务大厅——研-补助申请”模块提交专项困难补助申请。
\\如有任何问题请联系党委学生工作部、学生资助管理中心:
\\热线电话: 025-89680606(本科生)、025-89680029(研究生)。
\\详见:\url{https://mp.weixin.qq.com/s/0mJMwOCaAjWznPCwl5aXOA}

\section{浦口校区奋进音乐跑暨挑战接力赛}
活动时间:2025年4月2日—5月30日每天19:00—20:30
\\活动地点:南京大学浦口校区操场
\\活动、奖品及歌单等请见原推文
\\详见:\url{https://mp.weixin.qq.com/s/NGI01w-aw_8-dmpx__GCkQ}

\section{第十九届“挑战杯”竞赛志愿者培训第一期课程发布!}
1.培训时间:4.2(周三)16:00—18:00
\\2.培训地点:线上 腾讯会议 腾讯会议号:795-489-205 注意:请同学们使用南京大学腾讯会议账号(SSO登陆)参与培训。
\\3.培训内容:急救卫生培训
\\4.报名方式:扫码填写问卷报名
\\系列课程介绍
\\4.2(周三)16:00—18:00 线上 腾讯会议 急救卫生培训
\\4.9(周三)16:00—18:00 线上 腾讯会议 安全教育
\\4.13(周日)16:00—18:00 线上 腾讯会议 宣传技能培训
\\4.16(周三)16:00—18:00 线下,地点另行通知 心理团辅
\\4.23(周三)16:00—18:00 线上 腾讯会议 国赛赛制全流程介绍讲解
\\详见:\url{https://mp.weixin.qq.com/s/lw_YFzXW0iMjeOIUgYQ7Gg}

\section{人工智能信息素养知识大赛}
本活动由南京大学图书馆等单位主办,奖品与参与方式等请详阅链接。
\\时间:3.31-4.15
\\参赛对象:全体本科生
\\详见:\url{https://mp.weixin.qq.com/s/vPLa1WV8-7QtO5zHIR7PUQ}
\section{选拔报名通知 | 2025年本科生全球科考与科研训练项目“跨文化视角下‘世界记忆’与‘和平形象’构建”}
1. 报名条件:(1)南京大学2022、2023、2024级在读本科生,欢迎不同学科背景的同学报名;(2)政治立场坚定,在南大学习期间未受过任何纪律处分;(3)身心健康,具有集体意识与责任感,善于沟通表达,乐于合作奉献;(4)外语能力良好(英语/德语/丹麦语),学术基础扎实;(5)自媒体运营、摄影与视频制作、艺术策展等经验为加分项。
\\2. 报名方式:见链接
\\截止时间:2025年4月6日(星期日)23:59(以表格提交时间为准)
\\3. 选拔流程:预计录取15名本科生参加本项目。将根据候选人报名表进行首轮筛选,通过者参加面试,面试将评估学生思政、外语、技能、团队协作等综合能力,择优录取并公示。
\\4. 费用说明:根据《南京大学“本科生全球科考与科研训练项目”管理办法(试行)》(南本院〔2022〕51号)文件精神,该项目产生的费用由南京大学本科生院、外国语学院和学生个人共同承担。项目为入选学生在签证办理、国际旅行、境外交通及住宿等方面提供资助;根据经费情况和价格波动因素,参与该项目的学生预计需承担人均总费用的20\%-30\%。科考期间门票、伙食费、其他与本项目无关事项由学生自理。
\\5. 其它说明:2025年秋季学期8月25日开始,参加本次科考可能影响新学期第一周出勤,请同学们谨慎考虑,务必在报名前征得所在学院同意 。
\\详见:\url{https://mp.weixin.qq.com/s/kk9KibgvtoBl4-_y3PHf-w}
\section{院级活动}
\begin{tabular}{|>{\centering\arraybackslash}m{.3\textwidth}|m{.06\textwidth}|m{.06\textwidth}|}
\hline
    活动 & 开展时间 & 刊载时间\\
    \hline\hline
    文院剧本创作研讨会 & 9.30 & 3.2\\
    物院征集课程指南 & 6.15 & 3.3\\
    地海征集春日影 & 6.15 & 3.14\\
    社院学术节 & 4.18 & 3.25\\
    五院运动会 & 4.13 & 3.31\\
    电子南师春日交流 & 4.12 & 3.31\\
    五院乒乓球赛 & 4.19 & 3.31\\
    建城影展征集 & 4.16 & 3.31\\
    
    \hline
\end{tabular}
\subsection{活动预告 | 五院联谊运动会}
工管、国关、历史、软件、新传五院联谊运动会将于4月13日下午在炜华体育场举办,全校同学均可参与,详情请阅链接。
\\详见:\url{https://mp.weixin.qq.com/s/y4c-D_bq8cgtXSo030ciwg}


\subsection{电子学院与南师教科院春日交流活动}
活动时间:4月12日 13:30--18:00
\\活动地点:羊山公园东门入口处大草坪
\\活动人数:两院各40人
\\入场签到时间:13:30-14:00
\\内含多个活动以及茶话会。活动详情和报名方式见原推。
\\详见:\url{https://mp.weixin.qq.com/s/aDgdWieJA5OTEey7NZIKLQ}


\subsection{地学文化节系列活动丨大气、地学、地海、环境乒乓球赛}
活动时间与地点:4月19日14:00-18:00,南京大学仙林校区方肇周体育馆二楼乒乓球馆(开始前也可以对打练习)。
\\活动对象:大气、南赫、地学、地海、环院师生
\\奖项设置,报名方式等详见原推文。
\\详见:\url{https://mp.weixin.qq.com/s/pntuZpslUDjWtuaNtgLZ-Q}

\subsection{影展征集 | 四月雨韵}
四月,是雨的季节,是万物生长的乐章。本月影展以"雨之旋律"为主题,征集同学们在这个雨季里遇见的动人画面,通过摄影作品展现雨季的曼妙,传递另一种自然的情绪。
\\经筛选后入围的影像作品将于"NJU建城青年"公众号平台进行公开展览投票,作者可领取精美纪念品一份。期待大家踊跃投稿,用镜头捕捉雨的每一段旋律,每一幅画面,共同聆听这一季的清新与诗意!
\\征稿要求:1. 本期投稿作品需与"雨"相关,记录春天最打动你的雨天瞬间,每人仅限投稿一份;2. 每张/组作品应附有创作说明,分享这一瞬间背后的故事或感受,字数不限;3. 邮件主题处请填写"影展主题"(此次征集活动邮件主题请填写"雨之旋律");4. 不限作品形式,必须为本人原创;5. 提交图片为jpg或png格式,相片需要超过200万像素。每张大小控制在10MB以下。
\\征稿期限:即日起至4月16日晚24:00。
\\详见:\url{https://mp.weixin.qq.com/s/U5yqnBGM0rkWfdCk-CBN1g}
\section{社团活动}
\begin{tabular}{|>{\centering\arraybackslash}m{.3\textwidth}|m{.06\textwidth}|m{.06\textwidth}|}
    \hline
    社团活动 & 开展时间 & 刊载时间\\
    \hline\hline
    天文台开放日 & / & 1.6\\
    二剧招募 & 4.1 & 3.28\\
    重唱诗歌奖征稿 & 4.30 & 3.31\\
    \hline
\end{tabular}
%这里是写社团活动的,社团活动就是由社团主办、主要针对社团内部人员的活动。不要把非社团活动写在这里。
\subsection{男篮明日赛程}
4月1日
\\男篮院系杯小组赛
\\计科 vs 地科 
\\12 : 30 - 13 : 30
\\工程 vs 电子 
\\20 : 00 - 21 : 00
\\地点 : 一组团篮球场
\\详见:\url{https://mp.weixin.qq.com/s/9Ep14X13addDDUvuEQ3u8Q}
\subsection{征稿启事 | 第十届南京大学“重唱诗歌奖”}
参评对象:第十届 “重唱诗歌奖”参评对象为高校在读(专科、本科、硕士、博士)汉语新诗写作者;“重唱诗歌奖”已获奖者不得再次参评;诗歌奖评委不得参评。
\\时间安排
\\征稿时间:2025年4月1日—2024年4月31日
\\初审时间:2024年5月1日—2024年5月8日
\\复审时间:2025年5月8日—2024年5月20日
\\终审时间:2025年5月21日—2024年6月4日
\\提交要求与方式见原文。
\\详见:\url{https://mp.weixin.qq.com/s/vAqSlW-WVjrohOylmxrwAg}
\end{multicols}
\end{document}
