% HEAD BEGIN
\documentclass[letterpaper, 12pt]{article}
\usepackage{graphicx}
\usepackage{multicol}
\usepackage{anysize}
\usepackage{fontspec}
\usepackage[fontset=none]{ctex}
\usepackage{tabularx}
\PassOptionsToPackage{hyphens}{url}
\usepackage[breaklinks=true, colorlinks=true]{hyperref}
\expandafter\def\expandafter\UrlBreaks\expandafter{\UrlBreaks\do\a\do\b\do\c\do\d\do\e\do\f\do\g\do\h\do\i\do\j\do\k\do\l\do\m\do\n\do\o\do\p\do\q\do\r\do\s\do\t\do\u\do\v\do\w\do\x\do\y\do\z\do\A\do\B\do\C\do\D\do\E\do\F\do\G\do\H\do\I\do\J\do\K\do\L\do\M\do\N\do\O\do\P\do\Q\do\R\do\S\do\T\do\U\do\V\do\W\do\X\do\Y\do\Z}
% \let\oldurl\url
% \renewcommand{\url}[1]{\begin{sloppypar}\oldurl{#1}\end{sloppypar}}
\setlength\columnsep{30pt}
\marginsize{30pt}{30pt}{10pt}{20pt}
\setmainfont{TeX Gyre Bonum}
\setCJKmainfont[BoldFont=Noto Serif CJK SC Bold, ItalicFont=FandolKai]{Noto Sans CJK SC}
\setlength{\parindent}{0cm}
% \setCJKmonofont{Noto Sans CJK SC}
\begin{document}
\begin{center}
    \Huge\textbf{南哪大专醒前消息}
\end{center}
\vspace{4mm}
\hrule
\renewcommand\tabularxcolumn[1]{m{#1}}
\begin{tabularx}{\textwidth}{>{\hsize.2\hsize}X>{\hsize.6\hsize}X>{\hsize.2\hsize}X}
    \begin{flushleft}
        2024.10.13\, No.87
    \end{flushleft}
    &
    \begin{center}
        \textit{“克明峻德。”}
    \end{center}
    &
    \begin{flushright}
        \textbf{南京市栖霞区}
    \end{flushright}
\end{tabularx}
\vspace{-3.5mm}
\hrule
\vspace{4mm}
% HEAD END
\centerline{\huge\textbf{活动预告}}
\begin{multicols}{2}

\section{Deadline Ongoing}
\begin{tabular}{|c|c|c|}
    \hline
    消息(未见ddl的,不刊) & 截止日期 & 刊载日期\\
    \hline\hline
    仙林校史馆招募讲解员 & 10.30 & 9.12\\
    本科生暑期课程评教 & 10.31 & 9.19\\
    大创训练计划申报 & 11.18 & 9.24\\
    苏州校区音乐会 & 10.19 & 9.25\\
    第十九届大挑 & 10.15 & 9.30\\
    声谷创新基金 & 10.18 & 9.30\\
    鹰角校招宣讲 & 10.15 & 10.2\\
    大专戏曲知识竞赛 & 10.20 & 10.2\\
    EBSCO数据库检索大赛 & 11.20 & 10.3\\
    炜华音乐跑 & 12.8 & 10.4\\
    马院主题宣讲报名 & 10.25 & 10.5\\
    后革命鲁迅研究征文 & 11.10 & 10.8\\
    “南大新传”编辑部招新 & 10.20 & 10.10\\
    遵义精神宣讲团遴选 & 10.27 & 10.10\\
    八院联谊活动 & 10.14 & 10.10\\
    心协十月征稿 & 10.20 & 10.11\\
    江苏创青春大赛 & 10.14 & 10.11\\
    国际化科研素养课程 & 10.14 & 10.11\\
    乐跑 & 12.8 & 10.12\\
    健雄书院院服设计赛 & 10.20 & 10.12\\
    计院迎新晚会征集节目 & 10.25 & 10.12\\
    毓秀素拓友谊赛 & 10.19 & 10.12\\
    历史学院实习分享报名 & 10.15 & 10.13\\
    新生午餐会报名 & 10.14 & 10.13\\
    新传南新读书会 & 10.16 & 10.13\\

    \hline
\end{tabular}
\begin{tabular}{|c|c|c|}
    \hline
    消息(未见ddl的,不刊) & 截止日期 & 刊载日期\\
    \hline\hline
    CTF竞赛宣讲 & 10.19 & 10.13\\
    行知趣味羽球赛 & 10.19 & 10.13\\
    计院趣味定向赛 & 10.20 & 10.13\\
    林泉钢琴社线上分享 & 10.21 & 10.13\\
    林泉音乐会 & 10.19 & 10.13\\
    \hline
\end{tabular}

\section{讲座}
\begin{tabular}{|c|c|c|}
    \hline
    往期讲座 & 开展日期 & 刊载日期\\
    \hline\hline
    《聚合物的研发与...》 & 10.24 & 10.3\\
    《电池及电化学能...》 & 11.24 & 10.3\\
    《专利查新与规避...》 & 12.19 & 10.3\\
    《恋爱是门技术活》 & 10.14 & 10.8\\
    《对于人工智能时...》 & 10.16 & 10.9\\
    《卡夫卡、现代组...》 & 10.16 & 10.10\\
    《中美博弈及其对...》 & 10.15 & 10.10\\
    《跨代性与跨代平...》 & 10.16 & 10.10\\
    《关于西方社会再...》 & 10.16 & 10.11\\
    《楚国郢都的诗经》 & 10.16 & 10.11\\
    《走进ESG暨案例分...》 & 10.17 & 10.12\\
    《作为症状的极端...》 & 10.16 & 10.12\\
    《在范德华单层晶...》 & 10.15 & 10.12\\
      \hline
\end{tabular}\\\\

1.关于战争与生产的哲学思考\\
主讲人:Maurizio Lazzarato(著名哲学家)\\
主持人:蓝江(南京大学哲学学院教授)\\
时间:10月16日18:30\\
地点:哲学学院(薛光林楼)402室\\
内容:阐述政治经济学批判与阶级斗争理论,反思行动主体,将战争与内战纳入生产概念,回应革命力量与生产力、革命进程发展之间的关系等问题。
2.\\\\

\section{行知书院:趣味羽毛球赛}
时间:10月19日18:00~22:00\\
地点:南京大学鼓楼校区体育馆\\
对象:全体行知书院新生,以及新生导师与朋辈导师\\
项目:(团体)羽毛球接力赛、一人一拍、羽毛球3v3、(个人)羽毛球男女单打\\
奖励:凡报名者均可获羽毛球小挂件一个,还有玩偶挂件、卡皮巴拉羽毛球握把装饰、羽毛球型风扇、羽毛球包+羽毛球等你来拿\\
报名链接和qq群详见(团体比赛也可单人报名)\url{https://mp.weixin.qq.com/s/cAm2GXZF6-v6Eon8V7cWfg}

\section{党政机关实习分享会}
2024年10月16日 18:30

大美楼 404教室

有意参加的同学于10月15日(周二)18:00前扫描以下二维码填写问卷报名

为向我院学子提供更好的求职经验咨询与指导,院研会筹办了“青心向党 红心筑梦”系列活动,本次分享会邀请到了曾在党政机关实习过的优秀学姐来与同学们进行交流。

详见:\url{https://mp.weixin.qq.com/s/w-cC8Jdp52vzty4ugHahLA}
\section{订阅方式和加入编辑部}
编辑部招聘人才,用爱发电,工作轻松,详情可联系QQ:1329527951 客服小祥\\想订阅本消息或获取PDF版(便于查看超链接和往期),可加QQ群:\href{https://qm.qq.com/q/FGX1VYCrGS}{849644979}.
\section{高研院新生午餐会第四十场}
题目:“种咱们的园地要紧”:糖渍佛手、自我冒险与学术理想国\\
谈话人:叶子 南京大学文学院副教授\\
主持人:蒋浩伟 南京大学文学院师资博士后\\
时间:2024年10月16日(周三)12:20-13:20\\
地点:鼓楼校区逸夫馆9楼 高研院报告厅\\
抽签开始时间:10月14日12:30\\
截止时间:10月15日12:30\\

\section{第二期CTF竞赛宣讲}
CTF(Capture The Flag),译作夺旗赛,在网络安全领域中指的是网络安全技术人员之间进行技术竞技的一种比赛形式,它是一种非常流行的信息安全竞赛形式。\\
本次主讲团队为南京大学Trinity战队和网络攻防实战课程的授课教师陈健。\\
本期具体内容是Crypto模块与web模块,包括web入门内容和crypto:对称密码学、公钥密码学、椭圆曲线加密/格密码。本期宣讲以激发兴趣为主,建议对python,html,SQL进行初步了解。\\
此外,宣讲现场还设置有一些小挑战,完成现场挑战,你将获得一份小奖品作为奖励。\\
同时,你也将有机会加入Trinity战队成为其中一员!\\
本期宣讲活动将于10月19日的15点至17点在仙林校区基础实验楼乙区开展。\\
进入活动QQ群(220630242),进一步了解活动安排,如有调整将在活动群中告知。\\
原文链接\url{https://mp.weixin.qq.com/s/PQqhVl8cNUbSt98zFRlFWg}

\section{计算机学院趣味定向赛}
本次趣味定向赛中,有四条不同的路线,会根据问卷填写结果,依照神秘算法为你选择其中一条,途中还安排了一些会对越野过程产生重要影响的任务,使本次活动更富有趣味!\\
本次活动暂定于10月20日上午9:30-11:00在仙林校区举行,有计算机学院组(限计算机学院学生参加)和开放组(不限专业)两个组别。请报名的同学加入报名群,以获取更多信息。目前规划计算机学院组限50个人,开放组限10个人,人数限制会动态调整。\\
活动问卷填写和报名群详见链接\url{https://mp.weixin.qq.com/s/YRBBkCzrnbfd_DTHMd9G7A}
\section{南新读书会 | 下周预告}
新闻传播学院南新读书会将在10月16日19:00例行举办,地点为新闻传播学院(紫金楼)311室。22硕刘艺璇将分享兰登·温纳《自主性技术:作为政治思想主题的失控技术》,23硕张含融将分享莱辛《拉奥孔》。
\section{林泉钢琴社“驻足艺术前”线上分享}
活动时间:10.12-10.21\\
活动形式:在推文中扫码参与,在平台上上传自己的美术创作、摄影作品或音乐时刻,于平台内赞数最多的前三位可获得特殊奖品,凡参与者皆可抽精美文创。\\
详见:\url{https://mp.weixin.qq.com/s/w-V-Qwac922t3Zl8Lsar0w}
\section{林泉社“木叶送秋声”音乐会}
时间:10月19日18:30\\
地点:仙林校区大学生活动中心多功能厅\\
领票方式:转发推文\url{https://mp.weixin.qq.com/s/UTTC0cDPAjG4ZWPvoX4nzQ},凭转发记录可于10月18、19日中午12点在四五六食堂门口、九食堂门口免费领取门票和节目单。
\section{远程艺术支教报名}
南京大学学生筑梦社现通知支教事宜如下:
授课模式:\\
1.小学的多媒体设备与志愿者设备同时登陆classin软件,在预先软件建立好的教室里,进行互动式的教学授课。\\
2.志愿者通过classin教室里的教学工具进行PPT讲解授课。\\
每周一、三各一节美术课,上课时间15:20—16:00,每节课40分钟。\\
招募对象:南京大学全日制本科生及以上学历学生\\
具备条件:\\
1.教学能力较强\\
2.爱心与责任心\\
3.有儿童青少年服务经验者优先。\\
4.具备教授小学生美术的基础知识与素养。(有基本的了解和爱好即可)\\
5.普通话标准\\
提供课程帮助、活动证明、志愿时长、培训实习机会等。详见:\url{https://mp.weixin.qq.com/s/Pb1Fpt0O2ZNAvijB2BgAGQ}

\section{篮球明日赛程}
(10.14)
男篮院系杯小组赛 \\
数学vs化生 \\
19:00 - 20:30\\
地点:一组团篮球场\\




\end{multicols} 

\hrule
\vspace{4mm}
\centerline{\huge\textbf{参考消息}}
\begin{multicols}{2}
\section{挑战杯队伍招募人才}
本项目是关于矿工口述史的社会科学类调查报告,未来将参加红色专项赛道。团队已有五人,皆为新闻传播学院本科23级学生,已有两名指导老师,分别为马克思主义与历史社会学方向。有较为成熟的项目计划。现需招募1-3名队员,要求为本科23/24级,有良好的口语表达能力,与陌生人能够良好沟通;来自江苏中、北部、安徽、山东,并能熟练听说读写家乡方言,或掌握江淮官话或中原官话。如有航拍等技能,则更希望你能够参加我们的团队。最重要地,希望你清晰地理解“挑战杯”的性质、要求和难度,决不摆烂,愿意为此付出极多时间和精力,并有志于拿到较高奖项。若抱定参赛之决心,请联系QQ1872515737。报名时间紧张,仅等待到10月14日12:00,过时不候。
\end{multicols} 
\end{document}