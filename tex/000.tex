% HEAD BEGIN
\documentclass[letterpaper, 12pt]{article}
\usepackage{graphicx}
\usepackage{multicol}
\usepackage{anysize}
\usepackage{fontspec}
\usepackage[fontset=none]{ctex}
\usepackage{tabularx}
\PassOptionsToPackage{hyphens}{url}
\usepackage[breaklinks=true, colorlinks=true]{hyperref}
\expandafter\def\expandafter\UrlBreaks\expandafter{\UrlBreaks\do\a\do\b\do\c\do\d\do\e\do\f\do\g\do\h\do\i\do\j\do\k\do\l\do\m\do\n\do\o\do\p\do\q\do\r\do\s\do\t\do\u\do\v\do\w\do\x\do\y\do\z\do\A\do\B\do\C\do\D\do\E\do\F\do\G\do\H\do\I\do\J\do\K\do\L\do\M\do\N\do\O\do\P\do\Q\do\R\do\S\do\T\do\U\do\V\do\W\do\X\do\Y\do\Z}
% \let\oldurl\url
% \renewcommand{\url}[1]{\begin{sloppypar}\oldurl{#1}\end{sloppypar}}
\setlength\columnsep{30pt}
\marginsize{30pt}{30pt}{10pt}{20pt}
\setmainfont{TeX Gyre Bonum}
\setCJKmainfont[BoldFont=Noto Serif CJK SC Bold, ItalicFont=FandolKai]{Noto Sans CJK SC}
\setlength{\parindent}{0cm}
% \setCJKmonofont{Noto Sans CJK SC}
\begin{document}
\begin{center}
    \Huge\textbf{南哪大专醒前消息}
\end{center}
\vspace{4mm}
\hrule
\renewcommand\tabularxcolumn[1]{m{#1}}
\begin{tabularx}{\textwidth}{>{\hsize.2\hsize}X>{\hsize.6\hsize}X>{\hsize.2\hsize}X}
    \begin{flushleft}
        2024.10.2\, No.78
    \end{flushleft}
    &
    \begin{center}
        \textit{“休说鲈鱼堪脍,尽西风,季鹰归未?”}
    \end{center}
    &
    \begin{flushright}
        \textbf{苏州市高新区}
    \end{flushright}
\end{tabularx}
\vspace{-3.5mm}
\hrule
\vspace{4mm}
% HEAD END
\centerline{\huge\textbf{活动预告}}
\begin{multicols}{2}

\section{Deadline Ongoing}
\begin{tabular}{|c|c|c|}
    \hline
    消息(未见ddl的,不刊) & 截止日期 & 刊载日期\\
    \hline\hline
    仙林校史馆招募讲解员 & 10.30 & 9.12\\
    国优计划报名 & 10.7 & 9.19\\
    本科生暑期课程评教 & 10.31 & 9.19\\
    网易雷火大赛 & 10.7 & 9.22\\
    大创训练计划申报 & 11.18 & 9.24\\
    苏州校区音乐会 & 10.19 & 9.25\\
    外院国庆摄影征集 & 10.7 & 9.25\\
    雨花成长计划课堂报名 & 10.3 & 9.26\\
    港澳台生中华文化大赛 & 10.9 & 9.26\\
    心理中心征稿 & 10.10 & 9.28\\
    周末剧场 & 10.10 & 9.28\\
    历史学院国庆活动 & 10.7 & 9.28\\
    计院国庆桌游会 & 10.5 & 9.29\\
    台湾地区交换项目 & 10.7 & 9.29\\
    第十九届大挑 & 10.15 & 9.30\\
    声谷创新基金 & 10.18 & 9.30\\
    软院国庆桌游会 & 10.7 & 9.30\\
    物院观影会 & 10.3 & 9.30\\
    国际化处全媒体招新 & 10.8 & 9.30\\
    午餐读书会 & 10.10 & 9.30\\
    “周一剧!”第二期 & 10.5 & 9.30\\
    鹰角校招宣讲 & 10.15 & 10.2\\
    软院国庆活动 & 10.7 & 10.2\\
    计院文创设计 & 10.8 & 10.2\\
    育教征集国庆祝福 & 10.4 & 10.2\\

    \hline
\end{tabular}
\begin{tabular}{|c|c|c|}
    \hline
    消息(未见ddl的,不刊) & 截止日期 & 刊载日期\\
    \hline\hline
    大专戏曲知识竞赛 & 10.20 & 10.2\\
    MathGlue导员招募 & 10.7 & 10.2\\
    \hline
    \end{tabular}
\section{订阅方式和加入编辑部方式}
编辑部招聘人才,用爱发电,工作轻松,详情可联系QQ:1329527951 客服小祥\\想订阅本消息或获取PDF版(便于查看超链接),可加QQ群:\href{https://qm.qq.com/q/FGX1VYCrGS}{962626571}.
\section{讲座}
\begin{tabular}{|c|c|c|}
    \hline
    往期讲座 & 开展日期 & 刊载日期\\
    \hline\hline
    《南大外哲论谭Col...》 & 10.7 & 10.2\\
      \hline
\end{tabular}\\\\
1.论谭预告|南大外哲论谭 Colloquium VI:类比专题

报告时间:2024年10月7日,10: 00-18: 00
\\报告地点:哲学楼(薛光林楼),218室
\\腾讯会议:195-903-604
\\活动简介:南京大学哲学系外国哲学专业组织“南大外哲论谭 Colloquium”系列活动。活动面对高年级本科生,硕士生以及博士生。\\
活动内容:一位同学报告一篇学术论文,多位评议人进行点评,报告人需逐一进行回应。报告的同学需提前报名,学术论文需原创性成果,符合学术论文写作规范,作者有义务深度修改论文,达到能够进行点评的水准,组织者审核同意后报告人需提前一周将论文发送给各位评议人,线上报告用PPT。会努力邀请国内外相关领域的专家学者以及感兴趣的学生前来评议论文,同时欢迎旁听。\\
哲学院计划接收外国哲学方向的学术论文,欲提前报名可以发送邮件至liuxin10.03@nju.edu.cn;可以在刘鑫老师的办公时间(周五下午四点-六点,需提前预约)来办公室(哲学楼342房间)面议。此活动根据大家提交的论文情况将不定期举行。
\section{鹰角网络宣讲}
10月15日18:30,在仙林校区计算机111报告厅,鹰角网络开展校招宣讲会。\\
前往宣讲会现场签到,可领取独家订制纸袋及周边伴手礼。
\section{Math Glue数学学业辅导人员招募}
Math Glue数学学业辅导活动由南京大学学生会、数学学院团委、新生学院团委联合举办,针对大学数学公共课程及数学学院专业课程,面向全校招募数学基础扎实、具备辅导能力的同学担任朋辈领学员,帮助有需求的同学激发学习兴趣、改善学习方法、掌握课程重难点,为学生大学四年的学涯发展赋能。\\\\
1.朋辈领学员招募\\
招募对象:全校学生,本研不限,专业不限,数量不限。\\
技能要求:修读过大学数学相关课程并取得较好成绩,有一定的沟通表达能力和辅导教学能力。\\\\
2.运营志愿者招募\\
招募对象:全校学生,数量不限\\
主要任务:开展试题整理、通知联络、现场组织、反馈意见搜集。\\
报名截止时间:10月7日\\
报名方式详见\url{https://mp.weixin.qq.com/s/CUjUjwBIWS91ksMFmuvGKg}
\section{征集NJUers为新中国庆生的N种方式}
家国和乐,月圆人圆。在这共庆国庆的时分,南大育教公众号现发出盛情的邀请,征集NJUers为新中国庆生的N种方式。\\
征集对象:南京大学全体师生\\
截止时间:2024年10月4日24:00\\\\
征集内容及形式:\\
国庆假期活动记录照片1-2张+想说的话\\
在国庆期间,你经历了哪些感动瞬间、难忘回忆?对于敬爱的祖国,你有哪些心声想要表达?
\section{计院文创产品邀你设计}
可设计产品包括帆布包、手机支架、徽章、卫衣、卡套和鼠标垫。\\
征集截止日期 :10月8日晚23:59\\
作品要求、提交方式、奖品等详见\url{https://mp.weixin.qq.com/s/GiMKZnb318NRfm_y-qa-UA}

\section{第三届全国大学生戏曲知识竞赛}
报名时间:10.1-10.20\\
初赛:10.23\\
决赛:10.27\\
主办单位:广东艺职、开封科技学院、信阳农林学院等各校戏曲社团。
特点:以个人为单位报名,随机分组;涵盖多个剧种;邀河南卫视作为本次活动指导单位;邀苏州昆剧院优秀青年演员殷立人为昆曲知识指导老师。\\
参与对象:全国各大高校在校大学生戏曲爱好者。\\
报名方式、具体时间安排和比赛规则见原文:\url{https://mp.weixin.qq.com/s/pb8wR7nIOcMQaED9uaINOg}
\section{软件学院国庆活动}
同学们可个人或组队前往红色景点或有国旗的公共场所,拍照或者拍摄小视频打卡,并撰写小诗或者一段不少于100字的个人感想。\\
奖品包括小夜灯、笔记本、开国大典流麻等。\\
截止时间:2024年10月7日晚23:59\\
内容详见\url{https://mp.weixin.qq.com/s/ZvyomYHDe7hd4vuSqf5lQw}


\end{multicols} 
\end{document}