% HEAD BEGIN
\documentclass[letterpaper, 12pt]{article}
\newsavebox\colbbox
\usepackage{graphicx}
\usepackage{multicol}
\usepackage{anysize}
\usepackage{fontspec}
\usepackage[fontset=none]{ctex}
\usepackage{tabularx}
\usepackage{longtable}
\PassOptionsToPackage{hyphens}{url}
\usepackage[breaklinks=true, colorlinks=true]{hyperref}
\expandafter\def\expandafter\UrlBreaks\expandafter{\UrlBreaks\do\a\do\b\do\c\do\d\do\e\do\f\do\g\do\h\do\i\do\j\do\k\do\l\do\m\do\n\do\o\do\p\do\q\do\r\do\s\do\t\do\u\do\v\do\w\do\x\do\y\do\z\do\A\do\B\do\C\do\D\do\E\do\F\do\G\do\H\do\I\do\J\do\K\do\L\do\M\do\N\do\O\do\P\do\Q\do\R\do\S\do\T\do\U\do\V\do\W\do\X\do\Y\do\Z}
% \let\oldurl\url
% \renewcommand{\url}[1]{\begin{sloppypar}\oldurl{#1}\end{sloppypar}}
\setlength\columnsep{30pt}
\marginsize{30pt}{30pt}{10pt}{20pt}
\setmainfont{TeX Gyre Bonum}
\setCJKmainfont[BoldFont=Noto Serif CJK SC Bold, ItalicFont=FandolKai]{Noto Sans CJK SC}
\setlength{\parindent}{0cm}
% \setCJKmonofont{Noto Sans CJK SC}
\begin{document}
\begin{center}
    \Huge\textbf{南哪大专醒前消息}
\end{center}
\vspace{4mm}
\hrule
\renewcommand\tabularxcolumn[1]{m{#1}}
\begin{tabularx}{\textwidth}{>{\hsize.2\hsize}X>{\hsize.6\hsize}X>{\hsize.2\hsize}X}
    \begin{flushleft}
        2024.11.9\, No.113
    \end{flushleft}
    &
    \begin{center}
        \textit{“秉中持正、求新博闻。”}
    \end{center}
    &
    \begin{flushright}
        \textbf{南京市栖霞区}
    \end{flushright}
\end{tabularx}
\vspace{-3.5mm}
\hrule
\vspace{4mm}
% HEAD END
\centerline{\huge\textbf{活动预告}}
\begin{multicols}{2}
    \section{订阅方式和加入编辑部}  
编辑部招聘人才,用爱发电,工作轻松,详情可联系QQ:1329527951 客服小祥\\想订阅本消息或获取PDF版(便于查看超链接和往期),可加QQ群:\href{https://qm.qq.com/q/VXIW7fgsEe}{849644979}.
\section{Deadline Ongoing}
\setbox\colbbox\vbox{
\makeatletter\col@number\@ne
\begin{longtable}{|c|c|c|}
    \hline
    消息(未见ddl的,不刊) & 截止日期 & 刊载日期\\
    \hline\hline
    紫藤学刊征稿 & 12.15 & 10.22\\
    后革命鲁迅研究征文 & 11.10 & 10.8\\
    大创训练计划申报 & 11.18 & 9.24\\
    招生宣传创意征集大赛 & 11.18 & 10.21\\ 
    EBSCO数据库检索大赛 & 11.20 & 10.3\\
    文院征稿 & 11.20 & 10.20\\
    乐跑 & 12.6 & 10.12\\
    国际访学计划申报 & 11.22 & 10.22\\
    普通话测试网络报名 & 11.12 & 10.29\\
    南大会征募会设 & 11.15 & 11.1\\
    心协十一月征稿 & 11.10 & 11.2\\
    秉文心理短视频 & 11.25 & 11.3\\
    法学主题参会 & 11.11 & 11.4\\
    心协香囊活动 & 11.10 & 11.4\\
    高校联合徒步报名 & 11.10 & 11.5\\
    天文台车赛报名 & 11.12 & 11.5\\
    南大模联校内会报名 & 11.11 & 11.5\\
    公务员面试大赛报名 & 11.12 & 11.6\\
    炜华音乐会(地海院) & 11.10 & 11.6\\
    简历大赛 &11.17 & 11.7\\
    林下诗社讲座 & 11.10 & 11.8\\
    黑匣工作坊招募 & 11.12 & 11.8\\
    新生午餐会报名 & 11.11 & 11.8\\
    医保补参保 & 11.17 & 11.8\\
    新生午餐会报名 & 11.10 & 11.9\\
    NCQM2024报名 & 11.15 & 11.9\\
    鼓楼口语角 & 11.10 & 11.9\\
    博洽书会 & 11.15 & 11.9\\
    \hline
\end{longtable}
\unskip
\unpenalty
\unpenalty}\unvbox\colbbox
\end{multicols}
\hrule
\pagebreak
\begin{multicols}{2}

\section{讲座}
\begin{tabular}{|c|c|c|}
    \hline
    往期讲座 & 开展日期 & 刊载日期\\
    \hline\hline
    《电池及电化学能...》 & 11.24 & 10.3\\
    《专利查新与规避...》 & 12.19 & 10.3\\
    图书馆系列讲座 & 12.3 & 10.20\\
    《卢卡奇遗产中的...》 & 11.10 & 11.4\\
    《从微观数据到宏...》& 11.11 & 11.5\\
    《教室性别结构对...》 & 11.14 & 11.7\\
    《Learning from AI》 & 11.13 & 11.7\\
    《humanism in the...》 & 11.11 & 11.9\\
    《基于无人机的移...》 & 11.12 & 11.9\\
    《回到万隆》 & 11.11 & 11.9\\
    \hline
\end{tabular}

1.Humanism in the Age of AI\\
主讲人:William Schweiker;Edward L. Ryerson Distinguished Service Professor of Theological Ethics, University of Chicago\\
与谈人:Iddo Dickmann(狄一多) 南京大学哲学学院副教授\\
主持人:孟振华  南京大学哲学学院教授\\
时间:2024年11月11日(周一)18:30\\
地点:南京大学哲学学院401报告厅\\
详情见\url{https://mp.weixin.qq.com/s/ytQvfshvwe9zTxpL5_OnSg}\\

2.江苏省青年物理学家论坛——基于无人机的移动式光量子网络

报告时间:11月12日(周二)中午12点\\
报告地点:南京大学鼓楼校区唐仲英楼B501\\
报告人:谢臻达,南京大学电子科学与工程学院教授\\
直播链接:https://www.koushare.com/live/details/38421\\
详情与报名请查看\url{https://mp.weixin.qq.com/s/hw-xWKpBlYpEXZxdRyMoOg}\\

3.回到万隆:第三世界国际主义七十年

主讲人:陈光兴 台湾交通大学社会与文化研究所荣誉教授\\
主持人:胡大平 南京大学哲学学院教授\\
时间:2024年11月11日(周一)19:00-21:00\\
地点:鼓楼校区逸夫馆9楼高研院报告厅\\
具体链接:\url{https://mp.weixin.qq.com/s/EGnSpYdSiKGfT1Yg4s8cVQ}\\


\section{新生午餐读书会第三场}
1、本场活动时间为2024年11月13日(周三)中午12:15-13:40,地点在鼓楼校区南大出版社党建活动室。\\
2、读书会流程为两位同学发言各15分钟,提问讨论25分钟,老师评论导读30分钟。\\
3、本次活动向参加读书会活动的全体老师同学提供午餐。\\
旁听报名:本活动每次向全体新生开放20个旁听名额,对本主题感兴趣的同学可报名参加讨论。\\
本期报名抽签将于11月10日12:00开始
详情见\url{https://mp.weixin.qq.com/s/ITF6Gs2VoBWjKwHcILYu_g}
\section{计院掼蛋比赛}
活动对象:\\
南京大学计算机学院全体研究生\\
时间地点:\\
时间:2024年11月14日(周四)14:00-16:00\\
地点:计算机科学技术楼221会议室(全员线下参赛)\\
报名方式:\\
以队伍为单位,成功填写报名表即视为报名成功。限额30队(60人),先到先得\\
奖品包括小蓝鲸搪瓷杯、莫斯利安酸奶等\\
活动规则、报名表等详见\url{https://mp.weixin.qq.com/s/56w5uIw6zSF0QcHF0GkN6w}\\
\section{计院满帮集团开放日}
活动时间\\
2024年11月12日(周二)下午\\
活动地点\\
万博科技园A栋运满满(车辆往返接送)\\
参与对象\\
南京大学计算机学院学生\\
企业介绍与报名方式见\url{https://mp.weixin.qq.com/s/Yqs2lPE2JUkrtQzL8nNvZQ}\\
\section{NCQM 2024 Call for registration}
The Nanjing Conference on Quantum Materials (NCQM) is an annual conference series that provides a high-level forum for addressing a wide range of highly relevant and frontier topics in condensed matter physics, as well as Atomic, Molecular and Optical Physics. \\
Conference Website Link \url{https://physics.nju.edu.cn/NCQM/home/index.html}\\
Registraion\\
Please scan the QR code in the link below to register for the conference by November 15th (free of registration fee).\\
Dates\\
November 22nd - 25th , 2024\\
Registration on 22nd , full-day meetings on 23rd - 24th , and morning meeting on 25th .\\
Venue\\
The Purple Palace\\
Address\\
No.18 Huanling Road, Xuanwu District, Nanjing, Jiangsu\\
For more information, please visit \url{https://mp.weixin.qq.com/s/_-8ls0OeO1USEr3YsFPqfw}

\section{鼓楼口语角}
NJU英语俱乐部 发布\\
主题:Childbirth, Memorable ending, Hometown, Self-study\\
本周鼓楼校区口语角将于本周日(11月10日)晚7:00-8:30进行,地点位于新教-202。\\
其他详情与报名请查看\url{https://mp.weixin.qq.com/s/fd99lThJu-rdWYKa04jyWQ}\\

\section{区域国别研究院第1期“博洽书会”报名}
主题:“德国人的西藏幻象”\\
主讲嘉宾:赵光锐(南京大学国际关系学院副教授,国际政治系主任)\\
时间:2024年11月15日(周五)19:00-20:30\\
地点:南京大学区域国别研究院会议室(暂定)\\
活动简介:“博洽书会”每期围绕一本相关专著展开,以书著主题为当期讨论主题,时长约为1.5-2小时。每期由一名从事区域国别研究的教师(研究人员)担任主讲嘉宾,围绕个人近期作品进行30-40分钟的导读发言,分享研究心得,其余时间由同学们轮流发言讨论。每期活动邀请15人左右参加,为每位与会同学提供交流的机会。\\
详情与报名请查看\url{https://mp.weixin.qq.com/s/MRezFQ7ey68gSNbTdq7BEA}\\

\section{<SRTP整理>11.10-11.12学术文化活动}
周日(11.10)\\
1.卢卡奇遗产中的文档、手稿、书信与照片\\
2.美术电影中的视与听——从《哪吒闹海》到《宝莲灯》\\
周二(11.12)\\
认知与情感:哲学、精神分析与神经科学\\
周日(11.10)\\
数学学院“三室一厅”专项活动微积分I(第一层次)专场\\
周一(11.11)\\
如何与期刊编辑打“交道”\\
周二(11.12)\\
机器人学习简介\\
详情请查看\url{https://mp.weixin.qq.com/s/vlp0xqxP1V1dI8-ZtQezng}


\end{multicols} 

\end{document}