% HEAD BEGIN
\documentclass[letterpaper, 12pt]{article}
\newsavebox\colbbox
\usepackage{graphicx}
\usepackage{multicol}
\usepackage{anysize}
\usepackage{fontspec}
\usepackage[fontset=none]{ctex}
\usepackage{tabularx}
\usepackage{longtable}
\PassOptionsToPackage{hyphens}{url}
\usepackage[breaklinks=true, colorlinks=true]{hyperref}
\expandafter\def\expandafter\UrlBreaks\expandafter{\UrlBreaks\do\a\do\b\do\c\do\d\do\e\do\f\do\g\do\h\do\i\do\j\do\k\do\l\do\m\do\n\do\o\do\p\do\q\do\r\do\s\do\t\do\u\do\v\do\w\do\x\do\y\do\z\do\A\do\B\do\C\do\D\do\E\do\F\do\G\do\H\do\I\do\J\do\K\do\L\do\M\do\N\do\O\do\P\do\Q\do\R\do\S\do\T\do\U\do\V\do\W\do\X\do\Y\do\Z}
% \let\oldurl\url
% \renewcommand{\url}[1]{\begin{sloppypar}\oldurl{#1}\end{sloppypar}}
\setlength\columnsep{30pt}
\marginsize{30pt}{30pt}{10pt}{20pt}
\setmainfont{TeX Gyre Bonum}
\setCJKmainfont[BoldFont=Noto Serif CJK SC Bold, ItalicFont=FandolKai]{Noto Sans CJK SC}
\setlength{\parindent}{0cm}
% \setCJKmonofont{Noto Sans CJK SC}
\begin{document}
\begin{center}
    \Huge\textbf{南哪大专醒前消息}
\end{center}
\vspace{4mm}
\hrule
\renewcommand\tabularxcolumn[1]{m{#1}}
\begin{tabularx}{\textwidth}{>{\hsize.2\hsize}X>{\hsize.6\hsize}X>{\hsize.2\hsize}X}
    \begin{flushleft}
        2024.10.\, No.
    \end{flushleft}
    &
    \begin{center}
        \textit{“秉中持正、求新博闻。”}
    \end{center}
    &
    \begin{flushright}
        \textbf{南京市栖霞区}
    \end{flushright}
\end{tabularx}
\vspace{-3.5mm}
\hrule
\vspace{4mm}
% HEAD END
\centerline{\huge\textbf{活动预告}}
\begin{multicols}{2}
    \section{订阅方式和加入编辑部}  
编辑部招聘人才,用爱发电,工作轻松,详情可联系QQ:1329527951 客服小祥\\想订阅本消息或获取PDF版(便于查看超链接和往期),可加QQ群:\href{https://qm.qq.com/q/VXIW7fgsEe}{849644979}.
\section{Deadline Ongoing}
\setbox\colbbox\vbox{
\makeatletter\col@number\@ne
\begin{longtable}{|c|c|c|}
    \hline
    消息(未见ddl的,不刊) & 截止日期 & 刊载日期\\
    \hline\hline
    安邦征稿 & 1.12 & 11.16\\
    创意物理实验竞赛 & 12.21 & 11.15\\
    仙林通宵自习室 & 1.12 & 11.26\\
    全国大学生家史大赛 & 1.31 & 12.2\\
    金融消费者大赛 & 12.31 & 12.5\\
    花旗杯报名 & 1.3 & 12.6\\
    西安史学论坛征稿 & 3.20 & 12.9\\
    重唱英文评诗会 & 12.21 & 12.10\\
    叶嘉莹纪念征稿 & 12.25 & 12.10\\
    粤语课堂 & 12.22 & 12.11\\
    五院迎新晚会 & 12.21 & 12.12\\
    普通话测试报名 & 12.24 & 12.12\\
    本科评教 & 1.12 & 12.13\\
    排协雪球杯 & 12.28 & 12.13\\
    心协暖冬歌会 & 12.21 & 12.13\\
    12306学生优惠票 & 2.12 & 12.13\\
    高研院午餐会 & 12.18 & 12.14\\
    AI 朋辈就业分享 & 12.18 & 12.14\\
    校园迷你马拉松报名 & 12.20 & 12.14\\
    南北大联合读书会 & 12.19 & 12.15\\
    新传迎新晚会 & 12.21 & 12.15\\
    南新读书会 & 12.18 & 12.15\\
    外语社团联谊活动 & 12.21 & 12.15\\
    歌魅音乐会 & 12.22 & 12.15\\
    希音编程竞赛 & 12.21 & 12.15\\
    药石杯生化歌赛 & 12.22 & 12.15\\
    交响乐团室内乐 & 12.20 & 12.15\\
    南大新年音乐会 & 12.19 & 12.16\\
    flicker影映 & 12.21 & 12.16\\
    期末考试安排 & 1.12 & 12.17\\
    朝天宫民族团结研学 & 12.20 & 12.17\\
    南大迷你马拉松报名 & 12.20 & 12.17\\
    南大博物馆展览 & 6.16 & 12.17\\
    史院包饺子 & 12.21 & 12.17\\
    地海包饺子 & 12.20 & 12.17\\
    计院包饺子 & 12.21 & 12.17\\
    \hline
\end{longtable}
\unskip
\unpenalty
\unpenalty}\unvbox\colbbox
\end{multicols}
\hrule
\pagebreak
\begin{multicols}{2}

\section{讲座}
\begin{tabularx}{0.5\textwidth}{|X|X|X|}
    \hline
    名称 & 时间 & 地点\\
    \hline\hline
《中国马克思主义...》& 12.17 & 12.16\\\hline
《杨万里对苏轼诗...》 & 12.27 & 12.14\\\hline
《现代化进程中的...》 & 12.17 & 12.11\\\hline
《专利查新与规避...》 & 12.19 & 10.3\\\hline
《前向推理的一阶...》 & 12.18 & 12.10\\\hline
《非平衡任意子边...》 & 12.17 & 12.13\\\hline
《Microlocal...》 & 12.17 & 12.15\\\hline
《计算复杂性下界...》 & 12.19 & 12.16\\\hline
《二维半导体中的...》 & 12.19 & 12.16\\\hline
《On global...》 & 12.18 & 12.16\\\hline
《Designing...》 & 12.19 & 12.16\\\hline
《Mechanism...》 & 12.17 & 12.16\\\hline
物院学术交流会 & 12.21 & 12.16\\\hline
人工智能的社会影响 & 12.18 & 12.17\\\hline
世界藏学的回顾与批评 & 12.18 & 12.17\\\hline
大城市基层治理的特点与趋势 & 12.20 & 12.17\\\hline
\end{tabularx}
1.孙本文社会学论坛第292期\\
讲座题目:人工智能(AI)的社会影响\\
主讲人:谢宇 社会学家、美国国家科学院院士\\
讲座内容:本次讲座将聚焦于基于大语言模型的生成式人工智能的社会影响,分析讨论促进其技术发展的社会因素,并讨论其在扩大国际和国内社会不平等方面的潜在作用。\\
讲座时间:12月18日(周三)14:00-16:00\\
讲座地点:仙林社会学院110报告厅\\

2.世界藏学的回顾与批评\\
时间:2024年12月18日(周三)15:00\\
地点:仙林校区历史学院216室\\
主讲人:沈卫荣 清华大学人文与社会科学高等研究所教授\\
主持人:特木勒 南京大学民族与边疆研究中心教授\\

3.社会学院年度社会学本科专业学习交流报告会\\
活动主题:2024年社会学本科专业实习交流报告会\\
汇报主题:大城市基层治理的特点与趋势(成都组)\\
县城城乡融合发展中的产业转型与基层治理(建湖组)\\
时间:12月20日14:30-17:00\\
地点:仙林社会学院101报告厅\\


\section{【期末考试】2024-2025学年第一学期公共课期末考试安排}
考试时间:2024年12月30日—2025年1月12日
具体链接:\url{https://jw.nju.edu.cn/26/cc/c26263a730828/page.htm}


\section{朝天宫街道民族团结研学}
值南京大学“中华文化节暨民族团结进步宣传月”之际,一场特别的研学活动即将启程。欢迎来自各国、各地区、各民族同学与我们一起,沿朝天宫街道民族团结研学路线,感悟民族融合的历史韵味,触摸团结奋斗的时代脉搏。\\
活动时间:12月20日(周五) 9:00-12:30(包含乘车往返时间)\\
活动详情及报名信息:\url{https://mp.weixin.qq.com/s/1WrCKjbP7nKPdY5F2CCs_g}\\
\section{南京大学第六届校园迷你马拉松}
日期:2024年12月28日下午14:00\\
起点:南京大学仙林校区二源广场\\
报名要求:\\
1.竞赛组须是在校学生、教职工及南京大学校友。\\
2.竞赛组年龄须年满16周岁(2008年12月以前出生)。\\
3.健身组年龄不设限。\\
报名时间:2024年12月16日 9:00 至 2024年12月20日12:00\\
详细比赛及报名信息、相关注意事项:\url{https://mp.weixin.qq.com/s/SafThirA5YiE0JVnHlCfIw}\\
\section{南京大学博物馆展览}
1.安徽郎溪磨盘山遗址考古成果展\\
开展时间:2024.12.17~2025.6.16\\
展览地点:南京大学仙林校区星云楼1F\\
本次展览的磨盘山遗址考古成果是由南京大学主持完成的2023年度的全国十大考古新发现,主要揭示了距今5000-6000年间长江下游地区多元一体的面貌,反映了中华民族文明起源阶段小区域整合向大区域交融的转化过程。\\
2.新疆生产建设兵团军垦文化展\\
开展时间:2024.12.18\\
展览地点:南京大学仙林校区星云楼1F\\
2024年是新疆生产建设兵团成立70周年,为让广大学子和公众更加深刻地了解兵团精神,充分感受兵团在开发建设新疆、增进民族团结、推进社会进步、巩固西北边防等方面做出的重要贡献,丰富苏韵伊情文化内涵,特组织“屯垦戍边 保家卫国——新疆生产建设兵团第四师军垦文化展”。这也是新疆生产建设兵团专题展第一次走出新疆。\\
详情\url{https://mp.weixin.qq.com/s/MG0OfmVRtyUjEG4WoBA24w}\\

\section{篮协赛程预告}
「研究生男篮淘汰赛」\\
建城 vs 地科\\
12 : 00 - 14 : 00\\
地点:方肇周体育馆\\

\section{SRTP讲座整理}
周二(12.17)\\
1.与鲁迅在江南水师、陆师学堂相遇\\
2.现代化进程中的中国居民地位不一致问题研究\\
3.以乐观心:走进音乐疗愈\\
周三(12.18)\\
前向推理的一阶可定义性\\
周四(12.19)\\
1.二维半导体中的光-物质相互作用\\
2.“南雍博雅”共读会\\
详见原文:\url{https://mp.weixin.qq.com/s/lg4zB-YIWQGvtL-Zt52E5g}


\section{多院举行迎新晚会、音乐会}
南大青年汇总部分院系、协会迎新晚会、音乐会信息,汇总如下:\\
一、2025南京大学新年音乐会\\
二、“音既冬至”室内音乐会(Bravo吉他协会)\\
三、粤语课堂(南京大学学生粤语文化协会)\\
四、“一切的一切,花朵般绽开”2024冬英文评诗会(南京大学学生英语俱乐部、南京大学学生重唱诗社)\\
五、社会学院2024年度“社彩映未来,绘梦启新篇”迎新晚会\\
六、存心斋丨“心理一小口:轻阅读空间”——邀你共度阅读与治愈的午后(南京大学文学院存心斋)\\
七、南京大学法学院2024年迎新晚会\\
八、“光熠熠·政韶华”2024年政府管理学院与国际关系学院迎新晚会\\
九、五院联合迎新晚会(南京大学地理与海洋科学学院、南京大学地球科学与工程学院、南京大学环境学院、南京大学教育研究院、南京大学天文与空间科学学院)\\
十、“暖冬相伴,物耀星光”2024年度物理学院师生联欢晚会\\
十一、南京大学新闻传播学院2024年迎新晚会\\
十二、2024年南京大学台港澳青年迎新晚会\\
详情见:\url{https://mp.weixin.qq.com/s/OYZZC8QEXnTCcltjM-0nFQ}\\

\section{计算机学院冬至活动}
报名截止日期:12月18日下午18:30\\
活动时间:12月21日下午 2:00 - 5:00\\
活动地点:仙林校区教工第二食堂\\
点击原文进群报名\url{https://mp.weixin.qq.com/s/AEwwTBbYxyv22tJRDjWMzA}

\section{地海院包饺子活动}
包含包饺子和小游戏环节。\\
时间:2024年12月20日16:00-18:00\\
地点:学生第十食堂\\
面向对象:地理与海洋科学学院师生\\
原文扫码填写报名表报名,报名截止时间12月18日24:00\\
\url{https://mp.weixin.qq.com/s/i3AxKTJ7fQ1ITqOgcHek8g}

\section{哲学学院、历史学院、数学学院联合包饺子活动}
包饺子的过程中,还有预估饺子数量、添字游戏饺子皮接力赛等小游戏。\\
时间:12月21日星期六15:00-17:00\\
地点:南京大学仙林校区十食堂\\
参与对象:历史学院,数学学院,哲学学院全体本科生;每个学院预计20人。\\
进入原文加入报名群,群内填写报名链接,报名截止时间为12月18日。\\
\url{https://mp.weixin.qq.com/s/IwhLHXw5j5Zm8g6YhRqMtQ}\\

\end{multicols} 

\end{document}