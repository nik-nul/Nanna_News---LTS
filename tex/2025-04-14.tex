% HEAD BEGIN
\documentclass[letterpaper, 12pt]{article}
\newsavebox\colbbox
\usepackage{graphicx}
\usepackage{multicol}
\usepackage{anysize}
\usepackage{fontspec}
\usepackage[fontset=none]{ctex}
\usepackage{tabularx}
\usepackage{longtable}
\PassOptionsToPackage{hyphens}{url}
\usepackage[breaklinks=true, colorlinks=true]{hyperref}
\expandafter\def\expandafter\UrlBreaks\expandafter{\UrlBreaks\do\a\do\b\do\c\do\d\do\e\do\f\do\g\do\h\do\i\do\j\do\k\do\l\do\m\do\n\do\o\do\p\do\q\do\r\do\s\do\t\do\u\do\v\do\w\do\x\do\y\do\z\do\A\do\B\do\C\do\D\do\E\do\F\do\G\do\H\do\I\do\J\do\K\do\L\do\M\do\N\do\O\do\P\do\Q\do\R\do\S\do\T\do\U\do\V\do\W\do\X\do\Y\do\Z}
% \let\oldurl\url
% \renewcommand{\url}[1]{\begin{sloppypar}\oldurl{#1}\end{sloppypar}}
\setlength\columnsep{30pt}
\marginsize{30pt}{30pt}{10pt}{20pt}
\setmainfont{TeX Gyre Bonum}
\setCJKmainfont[BoldFont=Noto Serif CJK SC Bold, ItalicFont=FandolKai]{Source Han Sans SC}
\setlength{\parindent}{0cm}
% \setCJKmonofont{Noto Sans CJK SC}
\begin{document}
\begin{center}
    \Huge\textbf{南哪大专醒前消息}
\end{center}
\vspace{4mm}
\hrule
\renewcommand\tabularxcolumn[1]{m{#1}}
\begin{tabularx}{\textwidth}{>{\hsize.2\hsize}X>{\hsize.6\hsize}X>{\hsize.2\hsize}X}
    \begin{flushleft}
        2025.4.14\, No.220
    \end{flushleft}
    &
    \begin{center}
        \textit{“秉中持正、求新博闻。”}
    \end{center}
    &
    \begin{flushright}
        \textbf{南京市栖霞区}
    \end{flushright}
\end{tabularx}
\vspace{-3.5mm}
\hrule
\vspace{4mm}
% HEAD END
\centerline{\huge\textbf{活动预告}}
\begin{multicols}{2}
\section{订阅方式和加入编辑部}  
编辑部招聘人才,用爱发电,工作轻松,详情可联系QQ:1329527951 客服小千\\想订阅本消息或获取PDF版(便于查看超链接和往期),可加QQ群:\href{https://qm.qq.com/q/4HL41Nt3sQ}{466863272}.
\section{活动清单}
\setbox\colbbox\vbox{
\makeatletter\col@number\@ne
\begin{longtable}{|>{\centering\arraybackslash}m{.3\textwidth}|m{.06\textwidth}|m{.06\textwidth}|}
    \hline
    活动 & 开展时间 & 刊载时间\\
    \hline\hline
    南大版deepseek & / & 2.22\\
    悦读课程群 & / & 2.24\\
    eScience AI科研助手 & / & 3.11\\
    地科博物馆开放安排 & / & 3.22\\ 
    2025年分流和转专业政策通知 & / & 4.7\\
    乐跑 & 5.16 & 3.10\\
    本科生劳育实践 & 7.20 & 2.19\\
    银星杯论文赛 & 4.22 & 2.27\\
    高教社杯 & 4.25 & 3.5\\
    大文大理题目征集 & 期末 & 3.8\\
    5月免费上网 & ? & 3.9\\
    基础学科论坛 & 4.20 & 3.9\\
    外教社杯 & 5.27 & 3.12\\
    江苏创青春赛事 & 4.30 & 3.26\\
    南大数学竞赛 & 4.15 & 3.27\\
    AI素养大赛 & 4.15 & 3.31\\
    浦口音乐跑 & 5.30 & 3.31\\
    红会暑期项目招募 & 4.12 & 4.1\\
    程设大赛 & 4.26 & 4.2\\
    瑞声杯 & 4.20 & 4.4\\
    仙林校区志愿法律咨询 & / & 4.4\\
    青春活力大赛 & 5.17 & 4.7\\
    在校生自愿体检 & 6.20 & 4.8\\
    数智应用大赛 & 4.16 & 4.9\\
    南大购买WPS & / & 4.8\\
    24级程设大赛 & 4.27 & 4.11\\
    南书房支教招新 & 4.15 & 4.11\\
    法治情景剧策划大赛 & 4.23 & 4.11\\
    仙林猫鼠游戏 & 4.19 & 4.12\\
    EL程设大赛 & 4.27 & 4.13\\
    大挑志愿者课3 & 4.16 & 4.13\\
    全国行研大赛 & 4.20 & 4.13\\
    全国ESG大赛 & 4.17 & 4.13\\
    南大网双公开赛报名 & 4.18 & 4.13\\
    中美中心2025年证书项目 & 5.24 & 4.14\\
    心里剧本创作大赛 & 4.20 & 4.14\\
    \hline
\end{longtable}
\unskip
\unpenalty
\unpenalty}\unvbox\colbbox
\end{multicols}
\begin{multicols}{2}
\pagebreak

\section{讲座}
\begin{tabular}{|>{\centering\arraybackslash}m{.3\textwidth}|m{.06\textwidth}|m{.06\textwidth}|}
    \hline
    讲座 & 开展时间 & 刊载时间\\
    \hline\hline
    社交媒体分享实践的语用学研究 & 4.18 & 4.9\\\hline
    基于归纳程序合成的算法自动应用 & 4.15 & 4.10\\\hline
    面向推荐大模型的参数存储系统研究 & 4.15 & 4.10\\\hline
    特朗普2.0与中美日关系 & 4.16 & 4.10\\\hline
    Deepseek现象中的管理学 & 4.18 & 4.10\\\hline
    智能时代的中国式养老:理论与实践”学术研讨会 & 4.18-20 & 4.10\\\hline
    从感知到疗愈:人脑音乐加工机制 & 4.25 & 4.11\\\hline
    Accretion-generated rings:coplaner and polar structures & 4.18 & 4.11\\\hline
    Ionizing spotlight of Active Galactic Nucleus & 4.23 & 4.11\\\hline
    用法律知识为职场保驾护航 & 4.16 & 4.12\\\hline
    对早期中国研究中手稿获取伦理的批判 & 4.15 & 4.14\\\hline
    浅层地下热环境的全球趋势-影响与机遇 & 4.16 & 4.14\\\hline
    户籍分层对金融融入的影响 & 4.16 & 4.14\\\hline
    强化学习及应用研讨会 & 4.15 & 4.14\\\hline
\end{tabular}
%讲座预告写在这。用subsection
\subsection{麦笛(Dirk Meyer):古籍重现:对早期中国研究中手稿获取伦理的批判}
讲座语言:汉语
\\时间:2025年4月15日 周二 09:00-11:00
\\地点:南京大学鼓楼校区逸夫馆9楼高研院报告厅
\\详见:\url{https://mp.weixin.qq.com/s/DctvlEvhszH1ZwgpLLgtxg}

\subsection{国际访问学者|讲座预告:浅层地下热环境的全球趋势-影响与机遇}
主讲人 :Peter Bayer ,德国哈勒﹣维腾贝格大学教授,国际知名地热学、水文地质学与应用地质学专家。
\\时间:4月16日 10:00-12:00    
\\地点:朱共山楼AD26
\\详见:\url{https://mp.weixin.qq.com/s/NCa7jwbGOmBqdV6OMNkmxA}

\subsection{高研院仙林学术午餐会 李爱红助理研究员:户籍分层对金融融入的影响}
题目:人口流动与社会变迁背景下户籍分层对金融融入的影响
\\发言人:李爱红 高研院第21期驻院学者、社会学院助理研究员
\\主持人:胡翼青 高研院副院长、新闻传播学院教授
\\与谈人:高研院第21期驻院学者、高研院2025年度访问学者、 “悦读书社”同学
\\时间:2025年4月16日(周三)12:20
\\地点:仙林校区邵逸夫楼国际学院C308
\\详见:\url{https://mp.weixin.qq.com/s/enQWu6B64acn4hqDRyNd7A}

\subsection{论坛预告|强化学习及应用研讨会}
论坛时间:2025年4月15日(星期二)14:30—17:30
\\论坛地点:南京大学苏州校区(东区)  天枢楼303会议室
\\
\\详见:\url{https://mp.weixin.qq.com/s/Pvyav1eJ3C-irSR411pU8w}
\section{走近名企 | 第十三弹——SHEIN}
【主题】
\\“走近名企”系列活动之走进神秘的跨境电商独角兽企业SHEIN
\\【时间】
\\2025年4月17日14:00-16:00
\\【地点】
\\南京市雨花台区软件大道170-1号天溯科技园
\\【参访人数】
\\45人
\\【流程】
\\14:00-14:15 SHEIN产品展厅参观
\\14:15-14:45 SHEIN企业简介
\\(SHEIN发展历程、商业模式)
\\14:45-15:10 SHEIN人才需求探索
\\(跨境电商企业发展所需人才画像)
\\15:10-15:40 求职经验分享,如何做好求职准备(简历+面试内容揭秘)
\\15:40-16:00 自由交流,合影留念
\\
\\详见:\url{https://mp.weixin.qq.com/s/tKcyhQvuDK9S7klmf8C6eg}

\section{中美中心2025年证书项目}
联合证书项目是跨学科的研究生层次项目,学制一年。联合证书项目的目标是培养能从事中美双边事务和国际事务的高级人才。联合证书项目不分专业,学生可以选择有关国际政治、国际经济、国际法、中美历史文化、能源资源与环境等领域课程。学生在完成规定课程后,可以获得由两校校长联合签署的结业证书。
\\报名条件:1.入学时必须具有本科及以上学位证书;2.能以英语作为工作语言,具有较好的听说读写能力。
\\报名截至时间为2025年5月24日。
\\
\\详见:\url{https://mp.weixin.qq.com/s/uCevL2U61p5eemFh8nxZ_A}
\section{图书馆征集AI达人}
南大图书馆征集在利用AI工具辅助科研学习生活方面有独到的见解和心得的大神,有意者可以发送邮件至njuip@nju.edu.cn或直接在原文留言,会有随机奖品
\\详见:\url{https://mp.weixin.qq.com/s/eh0l_w4wyo94XLy87bxFhA}

\section{浦口校区春日手作}
慢下脚步触摸春日温度,用平凡草木书诗与乐谱,南京大学艺术学院(研会与新媒体中心)、南京大学党委学生工作部、南京大学图书馆邀您共赴手作游园会,用心与双手定格浦口春色。
\\活动时间:2025年4月19日11-17时
\\活动地点:南京大学浦口校区15食堂前
\\详见:\url{https://mp.weixin.qq.com/s/J6KVTwTTm94Zr58E1h6rmg}


\section{南大2025年度心理剧本创作大赛}
大赛主题:心光引路,剧写新生
\\参赛对象:南京大学全体师生 个人或团队不超过3人
\\剧本主题:
\\1)大学生成长困境:学业就业压力、身份认同等
\\2)情绪管理:焦虑情绪、抑郁情绪、孤独感调节等
\\3)生命教育:挫折应对、意义追寻等
\\4)亲密关系:家庭、恋爱、人际关系等
\\更多详情,请见原文
\\详见:\url{https://mp.weixin.qq.com/s/cKmS17bSWrRJHQao-pvRCQ}

\section{院级活动}
\begin{tabular}{|>{\centering\arraybackslash}m{.3\textwidth}|m{.06\textwidth}|m{.06\textwidth}|}
\hline
    活动 & 开展时间 & 刊载时间\\
    \hline\hline
    文院剧本创作研讨会 & 9.30 & 3.2\\
    物院征集课程指南 & 6.15 & 3.3\\
    地海征集春日影 & 6.15 & 3.14\\
    社院学术节 & 4.18 & 3.25\\
    五院乒乓球赛 & 4.19 & 3.31\\
    建城影展征集 & 4.16 & 3.31\\
    法院党建征文 & 5.20 & 4.2\\
    地学乒赛 & 4.19 & 4.2\\
    软院征集 & 4.20 & 4.4\\
    地学趣运会 & 4.26 & 4.9\\
    四院音乐节 & 5.11 & 4.7\\
    商院征集 & 5.5 & 4.8\\
    毓秀羽球 & 4.20 & 4.8\\
    大气设计 & 4.18 & 4.8\\
    文院诗歌 & 4.18 & 4.8\\
    化院摄影 & 4.15 & 4.9\\
    毓秀宿舍 & 4.16 & 4.10\\
    社院访企 & 4.16 & 4.11\\
    物院运动打卡 & 5.14 & 4.12\\
    大气留学分享会 & 4.15 & 4.12\\
    地学定向越野 & 4.19 & 4.12\\
    \hline
\end{tabular}

\subsection{南新读书会|下周预告}
本周的南新读书会将于4月16日晚19:00于新闻传播学院311室举行,23硕沙璨将分享列维纳斯《总体与无限》,24硕宋鑫铭将分享韩炳哲《爱欲之死》,欢迎全体师生参加。
\\详见:\url{https://mp.weixin.qq.com/s/Bxm9KtdIkf0f1vVGolN3yw}


\section{社团活动}
\begin{tabular}{|>{\centering\arraybackslash}m{.3\textwidth}|m{.06\textwidth}|m{.06\textwidth}|}
    \hline
    社团活动 & 开展时间 & 刊载时间\\
    \hline\hline
    天文台开放日 & / & 1.6\\
    重唱诗歌奖征稿 & 4.30 & 3.31\\
    足协体验 & 4.15 & 4.1\\
    轮滑社体验 & 4.17 & 4.1\\
    拳击社体验 & 4.22 & 4.1\\
    轮滑社体验 & 4.22 & 4.1\\
    定向赛 & 4.20 & 4.1\\
    体育舞蹈教学 & 4.25 & 4.1\\
    吉他社歌手招募 & 4.20 & 4.4\\
    吉他社春日音 & 4.26 & 4.4\\
    天健捐衣 & 4.20 & 4.13\\
    \hline
\end{tabular}
%这里是写社团活动的,社团活动就是由社团主办、主要针对社团内部人员的活动。不要把非社团活动写在这里。


\end{multicols}
\end{document}
