% HEAD BEGIN
\documentclass[letterpaper, 12pt]{article}
\newsavebox\colbbox
\usepackage{graphicx}
\usepackage{multicol}
\usepackage{anysize}
\usepackage{fontspec}
\usepackage[fontset=none]{ctex}
\usepackage{tabularx}
\usepackage{longtable}
\PassOptionsToPackage{hyphens}{url}
\usepackage[breaklinks=true, colorlinks=true]{hyperref}
\expandafter\def\expandafter\UrlBreaks\expandafter{\UrlBreaks\do\a\do\b\do\c\do\d\do\e\do\f\do\g\do\h\do\i\do\j\do\k\do\l\do\m\do\n\do\o\do\p\do\q\do\r\do\s\do\t\do\u\do\v\do\w\do\x\do\y\do\z\do\A\do\B\do\C\do\D\do\E\do\F\do\G\do\H\do\I\do\J\do\K\do\L\do\M\do\N\do\O\do\P\do\Q\do\R\do\S\do\T\do\U\do\V\do\W\do\X\do\Y\do\Z}
% \let\oldurl\url
% \renewcommand{\url}[1]{\begin{sloppypar}\oldurl{#1}\end{sloppypar}}
\setlength\columnsep{30pt}
\marginsize{30pt}{30pt}{10pt}{20pt}
\setmainfont{TeX Gyre Bonum}
\setCJKmainfont[BoldFont=Noto Serif CJK SC Bold, ItalicFont=FandolKai]{Noto Sans CJK SC}
\setlength{\parindent}{0cm}
% \setCJKmonofont{Noto Sans CJK SC}
\begin{document}
\begin{center}
    \Huge\textbf{南哪大专醒前消息}
\end{center}
\vspace{4mm}
\hrule
\renewcommand\tabularxcolumn[1]{m{#1}}
\begin{tabularx}{\textwidth}{>{\hsize.2\hsize}X>{\hsize.6\hsize}X>{\hsize.2\hsize}X}
    \begin{flushleft}
        2025.2.27\, No.177
    \end{flushleft}
    &
    \begin{center}
        \textit{“秉中持正、求新博闻。”}
    \end{center}
    &
    \begin{flushright}
        \textbf{南京市栖霞区}
    \end{flushright}
\end{tabularx}
\vspace{-3.5mm}
\hrule
\vspace{4mm}
% HEAD END
\centerline{\huge\textbf{活动预告}}
\begin{multicols}{2}
    \section{订阅方式和加入编辑部}  
编辑部招聘人才,用爱发电,工作轻松,详情可联系QQ:1329527951 客服小祥\\想订阅本消息或获取PDF版(便于查看超链接和往期),可加QQ群:\href{https://qm.qq.com/q/VXIW7fgsEe}{849644979}.
\section{Deadline Ongoing}
\setbox\colbbox\vbox{
\makeatletter\col@number\@ne
\begin{longtable}{|c|c|c|}
    \hline
    消息(未见ddl的,不刊) & 截止日期 & 刊载日期\\
    \hline\hline
    南大版deepseek & / & 2.22\\
    天文台开放日 & / & 1.6\\
    悦读课程群 & / & 2.24\\
    原创剧本联合孵化报名 & 3.20 & 1.10\\
    njumun代表报名 & 3.2 & 1.16\\
    课程补退选 & 3.2 & 2.19\\
    本科生劳育实践 & 7.20 & 2.19\\
    医保零星报销 & 3.31 & 2.19\\
    第二届大学生阅读分享活动 & 3.7 & 2.21\\
    心理中心助理招新 & 2.28 & 2.20\\
    招办全媒体招新 & 3.5 & 2.20\\
    交响乐团招新 & 3.7 & 2.20\\
    萌马音乐工作室招新 & 2.28 & 2.22\\
    秉文书院早晚自习 & 3.3 & 2.23\\
    图协招新 & 2.28 & 2.23\\
    “核真录”招新 & 3.2 & 2.24\\
    菁菁南数招募讲师 & 3.9 & 2.24\\
    新火星观影会 & 3.2 & 2.25\\
    车协骑行 & 3.1 & 2.25\\
    天协观测活动 & 3.2 & 2.26\\
    南书房支教队长团招募 & 3.3 & 2.26\\
    雨花斑斓课程 & 3.2 & 2.26\\
    半马志愿者招募 & 3.1 & 2.27\\
    南说喜剧 & 3.1 & 2.27\\
    银星杯论文赛 & 4.22 & 2.27\\
    部分课程增加名额 & 2.28 & 2.27\\
    金陵杯拟法赛 & 3.9 & 2.27\\
    
    \hline
\end{longtable}
\unskip
\unpenalty
\unpenalty}\unvbox\colbbox
\end{multicols}
\hrule
\pagebreak
\begin{multicols}{2}

\section{讲座}
\begin{tabular}{|>{\centering\arraybackslash}m{.3\textwidth}|m{.06\textwidth}|m{.06\textwidth}|}
    \hline
    讲座 & 开展时间 & 刊载时间\\

    \hline\hline
    从微观犯罪社会学分析到理解宏观社会空间分异   &3.6  &2.26 \\\hline
    大陆的起源 & 3.4 & 2.17\\\hline
    身体的重写本 & 3.7 & 2.25\\\hline
    探秘华为开发者大赛体系 & 2.28 & 2.25\\\hline
    党的创建和伟大建党精神 & 2.28 & 2.25\\\hline
    当代德语文学与视觉媒介的互动 & 3.3 & 2.27\\\hline
    古文字+AI研讨工作坊\\\hline
\end{tabular}
1.如诗如画:当代德语文学与视觉媒介的互动\\
主讲人:李双志  复旦大学德语系教授\\
主持人:陈民 南京大学德语系教授\\
讲座时间:2025年03月03日(周)19:30-21:30\\
讲座地点:南京大学仙林校区外国语学院(侨裕)会议室303\\
主办:南京大学外国语学院德语系 中德文化比较研究所\\

2.古文字+AI研讨工作坊\\
2月28日(星期五)14:00-16:00\\
讨论人 23级古文字强基班本科生\\
地点 南京大学仙林校区文学院433室\\
\section{志愿招募丨2025南京半马志愿者招募}
报名截止时间:3月1日晚20:00\\
服务时间:2025年3月16日(周日)\\
服务地点:南京奥体中心东门(起点志愿者)、江心洲梅子洲路(终点志愿者)\\
招募对象:南京大学在读学生(限3月16日早可以集中从鼓楼校区出发的同学)\\
详情:\url{https://mp.weixin.qq.com/s/jokgARXozJThKA0plgvu4Q}\\
\section{南说喜剧丨新学期第一次开放麦}
时间:3月1日(周六)晚 19:00\\
地点:敬文学生活动中心(大活)九楼\\
报名及链接:\url{https://mp.weixin.qq.com/s/-fUhHSAeuiOVqp6uYYE7uQ}\\
\section{第三十六届“银星杯”本科生学术论文竞赛正式启动}
参赛对象:南京大学在校本科生 可以个人形式或合作形式参赛(论文署名在三人以内),每人最多参与两篇论文的创作\\
论文类别:经济类、管理类的论文、科学报告或是研究报告,中英文不限\\
截稿时间:2025年4月22日24:00\\
竞赛答疑QQ群号:1016822931\\
赛事详情、投稿方式、推荐选题等请见原文
\url{https://mp.weixin.qq.com/s/fOjJaa_0lfRlmIQQAnn_Cg}\\

\section{南商乐学之微积分及Python集中自习“充电站”活动开启}
活动内容:本次活动主要采取自主学习和志愿者答疑相结合的模式,值班志愿者与报名同学一同自习,并随时解答同学们的疑问\\
活动时间:3月3日(下周一)开始,活动时间暂定每周一、周三、周五晚\\
活动地点:四栋自习室\\
活动报名与志愿者招募:本次活动仅限大四或延期同学,并面向商学院以及其他学院广泛招募志愿者\\
报名链接及活动QQ群二维码见原文\url{https://mp.weixin.qq.com/s/vVg6ldzwygMrDBAKlHdKag}\\

\section{部分课程增加名额的通知(三)}
增加的名额将于周五(2月28日)中午13:30放出。有需要的同学请及时参加补选。\\
链接:\url{https://jw.nju.edu.cn/5e/f1/c26263a745201/page.htm}\\

\section{第六届“金陵杯”全国高校模拟法庭竞赛}
2025年第六届“金陵杯”全国高校模拟法庭竞赛将于2025年3月-4月进行,由书状评审和言词辩论两部分组成。\\
初赛、复赛定于“腾讯会议”线上平台进行;\\
半决赛、决赛地点定于南京大学鼓楼校区,地址为江苏省南京市鼓楼区汉口路22号(如遇特殊情况,将另行通知)。\\
2025年第六届“金陵杯”全国高校模拟法庭竞赛采取邀请制,有意报名参赛的队伍需在3月9日24:00前将报名表提交至第六届组委会公邮:jinlingbei2025@163.com。\\
赛事全程采取匿名制。参赛队伍完成报名后,组委会将以适当方式确定赛队编号,并及时告知各赛队。在书状评审、言词辩论环节中,任何队伍不得暴露自己的学校。\\
详情见\url{https://mp.weixin.qq.com/s/wPvOilOIFfso_FVHcdp20w}\\

\section{2025年春季学期“大美汉字”通识课选课通知}
选课时间:2月28日13:30开选,开课两周内可补退选。\\
链接:\url{https://jw.nju.edu.cn/5f/fe/c26263a745470/page.psp}\\

\section{南京大学-东京大学“文化表象”通识课程第20期即将开课}
授课时间:2025年3月3日-21日每周一、二、四、五下午16:10~18:00\\
授课地点:南京大学仙林校区(逸B-507)\\

\section{蓝鲸有礼|暖春霸王餐福利}
南京大学学生会联合腾讯音乐招聘开展抽奖活动。本次抽奖面向全体南大师生,覆盖美食、健身、休闲、娱乐等等,中奖名额超过666+名。参与方式为转发“南京大学学生会”推文至朋友圈集8赞,再扫描推文内二维码参与抽奖。\\
此外腾讯音乐招聘再追加赞助500+500杯奶茶,仙林校区3月2日可到茶百道(万达茂店)【每人一杯、中杯茉莉奶绿】,鼓楼校区3月2日可到沪上阿姨·精选茶饮(鼓楼乐业村店)【每人一杯、中杯茉莉奶绿】,出示南大校园卡+本文朋友圈8个赞免费领取奶茶 限量赠送,领完截止~\\
链接:\url{https://mp.weixin.qq.com/s/dJgddZfWgVr5Tqn9JWAmnA}\\

\section{南大网协|教学班招新}
适合人群:零基础学员、初级学员、进阶学员、网球爱好者\\
课程人数:小班教学,每班10人左右\\
课程时间:单双周周日10:00-12:00\\
教学地点:南京大学仙林校区网球场\\
免费!免费!免费!\\
报名方式:QQ群950929081\\
入群关注后续通知及报名问卷\\
详情见推文:\url{https://mp.weixin.qq.com/s/5zujetsyRD2gjVFInh8CXw}\\
\end{multicols} 
\hrule
\vspace{4mm}
\centerline{\huge\textbf{参考消息}}
\begin{multicols}{2}
\section{南哪消息同学小文连载板块}
因收到小说投稿一篇,南哪消息现在开辟了同学小文连载板块。如想评论,可以发至邮箱:1329527951@qq.com,第二天会刊在此处。如想投稿渠道相同。
\section{《等待,遗忘》(5)}
金映樺\\

\newCJKfontfamily\fan{FandolFang}\fan

第三天\\

  第二次和林望舒见面是在L地铁站。附近的步行街沿途有很漂亮的银杏树,叶子在秋天变得灿烂。我匆匆走上出站的楼梯时,她已经在站外等我了。\\
  
  “抱歉,久等了...”奇怪,上次见面我是不是也说了这句话?\\
  
 “没事,我也刚到没多久。”她笑着摇了摇头。\\
 
  我们沿着步行街漫步,我注意到她今天穿得很休闲,比起上次见面少了几分疏离,我们似乎离成为朋友更近了一点。一阵风吹来,有点冷,银杏树的落叶好像一场又一场金黄色的雨,林望舒的头上落了几片叶子,她也闪着金色的光,耀眼闪亮。我帮她把叶子摘下,她轻声向我道谢。不用谢,不需要和我这么客气,我想。\\
  
  我们有上句没下句漫不经心地聊着天,她说我看起来不像学宗教学的学生,我问她什么样才看起来像?她认真想了想,她皱眉的样子也很美,像掌握了众生命运思考着要不要给人们赐福的神或者天使,我真是无药可救。\\
  
“穿着海青...戴着罗马领?像僧侣或者牧师,就是那种感觉。”她说话的时候没有看我,而是盯着零零散散飘落的银杏叶。\\

  我向她解释作为一个宗教研究者大多不应该是一位信徒,学者应该和他研究的对象保持一定的距离,这样才能保证学术研究的客观和中立。她点点头,很赞同的样子。我不自觉地向她说起在伦敦时偶然走进神秘学书店,店员都像从19世纪的画作中走出来那样,紫色天鹅绒帘子,店内似有似无的香薰,某种植物燃烧产生的烟雾,我或许就是在这样的时刻下了决心。\\
  
我没有告诉她,更直接的原因是我曾在在一本专著中读到作者写到自己亲身经历的故事,介绍自己是怎么开始信仰和研究菩萨的:在她还很小的时候,大概是在抗日战争结束后,他们一家人要从武汉坐船回家,一家人苦苦等待了三个月,好不容易等到了船票,然而在即将上船时,她的外婆突然毫无征兆地拒绝上船并坚持不让家人上船。她的母亲是一位无神论的新青年,对此不以为意,坚持要上船。因为如果这次放弃登船,不知道下次什么时候会再有船票了。然而他们无法违忤老人十分坚定的意志,最后一家人都没有登上这艘轮船。让人万万想不到的是,那艘轮船在离港不久就进入了日本人部署的雷区,被不幸炸沉了,船上的人全部遇难无一幸免,唯他们一家逃过了这场劫难。她的外婆,一位虔诚的信徒,解释说,她当时看见菩萨显现在长江之中,摆手示意她不要上船。这件事情给她尚幼小的心灵以极大的震撼,从此就对菩萨生起了坚定不移的敬信,以至于后来一辈子专心研究菩萨。能坦然说出自己的亲身经历,向众人共享本来只向他们一家人显现的神迹,这是何等的光明。将自己的信仰与学术前途结合在一起,这又是何其幸运。但是这好像太过私人,暴露了太多自我,也许她也对此不感兴趣,如果之后有机会...\\

她很认真地在听我说话,纵使我的言语没有任何意义。\\

“我带了拍立得...我能给你拍张照吗?”我脱口而出。\\

她愣了一下,下一秒笑得很开心,点点头答应了。她站在银杏树下,身后是步行街边的红砖墙,很像从上个世纪画报中走出来的那样。即刻成像的相片将这一刻定格,我选好了打印出来,她笑得很开心,看到她开心我也开心。我将相片递给她,她的红色围巾为此刻染上了童话般的色彩。\\

她向我道谢,我不好意思地摸摸鼻尖。我多么希望时间停在此刻。你真美丽。\\

\end{multicols} 

\end{document}