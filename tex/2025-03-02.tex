% HEAD BEGIN
\documentclass[letterpaper, 12pt]{article}
\newsavebox\colbbox
\usepackage{graphicx}
\usepackage{multicol}
\usepackage{anysize}
\usepackage{fontspec}
\usepackage[fontset=none]{ctex}
\usepackage{tabularx}
\usepackage{longtable}
\PassOptionsToPackage{hyphens}{url}
\usepackage[breaklinks=true, colorlinks=true]{hyperref}
\expandafter\def\expandafter\UrlBreaks\expandafter{\UrlBreaks\do\a\do\b\do\c\do\d\do\e\do\f\do\g\do\h\do\i\do\j\do\k\do\l\do\m\do\n\do\o\do\p\do\q\do\r\do\s\do\t\do\u\do\v\do\w\do\x\do\y\do\z\do\A\do\B\do\C\do\D\do\E\do\F\do\G\do\H\do\I\do\J\do\K\do\L\do\M\do\N\do\O\do\P\do\Q\do\R\do\S\do\T\do\U\do\V\do\W\do\X\do\Y\do\Z}
% \let\oldurl\url
% \renewcommand{\url}[1]{\begin{sloppypar}\oldurl{#1}\end{sloppypar}}
\setlength\columnsep{30pt}
\marginsize{30pt}{30pt}{10pt}{20pt}
\setmainfont{TeX Gyre Bonum}
\setCJKmainfont[BoldFont=Noto Serif CJK SC Bold, ItalicFont=FandolKai]{Noto Sans CJK SC}
\setlength{\parindent}{0cm}
% \setCJKmonofont{Noto Sans CJK SC}
\begin{document}
\begin{center}
    \Huge\textbf{南哪大专醒前消息}
\end{center}
\vspace{4mm}
\hrule
\renewcommand\tabularxcolumn[1]{m{#1}}
\begin{tabularx}{\textwidth}{>{\hsize.2\hsize}X>{\hsize.6\hsize}X>{\hsize.2\hsize}X}
    \begin{flushleft}
        2025.3.2\, No.179
    \end{flushleft}
    &
    \begin{center}
        \textit{“秉中持正、求新博闻。”}
    \end{center}
    &
    \begin{flushright}
        \textbf{南京市栖霞区}
    \end{flushright}
\end{tabularx}
\vspace{-3.5mm}
\hrule
\vspace{4mm}
% HEAD END
\centerline{\huge\textbf{活动预告}}
\begin{multicols}{2}
    \section{订阅方式和加入编辑部}  
编辑部招聘人才,用爱发电,工作轻松,详情可联系QQ:1329527951 客服小祥\\想订阅本消息或获取PDF版(便于查看超链接和往期),可加QQ群:\href{https://qm.qq.com/q/VXIW7fgsEe}{849644979}.
\section{Deadline Ongoing}
\setbox\colbbox\vbox{
\makeatletter\col@number\@ne
\begin{longtable}{|c|c|c|}
    \hline
    消息(未见ddl的,不刊) & 截止日期 & 刊载日期\\
    \hline\hline
    南大版deepseek & / & 2.22\\
    天文台开放日 & / & 1.6\\
    悦读课程群 & / & 2.24\\
    原创剧本联合孵化报名 & 3.20 & 1.10\\
    本科生劳育实践 & 7.20 & 2.19\\
    医保零星报销 & 3.31 & 2.19\\
    第二届大学生阅读分享活动 & 3.7 & 2.21\\
    招办全媒体招新 & 3.5 & 2.20\\
    交响乐团招新 & 3.7 & 2.20\\
    秉文书院早晚自习 & 3.3 & 2.23\\
    “核真录”招新 & 3.2 & 2.24\\
    菁菁南数招募讲师 & 3.9 & 2.24\\
    南书房支教队长团招募 & 3.3 & 2.26\\
    银星杯论文赛 & 4.22 & 2.27\\
    金陵杯拟法赛 & 3.9 & 2.27\\
    商院影绘 & 3.16 & 3.2\\
    文院剧本创作研讨会 & 9.30 & 3.2\\
    毓秀摄影 & 3.14 & 3.2\\
    计院学雷锋 & 3.4 & 3.2\\
    数院手工 & 3.10 & 3.2\\
    安邦体育日 & 3.7 & 3.2\\
    \hline
\end{longtable}
\unskip
\unpenalty
\unpenalty}\unvbox\colbbox
\end{multicols}
\hrule
\pagebreak
\begin{multicols}{2}

\section{讲座}
\begin{tabular}{|>{\centering\arraybackslash}m{.3\textwidth}|m{.06\textwidth}|m{.06\textwidth}|}
    \hline
    讲座 & 开展时间 & 刊载时间\\
    \hline\hline
    从微观犯罪社会学分析到理解宏观社会空间分异   &3.6  &2.26 \\\hline
    大陆的起源 & 3.4 & 2.17\\\hline
    身体的重写本 & 3.7 & 2.25\\\hline
    当代德语文学与视觉媒介的互动 & 3.3 & 2.27\\\hline
    国际法视域下的供应链安全国内立法 & 3.4 & 2.28\\\hline
    重新评估图神经网络在通用图任务中的潜力边界 & 3.3 & 2.28\\\hline
    中西比较视域下的儒家伦理学发展脉络 & 3.4 & 3.1\\\hline
    大陆的起源 & 3.4 & 3.2\\\hline
\end{tabular}
1.李明书:中西比较视域下的儒家伦理学发展脉络——兼论关怀伦理学视角下的儒家道德教育\\
主讲人:李明书,浙江大学哲学学院特聘研究员\\
讲座时间:3月4日 14:30-16:30\\
讲座地点:南京大学哲学学院211\\

2.“诚计划”第140期:葛荣峰教授主讲"大陆的起源"\\
主题:大陆的起源\\
主讲人:葛荣峰 南京大学地球科学与工程学院教授\\
时间:3月4日(周二)19::30-21:00\\
观看地址请见\url{https://mp.weixin.qq.com/s/h-r0CqSHRs7bVLGAxxmZaw} 
\section{挑战杯志愿者招募启动仪式}
活动时间 2025年3月5日(星期三)11:00-14:00

活动地点

鼓楼校区:南园喷泉广场

仙林校区:学生第四、五、六食堂前

浦口校区:学生食堂前

苏州校区:学生食堂一楼

活动对象 南京大学各校区全体同学

活动内容:同学们可以通过参与趣味问答、拼图挑战、留言互动等,获取纪念小奖品。

详见:\url{https://mp.weixin.qq.com/s/ZFaFq9-WCqr9zQFhoOuyzg}
\section{商学院校园美景影绘活动}
活动时间:2025年3月3日至3月16日\\
活动对象:南京大学商学院全体同学\\
摄影作品需为原创,彩色、黑白不限,单幅、组图(每组不超过6张)均可,作品需保留原始拍摄信息,不得进行过度后期处理(可进行简单的亮度、对比度、色彩饱和度调整);绘画作品形式不限,如水彩画、油画、素描、漫画等,同时欢迎其他创意作品形式。\\
参赛者需在3月16日前将作品打包上传至南大云盘\url{https://box.nju.edu.cn/u/d/6fdd2476bdd54e33817a/}(或扫描二维码上传)。作品命名规范为学号+姓名+作品名称,请大家留意命名规范,未按照规范命名的作品将视作无效提交。请另附上200字左右作品简介(包括作品名称、拍摄或创作地点、想要表达的主题等)。\\
详情见\url{https://mp.weixin.qq.com/s/lRcBHgR0VSQ7lHYxqt0Hkw}



\section{“当代剧本创作”学术研讨会会议}
会议主题\\
当代剧本创作的理论与实践\\
主要议题\\
1.戏剧剧场剧本创作\\
2.后戏剧剧场文本创作\\
3.现代戏曲剧本创作\\
4.编剧教学\\
5.影视剧本创作\\
6.当代剧本创作的其他相关议题\\
会议时间\\
2025年10月17日-19日\\
会议地点\\
南京大学仙林校区文学院\\
会议投稿\\
请于2025年7月31日前提交论文摘要(500字左右)及个人简介(包括姓名、单位、职称/职务、研究方向、联系方式)\\
摘要审核通过后,将于2025年8月15日前通知作者\\
完整论文提交截止日期为2025年9月30日\\
投稿邮箱:njuplaywriting@163.com\\
详见\url{https://mp.weixin.qq.com/s/y7G0kB6z4Da3lpES7AKw8g}


\section{计算机学院“新春影,故乡情”主题摄影大赛线上投票}
可进入链接观赏作品并投票\url{https://mp.weixin.qq.com/s/DMyP4XZGChT7ObODy1OJEg}


\section{毓琇书院|春光摄影大赛}
时间:2025年3月1日至3月14日(共计两周)\\
(评选结果会在3月25日由书院官微公布)\\
活动对象:毓琇书院全体师生及历届学生\\
作品要求:\\
1. 作品内容应紧扣主题“寻找春光”,可以是春天的自然景色、人文景观或日常生活中的春意瞬间;\\
2. 参赛作品需为原创,不得抄袭或侵犯他人版权;\\
3. 每位参赛者最多提交3幅作品;\\
提交链接、奖品及QQ群见\url{https://mp.weixin.qq.com/s/p9Sc24SL2NXfTvFgxcnczg}








\section{南京大学计算机学院2025学雷锋日活动}
记录下自己的“雷锋”故事,上传到原文二维码链接中。参与活动的同学可以收到一朵“小红花”,故事将被整理推送。\\
活动时间:3月4日16:30 一人一朵,发完即止\\
活动地点:计算机学院院楼\\
\url{https://mp.weixin.qq.com/s/Qapp2w49EvuG17xaUMMRKg}\\

\section{数学学院手工DIY活动}
活动内容为中药香囊的制作和风铃的制作。\\
时间:3月10日 晚上7:00\\
地点:待定\\
填写问卷提前报名,以便根据各校区的人数安排活动场地和采购相应物资。\\
报名截止时间:3月7日 24:00\\
原文扫码进入群聊,获取更多信息。\\
\url{https://mp.weixin.qq.com/s/XJkc_6kfaGWCEM7e3xqXZw}\\







\section{安邦书院|体育月}
活动地点:见各社团活动安排
1.拳击\\
时间:3月8日(周六)下午14:00-15:30\\
活动名额:35\\
2.飞盘\\
时间:3月9日(周日)下午15:00-17:00\\
活动名额:20\\
3.跆拳道\\
时间:3月9日(周日)下午16:30-18:30\\
活动名额:25\\
4.空手道\\
时间:3月7日(周五)晚上20:00-21:30\\
活动名额:25\\
5.定向越野\\
时间:3月9日(周日)上午8:00-12:00\\
活动名额:25\\
6.龙舟\\
时间:3月16日(周日)下午16:00-17:30\\
活动名额:20\\
各活动报名链接见\url{https://mp.weixin.qq.com/s/8TGcjDJ5lUet9Cp8ru6rxw}

\end{multicols} 
\hrule
\vspace{4mm}
\centerline{\huge\textbf{参考消息}}
\begin{multicols}{2}
\section{南哪消息同学小文连载板块}
因收到小说投稿一篇,南哪消息现在开辟了同学小文连载板块。如想评论,可以发至邮箱:1329527951@qq.com,第二天会刊在此处。如想投稿渠道相同。
\section{《等待,遗忘》(7)}
金映樺\\
\newCJKfontfamily\fan{FandolFang}\fan
第四天\\

我好喜欢她发的表情包,好喜欢她使用的颜文字,好喜欢拍立得定格下她的笑。回家之后我鬼使神差地再打印了一份那张拍立得放在窗台,她仿佛可以将这个世界照亮。她好像对哲学宗教或者其他形而上的东西很感兴趣,有时会向我提出一些思考和问题;有一次回复的时候被娘娘看见了,娘娘形容我好像在回复教授的邮件一样冥思苦想,好啦,好啦,我只是想在她面前表现得专业一点。\\

林望舒的讯息总给我一种莫名的熟悉感,或许是很像讨论课时助教的提问?我不知道,有的时候一些想法从我的脑海中掠过,我无法将它们串联起来,我一直知道自己不是靠理性,先天的也好社会建构的也罢,来理解这个世界的。\\

正当我放空脑袋仰着头瘫在沙发上时,手机的铃声响了。林望舒问我,“人死后会去哪?”这恰巧是我刚温习过的内容,我几乎不需要思考就给出回复。然而当这段对话也告一段落后我迟来地感到不安。就像第一次见面那样我感到她有心事,她有时离我很近,有时又离我很远。我唯一能做的似乎只有等待她开口的那一天。无意间我的视线扫过窗台上的相片,林望舒的笑容依旧,却让我想到了过去的一些事情,这种熟悉感让我有些心悸。\\

小学时爸妈去日本旅游回来给我带了寿司形状的橡皮擦,我放在笔袋的最上层,上课时就算看到心情也会变好。课间我拿出来向同学们展示,大家都觉得很可爱,于是问我可以送给他们吗?我无法拒绝,虽然很不舍也只好笑着说当然。后来我开始打篮球,每天都打到很晚才匆匆回家,爸妈为了诱惑我早点回家答应给我买那年限量款的联名篮球,我很喜欢那种清澈的湖蓝色;第二天我将球带去球场,大家看到了都觉得很酷,问我可以借给他们玩吗?我无法拒绝,虽然很不舍也只好笑着说当然。当我再次看到那个球的时候它已经变得破旧,湖蓝色褪去,再也不像从前鲜艳。还有好多这样的时刻。我发现我无法承受他人哪怕些许的失望,我总是下决心告诉自己下次不能这样,但是当下一个时刻来临,我被一股无名的力量控制,它让我笑着说当然。\\

因为很好奇福尔摩斯的故事我在初中转学去了伦敦,公寓的楼上很幸运地住着一个在UCL念SEF的哥哥,因为我的厨艺水平一团糟,他经常邀请我去蹭饭。我逐渐发现他好像并不喜欢经济或者统计(我也不喜欢),我们都喜欢一些其他的东西,伊特鲁里亚的雕塑和湿壁画、新现实主义和古埃及的多神论。他有时会给我看他写的一些文章或是简单的随笔,他让我感到安静、天才与温暖。有一年冬天我在某本书上看到“冬天一定要去柏林”,我还没有去过柏林,他说他之前有幸去过,我问他那是什么样,他想了想,笑着说那去一次不就好了?我们第二天一早就到了柏林。\\

但是这和书上写的不一样。灰蒙蒙的天幕像一张湿透的毛毯压在所有人头顶,空气里漂浮着工厂排出的烟和水雾,混成一团。我看不清远方,也分不清是白天还是夜晚。街角总是游荡着提着酒瓶的人,有的在清晨迷离的冷光里晃荡,有的在夜晚霓虹闪烁中沉默,他们似乎也不知道自己要去哪里。高纬度让天黑到来得特别早,灰色的天拖着疲倦的云彩,仿佛从未醒来。他看出了我的失落,摸摸我的头,笑着对我说:“小余淮,你发现了吗,人生比文学难。”\\

我从小一直嚷着央求爸妈给我生一个哥哥,这个无望的愿望在伦敦成真了,是不是这里真的有圣诞老人之类的存在?他说他有个妹妹比我小一两岁,但也差不多大。我问他他的妹妹会不会也来伦敦念书?他想了想,说也许会,因为她很喜欢他每次带回去的,动物形状的巧克力(我也喜欢)。\\

我永远也不会忘记他,永远也不会忘记那天一早他敲门问我要不要一起去白崖。我只在《国家地理》上看到过这里,所谓“英国最美的地方”。我匆匆披了件衣服,他开车带我去了Brighton。我们到的时候赶上了退潮,可以走在海滩上。早上的雾气还没有散去,其实完全看不到白崖,但是走在山崖下依然会觉得很壮观。过了一会我们向山上走去,天也开始放晴,雾气一点点散去,白崖一点一点在眼前展开,就像中国画一样美。他说这里就像世界尽头一样,我点点头,想到了易卜生。\\

我说我们改天去挪威看峡湾吧,他转过头笑着看着我,轻轻地说好。\\

我们在山上站了很久,我其实有些冷,他看到我在微微颤抖就拉着我跑回了车里,准备回伦敦。路上他突然问我将来想学什么,我没想好,冲他做了个鬼脸说一定不学数学、统计或者经济。他被逗笑了,笑了一会说让我选一个喜欢的去学就好,不要想那么多。我不明白他为什么突然说这个,还有好久呢,我可以到时候再咨询他。我问他还有几个月就要毕业了了,要回国了吗?他有片刻恍惚,侧过头看着我说他有其他地方想去。我大喊你要专心行驶啊大驾驶员,他又笑了起来,我也跟着他一起笑。\\

在我的公寓门口他突然抱了一下我,笑着让我不要太想他。我想说好啦,又不是再也见不到。但是那股我很久都没有感受到的不知名的力量再次控制住了我,我说不出话来。他的声音很轻,但是我永远也忘不了他对我说,“小余淮,新年快乐。”\\

对,那是一年的最后一天,12月31日,那天之后我再也没见过他。\\

我会永远记得他,他围着红围巾站在门口向我道别,我只是漫不经心地回应。那天晚上房东给所有人准备了派和礼物,无论怎么敲他的门也没有回应。我没有看到他,永远睡着的他,永远不会再回来的他,永远停在最后这天停在世界尽头的他。警笛或是救护车的声音中,我只记得我蜷缩在角落里愣住了,或许是在哭泣,或许是晕了过去,恍惚中爸妈将我带回了家。\\

我一直等着他回来,他答应了一起去挪威,为什么最后食言?我为什么这样迟钝,没有看出他究竟想去的是哪里?我感到一切都莫名其妙,我莫名其妙得到了一个哥哥,又莫名其妙失去了他。我消沉了很久,看着爸妈担心的神色我知道不能再这样下去了,我不会再想他。\\

对不起啊,林望舒。我骗了你。我现在不再喜欢,也再也没有办法喜欢上伦敦了。\\


\end{multicols} 

\end{document}