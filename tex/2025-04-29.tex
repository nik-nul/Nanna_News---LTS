% HEAD BEGIN
\documentclass[letterpaper, 12pt]{article}
\newsavebox\colbbox
\usepackage{graphicx}
\usepackage{multicol}
\usepackage{anysize}
\usepackage{fontspec}
\usepackage[fontset=none]{ctex}
\usepackage{tabularx}
\usepackage{longtable}
\PassOptionsToPackage{hyphens}{url}
\usepackage[breaklinks=true, colorlinks=true]{hyperref}
\expandafter\def\expandafter\UrlBreaks\expandafter{\UrlBreaks\do\a\do\b\do\c\do\d\do\e\do\f\do\g\do\h\do\i\do\j\do\k\do\l\do\m\do\n\do\o\do\p\do\q\do\r\do\s\do\t\do\u\do\v\do\w\do\x\do\y\do\z\do\A\do\B\do\C\do\D\do\E\do\F\do\G\do\H\do\I\do\J\do\K\do\L\do\M\do\N\do\O\do\P\do\Q\do\R\do\S\do\T\do\U\do\V\do\W\do\X\do\Y\do\Z}
% \let\oldurl\url
% \renewcommand{\url}[1]{\begin{sloppypar}\oldurl{#1}\end{sloppypar}}
\setlength\columnsep{30pt}
\marginsize{30pt}{30pt}{10pt}{20pt}
\setmainfont{TeX Gyre Bonum}
\setCJKmainfont[BoldFont=Noto Serif CJK SC Bold, ItalicFont=FandolKai]{Source Han Sans SC}
\setlength{\parindent}{0cm}
% \setCJKmonofont{Noto Sans CJK SC}
\begin{document}
\begin{center}
    \Huge\textbf{南哪大专醒前消息}
\end{center}
\vspace{4mm}
\hrule
\renewcommand\tabularxcolumn[1]{m{#1}}
\begin{tabularx}{\textwidth}{>{\hsize.2\hsize}X>{\hsize.6\hsize}X>{\hsize.2\hsize}X}
    \begin{flushleft}
        2025.4.29\, No.233
    \end{flushleft}
    &
    \begin{center}
        \textit{“秉中持正、求新博闻。”}
    \end{center}
    &
    \begin{flushright}
        \textbf{南京市栖霞区}
    \end{flushright}
\end{tabularx}
\vspace{-3.5mm}
\hrule
\vspace{4mm}
% HEAD END
\centerline{\huge\textbf{活动预告}}
\begin{multicols}{2}
\section{订阅方式和加入编辑部}  
编辑部招聘人才,用爱发电,工作轻松,详情可联系QQ:1329527951 客服小千\\想订阅本消息或获取PDF版(便于查看超链接和往期),可加QQ群:\href{https://qm.qq.com/q/4HL41Nt3sQ}{466863272}.
\section{活动清单}
\setbox\colbbox\vbox{
\makeatletter\col@number\@ne
\begin{longtable}{|>{\centering\arraybackslash}m{.3\textwidth}|m{.06\textwidth}|m{.06\textwidth}|}
    \hline
    活动 & 开展时间 & 刊载时间\\
    \hline\hline
    南大版deepseek & / & 2.22\\
    悦读课程群 & / & 2.24\\
    eScience AI科研助手 & / & 3.11\\
    地科博物馆开放安排 & / & 3.22\\ 
    2025年分流和转专业政策通知 & / & 4.7\\
    2025年转专业志愿填报通知 & / & 4.24\\
    乐跑 & 5.16 & 3.10\\
    本科生劳育实践 & 7.20 & 2.19\\
    大文大理题目征集 & 期末 & 3.8\\
    5月免费上网 & ? & 3.9\\
    外教社杯 & 5.27 & 3.12\\
    江苏创青春赛事 & 4.30 & 3.26\\
    浦口音乐跑 & 5.30 & 3.31\\
    仙林校区志愿法律咨询 & / & 4.4\\
    青春活力大赛 & 5.17 & 4.7\\
    在校生自愿体检 & 6.20 & 4.8\\
    南大购买WPS & / & 4.8\\
    中美中心2025年证书项目 & 5.24 & 4.14\\
    粤歌赛决赛 & 5.10 & 4.21\\
    汉字文化技能大赛 & 5.4 & 4.21\\ 
    校博岩画展 & 6.22 & 4.23\\
    CASHL“畅读”活动 & 5.23 & 4.24\\
    江苏高校凤凰读书节 & 6.15 & 4.24\\
    大挑志愿者考核 & 4.30 & 4.27\\
    丝路电商赛事 & 4.30 & 4.28\\
    图书馆征集春日影 & 5.10 & 4.28\\
    汉字知识竞赛 & 5.4 & 4.28\\
    在校生体检 & 5.7 & 4.29\\
    无偿献血 & 5.9 & 4.29\\
    新生创意大赛 & 5.5 & 4.29\\
    \hline
\end{longtable}
\unskip
\unpenalty
\unpenalty}\unvbox\colbbox
\end{multicols}
\begin{multicols}{2}
\pagebreak

\section{讲座}
\begin{tabular}{|>{\centering\arraybackslash}m{.3\textwidth}|m{.06\textwidth}|m{.06\textwidth}|}
    \hline
    讲座 & 开展时间 & 刊载时间\\
    \hline\hline
    从语言到智能 ⸺ 大语言模型的奥秘与应用 & 5.6 & 4.16\\\hline
    在半导体中一瞥“引力子”的身影 & 4.30 & 4.27\\\hline
    记忆与迷途——换一个角度读杜拉斯 & 4.30 & 4.29\\\hline
    创意大赛技术赋能工作坊 & 4.30 & 4.29\\\hline
    儿童脑智发育与人口神经科学 & 5.14 & 4.30\\\hline
    外语学科区域国别研究的路径探索和实践 & 4.30 & 4.29\\\hline
    景观研究的方法与逻辑 & 4.30 & 4.29\\\hline
    Social Simulation with Large Language Model-based Agents & 5.8 & 4.29\\\hline
\end{tabular}
\subsection{[SRTP]4.29-5.8(周二\textasciitilde{}下周四)学术文化活动概览} % 讲座 describer: Hikari
周三(4.30)

1.守正与创新:外语学科区域国别研究的路径探索和实践

2.景观研究的方法与逻辑:以南京燕子矶景观的形塑为例

3.记忆与迷途——换一个角度读杜拉斯

周四(5.8)

Social Simulation with Large Language Model-based Agents
\\详见:\url{https://mp.weixin.qq.com/s/EpnAppmOspHlpla70UcNJw}

\subsection{袁筱一:记忆与迷途——换一个角度读杜拉斯} % 讲座 describer: Hikari
主题:记忆与迷途——换一个角度读杜拉斯
\\时间:2025年4月30日(周三)15:00-17:00
\\主讲人:袁筱一 上海社会科学院文学所所长
\\与谈人:黄荭 南京大学法语系教授
\\地点:南京大学仙林校区仙Ⅱ-112教室
\\详见:\url{https://mp.weixin.qq.com/s/YHkdmb-NXLE0a39uZjDkNw}

\subsection{南京大学首届新生创意大赛技术赋能工作坊} % 讲座 describer: Think Young
时间:4月30日(周三) 16:00
\\地点:鼓楼校区田家炳报告厅
\\详见:\url{https://mp.weixin.qq.com/s/XhIo6Hk-qB_gtabTyd-3Pg}

\subsection{星成长丨满天星商知讲堂——从数据到AI:量化金融的实践} % 讲座 describer: Jolly
讲座主题:从数据到AI:DolphinDB驱动量化金融的高效实践
\\讲座时间:2025年5月7日(周三)下午16:00
\\讲座地点:南京大学仙林校区 仙2-104
\\主讲人:周小华 智臾科技(DolphinDB)创始人、CEO
\\报名请扫描推文内二维码
\\详见:\url{https://mp.weixin.qq.com/s/mokyob6HOYB4RPkZSZQBzg}

\subsection{潘菽心理学论坛第九十四期 | 儿童脑智发育与人口神经科学} % 讲座 describer: Jolly
演讲题目:儿童脑智发育与人口神经科学
\\讲演人:左西年 北京师范大学心理学部教授
\\时间:2025年5月14日(周三)下午2:30
\\地点:南京大学仙林校区社会学院河仁楼101室
\\
\\详见:\url{https://mp.weixin.qq.com/s/_LfRQhg7pZlqyYD8zcJz2w}


\section{再次开放在校学生体检预约} % 校级活动 describer: Hikari
开放时间:2025年05月07日中午12点。请有需要的同学及时预约。
\\体检地点:仙林校医院二楼或鼓楼校医院一楼
\\体检对象:所有在校学生(自愿参加;苏州校区另行通知)。
\\预约方式(任选一):
\\①手机端:微信搜索公众号“南京大学医院”→便民服务→自助服务→统一身份认证登录→点击左下方黑色箭头→预约系统→体检预约→2025年学生中期体检预约;
\\②PC端:打开https://ndyy.nju.edu.cn网页,登录操作同上;
\\用户名:学号;初始密码:身份证号(请及时修改)
\\四、体检项目:(请按需选择一种套餐)
\\基础套餐(免费):血压、身高、体重、腰围、肝功能(ALT)、尿酸、全胸片DR。
\\基础套餐+乙肝两项(优惠):血压、身高、体重、腰围、肝功能(ALT)、尿酸、全胸片DR、乙肝表面抗原、乙肝表面抗体。(建议有注射乙肝疫苗需求的或想了解抗体情况的同学选择此套餐)具体价格见预约界面。
\\升级套餐(优惠)(已覆盖基础套餐体检内容):内科、外科、五官科、血常规、生化七项(ALT、GLU、UA、TC、TG、BUN、Cr)、心电图、彩超(肝胆胰脾双肾)、全胸片(DR)、尿常规。
\\备注:升级套餐可用于实习或求职体检(具体看用人单位要求)具体价格看预约界面。
\\另可根据个人需要,现场可加查幽门螺杆菌检测、甲状腺超声(优惠)。
\\详见:\url{https://mp.weixin.qq.com/s/rMfFRREocC-4gKGLF5vsig}

\section{2025年上半年南京大学红十字会无偿献血周} % 校级活动 describer: Hikari
鼓楼校区
\\时间:5月9日(周五)9:00-16:00 
\\地点:逸夫馆报告厅
\\仙林校区
\\时间:5月8日(周四)9:00-16:00 
\\地点:敬文学生活动中心多功能厅
\\献血条件:
\\01 基本条件
\\年龄范围:18-55周岁(55周岁前献过血且身体条件允许者,可申请延长至60周岁)
\\体重标准:男性≥50kg,女性≥45kg(体重过轻可能导致献血后短暂不适)
\\健康状况:体温正常,无发热、咳嗽等症状血压正常(收缩压90-140mmHg,舒张压60-90mmHg);脉搏60-100次/分(运动员可放宽至≥50次/分)
\\02 暂缓或不宜献血的情况
\\眼部健康:高度近视(≥600度)或近期接受过眼科手术者,因献血时血压波动可能增加视网膜脱落风险。 
\\疾病史: 活动性肺结核或其他慢性感染性疾病;病毒性肝炎(乙肝、丙肝)、艾滋病、梅毒等传染性疾病;心血管疾病、严重贫血、白血病等血液系统疾病
\\03 特殊时期
\\女性生理期及前后三天(避免因铁流失加重身体负担)
\\感冒、急性肠胃炎痊愈未满一周;拔牙、小手术后未满半个月(需确保伤口完全愈合)
\\04 其他情况
\\妊娠期、哺乳期女性;有晕血、晕针史或体质虚弱者(建议咨询医生后再决定)
\\详见:\url{https://mp.weixin.qq.com/s/Lgyyyw_zpdbttfhzKH_ydg}

\section{党代会心声征集活动} % 校级活动 describer: Hikari
现面向南京大学全体师生员工、海内外校友、关心和支持南京大学发展的社会各界人士发起党代会心声征集活动。
\\征集内容包括但不限于:
\\回顾南京大学五年来改革发展新成就和奋进卓越新气象,或展望南京大学未来发展。
\\投稿方式:通过点击文末“阅读原文”、扫描二维码或将相关材料发送至邮箱news@nju.edu.cn(请注明联系方式)进行投稿。
\\详见:\url{https://mp.weixin.qq.com/s/9sR2CgeDzNZeTN9N7ZkDjw}

\section{南京大学台湾青年齐鲁文化寻根之旅} % 校级活动 describer: zty
招募对象:南京大学台湾籍学生
\\招募人数:30人
\\活动时间:2025年5月15日(星期四)-18日(星期天)
\\日程安排详见原文
\\报名方式:扫描原文中二维码进行报名,截止时间5月7日,名额有限,先到先得
\\详见:\url{https://mp.weixin.qq.com/s/xjy3vpdTOwhoPAygC8URsw}

\section{南大博物馆五一开放公告} % 校级活动 describer: Ando
南京大学博物馆五一假日期间:5月1日(周四)、5月2日(周五)、5月3日(周六),闭馆3天;5月4日(周日)、5月5日(周一),一楼和四楼展厅正常开放。
\\详见:\url{https://mp.weixin.qq.com/s/xaqs22aghAs6Kl17SDfpnA}

\section{南青帮推丨南京大学首届新生创意大赛报名开启!} % 校级活动 describer: Cirlpso
参赛对象:主要为南京大学2024级本科生。可以个人形式报名参赛,也可以团队形式参赛,以团队形式参赛的,人数不超过3人(一年级本科生人数不少于2/3),鼓励跨书院跨专业组队。每支队伍可邀请1名指导教师提供项目优化建议。
\\报名阶段:即日起至2025年5月5日。参赛团队在5月5日前点击链接提交队伍报名信息即可完成报名。为鼓励跨学科组队,我们设有赛事咨询群供大家组队、咨询;同时,我们将在报名期间举行一次技术分享会,届时将介绍有关本次比赛的更多信息并为大家介绍平台使用方法。报名链接:https://docs.qq.com/form/page/DRkxuV3ZjQXV0a0VW
\\初审评定:5月18日前需要提交作品,主办方将组织相关部门和专家对作品进行评选,并为进入决赛的队伍分配导师指导孵化。评审综合考虑项目原创性、实用性、技术可行性,评选出创意与价值并重的项目。
\\决赛路演:5月下旬至6月上旬期间(具体时间另行通知)进行决赛路演,需要各位同学制作PPT进行现场展示,评审老师们会从创造性,可实现性,问题针对性等方面进行现场评审和提问,决出最终奖项归属。
\\详见:\url{https://mp.weixin.qq.com/s/_RRyWaTrNVgYygYTAv3gLQ}
\section{院级活动}
\begin{tabular}{|>{\centering\arraybackslash}m{.3\textwidth}|m{.06\textwidth}|m{.06\textwidth}|}
\hline
    活动 & 开展时间 & 刊载时间\\
    \hline\hline
    文院剧本创作研讨会 & 9.30 & 3.2\\
    物院征集课程指南 & 6.15 & 3.3\\
    地海征集春日影 & 6.15 & 3.14\\
    法院党建征文 & 5.20 & 4.2\\
    四院音乐节 & 5.11 & 4.7\\
    商院征集 & 5.5 & 4.8\\
    物院运动打卡 & 5.14 & 4.12\\
    电院征集 & 5.11 & 4.22\\
    智院摄影 & 5.6 & 4.22\\
    商院征集 & 5.9 & 4.27\\
    新传读书 & 4.30 & 4.27\\
    \hline
\end{tabular}

\section{社团活动}
\begin{tabular}{|>{\centering\arraybackslash}m{.3\textwidth}|m{.06\textwidth}|m{.06\textwidth}|}
    \hline
    社团活动 & 开展时间 & 刊载时间\\
    \hline\hline
    天文台开放日 & / & 1.6\\
    重唱诗歌奖征稿 & 4.30 & 3.31\\
    红会一块走 & 5.20 & 4.21\\
    集庆折子戏 & 5.7 & 4.22\\
    九歌大会 & 5.11 & 4.27\\
    红会图书角 & 5.7 & 4.29\\
    街舞社路演 & 5.9 & 4.29\\
    排协鼓楼杯报名 & 5.5 & 4.29\\
    \hline
\end{tabular}
%这里是写社团活动的,社团活动就是由社团主办、主要针对社团内部人员的活动。不要把非社团活动写在这里。
\subsection{2025南大红会图书角系列活动} % 社团活动 describer: Hikari
(一)自愿捐书阶段(4月29日—5月5日)
\\志愿者自愿捐赠书籍,同时根据捐赠的书籍附上创作品(手绘卡片、明信片、手绘书签等)。红会工作人员将于五一假期结束后进行书籍收集、整理、筛选工作,具体时间和地点请加群关注后续通知。自愿捐书和创作品的志愿者,将获得额外的志愿时长奖励。
\\(二)选书阶段(5月5日—5月7日)
\\书单公布:校红会工作人员将在 QQ 群发布 100 本图书清单(仙林校区和鼓楼校区各50本),标注年级分类与内容简介,供志愿者选择。
\\图书认领:志愿者在群内留言认领书籍,每人可认领 1-3 本,先到先得,认领后需在 24 小时内确认,逾期释放名额。南大红会统一采购图书后,将通知志愿者到指定地点领取,领取时需登记个人信息与图书编号。
\\(三)内容创作阶段(5月7日-5月14日)
\\志愿者在1周内完成书信/明信片撰写及手绘卡片/书签制作,内容需原创,积极向上,符合小学生认知水平,注明志愿者姓名、联系方式及图书名称。
\\(四)回收与审核、整理阶段(5月14日开始)
\\校红会对书信、明信片、短视频等材料进行审核,重点关注内容是否贴合图书主题、是否存在敏感信息,不符合要求的作品需进行修改。红会将根据材料的数量、完成度、创意度、精美度等指标,给予志愿者相应志愿时长认证。
\\报名方式:有意向的志愿者加入活动 QQ 群(群号:1011350167),填写电子报名表,备注年级、专业及兴趣方向。
\\详见:\url{https://mp.weixin.qq.com/s/5Rx3OI9I8-nGP7F6BMDTgw}

\subsection{随机舞蹈|歌单由你定,路演征集中} % 社团活动 describer: Hikari
时间:5.9晚6点半
\\地点:炜华运动场
\\01 INSCHOOL校园随舞  
\\随舞歌单你来定:https://docs.qq.com/form/page/DWnFSdFhJRE9TdkJ5
\\路演招募中:https://docs.qq.com/form/page/DWk1IbkJIanZRRGxU
\\02 视频投稿
\\来到现场,投稿10秒以上随舞视频至抖音,加上tag \#抖音俱乐部与大学生们 \#南京大学抖音俱乐部 @南有音力 @抖音俱乐部。将于5.17日12点统计最高赞前6名发放奖品。                    
\\
\\详见:\url{https://mp.weixin.qq.com/s/-9-OIl72sPlWSjiVZUGNSw}

\subsection{报名|2025春季鼓楼杯} % 社团活动 describer: Hikari
活动时间:2025.5.17-5.18
\\活动地点:南京大学鼓楼校区排球场
\\报名方法 
\\1.本次活动为小型娱乐比赛,欢迎所有本科生、研究生、教职工、毕业校友参赛,谢绝外来人员组队。
\\2.高水平女排可以参赛,但需算在男生名额,且场上最多只能有一位高水平女排;
\\3.报名方法:团队报名,请队长在文中“鼓楼杯报名表”里扫码填写报名信息,报名时请务必真实,报名截止日期为5月5日晚24点(暂定);
\\4.报名费每人10元,用于排协购买奖品和支付裁判费;
\\5.娱乐为主,和气为贵,友谊第一;
\\6.所有参赛队员比赛结束均可获得精美礼品一份\textasciitilde{};
\\7.报名该赛事即所有参赛成员同意对自己安全负责,比赛中出现受伤情况,主办方会提供医疗帮助,但不承担相关责任。
\\比赛规则与特殊规则见原文。
\\详见:\url{https://mp.weixin.qq.com/s/9rUPHYx81ZBzsSBxLo94fQ}
\end{multicols}
\end{document}
