% HEAD BEGIN
\documentclass[letterpaper, 12pt]{article}
\newsavebox\colbbox
\usepackage{graphicx}
\usepackage{multicol}
\usepackage{anysize}
\usepackage{fontspec}
\usepackage[fontset=none]{ctex}
\usepackage{tabularx}
\usepackage{longtable}
\PassOptionsToPackage{hyphens}{url}
\usepackage[breaklinks=true, colorlinks=true]{hyperref}
\expandafter\def\expandafter\UrlBreaks\expandafter{\UrlBreaks\do\a\do\b\do\c\do\d\do\e\do\f\do\g\do\h\do\i\do\j\do\k\do\l\do\m\do\n\do\o\do\p\do\q\do\r\do\s\do\t\do\u\do\v\do\w\do\x\do\y\do\z\do\A\do\B\do\C\do\D\do\E\do\F\do\G\do\H\do\I\do\J\do\K\do\L\do\M\do\N\do\O\do\P\do\Q\do\R\do\S\do\T\do\U\do\V\do\W\do\X\do\Y\do\Z}
% \let\oldurl\url
% \renewcommand{\url}[1]{\begin{sloppypar}\oldurl{#1}\end{sloppypar}}
\setlength\columnsep{30pt}
\marginsize{30pt}{30pt}{10pt}{20pt}
\setmainfont{TeX Gyre Bonum}
\setCJKmainfont[BoldFont=Noto Serif CJK SC Bold, ItalicFont=FandolKai]{Source Han Sans SC}
\setlength{\parindent}{0cm}
% \setCJKmonofont{Noto Sans CJK SC}
\begin{document}
\begin{center}
    \Huge\textbf{南哪大专醒前消息}
\end{center}
\vspace{4mm}
\hrule
\renewcommand\tabularxcolumn[1]{m{#1}}
\begin{tabularx}{\textwidth}{>{\hsize.2\hsize}X>{\hsize.6\hsize}X>{\hsize.2\hsize}X}
    \begin{flushleft}
        2025.4.22\, No.227
    \end{flushleft}
    &
    \begin{center}
        \textit{“秉中持正、求新博闻。”}
    \end{center}
    &
    \begin{flushright}
        \textbf{南京市栖霞区}
    \end{flushright}
\end{tabularx}
\vspace{-3.5mm}
\hrule
\vspace{4mm}
% HEAD END
\centerline{\huge\textbf{活动预告}}
\begin{multicols}{2}
\section{订阅方式和加入编辑部}  
编辑部招聘人才,用爱发电,工作轻松,详情可联系QQ:1329527951 客服小千\\想订阅本消息或获取PDF版(便于查看超链接和往期),可加QQ群:\href{https://qm.qq.com/q/4HL41Nt3sQ}{466863272}.
\section{活动清单}
\setbox\colbbox\vbox{
\makeatletter\col@number\@ne
\begin{longtable}{|>{\centering\arraybackslash}m{.3\textwidth}|m{.06\textwidth}|m{.06\textwidth}|}
    \hline
    活动 & 开展时间 & 刊载时间\\
    \hline\hline
    南大版deepseek & / & 2.22\\
    悦读课程群 & / & 2.24\\
    eScience AI科研助手 & / & 3.11\\
    地科博物馆开放安排 & / & 3.22\\ 
    2025年分流和转专业政策通知 & / & 4.7\\
    乐跑 & 5.16 & 3.10\\
    本科生劳育实践 & 7.20 & 2.19\\
    高教社杯 & 4.25 & 3.5\\
    大文大理题目征集 & 期末 & 3.8\\
    5月免费上网 & ? & 3.9\\
    外教社杯 & 5.27 & 3.12\\
    江苏创青春赛事 & 4.30 & 3.26\\
    浦口音乐跑 & 5.30 & 3.31\\
    程设大赛 & 4.26 & 4.2\\
    仙林校区志愿法律咨询 & / & 4.4\\
    青春活力大赛 & 5.17 & 4.7\\
    在校生自愿体检 & 6.20 & 4.8\\
    南大购买WPS & / & 4.8\\
    24级程设大赛 & 4.27 & 4.11\\
    法治情景剧策划大赛 & 4.23 & 4.11\\
    EL程设大赛 & 4.27 & 4.13\\
    中美中心2025年证书项目 & 5.24 & 4.14\\
    春季学期创新训练计划结题考核通知 & 4.28 & 4.15\\
    植物微景观活动 & 4.23 & 4.16\\
    “天池杯”AI创新大赛 & 4.28 & 4.17\\
    紫砂壶体验课 & 4.24 & 4.18\\
    “食堂‘名厨’”“巡回”“展” & 4.24 & 4.18\\
    新生午餐读书会旁听 & 4.23 & 4.20\\
    大先生剧目招募演员 & 4.23 & 4.21\\
    粤歌赛决赛 & 5.10 & 4.21\\
    十大半决赛 & 4.26 & 4.21\\
    以书换树环保公益活动 & 4.25 & 4.21\\
    汉字文化技能大赛 & 5.4 & 4.21\\ 
    法治嘉年华 & 4.25 & 4.21\\
    \hline
\end{longtable}
\unskip
\unpenalty
\unpenalty}\unvbox\colbbox
\end{multicols}
\begin{multicols}{2}
\pagebreak

\section{讲座}
\begin{tabular}{|>{\centering\arraybackslash}m{.3\textwidth}|m{.06\textwidth}|m{.06\textwidth}|}
    \hline
    讲座 & 开展时间 & 刊载时间\\
    \hline\hline
    从感知到疗愈:人脑音乐加工机制 & 4.25 & 4.11\\\hline
    Ionizing spotlight of Active Galactic Nucleus & 4.23 & 4.11\\\hline
    软件发展与技术漫谈 & 4.29 & 4.16\\\hline
    DeepSeek: 从人工智能到大模型及应用思考 & 4.24 & 4.16\\\hline
    从语言到智能 ⸺ 大语言模型的奥秘与应用 & 5.6 & 4.16\\\hline
    智慧物流助力智能制造 & 4.23 & 4.17\\\hline
    急救技能抓住 “黄金 4 分钟” & 4.25 & 4.18\\\hline
    遗忘的耐候性 & 4.23 & 4.18\\\hline
    2025平安留学行前培训会 & 4.29 & 4.20\\\hline
    花语崛起:拉斐尔前派画作中的花草 & 4.25 & 4.20\\\hline
    《威尼斯商人》的当代隐喻 & 4.23 & 4.20\\\hline
    “法护青春,职路引航”就业季法律公益讲座 & 4.25 & 4.20\\\hline
    网络加速的分布式内存存储系统研究 & 4.23 & 4.21\\\hline
    子图匹配:算法与应用 & 4.23 & 4.21\\\hline
    非遗国家级艺术大师系列讲座 & 4.23 & 4.21\\\hline
    Diagrams and automorphisms on Lie algebras & 4.23 & 4.21\\\hline
    概念的辩证矛盾及其演化 & 4.23 & 4.21\\\hline
    基于机器学习和人工智能的监测大数据采集和分析 & 4.23 & 4.21\\\hline
    从家族记忆到文学名著——《红楼梦》的经典化之旅 & 4.23 & 4.21\\\hline
    外国考古和大众传播 & 4.23 & 4.21\\\hline
    人工智能时代的外语专业毕业论文指导交流会 & 4.24 & 4.21\\\hline
    离散环境下的机器人可靠性导航与移动目标搜索 & 4.24 & 4.22\\\hline
    十三陵水库——在工地上理解新民歌运动 & 4.25 & 4.22\\\hline
    新型拓扑磁结构构筑与电流精准操控 & 4.24 & 4.22\\\hline
    商王朝时期长江与黄河间的文明互鉴 & 4.25 & 4.22\\\hline
    试论萨特的知觉哲学 & 4.28 & 4.22\\\hline
    “大模型部署与使用” & 4.26 & 4.22\\\hline
    
\end{tabular}
\subsection{离散环境下的机器人可靠性导航与移动目标搜索} % 讲座 describer: instruconrevo
此为“软件新技术讲坛” 学术报告
\\时间:2025年4月24日(星期四) 15:00
\\地点:计算机科学技术楼221室
\\报告人:郭宏亮 副研究员 四川大学计算机学院人工智能系
\\详见:\url{https://mp.weixin.qq.com/s/UZXKFXZnqAS_vMTRhstlfA}

\subsection{穹顶讲座(二)| 范雪:十三陵水库——在工地上理解新民歌运动} % 讲座 describer: Ando
从“人定胜天”的豪迈到“笑破枯燥”的烟火气,新民歌将人力劳动转化为现代化工程的“民情”表达。重新发现被遮蔽的社会主义文学实验——这不仅是历史的回响,更是对当代劳动文化困境的深刻追问,时间因此有了再次开始的可能。
\\讲座时间:2025年4月25日(周五)15:00-17:00
\\讲座地点:南京大学仙林校区文学院423会议室
\\报名方式:有意向参与讲座的同学可以扫描下方二维码填写活动报名表,来到现场的同学更有机会优先加入穹顶暑期新民歌实践小分队哦。机会有限,先到先得,快来报名吧!
\\详见:\url{https://mp.weixin.qq.com/s/fYkth_f7YCSB6uMI-b59Rg}

\subsection{新型拓扑磁结构构筑与电流精准操控} % 讲座 describer: instruconrevo
物理学院学术报告会(第58期)
\\报告人:田明亮,安徽大学、中国科学院合肥物质科学研究院
\\时间:2025年4月24日(周四)15:30
\\地点:鼓楼校区唐仲英楼B501
\\详见:\url{https://mp.weixin.qq.com/s/OMUQ6HSWHhy20waX-nLK-g}

\subsection{跟着教授看展览之长江文明系列讲座第四期:尊罍之路} % 讲座 describer: instruconrevo
题目:尊罍之路:商王朝时期长江与黄河间的文明互鉴
\\主持人:赵东升
\\主讲人:唐际根
\\时间:2025.04.25(周五) 下午14:30-16:30
\\地点:南京大学仙林校区星云楼1F
\\详见:\url{https://mp.weixin.qq.com/s/JajRlXqgH5K8NruCrfG4Rg}

\subsection{实在与构造——试论萨特的知觉哲学} % 讲座 describer: LucyRiver
时间:4月25日 15:00-17:00
\\地点:哲学学院218室
\\主讲人:黄笛(华东师范大学副教授)
\\参与人:王恒(南京大学教授)、王士盛(南京大学助理研究员)
\\详见:\url{https://mp.weixin.qq.com/s/u3Ef0cPzRKyaB2YshepA9w}

\section{ITALK 大模型部署与使用} %  describer: instruconrevo
内容包括:
\\大模型部署技术结构:模型压缩·量化·剪枝
\\大模型使用方案:大模型发展生态·软件平台·流行的应用·DIY产品·学术界前景
\\主讲人:彭志玉
\\时间:4月26日19:00\textasciitilde{}20:30
\\活动群聊:114259438
\\地点:仙2-303 苏州校区同步教室待定
\\主办方:南京大学计算机学院学生会南京大学学生IT侠互助协会
\\详见:\url{https://mp.weixin.qq.com/s/WdKdPkdcsDrdWg2wsUrm2w}

\section{江苏卫视“荔枝ai”大学生辩论大会} %  describer: instruconrevo
时间:4月25日晚 19:00
\\地点:张心瑜剧场
\\辩题:小组作业人多好办事/人多拖后腿
\\对阵:南京大学(正方)江南大学(反方)
\\详见:\url{https://mp.weixin.qq.com/s/TtGIf4g4MV9nJZVFkjWwgQ}

\section{零基础舞蹈教学活动第三弹} % 校级活动 describer: Ando
在夏意渐浓的四月下旬,零基础舞蹈教学活动第三弹将在热烈期盼中如期而至,全程参与本次活动并按规定完成签到签退,可抵免一次校园跑打卡,具体信息如下:零基础华尔兹
\\活动时间:4月25日 周五 17:30-18:30(具体时间将因天气状况灵活调整)
\\活动地点:仙林校区炜华体育场
\\报名方式:扫描下方二维码或现场直接报名
\\详见:\url{https://mp.weixin.qq.com/s/o3zzA2v4mCFE_31g9sLzDA}

\section{【补采】2026届普通全日制本科毕业生图像采集春季学期补采通知} % 校级活动 describer: Cirlpso
1. 集中补采:
\\时间:4月26日(本周六)9:00-17:00,地点:仙林校区 逸A-117、逸B-101、逸B-105
\\2. 线上补采
\\无法参加集中采集的同学可于4月23日9:00至4月26日17:00自行采集。为保证照片质量,请同学尽量参加集中采集。
\\二、采集对象
\\未参加采集以及参加采集但图像不合格的(2026届普通全日制本科毕业生,包括四年制2022级和五年制2021级学生、2024级第二学士学位生)。
\\2021级因无法获取学信网“图像采集码”而未参加过毕业生图像信息采集的延期毕业生。
\\详见:\url{https://jw.nju.edu.cn/88/0f/c26263a755727/page.htm}
\section{院级活动}
\begin{tabular}{|>{\centering\arraybackslash}m{.3\textwidth}|m{.06\textwidth}|m{.06\textwidth}|}
\hline
    活动 & 开展时间 & 刊载时间\\
    \hline\hline
    文院剧本创作研讨会 & 9.30 & 3.2\\
    物院征集课程指南 & 6.15 & 3.3\\
    地海征集春日影 & 6.15 & 3.14\\
    法院党建征文 & 5.20 & 4.2\\
    地学趣运会 & 4.26 & 4.9\\
    四院音乐节 & 5.11 & 4.7\\
    商院征集 & 5.5 & 4.8\\
    物院运动打卡 & 5.14 & 4.12\\
    地海图书漂流 & 4.23 & 4.16\\
    文院茶话会 & 4.24 & 4.20\\
    希音杯 & 4.25 & 4.20\\
    开甲剧本杀 & 4.26 & 4.21\\
    法院研习班 & 4.27 & 4.22\\
    电院征集 & 5.11 & 4.22\\
    地海宣讲 & 4.27 & 4.22\\
    商院分享 & 4.26 & 4.22\\
    智院摄影 & 5.6 & 4.22\\
    软院分享 & 4.24 & 4.22\\
    \hline
\end{tabular}
\subsection{“职”引迷津 | 职场半年谈第三期——揭秘银行就业} % 院级活动 describer: Jolly
时间:2025年4月26日周六晚20:00-22:00
\\地点:线上腾讯会议(会议号请关注微信群通知)
\\扫描二维码填写报名问卷(微信群二维码在问卷内)
\\详见:\url{https://mp.weixin.qq.com/s/SsgzDBo-bbDGOdjj0ngmIw}

\subsection{电院五一vlog创意征集} % 院级活动 describer: instruconrevo
征集时间:即日起至5月11日
\\征集对象:电子科学与工程学院全体本科生与研究生 毓琇书院电子信息类大一新生
\\提交方式:有意参加的同学请先加入QQ群,并通过下方链接提交:https://table.nju.edu.cn/dtable/forms/83cc4170-c11a-4e04-bc53-3302bbfd0ce4/
\\专题包括:「劳动节」「回家・旅游」「学习・科研」
\\一等奖为罗技G102鼠标、CAMEL骆驼轻便旅游双肩包
\\详见:\url{https://mp.weixin.qq.com/s/MlBCpD5-oGId8bE9IauElw}

\subsection{地海院地理科普宣讲} % 院级活动 describer: instruconrevo
亮点包括地球知识科普、科技互动课堂、亲子共创时光、地球守护者签名仪式等
\\时间:2025年4月27日(周日)15:00-16:30
\\地点:南京市栖霞区西岗幼儿园仙林湖园
\\对象:幼儿亲子家庭
\\志愿者招募:
\\招募对象:南京大学地理与海洋科学学院全体在读学生
\\招募人数:10人
\\报名链接:https://table.nju.edu.cn/dtable/forms/1b7c378d-7f5f-48cd-a29e-1cb6614476e6/
\\志愿者群:1003586737
\\详见:\url{https://mp.weixin.qq.com/s/ANpvbw7rjeDBJz9R1y9fvQ}

\subsection{AI学院摄影大赛开启} % 院级活动 describer: Noname
投稿时间:4月22日-5月6日
\\投票时间:5月11日-5月18日
\\参赛赛道:
\\1.美景赛道:自然风光、校园景色等(不含人物)
\\2.人文活动赛道:游玩踏青、户外运动等(需包含人物)
\\参与方式:将作品发送至邮箱:njuai\_gradunion@163.com
\\参赛要求:活动面向人工智能学院全体老师及研究生,每人限每个赛道投1张照片(不可拼图,尺寸不限)
\\一等奖:教育超市50元代金券或相同价值文创
\\二等奖:教育超市20元代金券或相同价值文创
\\三等奖:文创(徽章/卡套/扇子/书签/鼠标垫)任选
\\详细参赛要求见原文
\\详见:\url{https://mp.weixin.qq.com/s/xjgVSZ7SnU12JWSwdf8aXQ}

\subsection{EL程序设计大赛交互组赛前辅导暨经验分享会} % 院级活动 describer: Noname
活动时间:4月24日(周四) 晚19:00
\\活动地点:鼓楼校区费彝民楼B822
\\详见:\url{https://mp.weixin.qq.com/s/ZAvELENIo48IIxI_SqResQ}

\subsection{“鉴定式案例分析方法”研习班(刑法专题)} % 社团活动 describer: instruconrevo
面向群体:南京大学法学院全日制本科生与研究生 社会科学试验班全日制本科生
\\时间:4月27日(周日) 5月11日(周日)
\\地点:南京大学仙林校区(教室待定)
\\报名截止至4月26日 中午12:00
\\QQ群、报名二维码及活动详细流程见原文
\\详见:\url{https://mp.weixin.qq.com/s/kgDkoM-Yn37X6tF15dsZaA}

\section{社团活动}
\begin{tabular}{|>{\centering\arraybackslash}m{.3\textwidth}|m{.06\textwidth}|m{.06\textwidth}|}
    \hline
    社团活动 & 开展时间 & 刊载时间\\
    \hline\hline
    天文台开放日 & / & 1.6\\
    重唱诗歌奖征稿 & 4.30 & 3.31\\
    体育舞蹈教学 & 4.25 & 4.1\\
    吉他社春日音 & 4.26 & 4.4\\
    汉服社摆摊 & 4.26 & 4.17\\
    车协科普 & 4.26 & 4.17\\
    心协团辅活动 & 4.23 & 4.17\\
    匿名评诗会 & 4.26 & 4.20\\
    红会一块走 & 5.20 & 4.21\\
    法院研习班 & 4.27 & 4.22\\
    集庆折子戏 & 5.7 & 4.22\\
    \hline
\end{tabular}
%这里是写社团活动的,社团活动就是由社团主办、主要针对社团内部人员的活动。不要把非社团活动写在这里。


\subsection{“粉墨寄情” | 南京大学集庆会馆二十二周年折子戏} % 社团活动 describer: instruconrevo
演出时间:2025年5月11日14:00
\\演出地点:南京大学仙林校区张心瑜剧场
\\派票时间及地点:5月7日(星期三)11:00—13:00 鼓楼校区南园小广场
\\5月9日(星期五)11:00—13:00 仙林校区四五六食堂前
\\领票攻略:关注微信公众号并转发推文即可领票
\\
\\详见:\url{https://mp.weixin.qq.com/s/PV94gKbVFFeRRmJvtDjjPQ}

\end{multicols}
\end{document}
