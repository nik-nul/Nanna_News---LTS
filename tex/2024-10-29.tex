% HEAD BEGIN
\documentclass[letterpaper, 12pt]{article}
\newsavebox\colbbox
\usepackage{graphicx}
\usepackage{multicol}
\usepackage{anysize}
\usepackage{fontspec}
\usepackage[fontset=none]{ctex}
\usepackage{tabularx}
\usepackage{longtable}
\PassOptionsToPackage{hyphens}{url}
\usepackage[breaklinks=true, colorlinks=true]{hyperref}
\expandafter\def\expandafter\UrlBreaks\expandafter{\UrlBreaks\do\a\do\b\do\c\do\d\do\e\do\f\do\g\do\h\do\i\do\j\do\k\do\l\do\m\do\n\do\o\do\p\do\q\do\r\do\s\do\t\do\u\do\v\do\w\do\x\do\y\do\z\do\A\do\B\do\C\do\D\do\E\do\F\do\G\do\H\do\I\do\J\do\K\do\L\do\M\do\N\do\O\do\P\do\Q\do\R\do\S\do\T\do\U\do\V\do\W\do\X\do\Y\do\Z}
% \let\oldurl\url
% \renewcommand{\url}[1]{\begin{sloppypar}\oldurl{#1}\end{sloppypar}}
\setlength\columnsep{30pt}
\marginsize{30pt}{30pt}{10pt}{20pt}
\setmainfont{TeX Gyre Bonum}
\setCJKmainfont[BoldFont=Noto Serif CJK SC Bold, ItalicFont=FandolKai]{Noto Sans CJK SC}
\setlength{\parindent}{0cm}
% \setCJKmonofont{Noto Sans CJK SC}
\begin{document}
\begin{center}
    \Huge\textbf{南哪大专醒前消息}
\end{center}
\vspace{4mm}
\hrule
\renewcommand\tabularxcolumn[1]{m{#1}}
\begin{tabularx}{\textwidth}{>{\hsize.2\hsize}X>{\hsize.6\hsize}X>{\hsize.2\hsize}X}
    \begin{flushleft}
        2024.10.29\, No.102
    \end{flushleft}
    &
    \begin{center}
        \textit{“克明峻德。”}
    \end{center}
    &
    \begin{flushright}
        \textbf{南京市栖霞区}
    \end{flushright}
\end{tabularx}
\vspace{-3.5mm}
\hrule
\vspace{4mm}
% HEAD END
\centerline{\huge\textbf{活动预告}}
\begin{multicols}{2}
    \section{订阅方式和加入编辑部}  
编辑部招聘人才,用爱发电,工作轻松,详情可联系QQ:1329527951 客服小祥\\想订阅本消息或获取PDF版(便于查看超链接和往期),可加QQ群:\href{https://qm.qq.com/q/VXIW7fgsEe}{849644979}.
\section{Deadline Ongoing}
\setbox\colbbox\vbox{
\makeatletter\col@number\@ne
\begin{longtable}{|c|c|c|}
    \hline
    消息(未见ddl的,不刊) & 截止日期 & 刊载日期\\
    \hline\hline
    紫藤学刊征稿 & 12.15 & 10.22\\
    仙林校史馆招募讲解员 & 10.30 & 9.12\\
    本科生暑期课程评教 & 10.31 & 9.19\\
    黑匣招募 & 11.1 & 10.19\\
    学位英语考试报名 & 11.3 & 10.17\\
    校运会 & 11.8 & 10.21\\
    后革命鲁迅研究征文 & 11.10 & 10.8\\
    大创训练计划申报 & 11.18 & 9.24\\
    招生宣传创意征集大赛 & 11.18 & 10.21\\ 
    EBSCO数据库检索大赛 & 11.20 & 10.3\\
    文院征稿 & 11.20 & 10.20\\
    乐跑 & 12.8 & 10.12\\
    国际访学计划申报 & 11.22 & 10.22\\
    毓琇书院宿舍评比 & 10.31 & 10.22\\
    南大献血周 & 10.31 & 10.24\\
    百团大战 & 11.3 & 10.26\\
    南新读书会 & 10.30 & 10.26\\
    联合国南大宣讲 & 10.30 & 10.27\\
    Passion公开课 & 10.29 & 10.27\\
    仙林草地音乐节 & 11.3 & 10.27\\
    教超打折 & 10.31 & 10.28\\
    青鸟分享会 & 10.31 & 10.28\\
    南大网球排位赛报名 & 11.1 & 10.28\\
    普通话测试网络报名 & 11.12 & 10.29\\
    全球学习交流展 & 11.4 & 10.29\\
    苏州走进南大系列活动 & 10.31 & 10.29\\
    健雄捡秋活动 & 11.3 & 10.29\\
    
    \hline
\end{longtable}
\unskip
\unpenalty
\unpenalty}\unvbox\colbbox
\end{multicols}
\hrule
\pagebreak
\begin{multicols}{2}

\section{讲座}
\begin{tabular}{|c|c|c|}
    \hline
    往期讲座 & 开展日期 & 刊载日期\\
    \hline\hline
    《电池及电化学能...》 & 11.24 & 10.3\\
    《专利查新与规避...》 & 12.19 & 10.3\\
    《与<自然>编辑对...》 & 10.30 & 10.16\\
    图书馆系列讲座 & 12.3 & 10.20\\
    《志工人力资源的...》 & 11.4 & 10.23\\
    《华人社会工作的...》 & 11.4 & 10.23\\
    《大模型技术的发...》 & 10.31 & 10.25\\
    《从华盛顿看台湾...》 & 10.30 & 10.26\\
    《拓扑材料和交变...》 & 10.30 & 10.26\\
    《作为社会互动的...》 & 10.30 & 10.27\\
    《新质生产力深度...》 & 10.30 & 10.28\\
    《从全球视角探讨...》 & 11.4 & 10.28\\
    《超导体中的竞争...》 & 10.31 & 10.28\\
    《浅谈欧拉数与组...》 & 10.30 & 10.29\\
    《瑞典电力和氢能...》 & 11.7 & 10.29\\
    \hline
\end{tabular}

1.数学学院本科生论坛(教师系列第85讲)\\
题目: 浅谈欧拉数与组合拓扑\\
报告人:于立\\
时间:10月30日(星期三) 16:00-17:30\\
地点:戊己庚四楼北\\
腾讯会议:870-7007-3326\\
摘要: 欧拉数是拓扑学中最基本的不变量,在拓扑学和几何学的研究中都有重要的地位。本报告将解释欧拉数作为组合流形不变量的一种本质特征,另外将介绍组合拓扑学的一些基本概念和事实。\\
\url{https://mp.weixin.qq.com/s/XyganFzVe4MFckhlEegk2A}\\
2.工程管理学院管理科学论坛\\
题目: 瑞典电力和氢能源市场展望\\
报告人:唐讴\\
时间:11月7日(星期四) 10:00-11:30\\
地点:协鑫楼108教室\\
摘要: 欧盟预计氢气将在未来的能源系统中发挥至关重要的作用,本讲座将提供两个氢气生产案例,分别是瑞典核电站和氢能源加油站。氢气生产将为瑞典的大型电力生产设施提供一个新的备选方案。\\
\url{https://mp.weixin.qq.com/s/2STXrFX8jMB1ZEQNgj7T_A}\\



\section{2024年普通话水平测试网络报名通知}
考试具体报名时间安排如下:\\
2024年11月14日(周四下午) 开放名额:120人\\
网络报名时间:即日起——2024年11月11日(额满为止)\\
2024年11月15日(周五下午) 开放名额:120人\\
网络报名时间:即日起——2024年11月12日(额满为止)\\
报到时间:按报名先后顺序下午14:00开始考试(以准考证上的具体报到时间为准,请携带身份证、准考证参加考试,届时在文学院4楼435房间依次排队入场)。 \\
具体链接:\url{https://jw.nju.edu.cn/fd/f6/c26263a720374/page.htm}\\




\section{南大2024长三角全球学习交流周}
集中展台:11月4日 杜厦图书馆报告厅外\\
南大师生平安留学交流会:11月4日10:30-11:30 杜厦图书馆大众书局\\
美国专场宣讲会:11月4日15:00-17:00 杜厦图书馆大众书局\\
加利福尼亚大学戴维斯分校:10月30日11:00-12:00 仙I-303\\
巴黎政治大学:10月30日17:00-18:00 仙I-303\\
墨尔本大学:10月31日19:00-20:00 地点由商学院通知\\
香港科技大学:11月5日16:00-18:00 仙II-114\\
具体链接:\url{https://mp.weixin.qq.com/s/yrCQyGQNzHlui2XBF9aegw}\\

% \section{“高新青年说”走进南大}
% 本次活动将介绍苏州高新区创新创业环境和青年人才政策,有企业家带来就业创业经验分享,围绕江苏选调、定岗特选等互动交流。现场还有苏绣传承人讲解和传授刺绣技艺,每一位同学均可沉浸式体验江南文化魅力,亲手制作一幅苏绣作品。\\
% 时间:10月31日(周四)15:00\\
% 地点:仙林校区学生就业指导中心303A\\
% 报名方式:进入南大青年公众号今日推文\url{https://mp.weixin.qq.com/s/b3_DgpZRxWCNhq3N54eAvQ},扫描文中二维码报名。\\

\section{物院午间学术交流会第八期}
时间:11月2日(周六)11:30-13:00\\
地点:鼓楼校区新教学楼405\\
1.报告人:梁杰辉,杜灵杰教授课题组2021级博士研究生\\
报告主题:Evidence for chiral graviton modes in fractional quantum Hall liquids(在分数量子霍尔液体中手性引力子模式的证据)\\
2.报告人:邵楷,陈伟教授课题组2021级博士研究生\\
报告主题:Non-Hermitian MoiréValley Filter(非厄米莫尔谷滤波器)\\
参与方式:活动可提供午餐,如需午餐,请扫码报名,午餐名额为35人,名额有限,先到先得。报名方式:填写推文中的报名问卷并扫码进群。成功报名午餐者会在活动举办前收到邮件通知,凭邮件通知获取凭证现场领取午餐。如无需午餐,可直接自行前往。\\
原文详见\url{https://mp.weixin.qq.com/s/8tHueNaHAbs-WwHcJrPYgQ}

\section{黑匣福利 | 卓翔导演影片放映交流会}
放映影片:

1.《一个武生》(2015/66分钟/彩色/普通话、粤语)\\
2.《一棵树》(2019/19分钟/彩色/韩语、印尼语、英语)\\
活动时间:11月4日(周一)19:00-21:30\\
活动地点:南京大学文学院报告厅\\
特邀导演:卓翔(导演/监制)\\
特邀嘉宾:沈晓平(南京艺术学院传媒学院副教授)\\
学术主持:杨柳(南京大学文学院副教授)\\
主办:南京大学黑匣子放映室、南京大学文学院\\
活动流程:现场观影——嘉宾对谈——提问交流\\
详情请查看\url{https://mp.weixin.qq.com/s/ymOZ85Hi_RwiPQ9vBRAFbA}
\section{排协秋季院系杯赛程预告}
时间:10月30日(周三)12:30\\
地点:方肇周副馆\\
女排小组赛:电子-文院

\section{第二十五届江平民商法学奖学金南京大学法学院评选}
2024年度江平民商法奖学金奖金金额为每名获奖者人民币1万元,南京大学法学院名额为2名。奖励对象为本院在校全日制本科生。\\
基本条件:1.法学院法学专业三年级、四年级学生。\\
2.已修习下列民商法课程中的至少四门课程:民法总论、物权法、债法、合同法、侵权责任法、婚姻家庭法、商法总论、公司法、保险法、破产法、证券法、知识产权法,并且其中一门课程的单科成绩在90分以上(含90分)或者两门课的单科成绩在85分以上(含85分)。\\
要求:2024年11月17日24:00之前向学院提交纸质材料和电子材料(纸质材料交至鼓楼校区西南楼319室张莉老师处,电子材料寄至jnzhangli@nju.edu.cn,咨询电话:83592112)\\
(一)初试:初试采取笔试方式,以客观题为主要题型,考试范围覆盖民商法课程的基本内容,时间为2024年11月24日(周日)8:30-12:00,与中国政法大学同步开考。评选工作小组根据考试成绩,取前5名参加复试。\\
(二)复试:复试以当堂作文或案例分析的形式进行。考试时间地点另行通知。\\
(三)评定:评选工作小组将两轮成绩加总,根据成绩名次评选出2024年度江平民商法学奖学金获奖学生。名单经公示无异议,报江平民商法学基金会审定。\\
(四)领奖:获奖学生由本院教师带队赴中国政法大学颁奖典礼现场领奖。
原链接:\url{https://mp.weixin.qq.com/s/kY4sS86LJ8BSNoEvLE6qBw}


\section{历史学院宣传技能内训}
【活动对象】:历史学院本科生\\
【内训时间】: 11月3日 10:30-11:30\\
【内训地点】: 仙I-315\\
【主要内容】: 公众号美编、平面设计、摄影摄像、文案写作\\
报名详见\url{https://mp.weixin.qq.com/s/SlaTe5RAazC8gwU5nQxiBg}

\section{校园苏州日走进南京大学系列活动邀请函}
PART.01\\
苏州创新创业环境推介会\\
活动时间:2024年10月31日14:00-14:30\\
活动地点:南京大学仙林校区国际会议中心紫金厅\\
PART.02\\
高层次人才招聘会\\
活动时间:2024年10月31日14:30-17:00\\
活动地点:南京大学仙林校区国际会议中心外广场\\
进入原文链接扫码可查看具体岗位信息\\
PART.03\\
“苏创”青春市集巡展\\
活动时间:2024年10月31日14:30-17:00\\
活动地点:南京大学仙林校区国际会议中心外广场\\
苏式漆扇DIY体验、香囊DIY体验、中医问诊、小清新文创打卡惊喜不断\\
其他相关活动:\\
1.重点学院对接会暨产学研合作对接会\\
时间:10月31日09::30-12:00\\
地点:南京大学软件学院费彝民楼B822报告厅\\
南京大学电子科学与工程学院东门厅\\
南京大学现代工程与应用科学学院X-301\\
2.“高新青年说”走进南京大学\\
时间:10月31日15:00-17:00\\
地点:南京大学仙林校区学生就业指导中心 303A\\
(需在下方链接中报名)\\
3.大学生职业生涯规划宣讲暨苏州校区专场实习就业双选会\\
时间:10月31日18:30-20:00\\
地点:南京大学苏州校区南雍楼西209\\
详情链接\url{https://mp.weixin.qq.com/s/RFls6wvyX5RUETMPhTi2xg}\\
“高新青年说”报名链接\url{https://mp.weixin.qq.com/s/b3_DgpZRxWCNhq3N54eAvQ}\\
软件学院对接会\url{https://mp.weixin.qq.com/s/1ZAlO5TzjPPzR-GNR0LUkQ}\\

\section{健雄书院:“捡起秋天的信”征集活动}
活动形式:赏秋景,拾秋物,品秋色\\
留下一张带有秋色的相片、制作一个含有秋意的小制品都是不错的“捡秋“打开方式\\
活动内容:本次活动为摄影征集活动\\
以“捡秋”为主题,鼓励大家到户外走走,用影像记录下身边的秋日;同时,你也可以拍下自己用落叶等“秋物”制作的小手工,我们也很期待看到你的创意和巧思\\
征集时间:即日起至11月3日23:59前\\
征集对象:健雄书院全体师生\\
拍摄地点:校内、校外风光均可,不局限于南大校园,可以是公园,也可以是风景区(在外出行请注意安全哦!)\\
征集要求:①主题为“捡秋”,拍摄类别不限②上传作品数量不限,单张作品大小不限,同学们可自由发挥③作品命名格式“学号+姓名”,多张照片请用文件夹上传,不需要压缩包④禁止抄袭和使用他人的摄影作品,若侵权,后果自行承担\\
奖励:①本次活动以“美育”录入五育系统,计入敦行成绩单②摄影活动结束后我们将在投稿作品中选择优秀作品于书院公众号推送中展示\\
提交链接见\url{https://mp.weixin.qq.com/s/mPVxWildoKC64IfFL_aryw}

\section{健雄书院:心理工作室-第一期小熊树洞话题征集}
你是否有过这样的时刻:\\
1学习压力大到喘不过气?\\
2人际关系让你头疼不已?\\
3情绪起伏不定,找不到出口?\\
我们有三个温馨的小角落:小熊树洞、林间来信和小熊驿站\\
这里,我们共同构建一个倾听与分享的社区,包括小熊树洞,林间来信,小熊驿站\\

(附:鼓楼校区心理咨询室地址:南园21舍202\\
预约电话:025-83597219
预约时间:周一至周五8:00-12:00、13:30-21:30\\
线上心理支持南大心理支持QQ号:2225806137(周一到周五18:30-21:00,可语音通话)\\
南大心理咨询邮箱:njuxlzx@nju.edu.cn)\\
加入树洞详见\url{https://mp.weixin.qq.com/s/jDIkfE_vIRwDBPtwqRfPHA}

\end{multicols} 
\hrule
\vspace{4mm}
\centerline{\huge\textbf{参考消息}}
\begin{multicols}{2}
\section{关于熄灯政策同后勤领导的交流}
四栋有同学发声,并在校园集市发布后续。现将原文展示如下:

「关于熄灯政策同后勤领导的交流

朋友们好,由于我上周在四栋贴大字报指责熄灯政策,后勤部党委书记找到我同我面谈。现在和各位朋友共享我从书记处得知的信息。

1.本科生熄灯2013年开始实行,仙林校区电路很早就完成了熄灯改造。但鼓楼校区由于22年才接受本科生,导致未能安装定时熄灯电路,所以鼓楼校区没有执行熄灯。

2.熄灯已经是2013年的政策了,后勤部门通过鄙人的大字报和同学们的反馈,已经意识到强制熄灯模式过时了。后勤处正在向上级反映,后续可能会以民主形式修订这一政策。

3.10月29日,校方已经召开了座谈会,重审2013年的熄灯文件。

4.在四栋电路修缮完毕前,不会执行熄灯:在改良版规定出台前,不会不执行熄灯。

我个人以为,改变强制熄灯是有希望的,还需要同学诸君同心同德,一并努力。」
\end{multicols} 
\hrule
\vspace{4mm}
% APPENDIX BEGIN
\centerline{\huge\textbf{附录}}
\begin{figure}[htbp]
    \centering
    \begin{minipage}[b]{0.32\textwidth}
        \centering
        % \includegraphics[width=0.5\textwidth]{Sample.png}
        \caption{Sample}
    \end{minipage}
    \begin{minipage}[b]{0.32\textwidth}
        \centering
        % \includegraphics[width=0.5\textwidth]{Sample.png}
        \caption{Sample}
    \end{minipage}
        \begin{minipage}[b]{0.32\textwidth}
        \centering
        % \includegraphics[width=0.5\textwidth]{Sample.png}
        \caption{Sample}
    \end{minipage}
\end{figure}
\end{document}