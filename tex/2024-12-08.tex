% HEAD BEGIN
\documentclass[letterpaper, 12pt]{article}
\newsavebox\colbbox
\usepackage{graphicx}
\usepackage{multicol}
\usepackage{anysize}
\usepackage{fontspec}
\usepackage[fontset=none]{ctex}
\usepackage{tabularx}
\usepackage{longtable}
\PassOptionsToPackage{hyphens}{url}
\usepackage[breaklinks=true, colorlinks=true]{hyperref}
\expandafter\def\expandafter\UrlBreaks\expandafter{\UrlBreaks\do\a\do\b\do\c\do\d\do\e\do\f\do\g\do\h\do\i\do\j\do\k\do\l\do\m\do\n\do\o\do\p\do\q\do\r\do\s\do\t\do\u\do\v\do\w\do\x\do\y\do\z\do\A\do\B\do\C\do\D\do\E\do\F\do\G\do\H\do\I\do\J\do\K\do\L\do\M\do\N\do\O\do\P\do\Q\do\R\do\S\do\T\do\U\do\V\do\W\do\X\do\Y\do\Z}
% \let\oldurl\url
% \renewcommand{\url}[1]{\begin{sloppypar}\oldurl{#1}\end{sloppypar}}
\setlength\columnsep{30pt}
\marginsize{30pt}{30pt}{10pt}{20pt}
\setmainfont{TeX Gyre Bonum}
\setCJKmainfont[BoldFont=Noto Serif CJK SC Bold, ItalicFont=FandolKai]{Noto Sans CJK SC}
\setlength{\parindent}{0cm}
% \setCJKmonofont{Noto Sans CJK SC}
\begin{document}
\begin{center}
    \Huge\textbf{南哪大专醒前消息}
\end{center}
\vspace{4mm}
\hrule
\renewcommand\tabularxcolumn[1]{m{#1}}
\begin{tabularx}{\textwidth}{>{\hsize.2\hsize}X>{\hsize.6\hsize}X>{\hsize.2\hsize}X}
    \begin{flushleft}
        2024.12.8\, No.138
    \end{flushleft}
    &
    \begin{center}
        \textit{“秉中持正、求新博闻。”}
    \end{center}
    &
    \begin{flushright}
        \textbf{南京市栖霞区}
    \end{flushright}
\end{tabularx}
\vspace{-3.5mm}
\hrule
\vspace{4mm}
% HEAD END
\centerline{\huge\textbf{活动预告}}
\begin{multicols}{2}
    \section{订阅方式和加入编辑部}  
编辑部招聘人才,用爱发电,工作轻松,详情可联系QQ:1329527951 客服小祥\\想订阅本消息或获取PDF版(便于查看超链接和往期),可加QQ群:\href{https://qm.qq.com/q/VXIW7fgsEe}{849644979}.
\section{Deadline Ongoing}
\setbox\colbbox\vbox{
\makeatletter\col@number\@ne
\begin{longtable}{|c|c|c|}
    \hline
    消息(未见ddl的,不刊) & 截止日期 & 刊载日期\\
    \hline\hline
    紫藤学刊征稿 & 12.15 & 10.22\\
    安邦征稿 & 1.12 & 11.16\\
    猫鼠大战 & 12.15 & 11.22\\
    防艾征集 & 12.10 & 11.22\\
    心理中心征稿 & 12.10 & 11.23\\
    物院摄影征集 & 12.9 & 11.24\\
    创意物理实验竞赛 & 12.21 & 11.15\\
    仙林通宵自习室 & 1.12 & 11.26\\
    防艾同伴教育 & 12.15 & 11.29\\
    防艾文艺作品征集 & 12.10 & 11.29\\
    南京高校戏曲交流 & 12.15 & 12.2\\
    全国大学生家史大赛 & 1.31 & 12.2\\
    说风解雨活动 & 12.9 & 12.4\\
    25年南大会学团报名 & 12.11 & 12.4\\
    女性劳动策展工坊 & 12.16 & 12.4\\
    防艾剧本杀 & 12.15 & 12.5\\
    金融消费者大赛 & 12.31 & 12.5\\
    药丸周边征稿 & 12.15 & 12.6\\
    南大演说家决赛 & 12.10 & 12.6\\
    四六级准考证打印 & 12.14 & 12.6\\
    花旗杯报名 & 1.3 & 12.6\\
    挑杯校园双选会 & 12.15 & 12.7\\
    南新读书会 & 12.11 & 12.7\\
    朋辈数模分享 & 12.15 & 12.7\\
    羊山公园环保市集 & 12.15 & 12.7\\
    新生午餐会报名 & 12.9 & 12.8\\
    C Through代码大赛 & 12.15 & 12.8\\
    数学公共课答疑报名 & 12.11 & 12.8\\
    \hline
\end{longtable}
\unskip
\unpenalty
\unpenalty}\unvbox\colbbox
\end{multicols}
\hrule
\pagebreak
\begin{multicols}{2}

\section{讲座}
\begin{tabular}{|c|c|c|}
    \hline
    往期讲座 & 开展日期 & 刊载日期\\
    \hline\hline
    《专利查新与规避...》 & 12.19 & 10.3\\
    basics on ... & 12.11 & 12.2\\
    《中国近代大学的...》 & 12.9 & 12.6\\
    《在线时尚零售的...》 & 12.11 & 12.6\\
    《中国文化中的点心》& 12.11 & 12.8\\
    《Adaptive...》 & 12.9 & 12.8\\
    《时空大数据高效...》 & 12.9 & 12.8\\
    《比较逻辑与社会...》 & 12.11 & 12.8\\
    \hline
\end{tabular}

1.Adaptive decision-making in mixed-agent systems\\
主讲人:Yang Li\\
时间:2024年12月9日 10:00-12:00\\
地点:南雍楼西区125报告厅\\
2.时空大数据高效计算\\
主讲人:商烁
时间:2024年12月9日 10:00-12:00\\
地点:南雍楼西区125报告厅\\
3.比较逻辑与社会科学方法论——从四因说、三因说、二因说的比较谈起\\
主讲人:王勇(疆生)教授\\
主持人:顿新国教授\\
时间:2024年12月11日(周三)10:10-12:00\\
地点:南京大学鼓楼校区教学楼(郑钢楼)101\\
4.高研院新生午餐会:徐兴无教授:中国文化中的点心

时间:2024年12月11日(周三)12:20-13:20

地点:鼓楼校区逸夫馆9楼,高研院报告厅

报名详见\url{https://mp.weixin.qq.com/s/WfzQiRNUbXiQK8ValMGrQQ}

5.比较逻辑与社会科学方法论——从四因说、三因说、二因说的比较谈起
主讲人:王勇(疆生) 教授
主持人:顿新国 教授
时间:2024年12月11日(周三)10:10-12:00
地点:南京大学鼓楼校区教学楼(郑钢楼)101 (讲座编辑:高松灯)


\section{高研院“新生午餐会” 第48场}
题目:中国文化中的点心\\
谈话人:徐兴无 南京大学文学院教授 南京大学高研院院长\\
时间:2024年12月11日(周三)12:20-13:20\\
地点:鼓楼校区逸夫馆9楼高研院报告厅\\
抽签开始时间:12月9日中午12:30\\
抽签结束时间:12月10日中午12:30\\
具体链接:\url{https://mp.weixin.qq.com/s/WfzQiRNUbXiQK8ValMGrQQ}

\section{C Through代码大赛}
参与对象:主要面向南京大学软件学院以及技术科学试验班2024级本科生,也鼓励其他院系学习过C语言课程的大一同学参加。\\
活动时间:12月15日(星期日)14:30-17:30\\
活动地点:线下举行,具体地点以报名后QQ群通知为准。\\
比赛的排名会将信息学竞赛选手和非信息学竞赛选手分开排名。\\
报名方式、奖品设置、比赛规则等见原文\url{https://mp.weixin.qq.com/s/oDgkukZEvrnqIl03It3dcQ}

\section{“论道”五校联合博士生学术论坛通知}
聚焦凝聚态物理、粒子与核物理和天文与天体物理三个方向。与会者现场参会,交流形式包括口头报告及海报。口头报告时间12分钟 + 3分钟提问。\\
时间:2024年12月17-18日(周二-周三)\\
地点:上海市浦东新区李所路1号 李政道研究所S500报告厅\\
诚邀物理与天文学科博士生参加,特别是已有优秀研究成果或创新性前沿研究进展的同学。\\
议程安排、报名方式、费用承担等见原文\url{https://mp.weixin.qq.com/s/_3G81LzfQ-othM6pCjCS_g}

\section{物理学院第三届“锦鲤杯”王者荣耀比赛}
【活动对象】\\
物理学院全日制在读本科生、硕士/博士研究生。\\
【比赛时间】\\
线上淘汰赛:2024年12月14日\\
线下决赛:2024年12月15日\\
【报名时间】\\
即日起至2024年12月12日23:55报名截止。\\
【报名方式】\\
(1)组队报名:以5人一队为单位报名,确定队名和队长之后,由队长填写问卷进行报名,报名成功后请加入比赛QQ群。\\
(2)个人报名:选手也可填写问卷进行个人报名,报名成功后请加入比赛QQ群。可在比赛群内自行组队,若仍无法组队成功的选手将随机分配队友。\\
其他要求和奖品设置等见原文\url{https://mp.weixin.qq.com/s/tsefn6jjyek0S2IizhxWFQ}

\section{计算机学院2024年迎新晚会展台活动}
含汉诺塔和毛球飞镖活动。\\
仙林校区\\
12月11、12日 11:30-13:00\\
五食堂前展台\\
鼓楼校区\\
12月11、12日 11:30-13:00\\
南园喷泉广场展台\\



\section{第一期数学公共课答疑辅导}
微积分I(第一层次) 12月14日18:30-20:30 馆1-102 助教周乾\\
线性代数(第一层次) 12月14日18:30-20:30 馆1-103 助教蒋韬\\
微积分I(第二层次) 12月15日18:30-20:30 教108 助教苏子浩\\
简明微积分 12月15日18:30-20:30 教118 助教王家文\\
报名截止时间:12月11日(周三)24:00\\
助教详细介绍与报名方式见原文\url{https://mp.weixin.qq.com/s/JEzniS7DEA-cMwjqV_kopQ}

\section{第十九届“挑战杯”校园双选会}
活动时间\\
2024年12月15日(星期日)14:00-17:00\\
活动地点\\
南京大学仙林校区敬文学生活动中心多功能厅\\
活动对象\\
有意参与“挑战杯”竞赛但未组队成功的学生个人/团体,目前已有项目但仍缺少“关键角色”的团队等\\
本次活动内容包括“赛事小讲堂”“项目博览馆”“自由交流站”\\
详细内容等见原文\url{https://mp.weixin.qq.com/s/44c8Xrb_fvLegfMXwBnKxA}
\section{第二十届“花旗杯”金融创新应用大赛报名}
本届竞赛主题:结合当今世界广为关注的热点领域和金融行业的具体需求,以团队为单位,提出具有商业化前景的金融科技解决方案。选题方向见原文\\
大赛日程安排大致如下:线上注册报名\rightarrow 线上提交项目中期报告\rightarrow 线上提交项目最终报告\rightarrow 线上预选\rightarrow 线上研讨\rightarrow 线下全国总决赛及颁奖典礼,具体流程与日期见原文\\
注册报名时间从即日起至2025年1月3日(中午12点前)\\
原文链接:\url{https://mp.weixin.qq.com/s/bf-RMkjNEn8-UQqR0YlpPw}\\
(消息编辑:llly)

\section{周末剧场二轮开票\&招募}
九乡河黑匣子 发布\\
项目:《失乐园通关指南》\\
购票方式:淘宝网店购买\\
时间:12月20-22日每日19:30,时长约90分钟\\
地点:黑匣子剧场\\
剧务招募:灯光执行 1名;音效执行 1名;场务 2名\\
招募详细要求、剧目与购票详情,请查看\url{https://mp.weixin.qq.com/s/NyRcMeOtQNG-PO5Krfr5nQ}\\
(编辑:Feanaro)
\end{multicols} 

\end{document}