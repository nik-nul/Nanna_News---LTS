% HEAD BEGIN
\documentclass[letterpaper, 12pt]{article}
\newsavebox\colbbox
\usepackage{graphicx}
\usepackage{multicol}
\usepackage{anysize}
\usepackage{fontspec}
\usepackage[fontset=none]{ctex}
\usepackage{tabularx}
\usepackage{longtable}
\PassOptionsToPackage{hyphens}{url}
\usepackage[breaklinks=true, colorlinks=true]{hyperref}
\expandafter\def\expandafter\UrlBreaks\expandafter{\UrlBreaks\do\a\do\b\do\c\do\d\do\e\do\f\do\g\do\h\do\i\do\j\do\k\do\l\do\m\do\n\do\o\do\p\do\q\do\r\do\s\do\t\do\u\do\v\do\w\do\x\do\y\do\z\do\A\do\B\do\C\do\D\do\E\do\F\do\G\do\H\do\I\do\J\do\K\do\L\do\M\do\N\do\O\do\P\do\Q\do\R\do\S\do\T\do\U\do\V\do\W\do\X\do\Y\do\Z}
% \let\oldurl\url
% \renewcommand{\url}[1]{\begin{sloppypar}\oldurl{#1}\end{sloppypar}}
\setlength\columnsep{30pt}
\marginsize{30pt}{30pt}{10pt}{20pt}
\setmainfont{TeX Gyre Bonum}
\setCJKmainfont[BoldFont=Noto Serif CJK SC Bold, ItalicFont=FandolKai]{Noto Sans CJK SC}
\setlength{\parindent}{0cm}
% \setCJKmonofont{Noto Sans CJK SC}
\begin{document}
\begin{center}
    \Huge\textbf{南哪大专醒前消息}
\end{center}
\vspace{4mm}
\hrule
\renewcommand\tabularxcolumn[1]{m{#1}}
\begin{tabularx}{\textwidth}{>{\hsize.2\hsize}X>{\hsize.6\hsize}X>{\hsize.2\hsize}X}
    \begin{flushleft}
        2025.2.19\, No.169
    \end{flushleft}
    &
    \begin{center}
        \textit{“秉中持正、求新博闻。”}
    \end{center}
    &
    \begin{flushright}
        \textbf{南京市栖霞区}
    \end{flushright}
\end{tabularx}
\vspace{-3.5mm}
\hrule
\vspace{4mm}
% HEAD END
\centerline{\huge\textbf{活动预告}}
\begin{multicols}{2}
    \section{订阅方式和加入编辑部}  
编辑部招聘人才,用爱发电,工作轻松,详情可联系QQ:1329527951 客服小祥\\想订阅本消息或获取PDF版(便于查看超链接和往期),可加QQ群:\href{https://qm.qq.com/q/VXIW7fgsEe}{849644979}.
\section{Deadline Ongoing}
\setbox\colbbox\vbox{
\makeatletter\col@number\@ne
\begin{longtable}{|c|c|c|}
    \hline
    消息(未见ddl的,不刊) & 截止日期 & 刊载日期\\
    \hline\hline
    南大博物馆展览 & 6.16 & 12.17\\
    ASC25报名 & 2.21 & 1.6\\
    天文台开放日 & / & 1.6\\
    原创剧本联合孵化报名 & 3.20 & 1.10\\
    阅读分享活动征稿 & 3.7 & 1.10\\
    njumun代表报名 & 3.2 & 1.16\\
    毓秀文创 & 2.20 & 2.6\\
    生科论文沙龙 & 2.22 & 2.6\\
    地海训练营 & 2.21 & 2.14\\
    返校注册 & 2.23 & 2.14\\
    课程补退选 & 3.2 & 2.19\\
    DIY课程报名 & 2.22 & 2.17\\
    南大育教新媒体招新 & 2.27 & 2.19\\
    本科生劳育实践 & 7.20 & 2.19\\
    医保零星报销 & 3.31 & 2.19\\
    信息通信产业链专项赛 & 2.23 & 2.19\\
    港理工暑课报名 & 2.21 & 2.19\\
    金法槌被模拟法庭大赛 & 2.23 & 2.19\\
    第二届大学生阅读分享活动 & 2.26 & 2.19\\
    \hline
\end{longtable}
\unskip
\unpenalty
\unpenalty}\unvbox\colbbox
\end{multicols}
\hrule
\pagebreak
\begin{multicols}{2}

\section{讲座}
\begin{tabularx}{0.5\textwidth}{|X|X|X|}
    \hline
    讲座 & 开展时间 & 刊载时间\\
    \hline\hline
Unconventional magnetism & 2.20 & 2.17\\\hline
人工微结构声学材料 & 2.20 & 2.17\\\hline
人机协同背景下高等外语教育的守正创新 & 2.27 & 2.17\\\hline
大陆的起源 & 3.4 & 2.17\\\hline
单杏花先进事迹宣讲会 & 2.20 & 2.17\\\hline
\end{tabularx}

\section{“国家安全教育”开课方案}
2024级未获得“国家安全教育”(课程号:00000160)学分的学生注意:2月23日前完成重修申请和缴费等工作,3月17日至3月19日,请学生及时登陆\url{http://nju.fanya.chaoxing.com}(或在移动端登录“学习通”)完成课程的相关学习注册事宜,查看课程信息和“课程须知”并进行确认,做好学习准备。同时登录教务平台,查询任课教师、班级信息。\\
课程详情:\url{https://jw.nju.edu.cn/5a/99/c26263a744089/page.psp}\\
\section{本科毕业生专业准出申请}
专业准出申请和信息核对于2025年2月19日-28日在线进行,请所有2025届毕业生(延长学习的学生)务必在规定时间内按照《专业准出申请操作指南》(见附件)填报专业准出信息,标准学制期满时能够达到申请辅修专业的辅修学士学位或者结业条件者,才需填写辅修准出专业。\\
详情及申请指南:\url{https://jw.nju.edu.cn/5a/9b/c26263a744091/page.htm}\\
\section{“南大育教”新媒体中心招新}
“南大育教”新媒体中心创建于2016年7月,主要负责南京大学党委学生工作部官方微信公众号与视频号“南大育教”的运营\\
面向全校本研生招新,详情:\url{https://mp.weixin.qq.com/s/0ox_Y7SWbLhE2rdC802-UQ}\\
截止时间:2025年2月27日23时59分
\section{本科生劳育实践}
编辑部提醒,原则上每位新生在基础实践模块至少完成10个小时的劳动时长;本学期原则上要求2022-2024级部分未达标的本科生在2025年7月20日前累计完成基础劳动实践10个小时的劳动时长,2021级毕业班本科生务必在本学期末实现基础劳动实践、学科劳动实践模块和总时长达标并获得“大学生劳动教育实践学分”,否则将影响如期毕业。\\
\section{医疗费用零星报销}
现根据医保有关部门通知,校医院将统一收取2024年度医保未报销的医疗费用发票,代交至南京市医保中心进行零星报销处理,请于2025年3月31日前提交相关材料\\
详情:\url{https://hospital.nju.edu.cn//ggtz/20250219/i308709.html}\\
\section{关于举办2025年江苏“创青春”新一代信息通信产业链专项赛的通知}
链接:\url{https://mp.weixin.qq.com/s/5V0yUK3yehxpt9jhnbAkTg}\\
领域:人工智能、物联网、新型显示、高性能集成电路、5G通信、云计算。\\
团队:不超过五人。\\
报名截止时间:2月23日。
\section{项目报名|本科|2025香港理工大学暑期课程报名}
2025年2月21日晚24:00截止\\
链接:\url{https://mp.weixin.qq.com/s/tk2z_skGJyYWLMy7pUfmSg}\\

\section{【21-24级劳育慕课重修同学关注】2025年春季学期《大学生劳动教育》慕课选修说明}
重修申请于2月23日截止\\
链接:\url{https://jw.nju.edu.cn/58/95/c26263a743573/page.htm}\\

\section{第十九届“金法槌杯”模拟法庭大赛报名选拔通知}
(一)案例设置:本次比赛均采用民事案例\\
(二)大赛议程:\\
1. 比赛时间:2025年3-4月\\
2. 初赛:案例分析\\
3. 决赛:实战型模拟法庭\\
选拔对象:南京大学法学院四年级、三年级、二年级本科生,将选拔4名同学组队代表我院参赛。\\
要求:报名同学应具有扎实的法律基础知识和较强的团队合作意识,既需要具备书写书状的能力,也需要具备上场比赛的素质。\\
报名方式:填写报名表(点击文末“阅读原文”获取报名表)发至邮箱 njulaw2024@163.com,邮件主题与附件名称均命名为【金法槌杯报名】+姓名”(例如【金法槌杯报名】+张三)。\\
截止日期:2025年2月23日12:00 前\\
详情见\url{https://mp.weixin.qq.com/s/rZZ5yibrz2PIvLZjBoxcyQ}\\

\section{中国出版集团第二届大学生阅读分享活动}
南京大学初选\\
征集时间:即日起至2月26日\\
征集范围:南京大学在校本、硕、博学生。\\
投稿要求:\\
1.以“中国好书”、中国出版集团好书和其他各类优秀出版物为主要阅读对象,撰写读后感、阅读小品或阅读故事等。\\
2. 作品格式:限文字作品。文章格式统一为WORD版本,文章标题使用华文中宋2号字体,正文使用3号仿宋字体,行间距为26磅。体裁包括读后感、阅读故事、读书小品等,字数为2000-4000字。每篇文章的署名作者人数不可超过2人。\\
3.版权要求:投稿作品均须为原创,不允许使用AI素材。\\
4.内容要求:各类投稿作品不得含有色情、暴力等内容,均须遵守国家有关法律、行政法规的规定,符合民族文化传统、公共道德价值、行业规范等要求。\\
投稿方式(校内初选):\\
请于2025年2月26日24:00前提交作品至南京大学学生悦读书社邮箱:irs@nju.edu.cn。
提交时需同时发送WORD和PDF版本,请将所有文件打包提交,文件名和邮件主题统一为“学号+姓名+第二届大学生阅读分享活动”\\
奖项设置及注意事项见推文:\url{https://mp.weixin.qq.com/s/YFXlBEX2zuy8upvHxiPJ9A}

\end{multicols} 
\end{document}