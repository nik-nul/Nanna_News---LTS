% HEAD BEGIN
\documentclass[letterpaper, 12pt]{article}
\newsavebox\colbbox
\usepackage{graphicx}
\usepackage{multicol}
\usepackage{anysize}
\usepackage{fontspec}
\usepackage[fontset=none]{ctex}
\usepackage{tabularx}
\usepackage{longtable}
\PassOptionsToPackage{hyphens}{url}
\usepackage[breaklinks=true, colorlinks=true]{hyperref}
\expandafter\def\expandafter\UrlBreaks\expandafter{\UrlBreaks\do\a\do\b\do\c\do\d\do\e\do\f\do\g\do\h\do\i\do\j\do\k\do\l\do\m\do\n\do\o\do\p\do\q\do\r\do\s\do\t\do\u\do\v\do\w\do\x\do\y\do\z\do\A\do\B\do\C\do\D\do\E\do\F\do\G\do\H\do\I\do\J\do\K\do\L\do\M\do\N\do\O\do\P\do\Q\do\R\do\S\do\T\do\U\do\V\do\W\do\X\do\Y\do\Z}
% \let\oldurl\url
% \renewcommand{\url}[1]{\begin{sloppypar}\oldurl{#1}\end{sloppypar}}
\setlength\columnsep{30pt}
\marginsize{30pt}{30pt}{10pt}{20pt}
\setmainfont{TeX Gyre Bonum}
\setCJKmainfont[BoldFont=Noto Serif CJK SC Bold, ItalicFont=FandolKai]{Noto Sans CJK SC}
\setlength{\parindent}{0cm}
% \setCJKmonofont{Noto Sans CJK SC}
\begin{document}
\begin{center}
    \Huge\textbf{南哪大专醒前消息}
\end{center}
\vspace{4mm}
\hrule
\renewcommand\tabularxcolumn[1]{m{#1}}
\begin{tabularx}{\textwidth}{>{\hsize.2\hsize}X>{\hsize.6\hsize}X>{\hsize.2\hsize}X}
    \begin{flushleft}
        2024.10.20\, No.93
    \end{flushleft}
    &
    \begin{center}
        \textit{“克明峻德。”}
    \end{center}
    &
    \begin{flushright}
        \textbf{南京市栖霞区}
    \end{flushright}
\end{tabularx}
\vspace{-3.5mm}
\hrule
\vspace{4mm}
% HEAD END
\centerline{\huge\textbf{活动预告}}
\begin{multicols}{2}
    \section{订阅方式和加入编辑部}  
编辑部招聘人才,用爱发电,工作轻松,详情可联系QQ:1329527951 客服小祥\\想订阅本消息或获取PDF版(便于查看超链接和往期),可加QQ群:\href{https://qm.qq.com/q/VXIW7fgsEe}{849644979}.
\section{Deadline Ongoing}
\setbox\colbbox\vbox{
\makeatletter\col@number\@ne
\begin{longtable}{|c|c|c|}
    \hline
    消息(未见ddl的,不刊) & 截止日期 & 刊载日期\\
    \hline\hline

    新生午餐会报名 & 10.21 & 10.19\\
    行知院服设计赛 & 10.21 & 10.15\\
    林泉钢琴社线上分享 & 10.21 & 10.13\\
    社院国际报告会 & 10.21 & 10.18\\
    有训运动会报名 & 10.21 & 10.17\\
     ‘满天星’调研大赛 & 10.22 & 10.16\\
    有训院服设计赛 & 10.22 & 10.16\\
    南新读书会 & 10.23 & 10.20\\
    歌魅放映会 & 10.23 & 10.20\\
    急救培训活动报名 & 10.24 & 10.17\\
    计院迎新晚会征集节目 & 10.25 & 10.12\\
    马院主题宣讲报名 & 10.25 & 10.5\\
    织围巾志愿者招募 & 10.26 & 10.20\\
    心协DIY活动 & 10.26 & 10.20\\
    心协流光影院 & 10.26 & 10.17\\
    校园今日说法大赛 & 10.26 & 10.17\\
    遵义精神宣讲团遴选 & 10.27 & 10.10\\
    青鸟剧场新戏招募 & 10.27 & 10.14\\
    体测 & 10.27 & 10.16\\
    鼓楼音乐跑 & 10.27 & 10.20\\
    普通话考试报名 & 10.28 & 10.14\\
    仙林校史馆招募讲解员 & 10.30 & 9.12\\
    本科生暑期课程评教 & 10.31 & 9.19\\
    黑匣招募 & 11.1 & 10.19\\
    学位英语考试报名 & 11.3 & 10.17\\
    后革命鲁迅研究征文 & 11.10 & 10.8\\
    大创训练计划申报 & 11.18 & 9.24\\
    EBSCO数据库检索大赛 & 11.20 & 10.3\\
    文院征稿 & 11.20 & 10.20\\
    乐跑 & 12.8 & 10.12\\
    
    \hline
\end{longtable}
\unskip
\unpenalty
\unpenalty}\unvbox\colbbox
\end{multicols}
\hrule
\pagebreak
\begin{multicols}{2}

\section{讲座}
\begin{tabular}{|c|c|c|}
    \hline
    往期讲座 & 开展日期 & 刊载日期\\
    \hline\hline
    《聚合物的研发与...》 & 10.24 & 10.3\\
    《电池及电化学能...》 & 11.24 & 10.3\\
    《专利查新与规避...》 & 12.19 & 10.3\\
    《第二现代性与儒...》& 10.21 &10.16\\
    《中国法律形象西...》 & 10.23 & 10.16\\
    《与<自然>编辑对...》 & 10.30 & 10.16\\
    《博士论文的选题...》 & 10.22 & 10.17\\
    《博士论文的选题...》 & 10.21 & 10.17\\
    《从文学话语到批...》 & 10.21 & 10.18\\
    《语言能力与前近...》 & 10.25 & 10.18\\
    《MATLAB与Python...》 & 10.21 & 10.18\\
    《物理信息神经网...》 & 10.23 & 10.18\\
    《国家图书馆的古...》 & 10.24 & 10.18\\
    《HKMW与当代德国...》 & 10.23 & 10.20\\
    图书馆系列讲座 & 12.3 & 10.20\\
    《中国电影有声转..》 & 10.22 & 10.20\\
    《则天文字在日本...》 & 10.23 & 10.20\\
    \hline
\end{tabular}

1.南京大学马克思主义学院国际学者讲座\\
题目:《马克思主义历史考证大辞典》(HKWM)与当代德国马克思主义\\
主讲人:Hauke Neddermann,柏林自由大学/柏林批判理论研究所研究员,《马克思主义历史考证大辞典》(HKWM)编辑\\
主持人:李乾坤\\
时间:2024.10.23(周三)18:30-20:30\\
地点:南京大学圣达楼211\\

2、南京大学第十九届读书节-图书馆知识讲座\\
注:由于该系列讲座较多,且已经由专门人员整理为图片,麻烦请感兴趣的同学直接点进链接查看\url{https://mp.weixin.qq.com/s/I4eBSUpyNRNHZPIXeM-9RA}\\

3.DIY研读研究课程“奇怪的电影”拓展讲座\\
主题:留声机 电话 无线电:中国电影有声转型的媒介技术与时空现代性\\
讲座人:罗婷,浙江大学传媒与国际文化学院“百人计划”研究员\\
主持人:杨鹏鑫,南京大学文学院副教授\\
时间:2024年10月22日(周二)16:00-18:00\\
地点:南京大学仙林校区文学院活水轩\\

4.武周女皇的东亚旅行:则天文字在日本及朝鲜半岛的流传
主讲人:梁晓虹

时  间:2024年10月23日星期三下午15:00-17:00

地  点:南京大学文学院221会议室

主  办:南京大学文学院 南京大学中国文学与东亚文明协同创新中心 教育部中华优秀传统文化专项课题(A)重大项目(尼山世界儒学中心):“隋唐历史文化认同与中华民族的发展研究”(23JDTCZ009)

\section{南新读书会|下周预告}
本周的南大新传读书会将于10月23日19:00在新闻传播学院311室举行,2023硕沙璨将分享Jacob Gaboury《Image Objects》,2024博王潇然将分享塞德里克·迪朗《技术封建主义》。\\
\section{心理协会|奶油胶DIY活动}
本次活动中,心协为大家准备了小镜子,同学们可以发挥自己的奇思妙想,围绕“快乐多巴胺,‘甜甜’奶油胶”这一主题,利用奶油胶装点小镜子。\\
活动时间:10月26日 下午15:00-16:00\\
活动地点:仙林校区3栋南青格庐\\
报名链接及详情,见\url{https://mp.weixin.qq.com/s/QtZUbykuyfYCyPWUujUq-A}\\
\section{“庆祝南京大学文学院创建110周年”诗歌、楹联在线征集}
征集要求
(一)大赛分为新诗、传统诗词与楹联三个组别。\\
(二)以“庆祝南京大学文学院创建110周年”为主题,选择与南大文学院历史、名家、风景、日常生活等相关的题材,创作诗歌或楹联。\\
(三)传统诗词的诗体、词牌不限,韵部不限。\\
征稿阶段:10月20日—11月20日\\
本次征集活动在“知鸿蒙”知识社交平台“南京大学文学院110周年院庆”专栏空间开展。参与者可自行使用手机或电脑,上传投稿内容。此外还征集回忆录、祝福语、学术展、照片、院史资料等。
\section{手织围巾为乡村学生送温暖活动志愿者招募}
南悦团队现通知志愿者招募事宜。

内容:

(一)志愿者招募阶段(即日起至2024年10月25日):志愿者可填写文末报名表进行报名。

(二)手织围巾阶段(2024.10.26-11.15):在这一时期每位志愿者可制作1-2条围巾,长度1-1.6米为宜。(材料需由志愿者自备)

(三)捐赠围巾阶段(2024.11.16-11.30):统一收集志愿者制作的围巾后,由唯爱公益进行围巾捐赠(围巾全部用优质的包装袋包装,唯爱公益提供,建议大学生手写一张明信片传递暖冬爱心祝福)

详见:\url{https://mp.weixin.qq.com/s/uRXJU-vXOdvdoxasulwqIQ}
(四)总结交流阶段(2024.12.15前)
\section{歌魅放映会|《律政俏佳人》}
时间:10月23日19:00\\
地点:仙一116\\
详见:\url{https://mp.weixin.qq.com/s/p_AUJmfx9KxUGHv-0fFlKQ}
\section{鼓楼音乐跑}
1.活动时间:2024年10月21日-10月27日\\
周一到周五:21:30-22:10\\
周末:19:30-21:00(如遇雨水天气或其他特殊情况将暂停一日)\\
2.活动地点:南京大学鼓楼校区苏浙体育场\\
3.惊喜:1)小蓝鲸们可在活动服务点处领取定制荧光手环(每日限量哦)为夜跑增一抹明亮\\
2)从周一到周日,我们为每天准备了不同的小礼品
周一 行知文创(帆布袋、文件夹)\\
周二 荧光发箍\\
周三 荧光发夹\\
周四 小星星手环\\
周五 荧光眼镜\\
周六周日 荧光发箍、发夹、眼镜、小星星手环、行知文创(文件夹)\\
还有定制的明信片,每天限量40张!\\
3)本周音乐跑活动开始后,在苏浙体育场入口处的展台旁,我们准备了荧光音乐跑的海报,每天跑完步后在海报上按荧光手印,拍照并发布朋友圈的小蓝鲸当晚即可获得小礼品一份!
该条朋友圈在一周内集赞满30个还可以10月26日-10月27日(周六周日)找我们领取定制钥匙扣一个!(每日限量40个,先到先得!)\\
4.打卡规则:凭借南大APP或其他运动APP的跑步记录可至活动服务点领取精美奖品数量有限,发完即止,先到先得~\\
5.荧光文艺表演\\
活动时间:2024年10月26日-27日(周六日)19:30-20:00活动地点:苏浙体育场草坪中央\\
歌单和礼品图案见\url{https://mp.weixin.qq.com/s/TF7QucLA6Pk9YOcueMwhLQ}

\end{multicols} 

\end{document}