% HEAD BEGIN
\documentclass[letterpaper, 12pt]{article}
\newsavebox\colbbox
\usepackage{graphicx}
\usepackage{multicol}
\usepackage{anysize}
\usepackage{fontspec}
\usepackage[fontset=none]{ctex}
\usepackage{tabularx}
\usepackage{longtable}
\PassOptionsToPackage{hyphens}{url}
\usepackage[breaklinks=true, colorlinks=true]{hyperref}
\expandafter\def\expandafter\UrlBreaks\expandafter{\UrlBreaks\do\a\do\b\do\c\do\d\do\e\do\f\do\g\do\h\do\i\do\j\do\k\do\l\do\m\do\n\do\o\do\p\do\q\do\r\do\s\do\t\do\u\do\v\do\w\do\x\do\y\do\z\do\A\do\B\do\C\do\D\do\E\do\F\do\G\do\H\do\I\do\J\do\K\do\L\do\M\do\N\do\O\do\P\do\Q\do\R\do\S\do\T\do\U\do\V\do\W\do\X\do\Y\do\Z}
% \let\oldurl\url
% \renewcommand{\url}[1]{\begin{sloppypar}\oldurl{#1}\end{sloppypar}}
\setlength\columnsep{30pt}
\marginsize{30pt}{30pt}{10pt}{20pt}
\setmainfont{TeX Gyre Bonum}
\setCJKmainfont[BoldFont=Noto Serif CJK SC Bold, ItalicFont=FandolKai]{Noto Sans CJK SC}
\setlength{\parindent}{0cm}
% \setCJKmonofont{Noto Sans CJK SC}
\begin{document}
\begin{center}
    \Huge\textbf{南哪大专醒前消息}
\end{center}
\vspace{4mm}
\hrule
\renewcommand\tabularxcolumn[1]{m{#1}}
\begin{tabularx}{\textwidth}{>{\hsize.2\hsize}X>{\hsize.6\hsize}X>{\hsize.2\hsize}X}
    \begin{flushleft}
        2024.11.23\, No.125
    \end{flushleft}
    &
    \begin{center}
        \textit{“Vis ex acta Deus ex machina.”\\“新闻赋能机械降神”}
    \end{center}
    &
    \begin{flushright}
        \textbf{南京市栖霞区}
    \end{flushright}
\end{tabularx}
\vspace{-3.5mm}
\hrule
\vspace{4mm}
% HEAD END
\centerline{\huge\textbf{活动预告}}
\begin{multicols}{2}
    \section{订阅方式和加入编辑部}  
编辑部招聘人才,用爱发电,工作轻松,详情可联系QQ:1329527951 客服小祥\\想订阅本消息或获取PDF版(便于查看超链接和往期),可加QQ群:\href{https://qm.qq.com/q/VXIW7fgsEe}{849644979}.
\section{Deadline Ongoing}
\setbox\colbbox\vbox{
\makeatletter\col@number\@ne
\begin{longtable}{|c|c|c|}
    \hline
    消息(未见ddl的,不刊) & 截止日期 & 刊载日期\\
    \hline\hline
    紫藤学刊征稿 & 12.15 & 10.22\\
    乐跑 & 12.6 & 10.12\\
    秉文心理短视频 & 11.25 & 11.3\\
    DIY课程学术论坛征稿 & 11.30 & 11.13\\
    国风歌曲演唱赛 & 12.1 & 11.13\\
    牡丹亭庆演 & 12.1 & 11.13\\
    普通话测试网络报名 & 11.30 & 11.16\\
    安邦征稿 & 1.12 & 11.16\\
    新传院迎新晚会征集 & 11.25 & 11.17\\
    秉文宿舍风采 & 12.1 & 11.17\\
    商院生涯论坛 & 11.24 & 11.18\\
    黑匣招募 & 11.25 & 11.18\\
    重唱诗社评诗会 & 11.24 & 11.19\\
    古琴社露天音乐会 & 11.24 & 11.19\\
    心协有奖征稿 & 11.25 & 11.19\\
    日俱影映 & 11.24 & 11.19\\
    法学院征诗活动 & 12.2 & 11.20\\
    计院乒赛 & 11.26 & 11.20\\
    歌魅影映 & 11.24 & 11.20\\
    平安留学交流会 & 12.3 & 11.20\\
    社计院联议 & 11.24 & 11.20\\
    音乐史分享会 & 11.24 & 11.20\\
    南商大树来信 & 11.24 & 11.21\\
    排球新生杯二次报名 & 11.24 & 11.21\\
    七院相亲活动 & 11.30 & 11.21\\
    青山线上支教报名 & 11.25 & 11.21\\
    计科工作坊 & 11.26 & 11.21\\
    黑匣工作坊 & 11.26 & 11.21\\
    朋导分享会二 & 11.24 & 11.22\\
    免费上网讲座 & 11.27 & 11.22\\
    猫鼠大战 & 12.15 & 11.22\\
    日本征文大赛 & 12.6 & 11.22\\
    萨勒姆的女巫 & 11.30 & 11.22\\
    案例分析大赛 & 11.25 & 11.22\\
    防艾征集 & 12.10 & 11.22\\
    爱心义卖 & 11.30 & 11.22\\
    银杏叶制作之旅 & 11.29 & 11.23\\
    法学朋导招募 & 11.29 & 11.23\\
    青山同行支教招募 & 11.25 & 11.23\\
    南选问答集赞 & 11.27 & 11.23\\
    建规迎新晚会 & 11.24 & 11.23\\
    心协信件盲盒 & 12.7 & 11.23\\
    港澳台晚会招募 & 11.27 & 11.23\\
    配音大赛招募 & 12.7 & 11.23\\
    物院研会系列活动 & 12.21 & 11.23\\
    心理中心征稿 & 12.10 & 11.23\\
    \hline
\end{longtable}
\unskip
\unpenalty
\unpenalty}\unvbox\colbbox
\end{multicols}
\hrule
\pagebreak
\begin{multicols}{2}

\section{讲座}
\begin{tabular}{|c|c|c|}
    \hline
    往期讲座 & 开展日期 & 刊载日期\\
    \hline\hline
    《电池及电化学能...》 & 11.24 & 10.3\\
    《专利查新与规避...》 & 12.19 & 10.3\\
    图书馆系列讲座 & 12.3 & 10.20\\
    《学术写作入门...》& 11.21 & 11.18\\
    《日本侵华与中国...》 & 11.24 & 11.20\\
    《文献资源检索...》 & 11.24 & 11.20\\
    《Adobe AI 讲座》 & 11.27 & 11.22\\
    《马克思的经济全球化》 & 11.27 & 11.23\\
    《法学研究类型和方法》 & 11.29 & 11.23\\
    《前沿情报捕捉...》 & 11.29 & 11.23\\
    《AI在设计中的参与...》 & 12.2 & 11.23\\
    \hline
\end{tabular}

\subsection{重思马克思的经济全球化思想}
主讲人:邹诗鹏 复旦大学哲学学院教授、教育部“长江学者”特聘教授\\
时间:11月27日 周三 15:00\\
地点:南京大学仙林校区圣达楼100B\\

\subsection{第二届法学院本科生学术写作竞赛开幕式暨“法学研究的类型和方法”专题讲座}
主讲人:陈坤 南京大学法学院教授、副院长\\
时间:2024年11月29日\\
地点:南京大学仙林校区敬文学生活动中心南青报告厅\\
详情与报名二维码请查看:\url{https://mp.weixin.qq.com/s/GZU_yH-5hxt4BClpn50GoA}\\

\subsection{AI达人分享沙龙}
第一场:\\
AI探索之旅:前沿情报捕捉与学术检索导航\\
分享人:魏同学\\
如何通过Prompt与AI互动?\\
分享人:吕同学\\
时间:11月27日(周三)下午16:45-17:45\\
地点:仙林校区图书馆校友之家小报告厅\\
第二场:\\
AI在设计中的参与及交互方式探究\\
分享人:夏同学\\
用大语言模型开发原创可专利软件:从测试到落地\\
分享人:何同学\\
时间:12月2日(周一)晚上18:30-19:30\\
地点:仙林校区图书馆校友之家小报告厅\\

\section{乐跑}
从明日(11月24日)算起,还有13次乐跑机会。

\section{银杏叶书签创作之旅、志愿者招募}
活动对象:新生学院全体同学\\ 
活动时间:11月29日(周五)中午12:00-14:00\\
活动地点:鼓楼南园喷泉广场\\
活动形式:同学们寻找自己心仪的银杏叶,现场制作并塑封银杏叶,鼓励同学们以文字、绘画等形式参与创作\\
%活动要求:1.可以在校园内自己寻找心仪的银杏树叶,轻轻擦拭叶片后,将其夹在书内自然风干,并使叶片保持干燥和平整。活动现场也会提供一定数量的树叶供同学们使用\\
%2.活动将为同学们提供水笔、颜料、印章、背景卡片等所需材料,并为大家塑封。DIY完成的银杏叶卡片可由同学们自行留存\\
%3.书写、绘画主题积极向上,或书写美好的未来愿景、积极奋进的理想和志向,或展现中华优秀传统文化等\\
%4.完成卡片制作后需要同学们按照要求上传个人作品照片,被精选的作品照片后续将有机会在公众号上推出\\
活动志愿者招募\\
诚邀有书法、绘画特长的2024级新生担任本次活动的志愿者,帮有需要的同学们在银杏叶上写字、绘画\\
志愿时间:11月29日(周五)12:00-14:00\\
志愿地点:鼓楼南园喷泉广场\\
活动结束后,每位志愿者会录入前期准备、活动现场相应的志愿服务时长。共招募10名志愿者。\\
志愿者报名链接见\url{https://mp.weixin.qq.com/s/ZHKTVH1om-EsRW8DCcBcow}
\section{南京大学法学院本科生朋辈导师招募计划}
为帮助法学院本科生更好地学习与生活,现决定面向法学院高年级本科生发布朋辈导师招募计划,以寝室为单位,每位朋辈导师对口2-4个低年级寝室进行问题解惑和学业引导。
报名与材料提交方式:詳見\url{https://mp.weixin.qq.com/s/1DKUozPAgcesuunPs7T8Kg}
\section{“青山同行”线上支教}
南大青协将与贵州平坝一中和宁夏泾源中学两所学校合作,对两所高中的学生开展有计划、有层次的周末一对一线上教学。\\
活动时间:11月下旬至中学学生寒假结束\\
活动形式:本期活动以线上云课堂授课及课后辅导为主。\\
收获:丰厚的志愿时长、志愿服务证明等\\
详情及报名方式:\url{https://mp.weixin.qq.com/s/SknpWQr4EkMMB6CZDaMVFA}\\
\section{“南选问答”}
“南选问答”活动以打破信息壁垒、体悟选调情怀为宗旨。本季“南选问答”将长期征集南大学子感兴趣且存疑的理论主题,并于后续通过微课堂推文或线上讲座的形式在公众号定期发布相关内容。
话题征集方式:\url{https://mp.weixin.qq.com/s/0KWR6CtT1KO8RCaBeTX8pg}\\
\section{建规学院迎新晚会}
时间:11.24 18:00\\
地点:鼓楼校区 科技馆报告厅1楼\\
\section{“镜界”征稿}
学生会摄影专栏“镜界”面向全体学生征稿\\
征集时间:11.20-11.27\\
精选作品将在学生会公众号发布\\
征集主题及投稿 详见:\url{https://mp.weixin.qq.com/s/1qYddJfsvQF5HC9HB4XGWw}
\section{AI达人经验分享活动}
第一场:\\
AI 探索之旅:前沿情报捕捉与学术检索导航\\
如何通过Prompt与AI互动?\\
时间:11月27日16:45-17:45\\
地点:仙林校区图书馆校友之家小报告厅\\
第二场:\\
AI在设计中的参与及交互方式探究\\
用大语言模型开发原创可专利软件:从测试到落地\\
时间:12月2日18:30-19:30\\
地点:仙林校区图书馆校友之家小报告厅\\
\section{致勇敢的我们 | 十一月征稿:和亲密的人吵架之后……}
征稿要求\\
讲述自己与亲密的人吵架之后的真实故事,可以分享期间的经历、思考、处理方式或咨询体验、受到的帮助和支持、印象深刻的一两件事等……\\
你的故事或许会成为他人的一束光。\\
征稿日期截至2024年12月10日。\\
经选用将提供稿费:根据长度每篇10-20元。\\
来稿要求\\
1.原创,真实。\\
2.100—500字\\
3.文章下方附上名字(不必真实),一个自己的头像(不必真实),简短的自我介绍。\\
可在\url{ https://www.wjx.cn/vm/rJVdnfx.aspx}投稿\\
详见\url{https://mp.weixin.qq.com/s/W54pXtW3lbcJ3o_b_jT68w}
\section{心协信件盲盒}
第二期信件盲盒聚焦大一新生的困惑与感想,推出信件盲盒之鼓楼特辑。\\
活动流程:1.投递与领取:本次活动仍采取盲盒交换的模式,报名参加的同学请前往鼓楼南青格庐领取信封并选择感兴趣的主题写作。2.领取与回信:两周之后我们会将信件随机分发给报名同学,鼓励大家积极回信(有心协特制小礼物)我们将在下一期活动收取信封时一同收取回信,届时备注“回信”分发小礼物。\\
参加方式:加入微信群,填写报名表单并登记笔名。最迟在12月7日前完成投递。\\
微信群见原文:\url{https://mp.weixin.qq.com/s/3_T7LcDug5ufPGHFDVS0OQ}
\section{篮球明日赛程}
女篮院系杯小组赛\\
地海vs政管\\
19:00-20:00\\
地点:一组团篮球场
\section{NEC英语角鼓楼}
本周鼓楼校区口语角将于本周日(11月24日)晚7:00-8:30进行,地点位于新教-201。\\
主题:Mental Calculation, Groups, Travel, Talent and Endeavour
报名与详情请查看\url{https://mp.weixin.qq.com/s/V7btd3jeFVthozjgx4gs_w}\\
\section{台港澳青年迎新晚会节目/主持人征集}
节目报名要求:\\
1.舞蹈类(街舞、现代舞、传统民族舞等)\\
2.声乐类(独唱、合唱、流行歌曲等)\\
3.乐器类(传统与民族乐器、西洋乐器等)\\
4.曲艺类(戏曲、大鼓、黄梅戏等)\\
5.语言类(相声、小品、舞台剧等)\\
6.创意类(魔术、走秀等其他创意节目)\\
7.其他题材积极向上,内容新颖有趣的节目\\
8.欢迎团体节目\\
主持人报名要求\\
1. 普通话标准流利,口齿清楚,有较强的语言表达能力\\
2. 形象佳,举止行为大方得体,气质亲和优雅\\
3. 具备一定的临场应变能力\\
报名详情请查看\url{https://mp.weixin.qq.com/s/PDRQAybB7e0XP07cWTkf7A}\\
\section{配音大赛海选报名}
日期:2024年12月7日(星期六)\\
时间:14:00-18:00\\
地点:仙林校区敬文学生活动中心 多功能厅\\
海选报名方式:将2分钟左右的自选配音片段(视频/音频皆可,格式不限,视频需有字幕,音频需有剧本)发至邮箱changgexingcv@163.com,邮件命名格式为圈名(如有)+姓名+QQ号+手机号+校区+作品。提交作品则视为参加比赛\\
报名、赛制、奖项等详情请查看\url{https://mp.weixin.qq.com/s/i1Lm3j8r2jjNRCK1vfP7bw}\\
\section{物院研究生会系列活动预告}
11月16日-11月17日:“辩物论理,慧颖争锋”辩论赛\\
11月21日:物院就业周——就业能力实训坊\\
11月29日:物院就业周——物理数学行业专场招聘会\\
11月30日:“悟道无穷”国奖专访\\
12月5日-12月7日:五校联盟博士生学术论坛\\
12月8日:物理学院第一届“掼军杯”掼蛋比赛\\
12月15日:物院第三届“锦鲤杯”王者荣耀比赛\\
12月16日:物院摄影展\\
12月21日:“Lunch Break”午间学术交流会(第九期)+“暖冬相伴,物耀星光”师生联欢晚会\\
详情请查看\url{https://mp.weixin.qq.com/s/zu_NlXX6C1eqmQlpHZ_sRw}\\
\end{multicols} 
\end{document}
