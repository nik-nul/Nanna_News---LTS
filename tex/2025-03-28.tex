% HEAD BEGIN
\documentclass[letterpaper, 12pt]{article}
\newsavebox\colbbox
\usepackage{graphicx}
\usepackage{multicol}
\usepackage{anysize}
\usepackage{fontspec}
\usepackage[fontset=none]{ctex}
\usepackage{tabularx}
\usepackage{longtable}
\PassOptionsToPackage{hyphens}{url}
\usepackage[breaklinks=true, colorlinks=true]{hyperref}
\expandafter\def\expandafter\UrlBreaks\expandafter{\UrlBreaks\do\a\do\b\do\c\do\d\do\e\do\f\do\g\do\h\do\i\do\j\do\k\do\l\do\m\do\n\do\o\do\p\do\q\do\r\do\s\do\t\do\u\do\v\do\w\do\x\do\y\do\z\do\A\do\B\do\C\do\D\do\E\do\F\do\G\do\H\do\I\do\J\do\K\do\L\do\M\do\N\do\O\do\P\do\Q\do\R\do\S\do\T\do\U\do\V\do\W\do\X\do\Y\do\Z}
% \let\oldurl\url
% \renewcommand{\url}[1]{\begin{sloppypar}\oldurl{#1}\end{sloppypar}}
\setlength\columnsep{30pt}
\marginsize{30pt}{30pt}{10pt}{20pt}
\setmainfont{TeX Gyre Bonum}
\setCJKmainfont[BoldFont=Noto Serif CJK SC Bold, ItalicFont=FandolKai]{Source Han Sans SC}
\setlength{\parindent}{0cm}
% \setCJKmonofont{Noto Sans CJK SC}
\begin{document}
\begin{center}
    \Huge\textbf{南哪大专醒前消息}
\end{center}
\vspace{4mm}
\hrule
\renewcommand\tabularxcolumn[1]{m{#1}}
\begin{tabularx}{\textwidth}{>{\hsize.2\hsize}X>{\hsize.6\hsize}X>{\hsize.2\hsize}X}
    \begin{flushleft}
        2025.3.27\, No.203
    \end{flushleft}
    &
    \begin{center}
        \textit{“秉中持正、求新博闻。”}
    \end{center}
    &
    \begin{flushright}
        \textbf{南京市栖霞区}
    \end{flushright}
\end{tabularx}
\vspace{-3.5mm}
\hrule
\vspace{4mm}
% HEAD END
\centerline{\huge\textbf{活动预告}}
\begin{multicols}{2}
\section{订阅方式和加入编辑部}  
编辑部招聘人才,用爱发电,工作轻松,详情可联系QQ:1329527951 客服小千\\想订阅本消息或获取PDF版(便于查看超链接和往期),可加QQ群:\href{https://qm.qq.com/q/4HL41Nt3sQ}{466863272}.
\section{活动清单}
\setbox\colbbox\vbox{
\makeatletter\col@number\@ne
\begin{longtable}{|>{\centering\arraybackslash}m{.3\textwidth}|m{.06\textwidth}|m{.06\textwidth}|}
    \hline
    活动 & 开展时间 & 刊载时间\\
    \hline\hline
    南大版deepseek & / & 2.22\\
    悦读课程群 & / & 2.24\\
    eScience AI科研助手 & / & 3.11\\
    地科博物馆开放安排 & / & 3.22\\ 
    乐跑 & 5.16 & 3.10\\
    本科生劳育实践 & 7.20 & 2.19\\
    医保零星报销 & 3.31 & 2.19\\
    银星杯论文赛 & 4.22 & 2.27\\
    高教社杯 & 4.25 & 3.5\\
    南辩院系杯 & 4.12 & 3.6\\
    大文大理题目征集 & 期末 & 3.8\\
    5月免费上网 & ? & 3.9\\
    基础学科论坛 & 4.20 & 3.9\\
    普通话测试 & 4.11 & 3.25\\
    外教社杯 & 5.27 & 3.12\\
    Python比赛 & 4.6 & 3.16\\
    本科生院征集大鸣大放 & 4.4 & 3.21\\
    两会知识竞赛 & 3.30 & 3.21\\
    纸鸢工作坊 & 4.3 & 3.22\\
    南大博篆刻体验课 & 4.2 & 3.23\\
    粤歌赛 & 4.12 & 3.24\\
    外词杯 & 3.31 & 3.25\\
    江苏创青春赛事 & 4.30 & 3.26\\
    石膏绘画活动 & 3.29 & 3.26\\
    悦读测试 & 4.6 & 3.27\\
    南大数学竞赛 & 4.15 & 3.27\\
    新生午餐会 & 3.30 & 3.28\\
    
    \hline
\end{longtable}
\unskip
\unpenalty
\unpenalty}\unvbox\colbbox
\end{multicols}
\begin{multicols}{2}
\pagebreak

\section{讲座}
\begin{tabular}{|>{\centering\arraybackslash}m{.3\textwidth}|m{.06\textwidth}|m{.06\textwidth}|}
    \hline
    讲座 & 开展时间 & 刊载时间\\
    \hline\hline
    如何影响消费者动机与心理健康支持 & 4.1 & 3.26\\\hline
    “图绎文心:清代文艺思想探研”工作坊 & 3.29 & 3.27\\\hline
    当本科生按下AI启动键 & 3.30 & 3.28\\\hline
    移民与流动研究的时间、情感和日常转向 & 4.1 & 3.28\\\hline
    AI: The Destruction of the Imagination? & 4.2 & 3.28\\\hline
\end{tabular}
%讲座预告写在这。用subsection
\subsection{移民与流动研究的时间、情感和日常转向}
时间:2025年4月1日 14:00—16:00
\\地点:南京大学仙林校区社会学院河仁楼401室
\\主讲人:郭未,南京大学社会学院社会工作与社会政策系教授、博导
\\更多内容见原推文
\\详见:\url{https://mp.weixin.qq.com/s/AHaNic1jqNqX6RgeYshoOg}

\subsection{Scott Lash教授:AI: The Destruction of the Imagination?}
主讲人:Scott Lash英国伦敦大学哥德斯密学院社会学教授文化研究中心主任
\\主持人: 周宪 南京大学艺术学院教授南京大学高研院名誉院长
\\时间:  2025年4月2日(周三)晚上18:30开始
\\地点:  仙林校区国际学院C308高研院报告厅
\\备注:英语演讲
\\详见:\url{https://mp.weixin.qq.com/s/tuo3tRPCqIGWzYSLHUlXqA}

\subsection{工作坊|智绘千年:当本科生按下AI启动键}
时间:2025年3月30日 周日 9:00-12:00
\\地点:历史学院223
\\主持人:南京大学历史学院副院长 王涛
\\与会嘉宾:金伯文 姚念达 姚全 殷洁
\\详见:\url{https://mp.weixin.qq.com/s/Vu9Nvp5K3hIiXamXUZ8PpQ}

%此处写校级活动,请不要把讲座、院级活动和社团活动写在这里orz orz orz


\section{【救在身边 校园守护】校医院应急救护培训征集}
目的:
\\1. 普及应急救护知识,提高师生安全意识和自救互救能力;
\\2. 掌握心肺复苏(CPR)+自动体外除颤器(AED)使用基本急救技能;
\\3. 培养师生在突发事件中的冷静应对能力,为校园安全提供保障。
\\ 培训内容:
\\1.急救基础知识:急救原则、急救流程、注意事项等;
\\2.心肺复苏(CPR):心脏骤停的识别与处理,胸外按压、人工呼吸的操作方法;
\\3.自动体外除颤器(AED)的使用
\\4.气道异物梗阻的应急处理;
\\5.创伤救护:止血、包扎、骨折固定等技能。请有意参与培训的学院/部门将联系方式、需求发送至邮箱:swwei2012@nju.edu.cn
\\详见:\url{https://hospital.nju.edu.cn//ggtz/20250328/i310538.html}
\section{高研院新生午餐会第五十一场}
题目:从敦煌到奈良:丝绸之路文化的东传
\\谈话人:刘东波 南京大学外国语学院特聘研究员、助理教授
\\主持人:徐志君 南京大学艺术学院副研究员
\\时间:2025年3月31日(周一)12:20-13:20
\\地点:鼓楼校区逸夫馆9楼高研院报告厅
\\本次抽签开始时间为3月29日(周六)中午12:30,请提前登录。人数限额为30人,先到先得
\\截止时间为3月30日中午12:30
\\详见:\url{https://mp.weixin.qq.com/s/bNnbOVlAwDPwPKZoBwR10Q}
\section{第十九届“挑战杯”竞赛志愿者培训课程}
四月培训课程安排:
\\急救卫生培训时间:4.2(周三)16:00—18:00
\\地点:线上腾讯会议
\\安全教育时间:4.9(周三)16:00—18:00
\\地点:线上腾讯会议
\\压力和情绪管理辅导:时间:4.16(周三)16:00—18:00
\\地点:线下,地点另行通知
\\国赛赛制全流程介绍讲解时间:4.23(周三)16:00—18:00
\\地点:线上腾讯会议
\\线上统一考核时间:4.27(周四)—4.28(周五)
\\地点:线上
\\培训规则:
\\1. 培训结束后,将统一组织志愿者考核、分组;
\\2. 已报名的志愿者至少参与一次培训方可参与志愿者考核;
\\3. 培训课程的参与数量将作为志愿者分组定岗的重要依据。
\\每次培训的报名推送都会提前3天在中午12点发布,推送发布时间即报名开始时间,问卷提交成功即为报名成功,培训记录以实际出勤记录为准。请大家及时关注公众号信息!
\\详见:\url{https://mp.weixin.qq.com/s/Ki76fj8qaR7NSheJpCGmoA}

\section{院级活动}
\begin{tabular}{|>{\centering\arraybackslash}m{.3\textwidth}|m{.06\textwidth}|m{.06\textwidth}|}
\hline
    活动 & 开展时间 & 刊载时间\\
    \hline\hline
    文院剧本创作研讨会 & 9.30 & 3.2\\
    物院征集课程指南 & 6.15 & 3.3\\
    地海征集春日影 & 6.15 & 3.14\\
    AI院影色舞 & 3.29 & 3.19\\
    商院羽球 & 3.29 & 3.19\\
    社院学术节 & 4.18 & 3.25\\
    生科栽培 & 3.30 & 3.25\\
    有训行知集体生日会 & 3.30 & 3.26\\
    地学研讨会 & 3.29 & 3.27\\
    电院趣运会 & 3.30 & 3.27\\
    马院春日音 & 3.30 & 3.28\\
    
    \hline
\end{tabular}
\subsection{炜华体育场“春日音乐节”}
马克思主义学院与现代工程与应用科学学院打破学科界限,携手打造「春日音乐节」。本次春日音乐节,不仅汇集了令人目不暇接的唱歌、舞蹈与器乐演奏等精彩节目,还加入了趣味横生的游戏环节及神秘的抽奖环节。具体节目单详见推文。
\\活动时间:2025年3月30日(周日)17:00-19:00
\\活动地点:南京大学仙林校区炜华体育场
\\详见:\url{https://mp.weixin.qq.com/s/kouzTQW5_inZZ126qcRi9g}

\section{社团活动}
\begin{tabular}{|>{\centering\arraybackslash}m{.3\textwidth}|m{.06\textwidth}|m{.06\textwidth}|}
    \hline
    社团活动 & 开展时间 & 刊载时间\\
    \hline\hline
    天文台开放日 & / & 1.6\\
    鸿新社捐书活动 & 3.30 & 3.17\\
    长歌行声演剧 & 3.29 & 3.19\\
    雁行南大x以伴云陪伴线上志愿者招募 & 3.30 & 3.24\\
    林泉CAC流行音乐会 & 3.30 & 3.26\\
    悲惨世界观影沙龙 & 3.29 & 3.26\\
    孤独症主题活动 & 3.30 & 3.27\\
    南苏摇联音乐节 & 3.29 & 3.28\\
    二剧招募 & 4.1 & 3.28\\
    悦读书社春季招新 & 3.30 & 3.28\\
    \hline
\end{tabular}
%这里是写社团活动的,社团活动就是由社团主办、主要针对社团内部人员的活动。不要把非社团活动写在这里。
\subsection{南苏摇联 首届“复苏”音乐节}
活动时间:
\\3.29 周六 19:00
\\活动地点:南京大学苏州校区体育场
\\具体曲目见原推
\\详见:\url{https://mp.weixin.qq.com/s/nFj_HkyzU6iogSAGgLDlRg}

\subsection{剧组招募 | 《凯撒切开》:关于即将召开的会议相关准备会议的讨论会}
演员招募
\\6人,性别不限,负责开会;分别饰演A、B、C、D、E、G。
\\不一定需要你此前有任何表演经验,只要你对剧场有玩耍热情与工作态度,就欢迎你来!欢迎男女老少各行各业五花八门千奇百怪的人来开会!
\\演出信息
\\2025年5月30日至6月1日于黑匣子。
\\注意事项
\\排练将主要在南京大学仙林校区内进行,预计每周有两到三次,每次半天的时间。
\\报名信息
\\截止时间:4月1日(周二)23:59;
\\面试时间:4月2日或4月3日,根据问卷统计情况而定;
\\报名方式和剧目具体内容详见原推
\\详见:\url{https://mp.weixin.qq.com/s/zp4ufk2rlBj4RqQ70NAFJg}

\subsection{曲目介绍 | “桃信缓歌行”流行音乐会}
本推文为对该音乐会曲目的介绍,具体内容请参见原推。
\\领票信息:转发此推文并凭转发记录可于3月29日(周六),30日(周日)两天中午11:30-12:50于仙林校区第五食堂门口展台处领取流行音乐会门票、节目单或于3.30演出当天下午6:30多功能厅门口补领
\\详见:\url{https://mp.weixin.qq.com/s/skKBVpjFIVb8VfyzOxAd5g}

\subsection{2025春季招新 | 悦读书社}
南京大学学生悦读书社是在南京大学人文社会科学高级研究院指导下的学生学术社团,负责南京大学DIY研读研究课程的筹备工作,定期策划各类讲座、读书沙龙和观影会等学术活动。
\\本次招新包括文化中心和研读中心两个部门。我们期待报名文化中心的你:对书社的各项工作怀抱兴趣,对DIY课程具有充分了解,有微信公众号推文编辑或海报制作等经验,熟练使用相关软件和工具。我们期待报名研读中心的你:拥有较浓厚的学术兴趣与较好的学术基础,参与过DIY课程、系列讲座等书社主办的活动,对DIY课程具有充分了解。
\\本次招新不限年级、专业,不设招新人数上限与下限。提交报名表后,书社将通过手机短信发送面试安排。请有意报名参与的同学于3月30日(周日)24:00前扫描二维码,填写报名表。
\\
\\详见:\url{https://mp.weixin.qq.com/s/usj9v2LWEydUyN4h3VPmRw}
\end{multicols}
\end{document}
