% HEAD BEGIN
\documentclass[letterpaper, 12pt]{article}
\newsavebox\colbbox
\usepackage{graphicx}
\usepackage{multicol}
\usepackage{anysize}
\usepackage{fontspec}
\usepackage[fontset=none]{ctex}
\usepackage{tabularx}
\usepackage{longtable}
\PassOptionsToPackage{hyphens}{url}
\usepackage[breaklinks=true, colorlinks=true]{hyperref}
\expandafter\def\expandafter\UrlBreaks\expandafter{\UrlBreaks\do\a\do\b\do\c\do\d\do\e\do\f\do\g\do\h\do\i\do\j\do\k\do\l\do\m\do\n\do\o\do\p\do\q\do\r\do\s\do\t\do\u\do\v\do\w\do\x\do\y\do\z\do\A\do\B\do\C\do\D\do\E\do\F\do\G\do\H\do\I\do\J\do\K\do\L\do\M\do\N\do\O\do\P\do\Q\do\R\do\S\do\T\do\U\do\V\do\W\do\X\do\Y\do\Z}
% \let\oldurl\url
% \renewcommand{\url}[1]{\begin{sloppypar}\oldurl{#1}\end{sloppypar}}
\setlength\columnsep{30pt}
\marginsize{30pt}{30pt}{10pt}{20pt}
\setmainfont{TeX Gyre Bonum}
\setCJKmainfont[BoldFont=Noto Serif CJK SC Bold, ItalicFont=FandolKai]{Source Han Sans SC}
\setlength{\parindent}{0cm}
% \setCJKmonofont{Noto Sans CJK SC}
\begin{document}
\begin{center}
    \Huge\textbf{南哪大专醒前消息}
\end{center}
\vspace{4mm}
\hrule
\renewcommand\tabularxcolumn[1]{m{#1}}
\begin{tabularx}{\textwidth}{>{\hsize.2\hsize}X>{\hsize.6\hsize}X>{\hsize.2\hsize}X}
    \begin{flushleft}
        2025.4.20\, No.225
    \end{flushleft}
    &
    \begin{center}
        \textit{“秉中持正、求新博闻。”}
    \end{center}
    &
    \begin{flushright}
        \textbf{南京市栖霞区}
    \end{flushright}
\end{tabularx}
\vspace{-3.5mm}
\hrule
\vspace{4mm}
% HEAD END
\centerline{\huge\textbf{活动预告}}
\begin{multicols}{2}
\section{订阅方式和加入编辑部}  
编辑部招聘人才,用爱发电,工作轻松,详情可联系QQ:1329527951 客服小千\\想订阅本消息或获取PDF版(便于查看超链接和往期),可加QQ群:\href{https://qm.qq.com/q/4HL41Nt3sQ}{466863272}.
\section{活动清单}
\setbox\colbbox\vbox{
\makeatletter\col@number\@ne
\begin{longtable}{|>{\centering\arraybackslash}m{.3\textwidth}|m{.06\textwidth}|m{.06\textwidth}|}
    \hline
    活动 & 开展时间 & 刊载时间\\
    \hline\hline
    南大版deepseek & / & 2.22\\
    悦读课程群 & / & 2.24\\
    eScience AI科研助手 & / & 3.11\\
    地科博物馆开放安排 & / & 3.22\\ 
    2025年分流和转专业政策通知 & / & 4.7\\
    乐跑 & 5.16 & 3.10\\
    本科生劳育实践 & 7.20 & 2.19\\
    银星杯论文赛 & 4.22 & 2.27\\
    高教社杯 & 4.25 & 3.5\\
    大文大理题目征集 & 期末 & 3.8\\
    5月免费上网 & ? & 3.9\\
    基础学科论坛 & 4.20 & 3.9\\
    外教社杯 & 5.27 & 3.12\\
    江苏创青春赛事 & 4.30 & 3.26\\
    浦口音乐跑 & 5.30 & 3.31\\
    程设大赛 & 4.26 & 4.2\\
    仙林校区志愿法律咨询 & / & 4.4\\
    青春活力大赛 & 5.17 & 4.7\\
    在校生自愿体检 & 6.20 & 4.8\\
    南大购买WPS & / & 4.8\\
    24级程设大赛 & 4.27 & 4.11\\
    法治情景剧策划大赛 & 4.23 & 4.11\\
    EL程设大赛 & 4.27 & 4.13\\
    中美中心2025年证书项目 & 5.24 & 4.14\\
    春季学期创新训练计划结题考核通知 & 4.28 & 4.15\\
    植物微景观活动 & 4.23 & 4.16\\
    “天池杯”AI创新大赛 & 4.28 & 4.17\\
    紫砂壶体验课 & 4.24 & 4.18\\
    “食堂‘名厨’”“巡回”“展” & 4.21-4 & 4.18\\
    新生午餐读书会旁听 & 4.23 & 4.20\\
    
    \hline
\end{longtable}
\unskip
\unpenalty
\unpenalty}\unvbox\colbbox
\end{multicols}
\begin{multicols}{2}
\pagebreak

\section{讲座}
\begin{tabular}{|>{\centering\arraybackslash}m{.3\textwidth}|m{.06\textwidth}|m{.06\textwidth}|}
    \hline
    讲座 & 开展时间 & 刊载时间\\
    \hline\hline
    从感知到疗愈:人脑音乐加工机制 & 4.25 & 4.11\\\hline
    Ionizing spotlight of Active Galactic Nucleus & 4.23 & 4.11\\\hline
    科技之巅:人机共智2030 & 4.22 & 4.16\\\hline
    软件发展与技术漫谈 & 4.29 & 4.16\\\hline
    DeepSeek: 从人工智能到大模型及应用思考 & 4.24 & 4.16\\\hline
    从语言到智能 ⸺ 大语言模型的奥秘与应用 & 5.6 & 4.16\\\hline
    现代中国诸社会思潮的地域起源 & 4.21 & 4.17\\\hline
    智慧物流助力智能制造 & 4.23 & 4.17\\\hline
    媒介见证的黄昏? & 4.21 & 4.17\\\hline
    急救技能抓住 “黄金 4 分钟” & 4.25 & 4.18\\\hline
    Integrated River Flow and Water Quality Modeling & 4.21 & 4.18\\\hline
    我国低空经济气象科技推进情况进展及展望 & 4.21 & 4.18\\\hline
    从点击到繁荣:中国电商发展的政治经济学 & 4.22 & 4.18\\\hline
    遗忘的耐候性 & 4.23 & 4.18\\\hline
    TrainMover: Live Migration for Resilient and Continuous ML Training & 4.22 & 4.18\\\hline
    2025平安留学行前培训会 & 4.29 & 4.20\\\hline
    结构化数据深度学习算法研究进展与展望 & 4.22 & 4.20\\\hline
    花语崛起:拉斐尔前派画作中的花草 & 4.25 & 4.20\\\hline
    《威尼斯商人》的当代隐喻 & 4.23 & 4.20\\\hline
    通过周期性驱动系统中的加权势能构造多体态 & 4.22 & 4.20\\\hline
    “法护青春,职路引航”就业季法律公益讲座 & 4.25 & 4.20\\\hline
    
\end{tabular}
\subsection{2025平安留学行前培训会} % 讲座 describer: Grey_Bubble
2025年江苏省领事保护进校园暨教育部平安留学培训会
\\时间:4月29日(周二)
\\地点:仙林校区杜厦图书馆一楼报告厅
\\活动内容:
\\1、中国公民海外安全与领事保护
\\2、留学与报国
\\3、出国留学心理调适及跨文化交流
\\4、遵纪守法留学及其相关法律问题
\\报名截止时间为4月24日(周四)晚24:00,报名二维码请见活动链接
\\详见:\url{https://mp.weixin.qq.com/s/bUvKBhgYZaZFda9WCrXjCg}

\subsection{结构化数据深度学习算法研究进展与展望} % 讲座 describer: nik_nul
时间:2025年4月22日(星期二) 19:00
\\腾讯会议ID:580-218-676
\\主讲人陈晋泰, 助理教授 香港科技大学(广州)
\\详见:\url{https://mp.weixin.qq.com/s/Yv69O4uvz1_4EiRjZXS_DA}

\subsection{人文艺术系列讲座第579期:花语崛起:拉斐尔前派画作中的花草} % 讲座 describer: asuka
地点:南京大学仙林校区教学楼仙2-306
\\时间:2025年4月25日16:00
\\主讲人:祺四,畅销书作家,独立策展人
\\主持人:杨秀娟,南京大学艺术学院副教授
\\
\\
\\详见:\url{https://mp.weixin.qq.com/s/lvgEJq8eONkLt8civDTmVQ}

\subsection{Kyoo Lee教授:《威尼斯商人》的当代隐喻} % 讲座 describer: asuka
主讲人:Kyoo Lee
\\主持人:从丛
\\时间:2025年4月23日(周三)下午16:00开始
\\地点:鼓楼校区逸夫馆9楼高研院报告厅
\\备注:英语演讲
\\详见:\url{https://mp.weixin.qq.com/s/nn6C0WU00qyL7fYN2UwnJA}

\subsection{通过周期性驱动系统中的加权势能构造多体态} % 讲座 describer: nik_nul
报告时间:4月22日(周二)中午12点
\\报告地点:南京大学鼓楼校区唐仲英楼B501
\\详见:\url{https://mp.weixin.qq.com/s/goMAniQYaffwFG2icNCI3w}

\subsection{“法护青春,职路引航”就业季法律公益讲座} % 讲座 describer: charlors
时间:2025年4月25日(周五)14:00-16:00
\\地点:仙林校区十食堂三楼就业中心303报告厅
\\主讲人:周长征 现任南京大学法学院副教授
\\详见:\url{https://mp.weixin.qq.com/s/Gmk7FJhlVMi0D4rtfXnv0Q}



\section{2025年春季学期新生午餐读书会第三场旁听报名} % 校级活动 describer: nik_nul
时间:4月23日(周三)12:15-13:40
\\地点:鼓楼校区新教501
\\物资:本场次读书会向招募参与同学发放文化纪念衫,领取时间、地点另行通知。
\\流程:两位同学发言各15分钟,老师领读30分钟,提问讨论25分钟。
\\主讲老师:历史学院 孙扬
\\阅读书目:
\\1)黄兴涛《重塑中华:近代中国“中华民族”观念研究》(北京师范大学出版社,2017年)。
\\汇报人:行知书院 林宝麟
\\2)陈蕴茜《崇拜与记忆:孙中山符号的建构与传播》(南京大学出版社,2009年)。
\\汇报人:秉文书院 汤淇钧
\\详见:\url{https://mp.weixin.qq.com/s/jllp10_5xX8HZ4_rYGuuUA}

\section{“矿蕴金陵 紫气东来——人与矿产资源的相互关系”特别展览} % 校级活动 describer: Ando
展览主题:矿蕴金陵 紫气东来——人与矿产资源的相互关系
\\时间:2025.4.20(开幕式)—2025.5.20 上午10:00-下午4:00
\\地点:南京大学鼓楼校区田家炳楼一楼多功能厅
\\参加4月20日10点启幕活动的前30位观众将分别获得南京(国际)矿博会“紫色矿物主题艺术特展”门票1张。2025南京矿博会将于2025年5月15-19日在
\\宁举办。
\\详见:\url{https://mp.weixin.qq.com/s/1arcOXZj1bkWrmAvtoRubw}

\section{NJU春日旧书市集开箱} % 校级活动 describer: HOllyWood
时间:4.21-4.23 10:00-18:00
\\地点:杜厦图书馆二楼大厅
\\购书即可解锁图书馆历代建筑章+多版本藏书章
\\ 彩蛋:购书满10本赠读书节限定帆布包(数量有限,先到先得)
\\详见:\url{https://mp.weixin.qq.com/s/X9-SaYEO2bB0BRVA79gAag}
\section{院级活动}
\begin{tabular}{|>{\centering\arraybackslash}m{.3\textwidth}|m{.06\textwidth}|m{.06\textwidth}|}
\hline
    活动 & 开展时间 & 刊载时间\\
    \hline\hline
    文院剧本创作研讨会 & 9.30 & 3.2\\
    物院征集课程指南 & 6.15 & 3.3\\
    地海征集春日影 & 6.15 & 3.14\\
    五院乒乓球赛 & 4.19 & 3.31\\
    法院党建征文 & 5.20 & 4.2\\
    地学趣运会 & 4.26 & 4.9\\
    四院音乐节 & 5.11 & 4.7\\
    商院征集 & 5.5 & 4.8\\
    物院运动打卡 & 5.14 & 4.12\\
    地海图书漂流 & 4.23 & 4.16\\
    文院茶话会 & 4.24 & 4.20\\
    希音杯 & 4.25 & 4.20\\
    \hline
\end{tabular}
\subsection{资讯|第十期“恰同学少年”师生交流会活动预告} % 院级活动 describer: asuka
形式:茶话会
\\地点:南京大学仙林校区(具体地点容后另行通知\textasciitilde{})
\\时间:2025年4月24日(周四)下午14:00-16:00
\\时长:2小时
\\参与人数:10人(面向文学院全体同学)
\\详见:\url{https://mp.weixin.qq.com/s/658Kd_SiI1dQh2BWHy44Qw}


\subsection{活动报名 | 第三届“希音杯”编程竞赛} % 院级活动 describer: nik_nul
活动对象:南京大学计算机学院、软件学院和人工智能学院全体在读本科生、研究生
\\活动时间:2025年4月27日(周日) 14:00-17:00
\\活动地点:南京大学仙林校区基础实验楼乙124
\\活动流程:
\\参赛者填写报名问卷进行报名,报名成功后,群聊二维码会通过邮件发送,请扫描该二维码加入赛事群
\\比赛开始前10分钟,参赛者自行携带电脑到达指定地点;届时举办方会向成功报名同学的邮箱发送赛题链接,参赛者通过点击链接、登录指定OJ平台参赛
\\比赛总计时间为2小时,预计共3-5道编程题,具体请以实际试题为准
\\本次比赛为个人赛,线下举行,需要参赛者自行携带电脑,并确保网络环境正常
\\比赛试题具有区分度,题目顺序与难度无关
\\本次比赛支持的编程语言为C++、Java和Go
\\参赛者需具备一定的程序设计基础,能够熟练使用给定的编程语言实现需求
\\比赛现场将为大家提供纸笔
\\正式比赛结束后,将在现场展示成绩排名,确定最终的名次,并进行颁奖
\\报名截止时间为北京时间2025年4月25日18:00。
\\报名问卷链接:https://wj.qq.com/s2/20676655/a4he/
\\比赛群二维码后续通过问卷中填写的邮箱发布,报名成功后请注意查收并加群。
\\活动奖品
\\一等奖:1名,1000元京东E卡
\\二等奖:3名,800元京东E卡
\\三等奖:6名,500元京东E卡
\\详见:\url{https://mp.weixin.qq.com/s/4qdO876dnnGHwsC4l9DM5g}

\subsection{南京大学大气科学学院2025年全球科考报名通知} % 院级活动 describer: nik_nul
内含“大湾区灾害性天气与城市防灾科考与科研训练”“中挪极地气候与环境科考与科研训练”“中芬赫尔辛基大气与地球系统科考与科研训练”等五个项目
\\1.报名对象:南京大学大气科学学院2022级优秀本科生(详见各项目简介),要求对大气及地球系统科学感兴趣,未受过任何纪律处分,具有较好的表达能力、外语能力和组织能力,团结同学,服从团队管理。
\\2. 原则上要求报名同学截止目前的必修课学分绩4.0以上。
\\3.根据项目实际,每项目或收取少量费用。
\\4. 报名方式:4月20日24:00之前报名,
\\报名链接:\url{https://table.nju.edu.cn/dtable/forms/42afb141-3672-44e6-ae0c-fef898e80b06/}
\\5.  选拔方式:根据报名表按照1:1.5比例进行第一轮筛选,通过者会尽快通知参加面试选拔。
\\6. 根据学校及学院相关要求,全球科考与科研训练项目每人至多参加一次(本科期间)。此外,享受过大气科学学院拔尖计划(海外研修)资助的同学不能报名。大气科学学院2025年大气与地球系统综合实习(国内外科考)只能选择参加一项。
\\注:按学校的规定,以上全球科考与科研训练项目均纳入暑期课程。顺利完成实习和所有学习、研究任务的同学,将获得1学分或可申请认定项目制课程。
\\7.如遇不可抗力因素,全球科考项目可能会取消。
\\详见:\url{https://mp.weixin.qq.com/s/x0LnVTdXu9VNwtcwIXOUag}

\subsection{南新读书会 | 下周预告} % 院级活动 describer: nik_nul
时间:2025年4月23日(周三)19:00
\\地点:南京大学新闻传播学院311室
\\书目:
\\1. 前浪后浪
\\分享人:马文 南京大学新闻传播学院助理研究员
\\2. 多种声音,一个世界
\\分享人:余清亦 2024级硕士研究生
\\详见:\url{https://mp.weixin.qq.com/s/BT0F8clIIOinUqLCbrmoIg}
\section{社团活动}
\begin{tabular}{|>{\centering\arraybackslash}m{.3\textwidth}|m{.06\textwidth}|m{.06\textwidth}|}
    \hline
    社团活动 & 开展时间 & 刊载时间\\
    \hline\hline
    天文台开放日 & / & 1.6\\
    重唱诗歌奖征稿 & 4.30 & 3.31\\
    拳击社体验 & 4.22 & 4.1\\
    轮滑社体验 & 4.22 & 4.1\\
    定向赛 & 4.20 & 4.1\\
    体育舞蹈教学 & 4.25 & 4.1\\
    吉他社歌手招募 & 4.20 & 4.4\\
    吉他社春日音 & 4.26 & 4.4\\
    天健捐衣 & 4.20 & 4.13\\
    毽球趣味赛 & 4.19 & 4.15\\
    流光影映 & 4.19 & 4.16\\
    汉服社摆摊 & 4.26 & 4.17\\
    车协科普 & 4.26 & 4.17\\
    招协招募 & 4.22 & 4.17\\
    摇联春日音 & 4.19 & 4.17\\
    心协团辅活动 & 4.23 & 4.17\\
    口语角 & 4.19 & 4.18\\
    匿名评诗会 & 4.26 & 4.20\\
    \hline
\end{tabular}
%这里是写社团活动的,社团活动就是由社团主办、主要针对社团内部人员的活动。不要把非社团活动写在这里。
\subsection{“紫丁香一生宣布,醒的目的是做梦”匿名评诗会} % 社团活动 describer: nik_nul
时间:2025年4月26日(周六)14:00
\\地点:新教404(南京大学鼓楼校区)
\\投稿邮箱:aichongchang@163.com(备注匿名评诗会,限投一首)
\\截稿日期:2025年4月25日(周五)12:00
\\详见:\url{https://mp.weixin.qq.com/s/q1dyW-MLBoaKcMFD7B15Vg}
\end{multicols}
\end{document}