% HEAD BEGIN
\documentclass[letterpaper, 12pt]{article}
\newsavebox\colbbox
\usepackage{graphicx}
\usepackage{multicol}
\usepackage{anysize}
\usepackage{fontspec}
\usepackage[fontset=none]{ctex}
\usepackage{tabularx}
\usepackage{longtable}
\PassOptionsToPackage{hyphens}{url}
\usepackage[breaklinks=true, colorlinks=true]{hyperref}
\expandafter\def\expandafter\UrlBreaks\expandafter{\UrlBreaks\do\a\do\b\do\c\do\d\do\e\do\f\do\g\do\h\do\i\do\j\do\k\do\l\do\m\do\n\do\o\do\p\do\q\do\r\do\s\do\t\do\u\do\v\do\w\do\x\do\y\do\z\do\A\do\B\do\C\do\D\do\E\do\F\do\G\do\H\do\I\do\J\do\K\do\L\do\M\do\N\do\O\do\P\do\Q\do\R\do\S\do\T\do\U\do\V\do\W\do\X\do\Y\do\Z}
% \let\oldurl\url
% \renewcommand{\url}[1]{\begin{sloppypar}\oldurl{#1}\end{sloppypar}}
\setlength\columnsep{30pt}
\marginsize{30pt}{30pt}{10pt}{20pt}
\setmainfont{TeX Gyre Bonum}
\setCJKmainfont[BoldFont=Noto Serif CJK SC Bold, ItalicFont=FandolKai]{Source Han Sans SC}
\setlength{\parindent}{0cm}
% \setCJKmonofont{Noto Sans CJK SC}
\begin{document}
\begin{center}
    \Huge\textbf{南哪大专醒前消息}
\end{center}
\vspace{4mm}
\hrule
\renewcommand\tabularxcolumn[1]{m{#1}}
\begin{tabularx}{\textwidth}{>{\hsize.2\hsize}X>{\hsize.6\hsize}X>{\hsize.2\hsize}X}
    \begin{flushleft}
        2025.4.2\, No.208
    \end{flushleft}
    &
    \begin{center}
        \textit{“秉中持正、求新博闻。”}
    \end{center}
    &
    \begin{flushright}
        \textbf{南京市栖霞区}
    \end{flushright}
\end{tabularx}
\vspace{-3.5mm}
\hrule
\vspace{4mm}
% HEAD END
\centerline{\huge\textbf{活动预告}}
\begin{multicols}{2}
\section{订阅方式和加入编辑部}  
编辑部招聘人才,用爱发电,工作轻松,详情可联系QQ:1329527951 客服小千\\想订阅本消息或获取PDF版(便于查看超链接和往期),可加QQ群:\href{https://qm.qq.com/q/4HL41Nt3sQ}{466863272}.
\section{活动清单}
\setbox\colbbox\vbox{
\makeatletter\col@number\@ne
\begin{longtable}{|>{\centering\arraybackslash}m{.3\textwidth}|m{.06\textwidth}|m{.06\textwidth}|}
    \hline
    活动 & 开展时间 & 刊载时间\\
    \hline\hline
    南大版deepseek & / & 2.22\\
    悦读课程群 & / & 2.24\\
    eScience AI科研助手 & / & 3.11\\
    地科博物馆开放安排 & / & 3.22\\ 
    乐跑 & 5.16 & 3.10\\
    本科生劳育实践 & 7.20 & 2.19\\
    银星杯论文赛 & 4.22 & 2.27\\
    高教社杯 & 4.25 & 3.5\\
    南辩院系杯 & 4.12 & 3.6\\
    大文大理题目征集 & 期末 & 3.8\\
    5月免费上网 & ? & 3.9\\
    基础学科论坛 & 4.20 & 3.9\\
    普通话测试 & 4.11 & 3.25\\
    外教社杯 & 5.27 & 3.12\\
    Python比赛 & 4.6 & 3.16\\
    本科生院征集大鸣大放 & 4.4 & 3.21\\
    纸鸢工作坊 & 4.3 & 3.22\\
    粤歌赛 & 4.12 & 3.24\\
    江苏创青春赛事 & 4.30 & 3.26\\
    悦读测试 & 4.6 & 3.27\\
    南大数学竞赛 & 4.15 & 3.27\\
    AI素养大赛 & 4.15 & 3.31\\
    浦口音乐跑 & 5.30 & 3.31\\
    红会暑期项目招募 & 4.12 & 4.1\\
    程设大赛 & 4.26 & 4.2\\
    主持人大赛报名 & 4.10 & 4.4\\
    春影摄影大赛 & 4.13 & 4.4\\
    奇绩创业宣讲课 & 4.11 & 4.4\\
    \hline
\end{longtable}
\unskip
\unpenalty
\unpenalty}\unvbox\colbbox
\end{multicols}
\begin{multicols}{2}
\pagebreak

\section{讲座}
\begin{tabular}{|>{\centering\arraybackslash}m{.3\textwidth}|m{.06\textwidth}|m{.06\textwidth}|}
    \hline
    讲座 & 开展时间 & 刊载时间\\
    \hline\hline
    秦汉玺印人名考析 & 4.9 & 3.31\\\hline
    先人后事 破局之道 & 4.11 & 3.3\\\hline
\end{tabular}
%讲座预告写在这。用subsection

\subsection{EMBA赋能讲座预告丨先人后事 破局之道}
讲座主题:先人后事 破局知道--仅三生物创业三年完成从零到一
\\讲座要点:高起点专业团队合作 科技与消费相辅相成 B2K2C独特品牌战略 匠心独运轻资产模式
\\主讲人:丁威 南京大学EMBA行业导师\&江苏仅三生物科技有限公司董事长
\\时间:4月11日(周五)18:30
\\地点:商学院安中楼109室\&南京大学EMBA视频号线上直播
\\报名:点击原文
\\详见:\url{https://mp.weixin.qq.com/s/iKrXuKPrm4ltKiCy7TbeoQ}


%此处写校级活动,请不要把讲座、院级活动和社团活动写在这里orz orz orz
\section{南京大学第十二届主持人大赛正式启动}
报名时间:4月3日-4月10日
\\初赛要求:参赛选手请录制参赛视频(需出镜,可自行制作手卡),用3分钟时间讲述属于你的南大故事,主题可以是:校内风景、校史故事、校园活动、社团文化等,内容积极向上,能够展现南大青年自信自强、刚健有为的精神风貌。视频时长3分钟左右,横屏拍摄为宜。
\\报名方式:本次大赛采用自主报名形式,参赛选手需自行下载并填写《参赛选手报名表》,下载链接:\url{https://box.nju.edu.cn/f/b1a7460ffd424e1899b1/?dl=1}
\\提交要求:于2025年4月10日18:00前将《参赛选手报名表》、作品视频、作品文稿打包发至:
\\\url{https://box.nju.edu.cn/u/d/d14b8675ba6846238a30/},压缩包命名为“主持人大赛+学院+姓名”。
\\若在报名中有任何疑问,欢迎加入选手咨询QQ群:377358135
\\详见:\url{https://mp.weixin.qq.com/s/DhaVJHUuq2F2gg0YMIC97w}

\section{2025中荷脑科学与生态学综合科考与科研训练项目选拔报名通知}
1。报名对象:南京大学在读2022、2023、2024级本科生12名左右(其中生命科学学院8名左右,其他院系4名左右);兄弟高校生物学背景(神经科学、生态学优先)本科生4名左右;飞越计划中学生2名。要求具有较好的专业知识、表达能力、外语(口语交流)能力、团队精神及组织能力。
\\2.报名方式:南京大学本科生:2025年4月10日24:00前,以“学号+姓名+荷兰科考”命名,发送报名表(附表2)至朱老师zhuyping@nju.edu.cn。兄弟高校本科生:由各高校遴选后将名单发给朱老师zhuyping@nju.edu.cn。飞越计划中学生:正在选拔中。
\\3. 选拔方式:根据报名表进行第一轮筛选,通过者暂定4月中下旬进行面试选拔(时间、地点另行通知)。
\\4. 项目费用:初步预算总花费约为3.5万元/人。南京大学本科生费用由南京大学本科生院、生命科学学院和学生个人共同承担,其中国际旅费、当地交通、住宿、保险费由中荷脑科学与生态学综合科考与科研训练项目经费以及个人共同承担,伙食费等其他费用则完全由个人自行承担;兄弟高校本科生费用由各高校和学生个人共同承担。飞越计划中学生费用自理。
\\详见:\url{https://jw.nju.edu.cn/7c/06/c26263a752646/page.htm}


\section{摄影大赛丨金陵寻春 “镜”中春影}
大赛主题:金陵寻春 “镜”中春影
\\参赛对象:南京大学全体师生
\\主办方:南京大学外国语学院新媒体中心
\\参赛要求:1. 参赛作品应当紧扣主题,拍摄形式与风格不限,摄影内容不限,摄影器材不限(可以是专业相机、普通相机、DV、手机等)。
\\2. 参赛作品应有创作理念的文字阐释,描绘捕捉到的金陵春日,表达迎接春天的感受。
\\3. 选手可对摄影作品做适当的后期处理,但参赛作品不可完全合成;摄影作品不添加显眼的文字或特殊图案。
\\4. 参赛作品必须真实原创。投稿具有清晰可辨人像的作品需经过被摄制者同意。如需对照片进行编辑处理,使用的字体、图案等素材不涉及版权争议。
\\提交截止时间:2025年4月13日24:00。参与方式、时间流程、评选规则、奖品设置详见推文。
\\详见:\url{https://mp.weixin.qq.com/s/7KEtzqci1bA8921GoW9mew}

\section{智能“职”通车 | 每周实习速递(六)}
1.荣耀新产业孵化部
\\招聘岗位:算法工程师,软件开发工程师等
\\招聘对象:全日制本科及以上在读,可长期实习者优先
\\工作地点:北京,上海,深圳
\\2.追觅科技
\\招聘岗位:总裁助理实习/正式员工招聘;追觅第四孵化器涵盖园林工具/割草机器人/泳池机器人/擦窗机器人等事业部
\\招聘对象:本科生,最好之后能够转正职,4月到岗
\\工作地点:苏州吴中区追觅科技园
\\薪资待遇:11-13k*15薪,六险一金,餐补房补
\\投递链接请点击原文
\\详见:\url{https://mp.weixin.qq.com/s/euZuvpm87f8FUnKY5kqAhw}

\section{学生创业的从0到1——奇绩创业公开宣讲课}
4月11日(周五)14:00,奇绩创坛投资与运营负责人「董科含」 将在 「南京大学」仙林校区择善楼107(I-107)线下分享,带来一堂创业实战课。
\\嘉宾将在4月11日上午10:00于南京大学仙林校区计算机科学技术楼225教室提前举办小型圆桌会议,为有创业想法/或者已经在创业的大学生提供一些解答,帮助其迭代创业idea!
\\报名方式和更多具体信息请见原推。
\\详见:\url{https://mp.weixin.qq.com/s/K1eHyziyG-_GDUS_iA8CsA}


\section{院级活动}
\begin{tabular}{|>{\centering\arraybackslash}m{.3\textwidth}|m{.06\textwidth}|m{.06\textwidth}|}
\hline
    活动 & 开展时间 & 刊载时间\\
    \hline\hline
    文院剧本创作研讨会 & 9.30 & 3.2\\
    物院征集课程指南 & 6.15 & 3.3\\
    地海征集春日影 & 6.15 & 3.14\\
    社院学术节 & 4.18 & 3.25\\
    五院运动会 & 4.13 & 3.31\\
    电子南师春日交流 & 4.12 & 3.31\\
    五院乒乓球赛 & 4.19 & 3.31\\
    建城影展征集 & 4.16 & 3.31\\
    法院党建征文 & 5.20 & 4.2\\
    地学乒赛 & 4.19 & 4.2\\
    匡计社商联谊 & 4.13 & 4.2\\
    
    \hline
\end{tabular}

\section{社团活动}
\begin{tabular}{|>{\centering\arraybackslash}m{.3\textwidth}|m{.06\textwidth}|m{.06\textwidth}|}
    \hline
    社团活动 & 开展时间 & 刊载时间\\
    \hline\hline
    天文台开放日 & / & 1.6\\
    重唱诗歌奖征稿 & 4.30 & 3.31\\
    印社讲座 & 4.9 & 4.1\\
    弓箭社体验 & 4.8 & 4.1\\
    飞镖社体验 & 4.8 & 4.1\\
    排协网协体验 & 4.10 & 4.1\\
    杨协体验 & 4.12 & 4.1\\
    足协体验 & 4.15 & 4.1\\
    轮滑社体验 & 4.17 & 4.1\\
    拳击社体验 & 4.22 & 4.1\\
    轮滑社体验 & 4.22 & 4.1\\
    飞盘大赛 & 4.13 & 4.1\\
    五子棋大赛 & 4.13 & 4.1\\
    定向赛 & 4.20 & 4.1\\
    体育舞蹈教学 & 4.25 & 4.1\\
    吉他社歌手招募 & 4.20 & 4.4\\
    吉他社春日音 & 4.26 & 4.4\\
    国学社寄明信片 & 4.14 & 4.4\\
    \hline
\end{tabular}
%这里是写社团活动的,社团活动就是由社团主办、主要针对社团内部人员的活动。不要把非社团活动写在这里。

\subsection{线上工作坊报名|SDGs、博物馆与优质教育}
活动时间:2025.4.8  19:00-21:00(周二)
\\参与方式:腾讯会议(线上)
\\活动人数:30人
\\年龄限制:青年群体(18-35岁)
\\表单报名截止时间:2025.4.6 22:00
\\有关活动内容和报名方式请参见原文
\\详见:\url{https://mp.weixin.qq.com/s/Jres4Uk526UVFgNBePSb4Q}


\subsection{吉他社“嘿,春”音乐会歌手招募}
报名链接:\url{https://table.nju.edu.cn/dtable/forms/33d4a646-c7d9-4739-98ae-d3d828407780/}
\\报名截止时间:2025.04.20
\\音乐会时间:2025.04.26
\\音乐会地点:仙林校区
\\详见:\url{https://mp.weixin.qq.com/s/VnkGMbW1A5LhWR9-zr-9XA}

\subsection{国学社山区留守儿童明信片寄语公益项目}
活动时间
\\线下:
\\写明信片时间:4.3-4.7
\\摊位收集时间:4.7-4.9  12:00-14:00
\\线上:
\\报名时间: 4月 7 日8:00 到 4月 14日22:00
\\活动时间: 4月 10日18:00到 4月 17日22:00
\\活动流程
\\线下展开写信活动摊位活动、线上展示活动作品
\\线下:
\\同学们自备明信片,并在明信片上写下与自己最有感触的一首诗词与解读故事作为寄语,并在规定时间将明信片送到活动摊位。
\\线上:
\\1.校内学生均可报名
\\2.在到梦空间报名活动后,注意先加活动通知群(QQ群号:455016001)。
\\3.活动开展(4月14日)之前,需将作品上传至爱心展示平台。
\\详见:\url{https://mp.weixin.qq.com/s/o3kKUE1EmTui-lhx1oOXrA}

\end{multicols}
\end{document}
