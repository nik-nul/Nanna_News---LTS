% HEAD BEGIN
\documentclass[letterpaper, 12pt]{article}
\newsavebox\colbbox
\usepackage{graphicx}
\usepackage{multicol}
\usepackage{anysize}
\usepackage{fontspec}
\usepackage[fontset=none]{ctex}
\usepackage{tabularx}
\usepackage{longtable}
\PassOptionsToPackage{hyphens}{url}
\usepackage[breaklinks=true, colorlinks=true]{hyperref}
\expandafter\def\expandafter\UrlBreaks\expandafter{\UrlBreaks\do\a\do\b\do\c\do\d\do\e\do\f\do\g\do\h\do\i\do\j\do\k\do\l\do\m\do\n\do\o\do\p\do\q\do\r\do\s\do\t\do\u\do\v\do\w\do\x\do\y\do\z\do\A\do\B\do\C\do\D\do\E\do\F\do\G\do\H\do\I\do\J\do\K\do\L\do\M\do\N\do\O\do\P\do\Q\do\R\do\S\do\T\do\U\do\V\do\W\do\X\do\Y\do\Z}
% \let\oldurl\url
% \renewcommand{\url}[1]{\begin{sloppypar}\oldurl{#1}\end{sloppypar}}
\setlength\columnsep{30pt}
\marginsize{30pt}{30pt}{10pt}{20pt}
\setmainfont{TeX Gyre Bonum}
\setCJKmainfont[BoldFont=Noto Serif CJK SC Bold, ItalicFont=FandolKai]{Noto Sans CJK SC}
\setlength{\parindent}{0cm}
% \setCJKmonofont{Noto Sans CJK SC}
\begin{document}
\begin{center}
    \Huge\textbf{南哪大专醒前消息}
\end{center}
\vspace{4mm}
\hrule
\renewcommand\tabularxcolumn[1]{m{#1}}
\begin{tabularx}{\textwidth}{>{\hsize.2\hsize}X>{\hsize.6\hsize}X>{\hsize.2\hsize}X}
    \begin{flushleft}
        2025.2.28\, No.178
    \end{flushleft}
    &
    \begin{center}
        \textit{“秉中持正、求新博闻。”}
    \end{center}
    &
    \begin{flushright}
        \textbf{南京市栖霞区}
    \end{flushright}
\end{tabularx}
\vspace{-3.5mm}
\hrule
\vspace{4mm}
% HEAD END
\centerline{\huge\textbf{活动预告}}
\begin{multicols}{2}
    \section{订阅方式和加入编辑部}  
编辑部招聘人才,用爱发电,工作轻松,详情可联系QQ:1329527951 客服小祥\\想订阅本消息或获取PDF版(便于查看超链接和往期),可加QQ群:\href{https://qm.qq.com/q/VXIW7fgsEe}{849644979}.
\section{Deadline Ongoing}
\setbox\colbbox\vbox{
\makeatletter\col@number\@ne
\begin{longtable}{|c|c|c|}
    \hline
    消息(未见ddl的,不刊) & 截止日期 & 刊载日期\\
    \hline\hline
    南大版deepseek & / & 2.22\\
    天文台开放日 & / & 1.6\\
    悦读课程群 & / & 2.24\\
    原创剧本联合孵化报名 & 3.20 & 1.10\\
    njumun代表报名 & 3.2 & 1.16\\
    课程补退选 & 3.2 & 2.19\\
    本科生劳育实践 & 7.20 & 2.19\\
    医保零星报销 & 3.31 & 2.19\\
    第二届大学生阅读分享活动 & 3.7 & 2.21\\
    招办全媒体招新 & 3.5 & 2.20\\
    交响乐团招新 & 3.7 & 2.20\\
    秉文书院早晚自习 & 3.3 & 2.23\\
    “核真录”招新 & 3.2 & 2.24\\
    菁菁南数招募讲师 & 3.9 & 2.24\\
    新火星观影会 & 3.2 & 2.25\\
    车协骑行 & 3.1 & 2.25\\
    天协观测活动 & 3.2 & 2.26\\
    南书房支教队长团招募 & 3.3 & 2.26\\
    雨花斑斓课程 & 3.2 & 2.26\\
    半马志愿者招募 & 3.1 & 2.27\\
    南说喜剧 & 3.1 & 2.27\\
    银星杯论文赛 & 4.22 & 2.27\\
    金陵杯拟法赛 & 3.9 & 2.27\\
    英语角 & 3.1 & 2.28\\
    \hline
\end{longtable}
\unskip
\unpenalty
\unpenalty}\unvbox\colbbox
\end{multicols}
\hrule
\pagebreak
\begin{multicols}{2}

\section{讲座}
\begin{tabular}{|>{\centering\arraybackslash}m{.3\textwidth}|m{.06\textwidth}|m{.06\textwidth}|}
    \hline
    讲座 & 开展时间 & 刊载时间\\
    \hline\hline
    从微观犯罪社会学分析到理解宏观社会空间分异   &3.6  &2.26 \\\hline
    大陆的起源 & 3.4 & 2.17\\\hline
    身体的重写本 & 3.7 & 2.25\\\hline
    当代德语文学与视觉媒介的互动 & 3.3 & 2.27\\\hline
    国际法视域下的供应链安全国内立法 & 3.4 & 2.28\\\hline
    重新评估图神经网络在通用图任务中的潜力边界 & 3.3 & 2.28\\\hline
\end{tabular}
1.国际法视域下的供应链安全国内立法\\
主讲人:沈伟(上海交通大学法学院特聘教授、博士生导师、美国纽约州执业律师)\\
与谈人:彭岳(南京大学法学院教授、博士生导师、南京大学法学院院长)\\
主要内容:以世界贸易组织原则和规则为标准,对一些国家的供应链安全立法进行审视。\\
讲座时间:2025年03月04日(周二)18:30\\
讲座地点:中美中心A106会议室\\
详见:\url{https://mp.weixin.qq.com/s/WPpAd5UegL1N1me1Zzuvlg}\\

2.重新评估图神经网络在通用图任务中的潜力边界\\
报告人:罗元凯 北京航空航天大学、香港理工大学(联合培养)博士生\\
时间:3月3日(周一)9:00\\
地点:计算机科学技术楼230室\\

\section{排超赛程预告}
南京广电猫猫VS广东台山和力\\
时间:3月2日19:30\\
地点:方肇周体育馆\\

\section{NEC预告}
仙林校区\\
时间:本周六(3月1日)晚7:00-8:30\\
地点:仙I-305。\\
鼓楼校区\\
时间:本周天(3月2日)晚7:00-8:30\\
地点:新教-202\\
详情与报名:\url{https://mp.weixin.qq.com/s/HXmpsEqOHMQi8C7Eawh7zg}\\




\end{multicols} 
\hrule
\vspace{4mm}
\centerline{\huge\textbf{参考消息}}
\begin{multicols}{2}
\section{南哪消息同学小文连载板块}
因收到小说投稿一篇,南哪消息现在开辟了同学小文连载板块。如想评论,可以发至邮箱:1329527951@qq.com,第二天会刊在此处。如想投稿渠道相同。
\section{《等待,遗忘》(6)}
金映樺\\

\newCJKfontfamily\fan{FandolFang}\fan

第四天\\

拍立得相片要多久才会褪色?放在阳光下或许只要几小时,放在桌子上或许需要几年,放在盒子里或许需要她的今生今世。那其实也不是很久,她想,就快了。\\

昨天的余淮和第一次见面时有些不同,或许他没有自己想的那么天真单纯,但是他滔滔不绝认真说着自己专业的样子真的很可爱,至少她缺少勇气向他人说出自己的爱好与归宿。余淮,余淮。他提到他曾经住在伦敦,她当时附和说伦敦真是美丽的城市,可惜这是一个好不走心的谎言。她多么希望自己衷心觉得那里美,但是她只能想起在伦敦时雾蒙蒙的天空与淅淅沥沥的小雨,以及,以及她已经很久没想起的那个人,她永远年轻的哥哥。\\

她想起小时候快过年时总会去寺庙,庙中没什么人上香,或许是在为初一的大拜做准备。哥哥带着她在在万佛堂叩拜,她好像听到了梵音,很寂静的声音,像突然下起雨。她想告诉哥哥这种声音,小小的她却无法掌握和词语的游戏,杂乱的话语让哥哥轻轻笑了起来,牵着她的手去写春联。她那时往往被打扮得像年画中的小人儿,暗红斜盘口,掐丝珐琅镯,拎着刚写好的福字,围着院子打转。她跟在哥哥后面,想要哥哥停下来。\\

后来哥哥去了伦敦念书,每次回来都会给她带一盒大大的巧克力。巧克力是动物形状,她总觉得盒子是囚笼,于是将巧克力动物都倒出来,要给予它们自由。母亲看到了总会叹气,却也不指责她,只是看着,但她知道是时候将伙伴们重新关进笼子。有一次哥哥看到了,走过来摸摸她的头,露出一个并不真心的笑容,“笼子最早也不是动物们的家。”她认同地点点头,想了想和哥哥分着吃掉了巧克力动物,甜甜的,这是大家最好的归宿。\\

哥哥总是被父亲指责,天赋不足、努力不够、还能做得更好吧?小小的幸福与大大的隐忍,终于有一天哥哥不再回来,是再也无法回来。哥哥的照片前花团锦簇,父亲母亲低声哭着,她不明白,只是听到形色匆匆的人们悄悄议论,看到脖子上的勒痕了吗?她突然明白,再无期待也就再无忧愁。
她长大后读了很多书,看到各种志怪小说中缢鬼的故事,偶尔会想起哥哥,他是不是也成了这样的鬼?父母当时买下了哥哥在伦敦租住的公寓,她在伦敦时也住在那,却没有像书中那些迂腐书生一样逢魔遇仙。她想知道人死后会去哪,上吊的人是不是果真会徘徊于阴阳之间?她无论如何也想不出答案。\\

发了一会呆,又想起余淮说的那家店,可惜自己没有这样的缘分遇见,也就不再有机会了。她突然想起,或许可以问问余淮。这两天他们偶尔聊聊宗教,偶尔聊聊生活,悄然无息间已经成了朋友了。\\

“人死后会去哪?”\\

余淮果然在线,看着对话框上的“Typing...”,她抿起了嘴角。\\

“前阴已谢,后阴未至,中阴现前。或许是指中阴身?”余淮似乎看出了她的不解,进一步解释道,“死亡被视为一个过渡阶段,称为‘中阴身’。”\\

她想起有部影片曾经说,自杀的人不能上天堂,“自杀的话也能经历这个过渡阶段吗?”\\

“可以是可以...但是自杀的人在此期间,貌似每七天就会重新体验一次自杀的恐惧?”余淮好像也不是很确定,“怎么啦?/转圈/”\\

看到余淮常发的这个企鹅转圈的emoji,她突然觉得一切也没什么大不了的,天堂、地狱、人间,在哪里或许都大差不离。或许还有转机,或许也没有。\\

“没事,随便问问\^\,\^”\\

她放下手机,闭上了眼睛。然而一阵铃声响起,是母亲。\\

“...我自己会安排...”她无力地回答。\\

“你不要干涉...”她无望地拒绝。\\

“我知道了。”她苦涩地答应。\\

母亲又一次自作主张定下了她和余淮见面的时间,她并不反感见到余淮,甚至有些期待,但这种命运被安排而自己无力地站在一旁的窒息感让她濒临崩溃,一切都不会更好了,所以也不会更差。哥哥当初是这么想的吗?她不知道。她只知道果然没有转机出现,神缄默不语,似乎也没有多看她一眼,就像当初抛弃哥哥那样。\\

她瘫倒在沙发上,再也读不进任何书了。\\

\end{multicols} 

\end{document}