% HEAD BEGIN
\documentclass[letterpaper, 12pt]{article}
\newsavebox\colbbox
\usepackage{graphicx}
\usepackage{multicol}
\usepackage{anysize}
\usepackage{fontspec}
\usepackage[fontset=none]{ctex}
\usepackage{tabularx}
\usepackage{longtable}
\PassOptionsToPackage{hyphens}{url}
\usepackage[breaklinks=true, colorlinks=true]{hyperref}
\expandafter\def\expandafter\UrlBreaks\expandafter{\UrlBreaks\do\a\do\b\do\c\do\d\do\e\do\f\do\g\do\h\do\i\do\j\do\k\do\l\do\m\do\n\do\o\do\p\do\q\do\r\do\s\do\t\do\u\do\v\do\w\do\x\do\y\do\z\do\A\do\B\do\C\do\D\do\E\do\F\do\G\do\H\do\I\do\J\do\K\do\L\do\M\do\N\do\O\do\P\do\Q\do\R\do\S\do\T\do\U\do\V\do\W\do\X\do\Y\do\Z}
% \let\oldurl\url
% \renewcommand{\url}[1]{\begin{sloppypar}\oldurl{#1}\end{sloppypar}}
\setlength\columnsep{30pt}
\marginsize{30pt}{30pt}{10pt}{20pt}
\setmainfont{TeX Gyre Bonum}
\setCJKmainfont[BoldFont=Noto Serif CJK SC Bold, ItalicFont=FandolKai]{Noto Sans CJK SC}
\setlength{\parindent}{0cm}
% \setCJKmonofont{Noto Sans CJK SC}
\begin{document}
\begin{center}
    \Huge\textbf{南哪大专醒前消息}
\end{center}
\vspace{4mm}
\hrule
\renewcommand\tabularxcolumn[1]{m{#1}}
\begin{tabularx}{\textwidth}{>{\hsize.2\hsize}X>{\hsize.6\hsize}X>{\hsize.2\hsize}X}
    \begin{flushleft}
        2024.12.19\, No.149
    \end{flushleft}
    &
    \begin{center}
        \textit{“秉中持正、求新博闻。”}
    \end{center}
    &
    \begin{flushright}
        \textbf{南京市栖霞区}
    \end{flushright}
\end{tabularx}
\vspace{-3.5mm}
\hrule
\vspace{4mm}
% HEAD END
\centerline{\huge\textbf{活动预告}}
\begin{multicols}{2}
    \section{订阅方式和加入编辑部}  
编辑部招聘人才,用爱发电,工作轻松,详情可联系QQ:1329527951 客服小祥\\想订阅本消息或获取PDF版(便于查看超链接和往期),可加QQ群:\href{https://qm.qq.com/q/VXIW7fgsEe}{849644979}.
\section{Deadline Ongoing}
\setbox\colbbox\vbox{
\makeatletter\col@number\@ne
\begin{longtable}{|c|c|c|}
    \hline
    消息(未见ddl的,不刊) & 截止日期 & 刊载日期\\
    \hline\hline
    安邦征稿 & 1.12 & 11.16\\
    创意物理实验竞赛 & 12.21 & 11.15\\
    仙林通宵自习室 & 1.12 & 11.26\\
    全国大学生家史大赛 & 1.31 & 12.2\\
    金融消费者大赛 & 12.31 & 12.5\\
    花旗杯报名 & 1.3 & 12.6\\
    西安史学论坛征稿 & 3.20 & 12.9\\
    重唱英文评诗会 & 12.21 & 12.10\\
    叶嘉莹纪念征稿 & 12.25 & 12.10\\
    粤语课堂 & 12.22 & 12.11\\
    五院迎新晚会 & 12.21 & 12.12\\
    普通话测试报名 & 12.24 & 12.12\\
    本科评教 & 1.12 & 12.13\\
    排协雪球杯 & 12.28 & 12.13\\
    心协暖冬歌会 & 12.21 & 12.13\\
    12306学生优惠票 & 2.12 & 12.13\\
    校园迷你马拉松报名 & 12.20 & 12.14\\
    新传迎新晚会 & 12.21 & 12.15\\
    外语社团联谊活动 & 12.21 & 12.15\\
    歌魅音乐会 & 12.22 & 12.15\\
    希音编程竞赛 & 12.21 & 12.15\\
    药石杯生化歌赛 & 12.22 & 12.15\\
    交响乐团室内乐 & 12.20 & 12.15\\
    flicker影映 & 12.21 & 12.16\\
    期末考试安排 & 1.12 & 12.17\\
    朝天宫民族团结研学 & 12.20 & 12.17\\
    南大迷你马拉松报名 & 12.20 & 12.17\\
    南大博物馆展览 & 6.16 & 12.17\\
    哲史数院包饺子 & 12.21 & 12.17\\
    地海包饺子 & 12.20 & 12.17\\
    计院包饺子 & 12.21 & 12.17\\
    马院包饺子 & 12.20 & 12.18\\
    软院包饺子 & 12.20 & 12.18\\
    商院包饺子 & 12.21 & 12.18\\
    文院包饺子 & 12.23 & 12.19\\
    茶话日和交流会 & 12.21 & 12.18\\
    期末自立营 & 12.23 & 12.18\\
    海岛放映 & 12.21 & 12.18\\
    辅导员满意度测评 & 12.24 & 12.19\\
    “悟理镜界”摄影投票 & 12.22 & 12.19\\
    南悦支教团队长报名 & 12.29 & 12.19\\
    建设兵团文化展 & 12.20 & 12.19\\
    博物馆志愿者招募 & 12.23 & 12.19\\
    
    \hline
\end{longtable}
\unskip
\unpenalty
\unpenalty}\unvbox\colbbox
\end{multicols}
\hrule
\pagebreak
\begin{multicols}{2}

\section{讲座}
\begin{tabular}{|>{\centering\arraybackslash}m{.3\textwidth}|m{.06\textwidth}|m{.06\textwidth}|}
    \hline
    讲座 & 开展时间 & 刊载时间\\
    \hline\hline
《杨万里对苏轼诗...》 & 12.27 & 12.14\\\hline
物院学术交流会 & 12.21 & 12.16\\\hline
大城市基层治理的特点与趋势 & 12.20 & 12.17\\\hline
犯罪变化与空间模式 & 12.20 & 12.18\\\hline
社会科学中的理论化工作 & 12.20 & 12.18\\\hline
如果我重写博士论文 & 12.20 & 12.18\\\hline
Active Learning of General Halfspaces & 12.21 & 12.18\\\hline
群上同调和Hlibert90 & 12.20 & 12.19\\\hline
贸易摩擦和创新驱动战略 & 12.20 & 12.19\\\hline
东亚视域下的卡夫卡 & 12.21 & 12.19\\\hline
\end{tabular}

1.数学学院本科生论坛(学生系列第53讲)\\
题目:群上同调和Hilbert 90\\
报告人:杜俊哲(23级)\\
时间:12月20日(星期五) 16:00-17:30\\
地点:仙林校区 仙I-303\\
腾讯会议:746-494-010\\
讲座简介见原文\url{https://mp.weixin.qq.com/s/ahk9dopXTTSwuRqbidB6Fg}\\

2.“满天星”学术讲堂

活动主题:贸易摩擦和创新驱动战略\\
活动时间:12月20日(周五)18:30-20:00\\
活动地点:仙林校区地理与海洋科学学院一楼报告厅\\
活动嘉宾:陈朴 中国人民大学经济学院国际经济系副教授,美国明尼苏达大学双城分校经济学博士,专注于宏观经济学研究。\\

3.高研院“东亚视域下的卡夫卡”线上工作坊

时间:2024年12月21日(周六)18:00-20:20\\
地点:线上Zoom会议\\
备注:工作语言为德语\\
具体链接:\url{https://mp.weixin.qq.com/s/XI0ybEDeF-_lO1RvmgIgiw}\\

\section{文院包饺子}
活动时间:12月23日 17:00-19:00

活动地点:仙林校区教工第一食堂

活动名额:50人左右

扫描二维码\url{https://mp.weixin.qq.com/s/i2Lr3JqhPbIH_hqclrSXOw}填写报名问卷,加入活动群。

报名截止时间:2024年12月20日中午12点
\section{开甲书院“平凡人的不凡路”长期征稿}
要求:文体不限,诗歌体裁字数不限,其他体裁字数在800-1000字左右。\\
如果您不想透露真实姓名,可以选择使用笔名。同时,您也可以自行决定是否提供书院或学院的名称。若需添加图片,请确保上传的是原图。稿件如被采用,我们会提前与您联系,并发送推文预览供您查看。\\
详见\url{https://mp.weixin.qq.com/s/sR50TYYXNeYfE68x1eaw6w}(信息小编:草昌思)

\section{辅导员工作满意度测评}
南大育教发布2024年度辅导员工作满意度测评工作,请同学们积极填写\\
本科生辅导员满意度测评:\url{https://wenjuan.nju.edu.cn/vm/mOwDSHZ.aspx}\\
研究生辅导员满意度测评:\url{https://wenjuan.nju.edu.cn/vm/PYXJoEv.aspx}\\
截止时间:12月24日24:00\\
详情:\url{https://mp.weixin.qq.com/s/494N6FJc78IJ9ixxzN3whw}\\



\section{第一届“悟理镜界”摄影展投票通道正式开启}
投票截止时间为12月22日晚十二点。获奖作品由微信公众号投票+评委投票产生(权重为1:5)。\\
第一期摄影展\url{https://mp.weixin.qq.com/s/9oyvSl5Ri_occhTlMNMG9w}\\
投票通道\&第二期摄影展\url{https://mp.weixin.qq.com/s/Kl3T_J-yfQ5jW4toWobpJA}\\

\section{一分钱暖心早餐套餐}
在寒风凛冽的冬日,在复习迎考的紧张时期,为了给专心备考的NJUers送上最真挚的关怀和鼓励,后勤服务集团将面向本校全体全日制学生倾情提供0.01元的暖心早餐套餐,助力大家跨越所有挑战!\\
活动时间:2024.12.20—2025.1.12每日早晨6:45—7:15(ps:以刷卡时间为准)\\\\
供应地点:学生第一食堂、学生第二食堂、学生第四食堂、学生第五食堂、学生第十食堂、学生第十一食堂、教工第二餐厅、学生第十五食堂。\\\\
具体供应窗口、供应份数、供应套餐详见\url{https://mp.weixin.qq.com/s/gFSg9BjQSwLYCRNd75-yVA}\\
温馨提示:请同学们带好校园卡,每人每天限领一份,不可代买代领。
\section{南悦乡村支教团队长训练营报名通知}
nju南悦之约 发布\\
时间:2024年12月-2025年2月底(暂定)(线上线下结合)\\
名额:约30人(视报名人数可上下浮动)\\
根据合格队长所具备的各项能力设置不同的主题式活动,如策划撰写、课程设计、情境模拟、项目申请、团队交流、小组合作等。\\
报名要求:\\
1、南京大学全日制在读本科生、研究生,院系专业年级不限。\\
2、有意向并有时间(2025年7月-8月)担任2025年暑期南悦乡村支教项目队长团成员(总队长、分队长等)\\
3、认真负责,有良好的学习能力,能完成训练营安排的各项活动。\\
报名截止日期:2024年12月29日24:00,面试将于寒假进行。\\
其他详细信息与报名流程详见\url{https://mp.weixin.qq.com/s/zipSbAEYXBar9QRQeMQWgg}\\

\section{新疆生产建设兵团第四师军垦文化展}
南京大学博物馆发布\\
开幕仪式时间:2024.12.20(明日)上午10点\\
展览内容:\\
第一单元 新疆生产建设兵团 \\
第二单元 伊犁古代——近代屯垦史 \\
第三单元 新疆兵团第四师革命战斗史 \\
第四单元 新疆兵团第四师屯垦戍边建城史 \\
详见\url{https://mp.weixin.qq.com/s/DaRODt9R-blNy_Cz36NF9A}
\section{2024年12月南京大学博物馆志愿者招募}
博物馆现有常设展1场,临时展2场,为更好地服务校内师生及社会公众,现分别针对3场展览面向校内招募志愿者,报名成功后将安排集中培训。\\
志愿服务类型:展览讲解、馆内其他业务\\
招募时间:即日起至2024年12月23日上午8点\\
集中培训时间:12月23日上午10点\\\\
招募要求:\\
1.南京大学在校生(校区、专业不限,文史、外语专业优先)。\\
2.具备可支配的充裕时间,保证按时出勤,接受本馆志愿培训,考核通过后正式上岗。\\
3.保证出勤时间,如遇特殊情况不能按时到岗,应提前说明。\\
4.本馆提供的任何资料仅限于志愿者工作时使用,严禁外传。\\\\
志愿者权益:\\
1.优先参与本馆各类展览策划及公众教育活动。\\
2.参加(不)定期举办的志愿者业务培训。\\
3.志愿服务计入劳育时长和志愿时长。\\
4.优先获取最新文创。\\
5.参与南京大学博物馆“年度优秀志愿者”评选,获奖者颁发证书、奖品。\\
6.开具南京大学博物馆实习证明。\\
7.免费提供工作期间的午餐。\\\\
申报方式详见\url{https://mp.weixin.qq.com/s/_TsIyCXeumtFqCEoWqwFcw}
\section{暖冬歌会预告|节目单介绍及入场须知}
南京大学学生心理协会 发布\\
时间:12月21日晚上18:30\\
地点:鼓楼大礼堂\\
主要节目:暖心故事、音乐表演、抽奖和神秘彩蛋等。现场发放歌词本,支持观众共唱\\
入场时间:18:00-18:20有门票的观众检票进场,18:20之后免票进场\\
另有线上活动直播间\\
其他详细入场须知、节目单、直播间等信息,详见\url{https://mp.weixin.qq.com/s/n0Ex2fR19M3GZTRDebZlrw}\\
\end{multicols} 

\end{document}