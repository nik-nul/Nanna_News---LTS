% HEAD BEGIN
\documentclass[letterpaper, 12pt]{article}
\newsavebox\colbbox
\usepackage{graphicx}
\usepackage{multicol}
\usepackage{anysize}
\usepackage{fontspec}
\usepackage[fontset=none]{ctex}
\usepackage{tabularx}
\usepackage{longtable}
\PassOptionsToPackage{hyphens}{url}
\usepackage[breaklinks=true, colorlinks=true]{hyperref}
\expandafter\def\expandafter\UrlBreaks\expandafter{\UrlBreaks\do\a\do\b\do\c\do\d\do\e\do\f\do\g\do\h\do\i\do\j\do\k\do\l\do\m\do\n\do\o\do\p\do\q\do\r\do\s\do\t\do\u\do\v\do\w\do\x\do\y\do\z\do\A\do\B\do\C\do\D\do\E\do\F\do\G\do\H\do\I\do\J\do\K\do\L\do\M\do\N\do\O\do\P\do\Q\do\R\do\S\do\T\do\U\do\V\do\W\do\X\do\Y\do\Z}
% \let\oldurl\url
% \renewcommand{\url}[1]{\begin{sloppypar}\oldurl{#1}\end{sloppypar}}
\setlength\columnsep{30pt}
\marginsize{30pt}{30pt}{10pt}{20pt}
\setmainfont{TeX Gyre Bonum}
\setCJKmainfont[BoldFont=Noto Serif CJK SC Bold, ItalicFont=FandolKai]{Noto Sans CJK SC}
\setlength{\parindent}{0cm}
% \setCJKmonofont{Noto Sans CJK SC}
\begin{document}
\begin{center}
    \Huge\textbf{南哪大专醒前消息}
\end{center}
\vspace{4mm}
\hrule
\renewcommand\tabularxcolumn[1]{m{#1}}
\begin{tabularx}{\textwidth}{>{\hsize.2\hsize}X>{\hsize.6\hsize}X>{\hsize.2\hsize}X}
    \begin{flushleft}
        2024.11.5\, No.109
    \end{flushleft}
    &
    \begin{center}
        \textit{“克明峻德。”}
    \end{center}
    &
    \begin{flushright}
        \textbf{南京市栖霞区}
    \end{flushright}
\end{tabularx}
\vspace{-3.5mm}
\hrule
\vspace{4mm}
% HEAD END
\centerline{\huge\textbf{活动预告}}
\begin{multicols}{2}
    \section{订阅方式和加入编辑部}  
编辑部招聘人才,用爱发电,工作轻松,详情可联系QQ:1329527951 客服小祥\\想订阅本消息或获取PDF版(便于查看超链接和往期),可加QQ群:\href{https://qm.qq.com/q/VXIW7fgsEe}{849644979}.
\section{Deadline Ongoing}
\setbox\colbbox\vbox{
\makeatletter\col@number\@ne
\begin{longtable}{|c|c|c|}
    \hline
    消息(未见ddl的,不刊) & 截止日期 & 刊载日期\\
    \hline\hline
    紫藤学刊征稿 & 12.15 & 10.22\\
    校运会 & 11.8 & 10.21\\
    后革命鲁迅研究征文 & 11.10 & 10.8\\
    大创训练计划申报 & 11.18 & 9.24\\
    招生宣传创意征集大赛 & 11.18 & 10.21\\ 
    EBSCO数据库检索大赛 & 11.20 & 10.3\\
    文院征稿 & 11.20 & 10.20\\
    乐跑 & 12.6 & 10.12\\
    国际访学计划申报 & 11.22 & 10.22\\
    普通话测试网络报名 & 11.12 & 10.29\\
    南大演说家 & 11.9 & 10.30\\
    南大演说家报名 & 11.9 & 10.30\\
    读书午餐会报名 & 11.6 & 11.1\\
    南大会征募会设 & 11.15 & 11.1\\
    心协十一月征稿 & 11.10 & 11.2\\
    秉文心理短视频 & 11.25 & 11.3\\
    扭泵音乐节 & 11.8 & 11.3\\
    法学主题参会 & 11.11 & 11.4\\
    《英国小史》分享会报名 & 11.6 & 11.4\\
    青鸟分享会报名 & 11.6 & 11.4\\
    心协香囊活动 & 11.10 & 11.4\\
    BRAVO草地音乐节 & 11.9 & 11.4\\
    高校联合徒步报名 & 11.10 & 11.5\\
    天文台车赛报名 & 11.12 & 11.5\\
    南大模联校内会报名 & 11.11 & 11.5\\
    里斯本丸沉没映后谈 & 11.6 & 11.5\\
    \hline
\end{longtable}
\unskip
\unpenalty
\unpenalty}\unvbox\colbbox
\end{multicols}
\hrule
\pagebreak
\begin{multicols}{2}

\section{讲座}
\begin{tabular}{|c|c|c|}
    \hline
    往期讲座 & 开展日期 & 刊载日期\\
    \hline\hline
    《电池及电化学能...》 & 11.24 & 10.3\\
    《专利查新与规避...》 & 12.19 & 10.3\\
    图书馆系列讲座 & 12.3 & 10.20\\
    《瑞典电力和氢能...》 & 11.7 & 10.29\\
    《组织动员如何影...》 & 11.6 & 10.30\\
    《信息与现代信息...》 & 11.6 & 10.31\\
    《卢卡奇1919与19...》 & 11.8 & 11.2\\
    《青年卢卡奇论马...》 & 11.8 & 11.2\\
    《比较文学是新人...》 & 11.6 & 11.3\\
    《比较文化研究与...》 & 11.8 & 11.3\\
    《数字化时代下如...》 & 11.6 & 11.4\\
    《中国历代龙形象...》 & 11.7 & 11.4\\
    《打开人文社科研...》 & 11.7 & 11.4\\
    《电影内外的里斯...》 & 11.6 & 11.4\\
    《卢卡奇遗产中的...》 & 11.10 & 11.4\\
    《开启中国式现代...》 & 11.6 & 11.5\\
    《Predictive M...》 & 11.7 & 11.5\\
    《史料场与问题域...》 & 11.8 & 11.5\\
    《一位晚清驻防旗...》& 11.7 & 11.5\\
    《从微观数据到宏...》& 11.11 & 11.5\\
    \hline
\end{tabular}

1.开启中国式现代化的共同富裕之路\\
主讲人:王岩(教育部长江学者特聘教授)\\
时间:11月6日16:00\\
地点:仙林校区恩玲剧场\\

2.一位晚清驻防旗人的生命经验——凤全与川边改制\\
主讲人:李文杰,华东师范大学历史学系教授\\
主持人:梁晨,南京大学历史学院教授\\
主题:一位晚清驻防旗人的生命经验——凤全与川边改制\\
时间:11月7日(周四)下午两点到四点\\
地点:仙林教学楼I-408\\

3.刘亚娟:史料场与问题域:以当代中国调干生研究为例\\
时间:2024年11月8日(周五) 10:00\\
地点:南京大学仙林校区 仙Ⅰ-317\\
主讲人:刘亚娟\\

4.梁晨:从微观数据到宏观历史——量化数据库与史学研究创新\\
时间:2024年11月11日,下午14:30-16:00\\
地点:教二304教室(线上线下结合,腾讯会议号:774-271-023)\\
主讲人:梁晨教授 南京大学历史学院\\
主持人:孙邦华教授 北京师范大学教育学部\\

5.【量化历史讲座】Predictive Modelling the Past\\
Predictive Modelling the Past: A New Machine Learning Method Applied to Seven Centuries of Wages\\
主讲人:Meredith Paker, Assistant Professor, Department of Economics, Grinnell College\\
讨论人:Yuqi Chen, PhD candidate, Department of History, Peking University (will join the Centre of Quantitative History as a Postdoctoral Fellow in December 2024)\\
时    间:2024年11月07日 10:00 - 11:30 (北京时间,星期四)\\
讲座语言:英文\\

\section{高校联合徒步活动}
侵华日军南京大屠杀遇难同胞纪念馆举办第八届 “感恩·南京安全区”国际和平徒步活动,特邀南京大学、南京航空航天大学、河海大学、南京师范大学、南京财经大学、南京审计大学、南京艺术学院、南京晓庄学院、金陵科技学院、南京传媒学院10所在宁高校,每校10名学生共同参与现场徒步。
时间:11月16日\\
详见:\url{https://mp.weixin.qq.com/s/U81zeFA7ExJUA1q1IAOg8g}\\
报名截止时间:11月10日12:00

\section{“非遗进校园”活动亮相南苏}
南京大学苏州校区将迎来一场独具魅力的“非遗进校园”活动。此次活动将分为五个精彩环节,分别为:评弹双人档教学体验、非遗苏式盘扣制作、苏式檀香扇制作、非遗缂丝体验以及非遗船点制作。\\
活动时间:2024年11月-12月初每周周末\\
单次时长:2小时\\
活动对象:南京大学苏州校区全体师生\\
活动报名方式及详情见原文:\url{https://mp.weixin.qq.com/s/_qrsw9CMbMnHjvl07M9kaQ}
\section{短期课程预告|认知与情感}
哲学学院“刘伯明讲座教授”Dorothea Olkowski将于本月在仙林校区开设《认知与情感:哲学、精神分析与神经科学》短期课程(英文),首节课程将于11月6日开课,本科生可直接登录选课平台选课,课程安排、群聊、助教联系方式等详阅\url{https://mp.weixin.qq.com/s/gauWvZ3QpUfi2Vj56QRkJQ}。

\section{明日赛程}
1.排协院系杯
时间:11月6日12:30\\
地点:方肇周副馆\\
女排小组赛:化学-电子\\
2.男篮院系杯小组赛\\
外院vs地科\\
时间:12:30-14:00\\
地点:一组团篮球场\\

\section{第四届南京大学天文台个人计时赛}
比赛时间:11月16日\\
比赛地点:仙林校区天文台\\
参赛人员:南大师生及校友\\
报名截止:11月12日24时\\
报名方式:详见推文,扫码加入参赛群,并填写报名问卷,名额有限先到先得。\\
\url{https://mp.weixin.qq.com/s/nFBy71FrnW2GfpiBFoLMhQ}

\section{2025年“南星梦想计划”正式启动}
进入“南京大学”小程序,首页点击“南星梦想计划”即可开始报名,正确填写相关信息后即进入相应中学团队。成功报名后,请按照小程序提示加入所在地区的“南星梦想计划”交流群,并及时与所在中学团队的成员取得联系,共同组队、筹备活动。报名截止日期为11月22日24:00。\\
详见\url{https://mp.weixin.qq.com/s/s7Ts_S3NBS_DiuEptqlSLA}

\section{NUBORN音乐节}
ROCK REPUBLIC南大摇滚联盟 发布\\
时间:11月8日18:30\\
地点:炜华体育场\\
歌单、乐队等详细信息请查看\url{https://mp.weixin.qq.com/s/KWSznt5PWWvQA5eXRjxFzA}\\

\section{2024年南京高校模拟联合国大会}
南大模联 发布\\
时间:12月7日-8日\\
地点:具体地点后续发布\\
会场1:国际移民组织理事会\\
议题:保障移徙工人的权利\\
工作语言:中文\\
会场2:United Nations Security Council\\
议题:Conflict-related Sexual Violence\\
工作语言:English\\
报名及其他详细情况请查看\url{https://mp.weixin.qq.com/s/QRgO3TSuWrqyAQBC3b087g}\\
\section{《里斯本丸沉没》映后谈}
11月6日,南京大学新生学院将特别邀请《里斯本丸沉没》导演方励开展映后谈,南京大学文学院副教授杨柳开展对谈。\\
观影信息\\
时间:2024年11月6日(星期三)11:30\\
地点:荔枝广场幸福蓝海影院(集中观影)\\
讲座信息\\
时间:2024年11月6日(星期三)16:10\\
地点:南京大学鼓楼校区田家炳多功能报告厅\\
有意愿参加观影和对谈的同学填写链接:\\
\url{https://table.nju.edu.cn/dtable/forms/a20313ab-3445-4005-8aa4-5e2f613e39f7/}\\
(参加观影的同学请留意后续短信通知,并请参加观影的同学尽量参加对谈,若已自主观影过的同学,欢迎报名下午对谈~)\\
电影内容等有关详情可见原文\url{https://mp.weixin.qq.com/s/hOXFIPhsvTbfqkR7B8bnIA}\\

\section{青鸟分享会}
分享主题:是制作人/舞监,不是打杂,谢谢\\
时间:11月10日,15:00~17:00\\
地点:文学院441\\
分享人:薛天乐、郑惟知\\
其他详细情况请查看\url{https://mp.weixin.qq.com/s/1IwSYOLNwMzge3X76Myv_A}|\\

\section{物院2024辩论赛预报名}
南雍物理 发布\\
活动时间:2024年11月16日—17日\\
活动对象:物理学院全体本科生研究生\\
活动地点:南京大学鼓楼校区\\
其他详细情况请查看\url{https://mp.weixin.qq.com/s/vpJ_bDDmFAb6IzWtd1dd1Q}\\
\end{multicols} 
\hrule
\vspace{4mm}
\centerline{\huge\textbf{参考消息}}
\begin{multicols}{2}
\section{南京大学校园大盗逍遥法外}
当事人在表白墙发布,现转载如下:

“本人于10月31日上午10时在逸B候车厅(监控盲区)遗留了一只价值3000元的乒乓球拍、一台笔记本电脑和一本书。下午3时返回时,惊愕地发现乒乓球拍已不翼而飞,而电脑和书仍安然无恙。此后,我迅速在校园报警中心备案、在仙林派出所做了笔录、并向校办老师反映了这一情况。

然而,令人失望的是,至今已经过去5天,所有相关部门对此案件的处理态度极为消极,没有任何新的进展,仿佛在默许盗窃行为的继续。更加令人震惊的是,11月1日,也就是我报案的第二天,这对盗窃者再次潜入校园逗留7小时(上午10时至下午5时),继续他们的盗窃行径,而校保安却毫无作为,任由校园大盗逍遥法外!

经过长时间的监控调查,我们发现盗窃者为两名50余岁的校外夫妻。他们自10月10日起几乎每天都在校园内进行偷窃,携带两个不透明的布袋,直到装满才离开。有时他们刷访客码进入,有时则尾随他人。这对夫妻极为狡猾,对我校的监控系统了如指掌,几乎只在室外活动,室内有监控的地方则很少出现。他们专挑自认为价值较小的物品下手,对电子产品则避而远之,显然是为了避免引起警方的注意。

据估计,这对夫妻多次盗窃的总价值已达数万元,民警指出,这已经构成了刑事犯罪,可能会面临“三年以上十年以下”的有期徒刑。\\

**发布此帖的目的有二:**

1. 提醒大家近期切勿在“逸B候车厅、网球场、游泳馆、食堂(这些是他们经常出没的地方)”遗留贵重物品。学校并非我们想象中的安全堡垒,你身边走过的陌生人可能就是潜藏的校外盗窃犯。

2. 校方对单个学生的案件重视程度不够,配合度极低。但此事绝不仅仅涉及我一个人,肯定还有多名同学的财产在10月遭到失窃。若近期有同学在监控盲区丢失过小额物品(几十至几千元,我估计他们不敢偷更多),请加入以下QQ群:275633809。\\

让我们携手努力,促使学校更加重视我们的财产安全及人身安全,早日将真凶绳之以法!我们的每一份努力,都将为校园带来一份平安!”\\

我是今日表白墙校园盗窃案的受害者本人。怀着激动的心情,我来向大家报告盗窃案的最新进展!在大家坚持不懈的关注和支持下,我们的集体行动迅速引起了校方的重视。发帖仅仅两小时,我就接到了三个紧急电话--来自校办老师、栖霞网管局和校园保卫处。在他们的迅速反应下,我在11点30分左右被告知我的球拍可能已经找到,可以去报警中心确认。经过确认,那确实是我丢失的那把球拍。但当我追问如何捉到真凶时,保安开始支支吾吾,显得非常心虚。他们最初试图搪塞我说是有人不小心拿错了,但在我坚持追问下,保卫处副处长终于承认,球拍确实是被那对涉嫌盗窃的夫妻偷的,只是他们至今未能将人抓获。

这里有两个疑点让我们不得不质疑:其
一,为什么一个案子拖了5天都无法破
解,而我一发帖子不到两小时就找到
了球拍?这表明案件并不复杂,而是他
们对“学生财产安全”的漠视。其二,既
然保卫处已经通过访客码查到了那对
夫妻的联系方式,并且他们还再次来
到学校归还了球拍,为什么我们校园
的保安团队和监控系统还未能将他们
绳之以法?

我将继续跟进此事,直到那对夫妻被
缉拿归案,正义得到伸张。对于时间后
续进展感兴趣的同学,欢迎加入我们
的行动群:
275633809。让我们共同关注,直到这
起案件得到妥善解决。”
\end{multicols} 

\end{document}