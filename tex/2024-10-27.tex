% HEAD BEGIN
\documentclass[letterpaper, 12pt]{article}
\newsavebox\colbbox
\usepackage{graphicx}
\usepackage{multicol}
\usepackage{anysize}
\usepackage{fontspec}
\usepackage[fontset=none]{ctex}
\usepackage{tabularx}
\usepackage{longtable}
\PassOptionsToPackage{hyphens}{url}
\usepackage[breaklinks=true, colorlinks=true]{hyperref}
\expandafter\def\expandafter\UrlBreaks\expandafter{\UrlBreaks\do\a\do\b\do\c\do\d\do\e\do\f\do\g\do\h\do\i\do\j\do\k\do\l\do\m\do\n\do\o\do\p\do\q\do\r\do\s\do\t\do\u\do\v\do\w\do\x\do\y\do\z\do\A\do\B\do\C\do\D\do\E\do\F\do\G\do\H\do\I\do\J\do\K\do\L\do\M\do\N\do\O\do\P\do\Q\do\R\do\S\do\T\do\U\do\V\do\W\do\X\do\Y\do\Z}
% \let\oldurl\url
% \renewcommand{\url}[1]{\begin{sloppypar}\oldurl{#1}\end{sloppypar}}
\setlength\columnsep{30pt}
\marginsize{30pt}{30pt}{10pt}{20pt}
\setmainfont{TeX Gyre Bonum}
\setCJKmainfont[BoldFont=Noto Serif CJK SC Bold, ItalicFont=FandolKai]{Noto Sans CJK SC}
\setlength{\parindent}{0cm}
% \setCJKmonofont{Noto Sans CJK SC}
\begin{document}
\begin{center}
    \Huge\textbf{南哪大专醒前消息}
\end{center}
\vspace{4mm}
\hrule
\renewcommand\tabularxcolumn[1]{m{#1}}
\begin{tabularx}{\textwidth}{>{\hsize.2\hsize}X>{\hsize.6\hsize}X>{\hsize.2\hsize}X}
    \begin{flushleft}
        2024.10.27\, No.100
    \end{flushleft}
    &
    \begin{center}
        \textit{“克明峻德。”}
    \end{center}
    &
    \begin{flushright}
        \textbf{南京市栖霞区}
    \end{flushright}
\end{tabularx}
\vspace{-3.5mm}
\hrule
\vspace{4mm}
% HEAD END
\centerline{\huge\textbf{活动预告}}
\begin{multicols}{2}
    \section{订阅方式和加入编辑部}  
编辑部招聘人才,用爱发电,工作轻松,详情可联系QQ:1329527951 客服小祥\\想订阅本消息或获取PDF版(便于查看超链接和往期),可加QQ群:\href{https://qm.qq.com/q/VXIW7fgsEe}{849644979}.
\section{Deadline Ongoing}
\setbox\colbbox\vbox{
\makeatletter\col@number\@ne
\begin{longtable}{|c|c|c|}
    \hline
    消息(未见ddl的,不刊) & 截止日期 & 刊载日期\\
    \hline\hline
    紫藤学刊征稿 & 12.15 & 10.22\\
    普通话考试报名 & 10.28 & 10.14\\
    仙林校史馆招募讲解员 & 10.30 & 9.12\\
    本科生暑期课程评教 & 10.31 & 9.19\\
    黑匣招募 & 11.1 & 10.19\\
    学位英语考试报名 & 11.3 & 10.17\\
    校运会 & 11.8 & 10.21\\
    后革命鲁迅研究征文 & 11.10 & 10.8\\
    大创训练计划申报 & 11.18 & 9.24\\
    招生宣传创意征集大赛 & 11.18 & 10.21\\ 
    EBSCO数据库检索大赛 & 11.20 & 10.3\\
    文院征稿 & 11.20 & 10.20\\
    乐跑 & 12.8 & 10.12\\
    国际访学计划申报 & 11.22 & 10.22\\
    毓琇书院宿舍评比 & 10.31 & 10.22\\
    南大献血周 & 10.31 & 10.24\\
    新生午餐会报名 & 10.28 & 10.26\\
    百团大战 & 11.3 & 10.26\\
    南新读书会 & 10.30 & 10.26\\
    联合国南大宣讲 & 10.30 & 10.27\\
    Passion公开课 & 10.29 & 10.27\\
    仙林草地音乐节 & 11.3 & 10.27\\
    \hline
\end{longtable}
\unskip
\unpenalty
\unpenalty}\unvbox\colbbox
\end{multicols}
\hrule
\pagebreak
\begin{multicols}{2}

\section{讲座}
\begin{tabular}{|c|c|c|}
    \hline
    往期讲座 & 开展日期 & 刊载日期\\
    \hline\hline
    《电池及电化学能...》 & 11.24 & 10.3\\
    《专利查新与规避...》 & 12.19 & 10.3\\
    《与<自然>编辑对...》 & 10.30 & 10.16\\
    图书馆系列讲座 & 12.3 & 10.20\\
    《志工人力资源的...》 & 11.4 & 10.23\\
    《华人社会工作的...》 & 11.4 & 10.23\\
    《困在历史中的卢...》 & 10.29 & 10.23\\
    《大模型技术的发...》 & 10.31 & 10.25\\
    《从华盛顿看台湾...》 & 10.30 & 10.26\\
    《拓扑材料和交变...》 & 10.30 & 10.26\\
    《教育机会与子女...》 & 10.28 & 10.26\\
    《作为社会互动的...》 & 10.30 & 10.27\\
    \hline
\end{tabular}

1.作为社会互动的论证\\
主讲人:鞠实儿(中山大学哲学系教授)\\
主持人:张建军(南京大学哲学学院教授)\\
时间:10月30日(周三)19:00-21:00\\
地点:哲学学院401报告厅\\
摘要:作为社会互动的论证,是隶属于一个或多个社会文化群体的个体成员,在相应社会文化背景和即时情景下,为了劝使其他成员对某事或某观点采取某种态度,依据所属社会文化群体规范生成的话语序列。\\\\

\section{仙林草地音乐节}
活动时间:2024年11月3日(周日)18:00

活动地点:南京大学仙林校区炜华体育场

活动主题:奋进航线

参与对象:南京大学全体师生

主办方:共青团南京大学委员会学生组织与文体活动部

转发其推送到朋友圈集满38个赞即可参与抽奖:\url{https://mp.weixin.qq.com/s/VkJusNoRomVweiowTyvc_Q}
\section{第五届联合国机构宣讲咨询活动南京大学专场}
活动介绍:为深化国际组织人才培养、大力向国际组织输送优秀人才,满足国家的全球治理人才需求及全球发展战略需要,国家留学基金管理委员会将在南京举办第五届联合国机构宣讲咨询(UN Job Fair)活动,让有志于国际事务的人才能够近距离接触和了解联合国及其相关机构的工作。\\
活动内容:来自联合国教科文组织、国际统一私法协会、联合国儿童基金会、联合国妇女署等4家国际组织的7位人力资源官员将来到南京大学开展国际组织政策宣讲及交流。届时将会有联合国官员与同学们进行现场互动交流,并有南京大学国际组织工作介绍与优秀南大学子国际组织实习经验分享等环节。\\
活动时间:2024年10月30日9:30-11:30\\
活动地点:南京大学学生就业指导中心303报告厅\\
报名和微信群二维码详见:\url{https://mp.weixin.qq.com/s/SFtehUrLY4lX4YTpeQOd-A}

\section{南青格庐 x Passion公开课 | Waacking}
来自南大passion街舞社\\
时间:10.29 周二 19:30-21:00\\
地点:鼓楼校区南园,南青格庐舞蹈房\\
舞种:Waacking\\
详情请查看:\url{https://mp.weixin.qq.com/s/B_gmop8DaG-NTIhlX3i1Lg}\\
\section{10.27-10.29(周日~周二)学术文化活动概览}
南京大学学生会学术创新部与SRTP学社现联合整理讲座信息如下:\\
\url{https://mp.weixin.qq.com/s/9gEy_U13_p5Hu7qEGnZRtw}\\
\end{multicols} 
\hrule
\vspace{4mm}
\centerline{\huge\textbf{参考消息}}
\begin{multicols}{2}
\section{《南哪消息》一百期祝辞}
消息编辑 小喇叭\\\\
可喜可贺,今天是南哪消息运行的第一百期。从五月到十月,从寥寥数位编辑和联络员到三十五位编辑,从部分院系的公众号消息到南哪大专的一切院系、部门、社团、非社团组织的公众号、网站、群聊,这份消息汇编已经太多地出乎我的预期。我们经历了很多挫折:参考消息版面常常为人诟病,然而主编究竟没有删除,一方面是主编的个人喜好;另一方面,我也想知道它的存在能否引入更多的舆论声音。亦有多位编辑离开南哪消息,仍能记得共事的时刻。无论如何,南哪消息终于是活到了第一百期,我们还是应当庆祝;虽然第一百期在报界并不吉利——不必提起《清议报》旧事。然而我还是要用梁公的祝辞结束我的祝辞:“校报章之良否,其率何如?一曰宗旨定而高,二曰思想新而正,三曰材料富而当,四曰报事确而速。”“报兮报兮,君之生涯,亘两周兮;君之声尘,遍五洲兮;君之责任,重且遒兮,君其自爱,罔俾羞兮。祝君永年,与国民同体兮。”\\

客服小祥注:南哪消息是由南京大学各年级本科生、研究生用爱发电建设的集约化信息载体,主要汇总和发布活动预告,旨在破除信息壁垒、增进同学福祉,并非任何形式的报刊。
\end{multicols} 
\end{document}