% HEAD BEGIN
\documentclass[letterpaper, 12pt]{article}
\usepackage{graphicx}
\usepackage{multicol}
\usepackage{anysize}
\usepackage{fontspec}
\usepackage[fontset=none]{ctex}
\usepackage{tabularx}
\PassOptionsToPackage{hyphens}{url}
\usepackage[breaklinks=true, colorlinks=true]{hyperref}
\expandafter\def\expandafter\UrlBreaks\expandafter{\UrlBreaks\do\a\do\b\do\c\do\d\do\e\do\f\do\g\do\h\do\i\do\j\do\k\do\l\do\m\do\n\do\o\do\p\do\q\do\r\do\s\do\t\do\u\do\v\do\w\do\x\do\y\do\z\do\A\do\B\do\C\do\D\do\E\do\F\do\G\do\H\do\I\do\J\do\K\do\L\do\M\do\N\do\O\do\P\do\Q\do\R\do\S\do\T\do\U\do\V\do\W\do\X\do\Y\do\Z}
% \let\oldurl\url
% \renewcommand{\url}[1]{\begin{sloppypar}\oldurl{#1}\end{sloppypar}}
\setlength\columnsep{30pt}
\marginsize{30pt}{30pt}{10pt}{20pt}
\setmainfont{TeX Gyre Bonum}
\setCJKmainfont[BoldFont=Noto Serif CJK SC Bold, ItalicFont=FandolKai]{Noto Sans CJK SC}
\setlength{\parindent}{0cm}
% \setCJKmonofont{Noto Sans CJK SC}
\begin{document}
\begin{center}
    \Huge\textbf{南哪大专醒前消息}
\end{center}
\vspace{4mm}
\hrule
\renewcommand\tabularxcolumn[1]{m{#1}}
\begin{tabularx}{\textwidth}{>{\hsize.2\hsize}X>{\hsize.6\hsize}X>{\hsize.2\hsize}X}
    \begin{flushleft}
        2024.10.3\, No.79
    \end{flushleft}
    &
    \begin{center}
        \textit{“渔阳鼙鼓动地来,惊破霓裳羽衣曲。\\九重城阙烟尘生,千骑万乘西南行。”}
    \end{center}
    &
    \begin{flushright}
        \textbf{苏州市高新区}
    \end{flushright}
\end{tabularx}
\vspace{-3.5mm}
\hrule
\vspace{4mm}
% HEAD END
\centerline{\huge\textbf{活动预告}}
\begin{multicols}{2}

\section{Deadline Ongoing}
\begin{tabular}{|c|c|c|}
    \hline
    消息(未见ddl的,不刊) & 截止日期 & 刊载日期\\
    \hline\hline
    仙林校史馆招募讲解员 & 10.30 & 9.12\\
    国优计划报名 & 10.7 & 9.19\\
    本科生暑期课程评教 & 10.31 & 9.19\\
    网易雷火大赛 & 10.7 & 9.22\\
    大创训练计划申报 & 11.18 & 9.24\\
    苏州校区音乐会 & 10.19 & 9.25\\
    外院国庆摄影征集 & 10.7 & 9.25\\
    港澳台生中华文化大赛 & 10.9 & 9.26\\
    心理中心征稿 & 10.10 & 9.28\\
    周末剧场 & 10.10 & 9.28\\
    历史学院国庆活动 & 10.7 & 9.28\\
    计院国庆桌游会 & 10.5 & 9.29\\
    台湾地区交换项目 & 10.7 & 9.29\\
    第十九届大挑 & 10.15 & 9.30\\
    声谷创新基金 & 10.18 & 9.30\\
    软院国庆桌游会 & 10.7 & 9.30\\
    国际化处全媒体招新 & 10.8 & 9.30\\
    午餐读书会 & 10.10 & 9.30\\
    “周一剧!”第二期 & 10.5 & 9.30\\
    鹰角校招宣讲 & 10.15 & 10.2\\
    软院国庆活动 & 10.7 & 10.2\\
    计院文创设计 & 10.8 & 10.2\\
    育教征集国庆祝福 & 10.4 & 10.2\\
    \hline
\end{tabular}
\begin{tabular}{|c|c|c|}
    \hline
    消息(未见ddl的,不刊) & 截止日期 & 刊载日期\\
    \hline\hline
    大专戏曲知识竞赛 & 10.20 & 10.2\\
    MathGlue导员招募 & 10.7 & 10.2\\
    健雄书院征集主题作品 & 10.7 & 10.3\\
    秉文书院早晚自习报名 & 10.5 & 10.3\\
    EBSCO数据库检索大赛 & 11.20 & 10.3\\
    Passion街舞公开课 & 10.9 & 10.3\\
    
    \hline
    \end{tabular}
\section{讲座}
\begin{tabular}{|c|c|c|}
    \hline
    往期讲座 & 开展日期 & 刊载日期\\
    \hline\hline
    《聚合物的研发与...》 & 10.24 & 10.3\\
    《电池及电化学能...》 & 11.24 & 10.3\\
    《专利查新与规避...》 & 12.19 & 10.3\\
    《ChatGPT和生成...》 & 10.9 & 10.3\\
      \hline
\end{tabular}\\\\
1.聚合物的研发与应用

主讲人:程小燕 博士

10月24日 线上

复制链接浏览器注册(推荐使用电脑):\url{https://www.cas.org/resources/webinar/polymer-forum}

2.电池及电化学能源材料

主讲人:杜德鑫 博士

11月14日 线上

复制链接浏览器注册(推荐使用电脑):\url{https://www.cas.org/resources/webinar/energy-materials-forum}\\
3.专利查新与规避专利侵权风险

主讲人:钱欣 博士

12月19日 线上

复制链接浏览器注册(推荐使用电脑):\url{https://www.cas.org/resources/webinar/patent-search-forum}\\
4.ChatGPT和生成式人工智能

主讲人 刘嘉 南京大学软件学院副院长、南京大学软件学院副教授、博士生导师、南京大学智能软件工程实验室副主任

主持人 张海燕 中美文化研究中心能源、资源与环境方向副教授

时间 2024年10月9日(周三)

地点 中美文化研究中心A106会议室

详见:\url{https://mp.weixin.qq.com/s/KXOghiq7f0sBH4pYv669DA}

\section{订阅方式和加入编辑部}
编辑部招聘人才,用爱发电,工作轻松,详情可联系QQ:1329527951 客服小祥\\想订阅本消息或获取PDF版(便于查看超链接),可加QQ群:\href{https://qm.qq.com/q/FGX1VYCrGS}{962626571}.
\section{健雄书院“AI”迎国庆主题作品征集活动}
活动对象:南京大学2024级全体本科新生\\
以庆祝新中国成立75周年为大主题,使用国产AI大模型与生成式人工智能应用软件,创作相关主题作品,同时鼓励建议融入南大历史、文化与元素。\\
投稿作品类型包括但不限于AI绘画作品、AI诗歌作品、AI音频创作、AI视频创作。\\
投稿作品须附200-400字的作品简介。\\
截止时间:2024年10月7日24:00\\
按照要求参与活动投稿即可录入五育“敦行成绩单”。作品征集及评审结束后,还将通过微信公众号对优秀作品进行展示,并向优秀作品创作者颁发纪念品奖励。详细内容见\url{https://mp.weixin.qq.com/s/gMHtYSBxT0c_bs8O2Nw9VA}
\section{秉文书院秋季“晨读晚修”}
晨读时间:10.8日起工作日7:00-10:00\\
地点:新教205\\
活动形式:本次晨读活动分为圆桌小组与个人独立早读两种形式,可在每天报名时选择自己心仪的形式开启早读并自主签到签退。\\
晚修时间:10.8日起工作日18:30-21:20(每 50分钟统一休息10分钟)\\
地点:新教205\\
活动形式:为保障自习质量,本次晚修共有3节晚修课堂供同学们自由选择,课间允许到场或退场,晚修进行时原则上不允许随意进出教室。
\\参加晨读晚修者可获得相关奖励。
\\本次活动采取自主报名,请感兴趣的同学于10.5日24:00前填写南大box报名表并加入qq群。\url{https://table.nju.edu.cn/dtable/forms/6a901071-ce13-4ee3-b4e0-a2d3594e21b6/}
\section{EBSCO数据库检索大赛}
南京大学图书馆现通知EBSCO数据库检索大赛事宜。\\
EBSCO提供的文献专业涉及:理、工、农、医、天、地、生、经济等全部学科,为全球10万多家图书馆提供服务。现推出主题“你的行动影响气候变化”数据库检索大赛。EBSCO设置了少量奖品名额。活动日期:2024年9月20日-11月20日。详见:\url{https://mp.weixin.qq.com/s/kwn7JRHanq9a4KgLRjxN3g}\\
\section{街舞公开课}
上课时间:10月9日晚7:00 - 8:30\\
地点:鼓楼校区南园南青格庐舞蹈房\\
舞种:Locking\\
Passion街舞社将面向全体学生开展舞蹈公开课,每周都会带来一次不同种类的舞蹈课程,无门槛。详情可见
\url{https://mp.weixin.qq.com/s/MVBHRJPyj2PU_3lF7OpV0A}\\
\section{弘雅书房九月合集更新}
南京大学图书馆邀您欣赏古今中外艺术之美,现更新弘雅书房九月合集。同时也欢迎大家访问图书馆主页Artlib世界艺术鉴赏库,探寻更多艺术知识。\\
更新内容:\\
《技艺同辉——中国古代艺术中的科技主题展》\\
《唯见长江天际流——长江流域文明艺术展》\\
《神的脑袋会发光——东西方神话人物背光文化展》\\
Artlib数据库访问路径:\\
1、PC端访问图书馆主页—>电子数据库—>中文—>46 世界艺术鉴赏库;\\
2、戳阅读原文一键跳转数据库主页。\\
也可关注公众号“弘雅书房”,查看相关内容。\\
\end{multicols} 


\hrule
\vspace{4mm}
% APPENDIX BEGIN
\centerline{\huge\textbf{附录}}
\begin{figure}[htbp]
    \centering
    \begin{minipage}[b]{0.32\textwidth}
        \centering
        \includegraphics[width=0.5\textwidth]{群链接.png}
        \caption{群链接}
    \end{minipage}

\end{figure}
\end{document}