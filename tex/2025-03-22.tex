% HEAD BEGIN
\documentclass[letterpaper, 12pt]{article}
\newsavebox\colbbox
\usepackage{graphicx}
\usepackage{multicol}
\usepackage{anysize}
\usepackage{fontspec}
\usepackage[fontset=none]{ctex}
\usepackage{tabularx}
\usepackage{longtable}
\PassOptionsToPackage{hyphens}{url}
\usepackage[breaklinks=true, colorlinks=true]{hyperref}
\expandafter\def\expandafter\UrlBreaks\expandafter{\UrlBreaks\do\a\do\b\do\c\do\d\do\e\do\f\do\g\do\h\do\i\do\j\do\k\do\l\do\m\do\n\do\o\do\p\do\q\do\r\do\s\do\t\do\u\do\v\do\w\do\x\do\y\do\z\do\A\do\B\do\C\do\D\do\E\do\F\do\G\do\H\do\I\do\J\do\K\do\L\do\M\do\N\do\O\do\P\do\Q\do\R\do\S\do\T\do\U\do\V\do\W\do\X\do\Y\do\Z}
% \let\oldurl\url
% \renewcommand{\url}[1]{\begin{sloppypar}\oldurl{#1}\end{sloppypar}}
\setlength\columnsep{30pt}
\marginsize{30pt}{30pt}{10pt}{20pt}
\setmainfont{TeX Gyre Bonum}
\setCJKmainfont[BoldFont=Noto Serif CJK SC Bold, ItalicFont=FandolKai]{Source Han Sans SC}
\setlength{\parindent}{0cm}
% \setCJKmonofont{Noto Sans CJK SC}
\begin{document}
\begin{center}
    \Huge\textbf{南哪大专醒前消息}
\end{center}
\vspace{4mm}
\hrule
\renewcommand\tabularxcolumn[1]{m{#1}}
\begin{tabularx}{\textwidth}{>{\hsize.2\hsize}X>{\hsize.6\hsize}X>{\hsize.2\hsize}X}
    \begin{flushleft}
        2025.3.21\, No.197
    \end{flushleft}
    &
    \begin{center}
        \textit{“秉中持正、求新博闻。”}
    \end{center}
    &
    \begin{flushright}
        \textbf{南京市栖霞区}
    \end{flushright}
\end{tabularx}
\vspace{-3.5mm}
\hrule
\vspace{4mm}
% HEAD END
\centerline{\huge\textbf{活动预告}}
\begin{multicols}{2}
\section{订阅方式和加入编辑部}  
编辑部招聘人才,用爱发电,工作轻松,详情可联系QQ:1329527951 客服小千\\想订阅本消息或获取PDF版(便于查看超链接和往期),可加QQ群:\href{https://qm.qq.com/q/4HL41Nt3sQ}{466863272}.
\section{活动清单}
\setbox\colbbox\vbox{
\makeatletter\col@number\@ne
\begin{longtable}{|>{\centering\arraybackslash}m{.3\textwidth}|m{.06\textwidth}|m{.06\textwidth}|}
    \hline
    活动 & 开展时间 & 刊载时间\\
    \hline\hline
    南大版deepseek & / & 2.22\\
    悦读课程群 & / & 2.24\\
    eScience AI科研助手 & / & 3.11\\
    地科博物馆开放安排 & / & 3.22\\ 
    乐跑 & 5.16 & 3.10\\
    本科生劳育实践 & 7.20 & 2.19\\
    医保零星报销 & 3.31 & 2.19\\
    银星杯论文赛 & 4.22 & 2.27\\
    高教社杯 & 4.25 & 3.5\\
    大创报名 & 3.23 & 3.6\\
    银星杯论文竞赛 & 4.22 & 3.6\\
    南辩院系杯 & 4.12 & 3.6\\
    大文大理题目征集 & 期末 & 3.8\\
    5月免费上网 & ? & 3.9\\
    基础学科论坛 & 4.20 & 3.9\\
    四六级报名 & 3.24 & 3.11\\
    普通话测试 & 3.28 & 3.12\\
    外教社杯 & 5.27 & 3.12\\
    心理中心全媒体招新 & 3.25 & 3.14\\
    Python比赛 & 4.6 & 3.16\\
    扎染志愿者招募 & 3.28 & 3.18\\
    中美中心开放日 & 3.26 & 3.19\\
    扎染体验 & 3.23 & 3.20\\
    本科生院征集大鸣大放 & 4.4 & 3.21\\
    两会知识竞赛 & 3.30 & 3.21\\
    纸鸢工作坊 & 4.3 & 3.22\\
    \hline
\end{longtable}
\unskip
\unpenalty
\unpenalty}\unvbox\colbbox
\end{multicols}
\begin{multicols}{2}
\pagebreak

\section{讲座}
\begin{tabular}{|>{\centering\arraybackslash}m{.3\textwidth}|m{.06\textwidth}|m{.06\textwidth}|}
    \hline
    讲座 & 开展时间 & 刊载时间\\
    \hline\hline
    陶行知对中国教育现代化问题的探索 & 3.24 & 3.7\\\hline
    Vinaya Revival on Baohua Mountain in Ming–Qing China & 3.25 & 3.18 \\\hline
    中美关系百年史 & 3.23 & 3.19\\\hline
    文学之都南京的前世今生 & 3.23 & 3.19\\\hline
    从数字化、网络化到AI驱动的新趋势 & 3.23 & 3.20\\\hline
    What Can Ecological Spatiotemporal Indicators Tell Us about the Resilience to Economic Crisis & 3.25 & 3.20\\\hline
    Eigenvector Spatial Filtering in Areal and Origin-Destination Data & 3.25 & 3.20\\\hline
    智能时代下文科何为 & 3.25 & 3.21\\\hline
    生活即田野,田野即生活 & 3.24 & 3.21\\\hline
    基层治理现代化语境下的居民社区责任 & 3.25 & 3.21\\\hline
    利用个人传感器评估人类对城市空间的反应 & 3.24 & 3.21\\\hline
    分数量子霍尔态的相变以及围绕任意子的密度波的特征 & 3.27 & 3.21\\\hline
    A Geometric Perspective on the Compressible Euler... & 3.26 & 3.22\\\hline
    Multivirate Poisson intensity estimation via low-rank.. & 3.24 & 3.22\\\hline
    铜氧化物超导体中配对密度波态和强电子 & 3.25 & 3.22\\\hline
    SDGs\&残障融合\&博物馆 & 3.23 & 3.22\\\hline
\end{tabular}
%讲座预告写在这
\subsection{A Geometric Perspective on the Compressible Euler...}
报告人:于品教授(清华大学)
\\时间:2025年3月26日(周三)16:30
\\地点:鼓楼校区西大楼308
\\详见:\url{https://mp.weixin.qq.com/s/B-9PhLfGR9wMx9CnJRe7qA}

\subsection{Multivirate Poisson intensity estimation via low-rank..}
主讲人:徐昊天
\\现场报告时间:北京时间2025年3月24日(周一)上午10:00-11:00
\\现场报告地点:鼓楼校区西大楼108报告厅
\\腾讯会议:602-855-215
\\详见:\url{https://mp.weixin.qq.com/s/U0SsnvTwd-knx9_Hpxrigw}

\subsection{铜氧化物超导体中配对密度波态和强电子-晶格相互作用的直接观测}
报告时间:3月25日(周二)中午12点
\\报告地点:南京大学鼓楼校区唐仲英楼B501
\\报告主题:Direct Visualization of Pair Density Wave States and Strong Electron-Lattice Interactions in the Cuprate Superconductor
\\报告人:杜增义
\\直播链接:\url{https://www.koushare.com/live/details/41321}
\\详见:\url{https://mp.weixin.qq.com/s/rCS6SfVny6komub38ewFPQ}

\subsection{Movers工作坊SDGs\&残障融合\&博物馆}
3月23日19:00-21:00\\
腾讯会议号:537-238-647(密码后续群内公布)
\\参与方式:填写问卷并加入微信群。活动限额30人,在群内公布入选名单
\\(问卷及微信参见微信推送)
\\详见:\url{https://mp.weixin.qq.com/s/_OBHAph-3RjGSBNqSbeCfA}

\subsection{[SRTP]3.23-3.25(周日~周二)学术文化活动概览}
周日(3.23)\\
1.合作与冲突:中美关系百年史\\
2.从数字化、网络化到AI驱动的新趋势(文娱消费案例分享与启示)\\
3.文学之都南京的前世今生\\
周一(3.24)\\
1.What Can Ecological Spatiotemporal Indicators Tell Us About the Resilience to Economic Crises\\
2.生活即田野,田野即生活\\
周二(3.25)\\
1.Vinaya Revival on Baohua Mountain in Ming-Qing China\\
2.Eigenvector Spatial Filtering in Areal and Origin-Destination Data\\
3.基层治理现代化语境下的居民社区责任\\
详见:\url{https://mp.weixin.qq.com/s/2paA_3iAiqDxOddELeBb7g}


\section{荧光夜跑 | 韵动青春,逐梦星辰}
活动内容:活动参与者加入活动QQ群603704965,并为自己创建一个包含年级、专业、姓名的群相册,用于上传跑步打卡记录。策划组将根据参与者打卡里程核算分数,每周核算一次。当周分数前五的参与者将会获得精美小礼品(每人限领一次),最终总分排名前十的同学将获得特别惊喜!活动过程中宿舍结伴跑步打卡、分享跑步音乐、分享跑步心得将会获得额外加分。
\\活动时间:3月24日--5月16日
\\活动地点:仙林校区炜华运动场
\\详见:\url{https://mp.weixin.qq.com/s/XKqLEACxev0NmFomHDGvTw}

\section{南大地球科学博物馆春季学期开放安排}
地球科学博物馆将于4月20日(周日)、4月22日(周二)地球日增加开放!
\\地点:南京大学鼓楼校区田家炳楼4楼(电梯上到3楼)
\\时间:周一、周三、周五上午9:00-11:30,下午14:00-16:30;
\\4月20日(周日)、4月22日(周二)上午9:00-11:30,下午14:00-16:30增加开放!
\\详见:\url{https://mp.weixin.qq.com/s/pY3uGpB5TUyjws44O0phMQ}

\section{第44届校园十大歌星赛}
阳春三月,草木蔓发,南京大学第44届校园十大歌星赛即将拉开帷幕。为海选咨询群和海选报名问卷的二维码已经发布,欢迎扫码报名,唱响你的高光时刻!海选赛段的比赛具体时间及赛制等信息将在后续推送及海选咨询群中陆续发布,更多相关内容敬请关注“南京大学学生会”公众号。
\\详见:\url{https://mp.weixin.qq.com/s/s7tbAUpkVe4Hct1y1qoEdg}

\section{南京大学研究生定向越野}
为丰富南大学子的课余生活,感受文体活动的魅力,南京大学研究生会联合人工智能学院研会主办,天文、哲学、生科、文学院和外院研究生会协办本次定向越野活动。
\\活动时间:2025年3月29日 14:00-16:00
\\活动地点:南京大学仙林校区
\\活动对象:南京大学全体学生
\\参与形式:每支参赛队伍最多由6人组成,分为单人组、双人组和多人组三条赛道,不同赛道分设不同奖项
\\活动内容和规则介绍、安全提示、活动奖品、活动报名信息详见“南大研会”公众号推文。
\\详见:\url{https://mp.weixin.qq.com/s/i_1RtiGJspCjgOypayaqGg}

\section{2025年“长望杯”气象达人邀请赛}
比赛流程:
\\阶段一:知识竞赛——海选赛
\\2025年3月22日20:00-21:00 时长60分钟
\\选手通过电脑/手机扫描问卷星二维码或点开链接答题,开卷,可参考任何资料以及互相交流。由活动负责人根据参考答案登录问卷星考试系统评定简答题和材料题得分。
\\| 比赛人员要求 |
\\高校本科生、研究生
\\| 比赛题型 |
\\判断题 单选题 多选题 简答题 材料题 总分150分
\\| 本阶段比赛奖励 |
\\一等奖1名,500元
\\二等奖5名,200元
\\三等奖20名,100元
\\阶段二:预报竞赛——技能赛
\\时间待定,预计在四月中上旬
\\详见原推
\\详见:\url{https://mp.weixin.qq.com/s/siEOQOimnOTeESjqtwalzw}

\section{清明纸鸢手作工坊}
时间:4月3日 14:00-16:00\\
地点:南京大学仙林校区敬文学生活动中心9楼\\
对象:南京大学全体在校师生(本次活动为留学生专场,现场配有英文支持。欢迎留学生同学及对跨文化交流感兴趣的同学参加!)\\
报名方法及须知:报名人数限30人,具体报名方法请见微信推送\\
详见:\url{https://mp.weixin.qq.com/s/9g_MU7m-LC8YqZQXN3nO6w}

\section{访企拓岗丨中兴通讯}
活动时间:3月28日(周五)下午
\\活动地点:中兴通讯南京研发中心
\\活动对象:在读硕博各年级同学
\\交通方式:集中乘坐安排的车辆往返
\\活动出发时间:3月28日(周五)下午 13:30
\\企业内活动安排:
\\1、5G行业展厅参观
\\2、公司介绍与业务分享
\\3、校友分享与答疑
\\参加活动的同学请扫描链接中二维码报名
\\报名截止时间:2025年3月25日下午18:00
\\本次开放日活动的名额为30人,名额有限,先到先得!快来报名吧!
\\详见:\url{https://mp.weixin.qq.com/s/uF9tAEIRLilgcjYbsloX4w}

%此处写校级活动,请不要把讲座、院级活动和社团活动写在这里orz orz orz


\section{院级活动}
\begin{tabular}{|>{\centering\arraybackslash}m{.3\textwidth}|m{.06\textwidth}|m{.06\textwidth}|}
\hline
    活动 & 开展时间 & 刊载时间\\
    \hline\hline
    文院剧本创作研讨会 & 9.30 & 3.2\\
    物院征集课程指南 & 6.15 & 3.3\\
    信地海环四院羽球赛 & 3.23 & 3.10\\
    电子学院腾讯简历面试指导 & 3.24 & 3.10\\
    地海征集春日影 & 6.15 & 3.14\\
    计院定向越野 & 3.22 & / \\
    秉文猫鼠游戏 & 3.23 & 3.18\\
    AI院影色舞 & 3.29 & 3.19\\
    商院羽球 & 3.29 & 3.19\\
    物院春游 & 3.22 & 3.21\\
    史院就业 & 3.25 & 3.21\\
    物院春游 & 3.22 & 3.21\\
    美团NJUAI专场空宣会 & 3.25 & 3.21\\
    字节跳动NJUAI专场空宣会 & 3.24 & 3.21\\
    物院访企 & 3.28 & 3.22\\
    \hline
\end{tabular}
%这里是写院级活动的,院级活动就是只限某院学生参加的活动,和由某院某部门主办、主要针对某院学生的活动。不要把对全校学生开放的活动写在这里。

\subsection{格物奋进 | “物耀启航”一日职场访企拓岗活动——波长光电参访邀请}
活动时间:2025年3月28日13:00-18:00
\\活动对象:物理学院在读学生
\\活动名额:30人
\\报名链接:\url{https://table.nju.edu.cn/dtable/forms/f4b7b420-6771-465e-9e5e-940ed0e98ddd/}
\\详见:\url{https://mp.weixin.qq.com/s/plPSgHjyIWM564lyEPsiRg}

\section{社团活动}
\begin{tabular}{|>{\centering\arraybackslash}m{.3\textwidth}|m{.06\textwidth}|m{.06\textwidth}|}
    \hline
    社团活动 & 开展时间 & 刊载时间\\
    \hline\hline
    天文台开放日 & / & 1.6\\
    相声社春季专场 & 3.22 & 3.17\\
    鸿新社捐书活动 & 3.30 & 3.17\\
    CAC观影 & 3.22 & 3.17\\
    心协卡牌招募 & 3.22 & 3.17\\
    知行古案今判 & 3.23 & 3.19\\
    心协流光影院 & 3.22 & 3.19\\
    长歌行声演剧 & 3.29 & 3.19\\
    乒协抽奖 & 3.24 & 3 19\\
    新火星影映 & 3.23 & 3.21\\
    鸿新社早起打卡 & 3.24 & 3.23\\
    \hline
\end{tabular}
%这里是写社团活动的,社团活动就是由社团主办、主要针对社团内部人员的活动。不要把非社团活动写在这里。
\subsection{ “早睡早起”打卡}
活动时间:3.24-3.30(共七天)
\\活动对象:全校学生
\\活动内容:同学们在活动期间坚持早睡早起,并在QQ群内以图片、音乐等方式进行打卡,与大家分享自律生活。
\\时间要求:需在早上8点前、晚上23点前打卡,在此时间段外的打卡不计入次数。
\\参与方式 :加入活动QQ群824533808
\\详见:\url{https://mp.weixin.qq.com/s/rNQIFncY6Pp577F4rj4W2w}
\subsection{院系杯辩论赛 | 3.22-3.23赛程预告}
3.23(周日) 14:00-15:30
\\地点:仙2-101
\\辩题:在当代心理学 / 命理学更是年轻人的治愈良方
\\对阵:商院辩论队 vs 匡医辩论队
\\3.23(周日) 16:00-17:30
\\地点:仙2-101
\\辩题:“娱乐至上”正在让互联网讨论氛围更松弛 / 更暴躁
\\对阵:社科辩论队 vs 软院辩论队
\\详见:\url{https://mp.weixin.qq.com/s/9sxRWaVt8Yq7PDQ_mxdZEA}
\end{multicols}
\end{document}
