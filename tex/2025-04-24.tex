% HEAD BEGIN
\documentclass[letterpaper, 12pt]{article}
\newsavebox\colbbox
\usepackage{graphicx}
\usepackage{multicol}
\usepackage{anysize}
\usepackage{fontspec}
\usepackage[fontset=none]{ctex}
\usepackage{tabularx}
\usepackage{longtable}
\PassOptionsToPackage{hyphens}{url}
\usepackage[breaklinks=true, colorlinks=true]{hyperref}
\expandafter\def\expandafter\UrlBreaks\expandafter{\UrlBreaks\do\a\do\b\do\c\do\d\do\e\do\f\do\g\do\h\do\i\do\j\do\k\do\l\do\m\do\n\do\o\do\p\do\q\do\r\do\s\do\t\do\u\do\v\do\w\do\x\do\y\do\z\do\A\do\B\do\C\do\D\do\E\do\F\do\G\do\H\do\I\do\J\do\K\do\L\do\M\do\N\do\O\do\P\do\Q\do\R\do\S\do\T\do\U\do\V\do\W\do\X\do\Y\do\Z}
% \let\oldurl\url
% \renewcommand{\url}[1]{\begin{sloppypar}\oldurl{#1}\end{sloppypar}}
\setlength\columnsep{30pt}
\marginsize{30pt}{30pt}{10pt}{20pt}
\setmainfont{TeX Gyre Bonum}
\setCJKmainfont[BoldFont=Noto Serif CJK SC Bold, ItalicFont=FandolKai]{Source Han Sans SC}
\setlength{\parindent}{0cm}
% \setCJKmonofont{Noto Sans CJK SC}
\begin{document}
\begin{center}
    \Huge\textbf{南哪大专醒前消息}
\end{center}
\vspace{4mm}
\hrule
\renewcommand\tabularxcolumn[1]{m{#1}}
\begin{tabularx}{\textwidth}{>{\hsize.2\hsize}X>{\hsize.6\hsize}X>{\hsize.2\hsize}X}
    \begin{flushleft}
        2025.4.24\, No.229
    \end{flushleft}
    &
    \begin{center}
        \textit{“秉中持正、求新博闻。”}
    \end{center}
    &
    \begin{flushright}
        \textbf{南京市栖霞区}
    \end{flushright}
\end{tabularx}
\vspace{-3.5mm}
\hrule
\vspace{4mm}
% HEAD END
\centerline{\huge\textbf{活动预告}}
\begin{multicols}{2}
\section{订阅方式和加入编辑部}  
编辑部招聘人才,用爱发电,工作轻松,详情可联系QQ:1329527951 客服小千\\想订阅本消息或获取PDF版(便于查看超链接和往期),可加QQ群:\href{https://qm.qq.com/q/4HL41Nt3sQ}{466863272}.
\section{活动清单}
\setbox\colbbox\vbox{
\makeatletter\col@number\@ne
\begin{longtable}{|>{\centering\arraybackslash}m{.3\textwidth}|m{.06\textwidth}|m{.06\textwidth}|}
    \hline
    活动 & 开展时间 & 刊载时间\\
    \hline\hline
    南大版deepseek & / & 2.22\\
    悦读课程群 & / & 2.24\\
    eScience AI科研助手 & / & 3.11\\
    地科博物馆开放安排 & / & 3.22\\ 
    2025年分流和转专业政策通知 & / & 4.7\\
    2025年转专业志愿填报通知 & / & 4.24\\
    乐跑 & 5.16 & 3.10\\
    本科生劳育实践 & 7.20 & 2.19\\
    高教社杯 & 4.25 & 3.5\\
    大文大理题目征集 & 期末 & 3.8\\
    5月免费上网 & ? & 3.9\\
    外教社杯 & 5.27 & 3.12\\
    江苏创青春赛事 & 4.30 & 3.26\\
    浦口音乐跑 & 5.30 & 3.31\\
    程设大赛 & 4.26 & 4.2\\
    仙林校区志愿法律咨询 & / & 4.4\\
    青春活力大赛 & 5.17 & 4.7\\
    在校生自愿体检 & 6.20 & 4.8\\
    南大购买WPS & / & 4.8\\
    24级程设大赛 & 4.27 & 4.11\\
    EL程设大赛 & 4.27 & 4.13\\
    中美中心2025年证书项目 & 5.24 & 4.14\\
    春季学期创新训练计划结题考核通知 & 4.28 & 4.15\\
    “天池杯”AI创新大赛 & 4.28 & 4.17\\
    粤歌赛决赛 & 5.10 & 4.21\\
    十大半决赛 & 4.26 & 4.21\\
    以书换树环保公益活动 & 4.25 & 4.21\\
    汉字文化技能大赛 & 5.4 & 4.21\\ 
    法治嘉年华 & 4.25 & 4.21\\
    红色保密主题互动 & 4.25 & 4.23\\
    校博岩画展 & 6.22 & 4.23\\
    鼓楼交通志愿者报名 & 4.25 & 4.23\\
    六月免费上网 & 4.28 & 4.23\\
    CASHL“畅读”活动 & 5.23 & 4.24\\
    天池杯 & 4.28 & 4.24\\
    CASHL畅读福利月文创发放 & 4.28 & 4.24\\
    江苏高校凤凰读书节 & 6.15 & 4.24\\

    \hline
\end{longtable}
\unskip
\unpenalty
\unpenalty}\unvbox\colbbox
\end{multicols}
\begin{multicols}{2}
\pagebreak

\section{讲座}
\begin{tabular}{|>{\centering\arraybackslash}m{.3\textwidth}|m{.06\textwidth}|m{.06\textwidth}|}
    \hline
    讲座 & 开展时间 & 刊载时间\\
    \hline\hline
    从感知到疗愈:人脑音乐加工机制 & 4.25 & 4.11\\\hline
    软件发展与技术漫谈 & 4.29 & 4.16\\\hline
    从语言到智能 ⸺ 大语言模型的奥秘与应用 & 5.6 & 4.16\\\hline
    急救技能抓住 “黄金 4 分钟” & 4.25 & 4.18\\\hline
    2025平安留学行前培训会 & 4.29 & 4.20\\\hline
    花语崛起:拉斐尔前派画作中的花草 & 4.25 & 4.20\\\hline
    “法护青春,职路引航”就业季法律公益讲座 & 4.25 & 4.20\\\hline
    十三陵水库——在工地上理解新民歌运动 & 4.25 & 4.22\\\hline
    商王朝时期长江与黄河间的文明互鉴 & 4.25 & 4.22\\\hline
    试论萨特的知觉哲学 & 4.28 & 4.22\\\hline
    “大模型部署与使用” & 4.26 & 4.22\\\hline
    Better Min-wise Hash Families from Pseudorandomness for Combinatorial Rectangles & 4.25 & 4.23\\\hline
    Efficient Symbolic Execution Based on Static Analysis & 4.25 & 4.23\\\hline
    支付意愿: 采用颠覆性创新的网络效应研究 & 4.28 & 4.23\\\hline
    近代城市的形成及不同政治主体的城市规划 & 4.25 & 4.23\\\hline
    “不体物”——论韩、欧、苏咏物诗的一种新范式 & 4.25 & 4.23\\\hline
    以综合客运枢纽为节点的联程出行智能服务 & 4.25 & 4.23\\\hline
    材料回用的机会和挑战 & 4.28 & 4.23\\\hline
    Precipitation Response to Global Warming & 4.25 & 4.24\\\hline
    Drought under Global Warming & 4.29 & 4.24\\\hline
    唐代国史修撰谈片 & 4.29 & 4.24\\\hline
\end{tabular}
\subsection{国际访问学者|讲座预告:气候动力学系列讲座} % 讲座 describer: Cirlpso
报告题目:Precipitation Response to Global Warming
\\报告人:Aiguo Dai 杰出教授
\\报告时间及地点:2025年4月25日(周五)10:00-11:30仙林校区大气楼D103
\\
\\报告题目:Drought under Global Warming
\\报告人:Aiguo Dai 杰出教授
\\报告时间及地点:2025年4月29日(周二)14:00-15:30仙林校区大气楼D103
\\详见:\url{https://mp.weixin.qq.com/s/mElKZe2d_e8MY5Y49uMs9w}

\subsection{唐朝的自画像——唐代国史修撰谈片} % 讲座 describer: instruconrevo
主题:唐朝的自画像——唐代国史修撰谈片
\\时间:2025年4月29日(周二) 19:00-21:00
\\主讲人:李南晖 中山大学中文系副教授
\\主持人:赵庶洋 南京大学文学院副教授
\\地点:南京大学仙林校区文学院活水轩
\\详见:\url{https://mp.weixin.qq.com/s/URTuQRoBPDDh0HHcZEyd7A}

\section{CASHL“畅读”活动} %  describer: instruconrevo
2025年4月23日是第30个 “世界读书日”。值此“世界读书日”到来之际,CASHL(中国高校人文社会科学文献中心)将于2025年4月23日-5月23日面向馆际互借成员馆用户,全面推出“畅读”原版外文图书免费借阅活动,以满足广大用户对人文社科外文图书的借阅需求。
\\尚未开通CASHL馆际互借服务的成员馆用户只能复制图书部分章节,无法借阅全本图书。
\\详细加入办法,请访问:http://www.cashl.edu.cn/node/42
\\借阅范围:图书免费借阅范围为35家服务馆馆藏(含上海图书馆)
\\详见:\url{https://mp.weixin.qq.com/s/nRD1GDsfLApX0xzBFTBldw}
\section{CASHL畅读福利月文创发放} % 校级活动 describer: nik_nul
到场师生可获:
\\• 基础礼:图书馆钥匙扣
\\• 隐藏礼:前20名提交申请者赠图书馆帆布袋一个
\\时间:
\\2025年4月28日 9:00-17:00
\\地点:
\\仙林校区:杜厦图书馆正门
\\鼓楼校区:图书馆大厅咨询台
\\领取方式:
\\关注开世览文公众号,通过联合认证,点击https://ill.calis.edu.cn/reader/index.html?tenant=a000628,选择“馆际互借与文献传递” 点击进入,输入校园一卡通的账号与密码,即CALIS统一登录账号密码,补全信息,提交注册,等待馆员审核,馆员审核后,请开始提交申请2—3条,截图兑换礼品
\\详见:\url{https://mp.weixin.qq.com/s/B2Gfew03Cpk5suplx6aO4w}
\section{南京大学“天池杯”AI创新大赛启动} %  describer: HOllyWood
大赛主题:智汇南雍·AI创想家
\\大语言模型LLM技术等通用人工智能应用,让人工智能迅速走进了我们每一个人的生活,深刻改变了人们与世界互动的方式。请同学们围绕AI对未来学习方式、科学研究、生活方式的影响,创设具体AI应用场景,开发智能应用。
\\赛程安排:
\\1. 报名(4月14日-4月28日)
\\2. 初赛(4月28日-5月6日)
\\3. 复赛(5月6日-5月15日)
\\4. 决赛路演\&颁奖典礼(5月下旬)
\\详见:\url{https://mp.weixin.qq.com/s/Bbn5vymmKslD8SFoJmwlhA}



\section{第三届江苏高校凤凰读书节} % 校级活动 describer: nik_nul
第三届江苏高校凤凰读书节内含书评大赛、短视频讲书大赛、高校文创设计大赛和精品图书展销四个项目,奖品是数百至数千元新华书店书卡,截止时间自6月15日至12月31日不等。活动非常复杂,有意向者请查看原推文
\\详见:\url{https://mp.weixin.qq.com/s/V3nJ6Zkk1i75hVoF84OnLw}

\section{关于本学期第二次悦读经典计划测试的通知} % 校级活动 describer: Cirlpso
鉴于部分同学未能及时参与本学期第一次悦读测试,经研究决定,本学期再增开一次悦读测试,时间为2025年4月27日-5月5日。测试面向2021级及以前学生开放,请需要测试的同学提前复习(测试书目见附件)。
\\测试形式为在线网络测试,时长60分钟。题库网址为:https://eztest.org/manager/student/6609/enroll/list/。测试包括客观题和主观题两个部分,主观题为开放式简答题,客观题与该单元所有书目相关。学生报名时选择该单元某一书目,即表示主观题与所选书目相关。每个单元只可选择一本书目答题,仅有一次答题机会,请想要通过参加测试来认定悦读学分的同学谨慎选择。答题完成后,需等相关书目的悦读导师批阅后才能看到成绩(教务系统显示成绩)。60分及格,及格的成绩可以用来认定悦读学分。
\\请注意:
\\1.请选择自身所缺单元作答,错选答题无效。
\\2.严禁使用AI工具作答,一经发现成绩作废并取消后续参考资格。
\\3.本次测试为本学期最后一次开放,后续不再安排补测,请务必珍惜机会。
\\详见:\url{https://jw.nju.edu.cn/89/cc/c26263a756172/page.htm}

\section{2025年学科、专业准入志愿填报通知} % 校级活动 describer: Cirlpso
4月28日9:00-5月9日24:00,2024级学生网上填报学科、专业准入志愿(包括跨大类/院系专业准入志愿);2023级有跨院系专业准入意愿的学生,填报跨院系专业准入志愿。志愿填报操作手册详见:https://jw.nju.edu.cn/69/54/c24683a616788/page.psp。申请跨大类(院系)专业准入的学生,还需按照志愿填报操作手册的指导,下载《跨大类(院系)专业准入申请表》。
\\5月12日下午17:00前,2024级填报跨大类(院系)专业准入志愿的学生将申请表交给所在书院教务员,2023级填报跨院系专业准入志愿的学生将申请表交给所在院系教务员。
\\5月7日(星期三)12:30-15:00在鼓楼校区新教学楼组织“2025年学科分流、专业准入填报咨询会”,同学可前往咨询,地点详见附件。
\\填报过程中,相关问题也可咨询书院及院系教务老师,联系方式可查看:南京大学院系、教学单位和大类本科教务员通讯录
\\详见:\url{https://jw.nju.edu.cn/89/34/c26263a756020/page.htm}

\section{南京大学2025年度心理短视频大赛} % 校级活动 describer: charlors
参赛选题:
\\1、暖心校园记录
\\2、自我成长分享
\\3、心理短剧展示
\\4、创意心理科普
\\5、人工智能专题
\\参赛要求:
\\1.视频形式:形式不限,情景剧、真人讲解、动画等均可。保证视频画质、音质清晰。鼓励参赛者发挥创造力。
\\2.视频时长:3-10分钟。
\\3.技术要求:要求提交视频(MP4 格式),配有字幕为宜,附带 1200*500 像素封面一张。
\\4.参赛方式:以各院系心理健康工作室为独立参赛单位,请各院系发动心理委员及广大同学自主创作、拍摄、制作优秀短视频作品。我校学生和教师也可以自由组队或以个人身份,独立拍摄作品参赛。
\\5.选送数量:每个心理健康工作室最多可选送10项优秀作品参赛。\\
各院系心理健康工作室或参赛个人于5月28日前将以下三份参赛材料:1、大赛报名汇总表;2、视频作品(mp4格式);3、封面图片。上传至南大云盘:

https://box.nju.edu.cn/u/d/c98c7d122d8242e6a3e9/
\\详见:\url{https://mp.weixin.qq.com/s/GIyz7qGIE4sBvt2m51Xc5g}

\section{院级活动}
\begin{tabular}{|>{\centering\arraybackslash}m{.3\textwidth}|m{.06\textwidth}|m{.06\textwidth}|}
\hline
    活动 & 开展时间 & 刊载时间\\
    \hline\hline
    文院剧本创作研讨会 & 9.30 & 3.2\\
    物院征集课程指南 & 6.15 & 3.3\\
    地海征集春日影 & 6.15 & 3.14\\
    法院党建征文 & 5.20 & 4.2\\
    地学趣运会 & 4.26 & 4.9\\
    四院音乐节 & 5.11 & 4.7\\
    商院征集 & 5.5 & 4.8\\
    物院运动打卡 & 5.14 & 4.12\\
    地海图书漂流 & 4.23 & 4.16\\
    文院茶话会 & 4.24 & 4.20\\
    希音杯 & 4.25 & 4.20\\
    开甲剧本杀 & 4.26 & 4.21\\
    法院研习班 & 4.27 & 4.22\\
    电院征集 & 5.11 & 4.22\\
    地海宣讲 & 4.27 & 4.22\\
    商院分享 & 4.26 & 4.22\\
    智院摄影 & 5.6 & 4.22\\
    软院分享 & 4.24 & 4.22\\
    艺院竹编 & 4.26 & 4.23\\
    \hline
\end{tabular}



\section{社团活动}
\begin{tabular}{|>{\centering\arraybackslash}m{.3\textwidth}|m{.06\textwidth}|m{.06\textwidth}|}
    \hline
    社团活动 & 开展时间 & 刊载时间\\
    \hline\hline
    天文台开放日 & / & 1.6\\
    重唱诗歌奖征稿 & 4.30 & 3.31\\
    体育舞蹈教学 & 4.25 & 4.1\\
    吉他社春日音 & 4.26 & 4.4\\
    汉服社摆摊 & 4.26 & 4.17\\
    车协科普 & 4.26 & 4.17\\
    心协团辅活动 & 4.23 & 4.17\\
    匿名评诗会 & 4.26 & 4.20\\
    红会一块走 & 5.20 & 4.21\\
    法院研习班 & 4.27 & 4.22\\
    集庆折子戏 & 5.7 & 4.22\\
    自然阅读论坛 & 4.27 & 4.24\\
    辩院系杯决赛 & 4.26 & 4.24\\
    \hline
\end{tabular}
%这里是写社团活动的,社团活动就是由社团主办、主要针对社团内部人员的活动。不要把非社团活动写在这里。
\subsection{院系杯决赛} % 社团活动 describer: nik_nul
辩题:见过了霍格沃茨的烟火,要/不要接受自己只是一个麻瓜
\\时间:4月26日(周六)19:00-21:00
\\地点:仙林校区地球科学与工程学院报告厅
\\对阵双方:化生辩论队 VS 人文辩论队
\\领票指南:
\\请将原推送转发至微信朋友圈或QQ空间,凭集赞截图发送至“南大辩论”公众号后台领取门票;
\\集赞满50可获得VIP门票(第四至六排,固定座位),集赞满30可获得普通票(第六排及以后,不设固定座位)。VIP门票数量有限,先到先得!
\\截图须清晰完整包含头像昵称,每个账号仅限领取一次,不得代领。
\\门票派发:
\\鼓楼:4.25 11:00-13:00 南园喷泉广场
\\仙林:4.26 比赛前 地科报告厅门口
\\详见:\url{https://mp.weixin.qq.com/s/WWLzytHjUxLHAdnG3EXI2w}

\subsection{篮球赛} % 社团活动 describer: nik_nul
半决赛:
\\南京大学VS金陵科技学院
\\时间: 4月25日 14:00 - 15:30
\\地点: 南京江北新区大厂体育馆
\\三四名决赛:
\\时间: 4月26日 15:00 - 16:30
\\地点: 南京江北新区大厂体育馆
\\冠亚军决赛:
\\时间: 4月26日 18:30 - 20:00
\\地点: 南京江北新区大厂体育馆
\\详见:\url{https://mp.weixin.qq.com/s/CEr0tqbqdn6F37ZDyK1bWg}

\subsection{心游天际 纸见自然——自然阅读论坛} % 讲座 describer: nik_nul
时间:4.27周日上午十点
\\地点:仙林校区杜厦图书馆校友之家报告厅(大众书局旁)
\\详见:\url{https://mp.weixin.qq.com/s/-kGhfkO83nW9XKIXvVOG-Q}
\end{multicols}
\end{document}
