% HEAD BEGIN
\documentclass[letterpaper, 12pt]{article}
\usepackage{graphicx}
\usepackage{multicol}
\usepackage{anysize}
\usepackage{fontspec}
\usepackage[fontset=none]{ctex}
\usepackage{tabularx}
\PassOptionsToPackage{hyphens}{url}
\usepackage[breaklinks=true, colorlinks=true]{hyperref}
\expandafter\def\expandafter\UrlBreaks\expandafter{\UrlBreaks\do\a\do\b\do\c\do\d\do\e\do\f\do\g\do\h\do\i\do\j\do\k\do\l\do\m\do\n\do\o\do\p\do\q\do\r\do\s\do\t\do\u\do\v\do\w\do\x\do\y\do\z\do\A\do\B\do\C\do\D\do\E\do\F\do\G\do\H\do\I\do\J\do\K\do\L\do\M\do\N\do\O\do\P\do\Q\do\R\do\S\do\T\do\U\do\V\do\W\do\X\do\Y\do\Z}
% \let\oldurl\url
% \renewcommand{\url}[1]{\begin{sloppypar}\oldurl{#1}\end{sloppypar}}
\setlength\columnsep{30pt}
\marginsize{30pt}{30pt}{10pt}{20pt}
\setmainfont{TeX Gyre Bonum}
\setCJKmainfont[BoldFont=Noto Serif CJK SC Bold, ItalicFont=FandolKai]{Noto Sans CJK SC}
\setlength{\parindent}{0cm}
% \setCJKmonofont{Noto Sans CJK SC}
\begin{document}
\begin{center}
    \Huge\textbf{南哪大专醒前消息}
\end{center}
\vspace{4mm}
\hrule
\renewcommand\tabularxcolumn[1]{m{#1}}
\begin{tabularx}{\textwidth}{>{\hsize.2\hsize}X>{\hsize.6\hsize}X>{\hsize.2\hsize}X}
    \begin{flushleft}
        2024.10.10\, No.84
    \end{flushleft}
    &
    \begin{center}
        \textit{“阅报愈多者其人愈智,报馆愈多者其国愈强。”\\——梁启超}
    \end{center}
    &
    \begin{flushright}
        \textbf{南京市栖霞区}
    \end{flushright}
\end{tabularx}
\vspace{-3.5mm}
\hrule
\vspace{4mm}
% HEAD END
\centerline{\huge\textbf{活动预告}}
\begin{multicols}{2}

\section{Deadline Ongoing}
\begin{tabular}{|c|c|c|}
    \hline
    消息(未见ddl的,不刊) & 截止日期 & 刊载日期\\
    \hline\hline
    仙林校史馆招募讲解员 & 10.30 & 9.12\\
    本科生暑期课程评教 & 10.31 & 9.19\\
    大创训练计划申报 & 11.18 & 9.24\\
    苏州校区音乐会 & 10.19 & 9.25\\
    第十九届大挑 & 10.15 & 9.30\\
    声谷创新基金 & 10.18 & 9.30\\
    鹰角校招宣讲 & 10.15 & 10.2\\
    大专戏曲知识竞赛 & 10.20 & 10.2\\
    EBSCO数据库检索大赛 & 11.20 & 10.3\\
    炜华音乐跑 & 12.8 & 10.4\\
    马院主题宣讲报名 & 10.25 & 10.5\\
    NJU MAJOR & 10.13 & 10.8\\
    后革命鲁迅研究征文 & 11.10 & 10.8\\
    街舞社开放活动 & 10.11 & 10.8\\
    心协黑胶唱片活动 & 10.13 & 10.8\\
    黑匣子对谈招募 & 10.11 & 10.8\\
    鼓楼草地音乐节前瞻 & 10.11 & 10.9\\
    鼓楼草地音乐节 & 10.13 & 10.9\\
    重唱诗社匿名评诗会 & 10.13 & 10.9\\
    “南大新传”编辑部招新 & 10.20 & 10.10\\
    遵义精神宣讲团遴选 & 10.27 & 10.10\\
    历史学院Photoshop培训 & 10.13 & 10.10\\
    八院联谊活动 & 10.14 & 10.10\\
    
    \hline
\end{tabular}
\section{订阅方式和加入消息编辑部}
编辑部招聘人才,用爱发电,工作轻松,详情可联系QQ:1329527951 客服小祥\\想订阅本消息或获取PDF版(便于查看超链接),可加QQ群:\href{https://qm.qq.com/q/FGX1VYCrGS}{849644979}.
\section{讲座}
\begin{tabular}{|c|c|c|}
    \hline
    往期讲座 & 开展日期 & 刊载日期\\
    \hline\hline
    《聚合物的研发与...》 & 10.24 & 10.3\\
    《电池及电化学能...》 & 11.24 & 10.3\\
    《专利查新与规避...》 & 12.19 & 10.3\\
    《恋爱是门技术活》 & 10.14 & 10.8\\
    《宋代佛教书籍史》 & 10.11 & 10.8\\
    《阿赫迈底亚教派...》 & 10.11 & 10.8\\
    《对于人工智能时...》 & 10.16 & 10.9\\
    《职普比大体相当...》 & 10.11 & 10.9\\
    《异域的反思与开...》 & 10.11 & 10.9\\
    《中国古代文学中...》 & 10.12 & 10.9\\
    《揭开量化投资的...》 & 10.12 & 10.10\\
    《卡夫卡、现代组...》 & 10.16 & 10.10\\
    《中美博弈及其对...》 & 10.15 & 10.10\\
    《跨代性与跨代平...》 & 10.16 & 10.10\\
      \hline
\end{tabular}\\\\
1.星成长|满天星商知讲堂\\
主题:揭开量化投资的神秘面纱\\
主讲人:温馨儿(上海优宗投资有限公司总经理助理);赵龙(南京大学商学院2014届经济学系硕士,彤天基金管理有限公司基金经理)\\
主持人:耿强,南京大学商学院经济学系教授,人口研究所所长\\
时间:2024.10.12(周六)18:30-20:30\\
地点:南京大学仙林校区 仙\uppercase\expandafter{\romannumeral2}-104\\
报名方式:面向全校同学开放,请有意向者填写报名链接或扫描二维码\\
报名链接:\url{https://table.nju.edu.cn/dtable/forms/111d7592-706f-498b-9abd-93a44ef90adf/}\\
注:本次活动可作为项目制课程的过程性学习\\
\url{https://mp.weixin.qq.com/s/-o9mMzWCMIdI1JPV0o83cA}\\

2.卡夫卡、现代组织理论和晚期哈布斯堡帝国\\
主讲人:Benno Wagner(德国锡根大学教授)\\
主持人:卢盛舟(南京大学外国语学院副教授、高研院第20期驻院学者)\\
时间:2024年10月16日(周三)16:00-17:30\\
链接:\url{https://mp.weixin.qq.com/s/GdLiHo4BisGj0nZN8220Uw}\\

3.中美博弈及其对世界秩序的冲击\\
题目:The Sino-US Competition and Its Impact on the World Order\\
主讲人:赵全胜 ZHAO Quansheng\\
主讲人介绍:现任美国美利坚大学国际关系学院教授、美国美利坚大学亚洲研究理事会主席,同时还是北京大学中外人文交流研究基地学术委员和美中关系全国委员会委员\\
主持人:华涛 HUA Tao(南京大学历史学院教授,中美文化研究中心历史学教授)\\
时间:2024.10.15(周二)Tuesday,October 15,2024,北京时间 18:30-20:00\\
地点:中美文化研究中心A106会议室\\
注:本讲座使用中文\\
链接:\url{https://mp.weixin.qq.com/s/uwIonwHxieMbeO3A7M8ejg}\\

4.跨代性与跨代平衡问题的哲学思考\\
主讲人:Tiziana Andina(都灵大学哲学系教授)\\
时间:2024年10月16日10:30\\
地点:哲学学院(薛光林楼)402室\\

\section{小蓝鲸草地音乐节表演阵容大揭秘}
10月13日19:00,在鼓楼校区苏浙体育场,2024年小蓝鲸草地音乐节将如期而至。校园十大歌星赛歌手与学生社团的舞者携手亮相,歌手、社团与主持天团的豪华阵容参见\url{https://mp.weixin.qq.com/s/dO1-0ynFP60UaKJE3q8cRw}\\
除此之外,学生会还为NJUers精心准备了抽奖环节,将在音乐节当天12:00开奖。转发该微信公众号推文,集赞满30个并参与抽奖,就有机会获得精美礼品一份。抽奖方式见推文链接。
\section{关于2024小蓝鲸草地音乐节操场使用情况的说明}
南京大学学生会将于2024年10月13日19:00在鼓楼校区苏浙体育场举办“和鸣盛世曲,奋进逐鲸程”2024年小蓝鲸草地音乐节。\\%,携校园歌星和各大社团,为您奉上一场不虚此行的视听盛会!\\
10月12日12:00-21:00以及10月13日13:00-22:00,学生会需要借用苏浙体育场进行布台、彩排和演出,向同学们表示歉意。\\
%考虑到草地音乐节演出的特殊性,10月12日12:00-21:00以及10月13日13:00-22:00,学生会需要借用苏浙体育场进行布台、彩排和演出。从设备调试到演出结束期间,学生会将尽可能在保证活动顺利、高质量开展的情况下降低音量,减少对您生活、学习的干扰。学生会对因音乐节开展而给您带来的不便表示歉意,希望能得到您的理解与支持;同时,也希望您能广为告知,并与您的朋友提前调整安排。学生会也盛情邀请您,共赴这场初秋音乐之约,印刻下独特的校园记忆!\\\\
%南京大学学生会
%\section{南京大学物理学院2025年度“申请-考核制”博士招生师生互选信息交流平台启用通知}
%有意向报考物理学院“申请-考核制”博士研究生的同学请看详情\url{https://mp.weixin.qq.com/s/mmVJb6954ng6svoiQ3OMWw}
\section{“缘聚八院 谊留长远”——八院联谊活动}
计算机学院、化学化工学院、外国语学院、文学院、信息管理学院、哲学学院、人工智能学院、马克思主义学院,八大学院携手举办联谊活动。\\
有chiikawa毛绒拖鞋、卡皮巴拉小夜灯、花束积木、公仔玩偶等奖品。\\
活动时间与内容:\\
10.19 14:00-17:30 精彩游戏节\\
10.19 19:00-22:30 趣味桌游会\\
10.21-10.27 “七日情侣”组队打卡活动\\
活动地点:计算机科学与技术楼\\
参与对象:八学院全体在读研究生\\
报名截止时间:10月14日 24:00\\
有意报名的同学可以进入链接填写问卷并加群\url{https://mp.weixin.qq.com/s/qL3QcbiRtn9hVpnUTN_H9w}
\section{篮协院系杯小组赛}
日期:10月11日\\
地点:仙林校区一组团篮球场\\
12:30-14:00 商院vs计科\\
18:30-20:00 外院vs新传
\section{“南京论坛2024”招募志愿者}
“南京论坛2024”——以国际交流合作为主题的专业论坛,将于2024年10月18日开幕。该论坛目前迄今已举办六届,是江苏省和南京市的一张重要国际名片。\\
现招募设备调试、维护、管理人员4人,文字工作16人,会场接待与会务安排10人。\\
参与方式:点击链接\url{https://table.nju.edu.cn/dtable/forms/e8952881-79e3-4336-98de-c0893e8cc98e/}填写问卷,并加入QQ群:527671370。\\
\section{遴选2024年全国大学生遵义会议精神志愿宣讲团}
遴选对象:面向全国高校遴选1000支大学生遵义会议精神志愿宣讲团队,每支团队招募10名志愿者。\\
服务时间:2024年10月—2025年4月集中开展,后续将常态化开展志愿宣讲。\\
具体信息请关注推文\url{https://mp.weixin.qq.com/s/fkDDcu8_AiGtND6WBOzaww}。\\截止时间:10.27.
\section{马院志愿服务队招新}
南京大学马克思主义学院“南马微芒志愿服务队”招募志愿者,计划线上线下相结合地在学校与社区进行红色宣讲。读者可扫描附录中的二维码加入交流群、填写报名表。报名截止日期为10月10日。具体活动内容详见:\url{https://mp.weixin.qq.com/s/taN9CMmpDa6iuzD5EzJOPw}。
\section{历史学院Photoshop技能培训}
本次培训计划于2024年10月13日20:00—21:00以线上会议的方式进行。培训面向历史学院全体学生。欢迎所有对宣传工作感兴趣、想提升自身宣传技能的同学们可参与腾讯会议号:925-501-116
\section{“南大新传”编辑部招新}
“南大新传”微信公众号编辑部招募采写、美工、运营相关同学,不限年级、专业。读者可以扫描附录中的二维码加入交流群、填写报名表。报名截止日期为10月20日。往期作品等请阅:\url{https://mp.weixin.qq.com/s/k_812eccWy-lDs3mR70bVA}。\\
本报编辑注:有且仅有新闻学专业本科生可以从中获得学分。
\section{NOVA招新}
南京大学学生智能数据决策工作室——NOVA招新开启。加入NOVA,你将获得——\\
1. 实践与挑战:你将参与到校园卡自助平台、室友匹配系统、智能问答平台等实际项目中。每一个项目都是一个全新的挑战,每一次突破都将是你成长的见证。\\
2. 顶尖技术指导:优秀学长学姐共同指导,你将获得最前沿的技术培训,接触AI、大数据等热门领域,真正让理论与实践结合起来,提升你的技术能力。\\
3. 广阔的发展空间:无论你是对技术开发、数据分析、产品设计、活动策划或其他方向感兴趣,NOVA都将为你提供广阔的舞台。\\
详见\url{https://mp.weixin.qq.com/s/Nu9AMVnmUpa-G6zDg1hIjw}
\end{multicols} 

\hrule
\vspace{4mm}
% APPENDIX BEGIN
\centerline{\huge\textbf{附录}}
\begin{figure}[htbp]
    \centering
    \begin{minipage}[b]{0.32\textwidth}
        \centering
        \includegraphics[width=0.5\textwidth]{Group_Ma.png}
        \caption{马院志愿服务队群聊}
    \end{minipage}
    \begin{minipage}[b]{0.32\textwidth}
        \centering
        \includegraphics[width=0.5\textwidth]{Questionnare_Ma.png}
        \caption{马院志愿服务队报名问卷}
    \end{minipage}
    \begin{minipage}[b]{0.32\textwidth}
        \centering
        \includegraphics[width=0.5\textwidth]{Group_JC.png}
        \caption{“南大新传”招新交流群}
    \end{minipage}
    \begin{minipage}[b]{0.32\textwidth}
        \centering
        \includegraphics[width=0.5\textwidth]{Questionnare_JC.png}
        \caption{“南大新传”报名问卷}
    \end{minipage}
\end{figure}
\end{document}