% HEAD BEGIN
\documentclass[letterpaper, 12pt]{article}
\newsavebox\colbbox
\usepackage{graphicx}
\usepackage{multicol}
\usepackage{anysize}
\usepackage{fontspec}
\usepackage[fontset=none]{ctex}
\usepackage{tabularx}
\usepackage{longtable}
\PassOptionsToPackage{hyphens}{url}
\usepackage[breaklinks=true, colorlinks=true]{hyperref}
\expandafter\def\expandafter\UrlBreaks\expandafter{\UrlBreaks\do\a\do\b\do\c\do\d\do\e\do\f\do\g\do\h\do\i\do\j\do\k\do\l\do\m\do\n\do\o\do\p\do\q\do\r\do\s\do\t\do\u\do\v\do\w\do\x\do\y\do\z\do\A\do\B\do\C\do\D\do\E\do\F\do\G\do\H\do\I\do\J\do\K\do\L\do\M\do\N\do\O\do\P\do\Q\do\R\do\S\do\T\do\U\do\V\do\W\do\X\do\Y\do\Z}
% \let\oldurl\url
% \renewcommand{\url}[1]{\begin{sloppypar}\oldurl{#1}\end{sloppypar}}
\setlength\columnsep{30pt}
\marginsize{30pt}{30pt}{10pt}{20pt}
\setmainfont{TeX Gyre Bonum}
\setCJKmainfont[BoldFont=Noto Serif CJK SC Bold, ItalicFont=FandolKai]{Noto Sans CJK SC}
\setlength{\parindent}{0cm}
% \setCJKmonofont{Noto Sans CJK SC}
\begin{document}
\begin{center}
    \Huge\textbf{南哪大专醒前消息}
\end{center}
\vspace{4mm}
\hrule
\renewcommand\tabularxcolumn[1]{m{#1}}
\begin{tabularx}{\textwidth}{>{\hsize.2\hsize}X>{\hsize.6\hsize}X>{\hsize.2\hsize}X}
    \begin{flushleft}
        2025.2.22\, No.172
    \end{flushleft}
    &
    \begin{center}
        \textit{“秉中持正、求新博闻。”}
    \end{center}
    &
    \begin{flushright}
        \textbf{南京市栖霞区}
    \end{flushright}
\end{tabularx}
\vspace{-3.5mm}
\hrule
\vspace{4mm}
% HEAD END
\centerline{\huge\textbf{活动预告}}
\begin{multicols}{2}
    \section{订阅方式和加入编辑部}  
编辑部招聘人才,用爱发电,工作轻松,详情可联系QQ:1329527951 客服小祥\\想订阅本消息或获取PDF版(便于查看超链接和往期),可加QQ群:\href{https://qm.qq.com/q/VXIW7fgsEe}{849644979}.
\section{Deadline Ongoing}
\setbox\colbbox\vbox{
\makeatletter\col@number\@ne
\begin{longtable}{|c|c|c|}
    \hline
    消息(未见ddl的,不刊) & 截止日期 & 刊载日期\\
    \hline\hline
    南大版deepseek & / & 2.22\\
    天文台开放日 & / & 1.6\\
    原创剧本联合孵化报名 & 3.20 & 1.10\\
    njumun代表报名 & 3.2 & 1.16\\
    生科论文沙龙 & 2.22 & 2.6\\
    返校注册 & 2.23 & 2.14\\
    课程补退选 & 3.2 & 2.19\\
    南大育教新媒体招新 & 2.27 & 2.19\\
    本科生劳育实践 & 7.20 & 2.19\\
    医保零星报销 & 3.31 & 2.19\\
    信息通信产业链专项赛 & 2.23 & 2.19\\
    金法槌杯模拟法庭大赛 & 2.23 & 2.19\\
    第二届大学生阅读分享活动 & 3.7 & 2.21\\
    心理中心助理招新 & 2.28 & 2.20\\
    招办全媒体招新 & 3.5 & 2.20\\
    交响乐团招新 & 3.7 & 2.20\\
    歌魅剧务招募 & 2.26 & 2.21\\
    萌马音乐工作室招新 & 2.28 & 2.22\\
    
    \hline
\end{longtable}
\unskip
\unpenalty
\unpenalty}\unvbox\colbbox
\end{multicols}
\hrule
\pagebreak
\begin{multicols}{2}

\section{讲座}
\begin{tabular}{|>{\centering\arraybackslash}m{.3\textwidth}|m{.06\textwidth}|m{.06\textwidth}|}
    \hline
    讲座 & 开展时间 & 刊载时间\\
    \hline\hline
    人机协同背景下高等外语教育的守正创新 & 2.27 & 2.17\\\hline
    大陆的起源 & 3.4 & 2.17\\\hline
    年号勘文中所见日本的类书利用 & 2.24 & 2.20\\\hline
    中国中古《孙子算经》在日本的受容 & 2.25 & 2.20\\\hline
    南京“世界文学之都”的前世今生 & 2.27 & 2.20\\\hline
    复杂异构大数据治理与分析关键技术及应用 & 2.25 & 2.20\\\hline
    香港大学经管学院硕士课程高校专场线下宣讲会 & 2.27 & 2.20\\\hline
    电子平带材料中的关联与拓扑 & 2.25 & 2.21\\\hline
    因明与逻辑文化学 & 2.24 & 2.21\\\hline
    2025香港大学暑期课程宣讲会 & 2.26 & 2.21\\\hline
\end{tabular}


\section{南大版DeepSeek上线}
南京大学信息化中心正式部署多款大语言模型供师生使用,包括Qwen,DeepSeek V3,DeepSeek R1等。校园网环境访问\url{https://chat.nju.edu.cn}并登录后即可使用。详见\url{https://mp.weixin.qq.com/s/pFgOQW4xTCQaDDiTxrbBLA}\\(有部分模型并非DeepSeek研发,但原标题如此,这里原样呈现)

\section{ 萌马音乐工作室招新}
萌马音乐工作室是一个由南京大学在读大学生组成的音乐团体,成员包括全校最顶尖的乐手和歌手。工作室在2015年即推出首张专辑《以梦为马》,并在黑匣子剧场首次举办“十萬人”演出。此后“十萬人”系列在恩玲剧场等地一次次奏响,成为众多南大毕业生重要的毕业回忆,也成为校园音乐文化的重要品牌。萌马不仅是一个音乐团队,也是一群志同道合者的音乐梦想。本次招募的职位包括:活动策划、平面设计、摄影摄像、视频制作及新媒体运营等。具体工作内容、工作要求及报名方式详见公众号文章。报名截止时间为2月28日。\\
链接:\url{https://mp.weixin.qq.com/s/JzUG18coqsix4h3i-1ZSdw}\\

\section{暑期项目 | 2025东吴大学暑期研修班}
东吴大学将于2025年7月、8月开设为期3周的「暑期研修班」。\\
招生对象:南大在读本科生、研究生\\
课程时间:2025年7月20日(周日)至2025年8月10日(周日)\\
住宿地点:东吴大学外双溪校区学生宿舍\\
上课地点:东吴大学外双溪校区
费用总额(不含往返机票费及在台生活费)68,000新台币\\
报名方式:扫描二维码填写报名信息,后续将由台港澳办视报名情况组织相关遴选,请等待进一步通知。\\
详情见推文:\url{https://mp.weixin.qq.com/s/8n4TpMoe7TfhlYd70fmAXQ}

\section{暑期项目 | 2025澳门大学科技学院优秀大学生暑期研习营}
课程时间:2025年7月8日至7月10日(三天)\\
报名条件\\
1. 就读于土木及环境、电脑及资讯科学、电机及电脑工程、机电工程、数学、海洋科学与技术等相关专业;\\
2. 成绩平均80分以上,或班级排名前30\%,或4分制下GPA≥3;\\
3. 有意报读2026/2027学年澳门大学科技学院硕士或博士课程。\\
符合以下任一项要求将有机会优先录取:\\
1) 获澳大学系教授推荐;\\
2) 获科技学院合作的大学推荐;\\
3) 来自国际知名院校及“双一流”建设高校的优秀学生。\\
费用信息:由澳门大学提供夏令营期间在校住宿费,以及7月9日的欢迎午餐。\\
自费部分:\\
• 进出澳门的有效旅游证件、签证费用及其他相关费用(详情请参阅澳门入境处的入境须知)
• 往返澳门的交通安排及其相关费用• 自由活动时的所有费用• 外游保险费用\\
报名方式:扫描二维码填写报名信息,后续将由台港澳办视报名情况组织相关遴选,请等待进一步通知。\\
详情见推文:\url{https://mp.weixin.qq.com/s/U-vivEQiH8Q-VV8CySh4zw}




\end{multicols} 
\hrule
\vspace{4mm}
\centerline{\huge\textbf{参考消息}}
\begin{multicols}{2}
\section{南哪消息同学小文连载板块}
因收到小说投稿一篇,南哪消息现在开辟了同学小文连载板块。如想评论,可以发至邮箱:1329527951@qq.com,第二天会刊在此处。如想投稿渠道相同。
\section{《等待,遗忘》(1)}
金映樺\\

第一天

  从信箱中拿出昨天的China daily时,天还是灰的。她想到读过的“你灰发的书拉密”,是了,都一样的,灰不是灰色的灰,而是灰烬的灰。毁灭的气味,赤裸地衰颓。奇怪的是,报纸页眉的底色是一种饱和度很高的蓝色,而这颜色莫名令她熟悉。在哪里看到过?她闭上眼,努力回忆,眉头因此无意识地皱起。短暂的静默后她选择放弃:你要知道放弃是一种美德,正如失败是成功之母云云。\\
  
  她在餐桌前坐下,将报纸摊开,铃声同时响起。还记得初中上补习班,路有些远,但是因为正值叛逆期偏要特立独行,她拒绝了安排好的司机接送,要求自己走路上下学。手机往往被扔在双肩包里,随着前进的步伐颠来倒去。当她终于开始厌倦,或是后悔,电话铃声便会如约响起。就不去管它,让它响着,这很酷,反正她从前是那么想的,也只会是父母来电询问抵达与否。通话会持续多久?两句话还是三句?总归不会比响铃的时间更长。但如今,她早已过了叛逆期,变得温驯,遂丧失了拒绝的权利——她必须将手机从包中拿出来了。\\
  
  “我在吃早餐呢...”
  
  “没有...只是起晚了...”
  
  “我说了我不想......我不是......“
  
  ”...好,我知道了...”
  
  “我会去的。”\\
  
  她微微叹了口气,等着母亲将通讯挂断,等待,必须等待,她从小被教育这是礼貌。中午12:00,F餐厅。时间真的是很廉价的东西,至少她的时间如此。如此轻易地被交付,如此任性地被占用。主观意愿是脆弱的,不堪一击,就好像她明明说了“我不想“,却还是必须去。她刹那间想抛开所受的一切教育指导不谈,大骂一句”去他的朋友家儿子“,这是否会让她更轻松?但是不行,至少她不行,因为要慎独。“君子慎独,不欺暗室。卑以自牧,含章可贞。”从古至今,人们总固执地认定孩子是一张白纸,定然喜听大人之言,所以谕教宜早。可是,为什么白一定是不谙世事、天真稚嫩的代名词,为什么世俗美丑都被过早地框定,而她只能日复一日地循规蹈矩?她下意识地咬住下唇,那唇于是毫无血色,终苍白如纸了。\\
  
  收起报纸,她忽然想到究竟是在哪里看到这样的蓝色了。是去年新出的策兰诗集,以那首最著名的诗为题。出版社总要给书取上一个名字:以作家代表作为题、以一句脍炙人口的名句为题、以翻拍成著名电影的文章为题。她无意识地想起还在上学的时候,每天早上父亲总是早早地坐在餐桌边,啜饮着属于他的茶。他会看看报纸,往往只看财经版,因为头条只会徒添烦恼,带不来更大的财富。当父亲喝完那杯茶,报纸便也看完了,而她得出门了。这时父亲总会发表一番对今日新闻的看法,常用词是:资本、政策以及价值。是了,资本家。所以才要取各种名字来诱导读者消费;sacrilegious,她想,是对文本的亵渎。\\
  
  不,不,问题不在这儿。问题在于:为什么是蓝色的?可以是黑的白的灰的棕色的为什么偏偏是蓝色?深沉,她想,该是一种可以容纳尤其多的阴影,吞吐出光芒的颜色。其实用金色也挺不错,“你金发的玛格丽特”,一举多得的好事,关联了书名诗题和封面颜色。她叹了口气。\\
  
  她忽然想到,如果这是一篇作文,老师会告诉她:“啜饮那杯茶”可以换成,“啜茗”。\\
  
  微信上母亲的消息不断,反复提醒时间,委婉地建议她穿上次自己代买的套装。母亲永远是温婉的,不会直接说这是错的那不对,她会无言地看着你,眼中酝酿着的失望让她无力。她会回复“好的”,简单的两个字,但她也必须如此回复。\\
  
  套装是黑色的,简约而不失时尚,来自母亲钟爱的品牌。她懊恼地换下白色T恤和牛仔裤,将自己托付给她尚不熟悉的衣物。兴许是被提点过太多次,她不自觉地挺直了腰,一扫方才的慵懒;她将头发盘起,因为母亲曾教她说要露出脖颈才会显得人精神。看向镜中的自己,效果意外的不错。只是,她惶恐地发现,自己和多年前的母亲有八成相似。\\
  
司机已经在门外等了。她微笑着打了声招呼,“王叔”。坐进车里,许久不闻的香水味让她有些想吐。是太过浓郁而令人恶心,还是太过熟悉而令人烦躁,不是那么容易分辨,此刻她也无心去分清。车在红灯前缓缓停下。恍惚间她想自己该多相信一点所谓因果所谓代价,或者用科学的说法,物质守恒能量守恒。果真得到什么也会失去些什么。她不知怎么的想起了博尔赫斯。“时间有无数系列,背离的、汇合的和平行的时间织成一张不断增长、错综复杂的网。”“由互相靠拢、分歧、交错或者永远互不干扰的时间织成的网络包含了所有的可能性。”\\

那么,请允许我说:在大部分时间里,我们并不存在;在某些时间,有他而没有我;在另一些时间,有我而没有他;再有些时间,我们都存在。目前这个时刻,无可抗的必然使我来到这里;在另一个时刻,无论是谁穿过了花园,都会发现我已死去:再在另一个时刻,我说着目前所说的话,不过我是个错误,是个幽灵。\\

  车停在F餐厅前,她搭着司机的手迈出了车门。\\
  
  11:50。黄金十分钟。她想,既然都说情比金坚,那么,就这一次也好,不管是谁,请让我爱上他,俗气地热烈地牺牲般地无可救药地爱上他。\\
  
  她已经看到了结局:母亲会挑一个好日子,让她嫁给他。

\end{multicols} 

\end{document}
