% HEAD BEGIN
\documentclass[letterpaper, 12pt]{article}
\newsavebox\colbbox
\usepackage{graphicx}
\usepackage{multicol}
\usepackage{anysize}
\usepackage{fontspec}
\usepackage[fontset=none]{ctex}
\usepackage{tabularx}
\usepackage{longtable}
\PassOptionsToPackage{hyphens}{url}
\usepackage[breaklinks=true, colorlinks=true]{hyperref}
\expandafter\def\expandafter\UrlBreaks\expandafter{\UrlBreaks\do\a\do\b\do\c\do\d\do\e\do\f\do\g\do\h\do\i\do\j\do\k\do\l\do\m\do\n\do\o\do\p\do\q\do\r\do\s\do\t\do\u\do\v\do\w\do\x\do\y\do\z\do\A\do\B\do\C\do\D\do\E\do\F\do\G\do\H\do\I\do\J\do\K\do\L\do\M\do\N\do\O\do\P\do\Q\do\R\do\S\do\T\do\U\do\V\do\W\do\X\do\Y\do\Z}
% \let\oldurl\url
% \renewcommand{\url}[1]{\begin{sloppypar}\oldurl{#1}\end{sloppypar}}
\setlength\columnsep{30pt}
\marginsize{30pt}{30pt}{10pt}{20pt}
\setmainfont{TeX Gyre Bonum}
\setCJKmainfont[BoldFont=Noto Serif CJK SC Bold, ItalicFont=FandolKai]{Source Han Sans SC}
\setlength{\parindent}{0cm}
% \setCJKmonofont{Noto Sans CJK SC}
\begin{document}
\begin{center}
    \Huge\textbf{南哪大专醒前消息}
\end{center}
\vspace{4mm}
\hrule
\renewcommand\tabularxcolumn[1]{m{#1}}
\begin{tabularx}{\textwidth}{>{\hsize.2\hsize}X>{\hsize.6\hsize}X>{\hsize.2\hsize}X}
    \begin{flushleft}
        2025.4.30\, No.234
    \end{flushleft}
    &
    \begin{center}
        \textit{“秉中持正、求新博闻。”}
    \end{center}
    &
    \begin{flushright}
        \textbf{南京市栖霞区}
    \end{flushright}
\end{tabularx}
\vspace{-3.5mm}
\hrule
\vspace{4mm}
% HEAD END
\centerline{\huge\textbf{活动预告}}
\begin{multicols}{2}
\section{订阅方式和加入编辑部}  
编辑部招聘人才,用爱发电,工作轻松,详情可联系QQ:1329527951 客服小千\\想订阅本消息或获取PDF版(便于查看超链接和往期),可加QQ群:\href{https://qm.qq.com/q/4HL41Nt3sQ}{466863272}.
\section{活动清单}
\setbox\colbbox\vbox{
\makeatletter\col@number\@ne
\begin{longtable}{|>{\centering\arraybackslash}m{.3\textwidth}|m{.06\textwidth}|m{.06\textwidth}|}
    \hline
    活动 & 开展时间 & 刊载时间\\
    \hline\hline
    南大版deepseek & / & 2.22\\
    悦读课程群 & / & 2.24\\
    eScience AI科研助手 & / & 3.11\\
    地科博物馆开放安排 & / & 3.22\\ 
    2025年分流和转专业政策通知 & / & 4.7\\
    2025年转专业志愿填报通知 & / & 4.24\\
    乐跑 & 5.16 & 3.10\\
    本科生劳育实践 & 7.20 & 2.19\\
    大文大理题目征集 & 期末 & 3.8\\
    5月免费上网 & ? & 3.9\\
    外教社杯 & 5.27 & 3.12\\
    浦口音乐跑 & 5.30 & 3.31\\
    仙林校区志愿法律咨询 & / & 4.4\\
    青春活力大赛 & 5.17 & 4.7\\
    在校生自愿体检 & 6.20 & 4.8\\
    南大购买WPS & / & 4.8\\
    中美中心2025年证书项目 & 5.24 & 4.14\\
    粤歌赛决赛 & 5.10 & 4.21\\
    汉字文化技能大赛 & 5.4 & 4.21\\ 
    校博岩画展 & 6.22 & 4.23\\
    CASHL“畅读”活动 & 5.23 & 4.24\\
    江苏高校凤凰读书节 & 6.15 & 4.24\\
    图书馆征集春日影 & 5.10 & 4.28\\
    汉字知识竞赛 & 5.4 & 4.28\\
    在校生体检 & 5.7 & 4.29\\
    无偿献血 & 5.9 & 4.29\\
    新生创意大赛 & 5.5 & 4.29\\

    \hline
\end{longtable}
\unskip
\unpenalty
\unpenalty}\unvbox\colbbox
\end{multicols}
\begin{multicols}{2}
\pagebreak

\section{讲座}
\begin{tabular}{|>{\centering\arraybackslash}m{.3\textwidth}|m{.06\textwidth}|m{.06\textwidth}|}
    \hline
    讲座 & 开展时间 & 刊载时间\\
    \hline\hline
    从语言到智能 ⸺ 大语言模型的奥秘与应用 & 5.6 & 4.16\\\hline
    儿童脑智发育与人口神经科学 & 5.14 & 4.30\\\hline
    Social Simulation with Large Language Model-based Agents & 5.8 & 4.29\\\hline
    Dark Side of AI Use: Firm-Level Evidence on Corporate Performance & 5.8 & 4.30\\\hline
    基于专利大数据的全球创新集群识别与全球创新网络分析 & 5.8 & 5.1\\\hline
    中国发明专利对全球专利引用研究 & 5.8 & 4.30\\\hline
    技术与产品关联视角下的区域产品创新 & 5.8 & 4.30\\\hline
    Short-selling Profitability, Stock Lending Fees, and Asset Pricing Anomalies & 5.8 & 4.30\\\hline
    BIM创新主题演化及技术成熟度分析 & 5.7 & 4.30\\\hline
\end{tabular}
\subsection{Dark Side of AI Use: Firm-Level Evidence on Corporate Performance} % 讲座 describer: Ando
主讲人:肖亚军  副教授
\\主持人:李昊骅  副教授
\\时间:5月8日(周四)10:00-12:00
\\地点:北楼105报告厅
\\详见:\url{https://mp.weixin.qq.com/s/sfSyxpEeIx7Eg9PXbimV6g}

\subsection{尹德云 | 基于专利大数据的全球创新集群识别与全球创新网络分析} % 讲座 describer: Jolly
主题:基于专利大数据的全球创新集群识别与全球创新网络分析
\\报告人:尹德云 哈尔滨工业大学(深圳)经济与管理学院副教授
\\时间:2025年5月8日 10:00
\\地点:安中楼1412 
\\详见:\url{https://mp.weixin.qq.com/s/SsHEY1vRAZgQolT_n4gA3g}

\subsection{史冬波 | 中外技术创新互动:中国发明专利对全球专利引用研究} % 讲座 describer: Jolly
主题:Chinese Patent Front-page and In-text Citations to Worldwide Patents(中外技术创新互动:中国发明专利对全球专利引用研究)
\\报告人:史冬波 上海交通大学国际与公共事务学院副教授
\\时间:2025年5月8日 12:30
\\地点:安中楼1412 
\\详见:\url{https://mp.weixin.qq.com/s/IycBaUxPY6-wK6isQpXl-Q}

\subsection{党建伟 | 技术与产品关联视角下的区域产品创新} % 讲座 describer: Jolly
主题:技术与产品关联视角下的区域产品创新
\\报告人:党建伟 同济大学上海国际知识产权学院副院长、副教授
\\时间:2025年5月8日 14:00
\\地点:安中楼1412 
\\
\\详见:\url{https://mp.weixin.qq.com/s/NFSkF9e4TP6LMZUfFq6KIQ}

\subsection{Short-selling Profitability, Stock Lending Fees, and Asset Pricing Anomalies} % 讲座 describer: Hikari
报告人:傅成博 副教授
\\主持人:苏彤 副研究员
\\时间:5月8日(周四) 14:30-16:30
\\地点:北楼105报告厅
\\详见:\url{https://mp.weixin.qq.com/s/KnAnbFFM8WtiP_WoGbZOiA}

\subsection{BIM创新主题演化及技术成熟度分析} % 讲座 describer: Hikari
报告人:王新成 助理教授
\\主持人:孙大鑫 副研究员
\\时间:5月7日(周三)10:00-12:00
\\地点:北楼105
\\详见:\url{https://mp.weixin.qq.com/s/ZkA9kHnLu7hQ-ppKOJQULw}

\section{院级活动}
\begin{tabular}{|>{\centering\arraybackslash}m{.3\textwidth}|m{.06\textwidth}|m{.06\textwidth}|}
\hline
    活动 & 开展时间 & 刊载时间\\
    \hline\hline
    文院剧本创作研讨会 & 9.30 & 3.2\\
    物院征集课程指南 & 6.15 & 3.3\\
    地海征集春日影 & 6.15 & 3.14\\
    法院党建征文 & 5.20 & 4.2\\
    四院音乐节 & 5.11 & 4.7\\
    商院征集 & 5.5 & 4.8\\
    物院运动打卡 & 5.14 & 4.12\\
    电院征集 & 5.11 & 4.22\\
    智院摄影 & 5.6 & 4.22\\
    商院征集 & 5.9 & 4.27\\
    物院研讨会 & 5.11 & 4.30\\
    健雄摄影 & 5.20 & 4.30\\
    工管学实杯 & 5.8 & 4.30\\
    开甲许愿 & 5.10 & 4.30\\
    \hline
\end{tabular}
\subsection{研讨会预告:聚焦光学,“精测”未来} % 院级活动 describer: Ando
时间: 5月11日(星期日)下午 15:00
\\地点: 物理楼356
\\报告主题: 光学超精密检测技术及应用
\\主讲人: 闫力松(华中科技大学 / 上海精测半导体)
\\请扫描下方二维码或点击链接,填写问卷完成报名。
\\详见:\url{https://mp.weixin.qq.com/s/HlcVG1a2muAIomzGpma8Mg}

\subsection{健雄书院原创摄影\&Vlog作品大赛邀你共赴创作之约} % 院级活动 describer: Noname
摄影\&Vlog作品主题不限,可以围绕自然与季节、劳动与成长、温馨情感、青春与活力、文化与节日、健康生活、创意与艺术等主题
\\征集对象 :
\\健雄书院全体师生
\\征集时间 :
\\即日起至2024年5月20日23:59
\\优秀作品将在书院公众号专题展播,并获赠定制神秘礼品,其他参赛者以“美育”录入五育系统,计入敦行成绩单。
\\详见:\url{https://mp.weixin.qq.com/s/7xgrASEFSzb6UMnGNc5puA}

\subsection{南京大学第四届“学实杯”学术论文竞赛截稿时间提醒} % 院级活动 describer: Hikari
考虑到部分同学需求,将征稿截止时间延期至5月8日晚22:00
\\01 参赛流程
\\1. 投稿截止日期:2025年5月8日晚22:00
\\2. 专家评审阶段:2025年5月9日-5月18日
\\3. 现场展示及颁奖:暂定2025年5月21日(周三)
\\02 论文提交方式
\\请参赛者将学实杯参赛信息登记表、参赛论文、诚信承诺书(含电子签名)于截止日期前通过电子邮件提交至“学实杯”论文竞赛投稿邮箱。
\\邮件命名为:“第四届学实杯投稿 + 组别 + 论文类别代号 + 论文题目 + 第一作者所在学院 + 第一作者姓名”
\\例:“第四届学实杯投稿-硕士研究生-F01-论文题目-工程管理学院-姓名”
\\本届“学实杯”学术论文竞赛投稿邮箱为:njuxueshibei2025@163.com
\\点击“阅读原文”查看附件
\\详见:\url{https://mp.weixin.qq.com/s/KYMD1pGly6Z88VupXup8Vg}
\subsection{开甲书院第四届“新语心愿”活动启动啦} %  describer: Noname
【许愿人】(想要许愿的同学)在心愿池匿名投放心愿。【心愿实现人】(想要实现他人愿望的同学)抽取心愿、实现心愿。【许愿人】反馈活动感受。
\\【许愿人】许愿在即日起至5月10日24:00前。
\\【心愿领取人】需在5月10日24:00前领取心愿,领取心愿后一周内完成许愿人心愿。
\\【许愿人】需于心愿实现3天内向工作组完成活动反馈。
\\活动计入“敦行成绩单”【美育】项目。
\\详见:\url{https://mp.weixin.qq.com/s/JA9A2Jmhj8NFauTI84crVQ}
\section{社团活动}
\begin{tabular}{|>{\centering\arraybackslash}m{.3\textwidth}|m{.06\textwidth}|m{.06\textwidth}|}
    \hline
    社团活动 & 开展时间 & 刊载时间\\
    \hline\hline
    天文台开放日 & / & 1.6\\
    红会一块走 & 5.20 & 4.21\\
    集庆折子戏 & 5.7 & 4.22\\
    九歌大会 & 5.11 & 4.27\\
    红会图书角 & 5.7 & 4.29\\
    街舞社路演 & 5.9 & 4.29\\
    排协鼓楼杯报名 & 5.5 & 4.29\\
    紫藤钱币研学 & 5.11 & 4.30\\
    \hline
\end{tabular}
%这里是写社团活动的,社团活动就是由社团主办、主要针对社团内部人员的活动。不要把非社团活动写在这里。

\subsection{古迹探寻|古钱币研学活动预告} % 社团活动 describer: Hikari
活动时间:5月11日下午2:00
\\活动地点:永银钱币博物馆
\\报名方式:扫描二维码添加活动群
\\(具体安排见后续通知)
\\详见:\url{https://mp.weixin.qq.com/s/0ycB8jlJp7xfC4jHJk-_KQ}
\end{multicols}
\end{document}
