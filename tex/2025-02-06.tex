% HEAD BEGIN
\documentclass[letterpaper, 12pt]{article}
\newsavebox\colbbox
\usepackage{graphicx}
\usepackage{multicol}
\usepackage{anysize}
\usepackage{fontspec}
\usepackage[fontset=none]{ctex}
\usepackage{tabularx}
\usepackage{longtable}
\usepackage{ulem}
\PassOptionsToPackage{hyphens}{url}
\usepackage[breaklinks=true, colorlinks=true]{hyperref}
\expandafter\def\expandafter\UrlBreaks\expandafter{\UrlBreaks\do\a\do\b\do\c\do\d\do\e\do\f\do\g\do\h\do\i\do\j\do\k\do\l\do\m\do\n\do\o\do\p\do\q\do\r\do\s\do\t\do\u\do\v\do\w\do\x\do\y\do\z\do\A\do\B\do\C\do\D\do\E\do\F\do\G\do\H\do\I\do\J\do\K\do\L\do\M\do\N\do\O\do\P\do\Q\do\R\do\S\do\T\do\U\do\V\do\W\do\X\do\Y\do\Z}
% \let\oldurl\url
% \renewcommand{\url}[1]{\begin{sloppypar}\oldurl{#1}\end{sloppypar}}
\setlength\columnsep{30pt}
\marginsize{30pt}{30pt}{10pt}{20pt}
\setmainfont{TeX Gyre Bonum}
\setCJKmainfont[BoldFont=Noto Serif CJK SC Bold, ItalicFont=FandolKai]{Noto Sans CJK SC}
\setlength{\parindent}{0cm}
% \setCJKmonofont{Noto Sans CJK SC}
\begin{document}
\begin{center}
    \Huge\textbf{南哪大专醒前消息}
\end{center}
\vspace{4mm}
\hrule
\renewcommand\tabularxcolumn[1]{m{#1}}
\begin{tabularx}{\textwidth}{>{\hsize.2\hsize}X>{\hsize.6\hsize}X>{\hsize.2\hsize}X}
    \begin{flushleft}
        2024.2.6\, No.166
    \end{flushleft}
    &
    \begin{center}
        \textit{“秉中持正、求新博闻。”}
    \end{center}
    &
    \begin{flushright}
        \textbf{南京市栖霞区}
    \end{flushright}
\end{tabularx}
\vspace{-3.5mm}
\hrule
\vspace{4mm}
% HEAD END
\centerline{\huge\textbf{活动预告}}
\begin{multicols}{2}
    \section{订阅方式和加入编辑部}  
编辑部招聘人才,用爱发电,工作轻松,详情可联系QQ:1329527951 客服小祥\\想订阅本消息或获取PDF版(便于查看超链接和往期),可加QQ群:\href{https://qm.qq.com/q/VXIW7fgsEe}{849644979}.
\section{Deadline Ongoing}
\setbox\colbbox\vbox{
\makeatletter\col@number\@ne
\begin{longtable}{|c|c|c|}
    \hline
    消息(未见ddl的,不刊) & 截止日期 & 刊载日期\\
    \hline\hline
    西安史学论坛征稿 & 3.20 & 12.9\\
    12306学生优惠票 & 2.12 & 12.13\\
    南大博物馆展览 & 6.16 & 12.17\\
    挂职干部选拔 & 2.13 & 12.31\\
    ASC25报名 & 2.21 & 1.6\\
    天文台开放日 & / & 1.6\\
    开甲书院科研作坊 & 2.17 & 1.6\\
    PL读书会 & 2.12 & 1.9\\
    春季选课 & 2.7 & 1.9\\
    原创剧本联合孵化报名 & 3.20 & 1.10\\
    阅读分享活动征稿 & 3.7 & 1.10\\
    njumun代表报名 & 3.2 & 1.16\\
    大气院寒假打卡 & 2.16 & 1.20\\
    冬季代码2025报名 & 2.8 & 1.26\\
    毓秀文创 & 2.20 & 2.6\\
    开甲支教志愿者 & 2.10 & 2.6\\
    计院摄影 & 2.16 & 2.6\\
    数院捐书 & 2.16 & 2.6\\
    生科论文沙龙 & 2.22 & 2.6\\
    \hline
\end{longtable}
\unskip
\unpenalty
\unpenalty}\unvbox\colbbox
\end{multicols}
\hrule
\pagebreak
\begin{multicols}{2}

\section{讲座}
\begin{tabularx}{0.5\textwidth}{|X|X|X|}
    \hline
    讲座 & 开展时间 & 刊载时间\\
    \hline\hline
/ & / & /\\\hline
\end{tabularx}

1.“诚计划”第137期:《以新质生产力增强发展新动能》\\
主讲人:沈坤荣教授 \\
时间:\sout{2025年2月11日(周二)晚19:30 - 21:00}因故暂停,择期再播\\
平台:南京大学“暾学堂”\\
主讲人介绍、内容简介、观看方式见推文:\url{https://mp.weixin.qq.com/s/e7kieQYghx01Qo9gRZipHw}\\
讲座延期公告:\url{https://mp.weixin.qq.com/s/s_qODDi-JbxkeXwD28335Q}




\section{南京大学毓琇书院 2025 年文创设计大赛}
参与对象:2024级以及历届毓琇书院学生\\
作品主题:本次文创设计大赛以“灵毓琇意,文创新程”为题,希望参赛作品能够围绕南大文化和新工科特色,展现毓琇学子奋勇争先、敢于进取的奋进精神。\\
作品要求:作品种类不限,鼓励同学们充分发挥想象力。包括但不限于帆布包、钥匙扣、明信片、胸针、书签、冰箱贴、贴纸、笔、文件夹、文化衫、鼠标垫……\\
截止日期:2025年2月20日\\
作品详细要求与提交方式等见推文:\url{https://mp.weixin.qq.com/s/sK5D7leEyIU_pOxQBMFfmQ}\\



\section{开甲书院“IT启明星”志愿服务活动第五期开始招募}
2025年将与南京大学第26届研究生支教团云南双柏分队合作,在云南省双柏县第一中学和双柏县妥甸中学开展线上教学,把计算机领域的知识带给云南省双柏县的孩子们,助力他们的未来发展。\\
活动内容\\
1.活动形式为线上教学\\
2.主题:与计算机或人工智能领域相关即可,志愿者可自由拟定。\\
志愿者要求\\
1. 招募对象:南京大学在校生,开甲书院、人工智能学院、计算机学院、软件学院等相关学院学生优先;\\
2. 人数需求:拟招募20-30名志愿者;\\
3. 教学内容需生动有趣,符合中学生心理和思维接受能力;\\
4. 志愿者需自行拟定课堂主题,在正式授课前完成相关内容的准备和课件的制作;\\
5. 直播过程中志愿者需要具备较好的临场应变能力;\\
6. 项目设置课程试讲环节,试讲通过后才可正式授课。\\
报名方式\\
1.扫描二维码加入QQ群\\
2. 点击下方链接填写报名表格,截止时间为2月10日24:00。\url{https://table.nju.edu.cn/dtable/forms/29397e68-974b-471c-99c3-570cc06421d7/}\\
 活动详细内容、志愿者招募要求见推文:\url{https://mp.weixin.qq.com/s/nVDMoRNdWcqwOrO_FLoL-A}\\


\section{计算机学院|“新春影,故乡情”摄影大赛}
征集时间:2.16晚23:59截止\\
征集对象:南京大学计算机学院全体师生\\
征集范围:包括但不限于家乡春节风俗、年夜饭、春节街景等,内容符合主题、积极健康即可(可以参考朋友圈的分享形式)。\\
图文格式:图片(1-9张,可以是拼接长图)+ 文字说明(可选项)\\
活动详情及参与方式见推文:\url{https://mp.weixin.qq.com/s/5os2m-1oC1EAhiQ-9BYNWw}\\

\section{“唤醒沉睡的书籍,传递温暖的情谊”——关于向渡江初中捐书的倡议}
时值寒假,南京大学数学学院拟启动第二批捐书活动,所捐书籍可由您所在地直接寄往渡江初中。若您家中有适合七-九年级孩子的闲置读物,欢迎捐赠,新旧不论。\\
活动时间:即日起至2月16日\\
活动QQ群号:747331847\\
具体捐书安排\\
1.书籍捐赠\\
捐赠对象:江西省赣州市龙南市渡江镇渡江初级中学图书资料室,阅读对象为七至九年级学生。\\
捐赠书目:不限,优先征集小说、科普、人物传记等适合初中生阅读的优质图书。新旧不限,如是旧书,希望较整洁、六成新以上。书目可参考教育部中小学生阅读指导目录的初中段部分(仅参考,不限此清单)\url{http://www.moe.gov.cn/jyb_xwfb/gzdt_gzdt/s5987/202004/W020200422556593462993.pdf}
捐赠方式:捐赠者自行邮寄并填写活动问卷(问卷链接在活动QQ群中公布),由数学学院报销邮费并奖励志愿时长。\\
2.书信交流\\
活动内容:捐赠者可为渡江初中的孩子们写一封信,分享个人成长故事或书籍推荐心得,与捐赠书籍一同邮寄给孩子们。\\
活动收获:将书信拍照记录,并填写信息收集问卷(问卷链接在活动QQ群中公布),可获得相应志愿时长奖励。优秀书信有机会在学院微信公众号等平台展示(在征得捐赠者意向的前提下),并奖励额外志愿时长。\\
3.喜迎新春\\
活动内容:捐赠者可为渡江初中的孩子们送上新春贺卡,喜迎蛇年,传递祝福,与捐赠书籍一同邮寄给孩子们。\\
活动收获:将贺卡拍照记录,并填写信息收集问卷(问卷链接在活动QQ群中公布),可获得相应志愿时长。优秀作品有机会在学院微信公众号等平台展示(在征得捐赠者意向的前提下),并奖励额外志愿时长。\\
活动背景及详细内容见推文:\url{https://mp.weixin.qq.com/s/L3hXZGjbyh--lcPapno9sA}\\

\section{生科院秉志科学沙龙初赛}
赛道主题:“创新与实践”\\
参赛者可自由组成1~4人的团队,参赛团队需根据自己的实验过程及结果完成论文,要求结论必须有实验支撑。\\
初赛阶段参赛队伍需要提交一篇论文,评委老师将根据论文质量选出优秀作品进入复赛。\\
初赛论文提交截止时间:2月22日24:00 \\
详细内容见原文\url{https://mp.weixin.qq.com/s/-NokUMG2-5GIGXzVocwAWg}\\

\section{二剧帮推 新春讲座预告}
主题:学戏剧到底能吃上饭吗?\\
2/4至2/7共有四场,均为晚上八点\\
具体讲座内容及嘉宾等见原文:\url{https://mp.weixin.qq.com/s/il5QNwcgKtXwQ9PDgumNlQ}

\section{数学学院新年活动}
含新春祝福征集、猜灯谜、新春民俗征集活动。\\
\url{https://mp.weixin.qq.com/s/Hx6zOg4p64X-HJOUB6iJrQ}

\end{multicols} 

\end{document}