% HEAD BEGIN
\documentclass[letterpaper, 12pt]{article}
\usepackage{graphicx}
\usepackage{multicol}
\usepackage{anysize}
\usepackage{fontspec}
\usepackage[fontset=none]{ctex}
\usepackage{tabularx}
\PassOptionsToPackage{hyphens}{url}
\usepackage[breaklinks=true, colorlinks=true]{hyperref}
\expandafter\def\expandafter\UrlBreaks\expandafter{\UrlBreaks\do\a\do\b\do\c\do\d\do\e\do\f\do\g\do\h\do\i\do\j\do\k\do\l\do\m\do\n\do\o\do\p\do\q\do\r\do\s\do\t\do\u\do\v\do\w\do\x\do\y\do\z\do\A\do\B\do\C\do\D\do\E\do\F\do\G\do\H\do\I\do\J\do\K\do\L\do\M\do\N\do\O\do\P\do\Q\do\R\do\S\do\T\do\U\do\V\do\W\do\X\do\Y\do\Z}
% \let\oldurl\url
% \renewcommand{\url}[1]{\begin{sloppypar}\oldurl{#1}\end{sloppypar}}
\setlength\columnsep{30pt}
\marginsize{30pt}{30pt}{10pt}{20pt}
\setmainfont{TeX Gyre Bonum}
\setCJKmainfont[BoldFont=Noto Serif CJK SC Bold, ItalicFont=FandolKai]{Noto Sans CJK SC}
\setlength{\parindent}{0cm}
% \setCJKmonofont{Noto Sans CJK SC}
\begin{document}
\begin{center}
    \Huge\textbf{南哪大专醒前消息}
\end{center}
\vspace{4mm}
\hrule
\renewcommand\tabularxcolumn[1]{m{#1}}
\begin{tabularx}{\textwidth}{>{\hsize.2\hsize}X>{\hsize.6\hsize}X>{\hsize.2\hsize}X}
    \begin{flushleft}
        2024.10.11\, No.84
    \end{flushleft}
    &
    \begin{center}
        \textit{“克明峻德。”}
    \end{center}
    &
    \begin{flushright}
        \textbf{南京市栖霞区}
    \end{flushright}
\end{tabularx}
\vspace{-3.5mm}
\hrule
\vspace{4mm}
% HEAD END
\centerline{\huge\textbf{活动预告}}
\begin{multicols}{2}
\section{Deadline Ongoing}
\begin{tabular}{|c|c|c|}
    \hline
    消息(未见ddl的,不刊) & 截止日期 & 刊载日期\\
    \hline\hline
    仙林校史馆招募讲解员 & 10.30 & 9.12\\
    本科生暑期课程评教 & 10.31 & 9.19\\
    大创训练计划申报 & 11.18 & 9.24\\
    苏州校区音乐会 & 10.19 & 9.25\\
    第十九届大挑 & 10.15 & 9.30\\
    声谷创新基金 & 10.18 & 9.30\\
    鹰角校招宣讲 & 10.15 & 10.2\\
    大专戏曲知识竞赛 & 10.20 & 10.2\\
    EBSCO数据库检索大赛 & 11.20 & 10.3\\
    炜华音乐跑 & 12.8 & 10.4\\
    马院主题宣讲报名 & 10.25 & 10.5\\
    NJU MAJOR & 10.13 & 10.8\\
    后革命鲁迅研究征文 & 11.10 & 10.8\\
    心协黑胶唱片活动 & 10.13 & 10.8\\
    鼓楼草地音乐节 & 10.13 & 10.9\\
    重唱诗社匿名评诗会 & 10.13 & 10.9\\
    “南大新传”编辑部招新 & 10.20 & 10.10\\
    遵义精神宣讲团遴选 & 10.27 & 10.10\\
    历史学院Photoshop培训 & 10.13 & 10.10\\
    八院联谊活动 & 10.14 & 10.10\\
    心协十月征稿 & 10.20 & 10.11\\
    江苏创青春大赛 & 10.14 & 10.11\\
    NEC口语角 & 10.13 & 10.11\\
    国际化科研素养课程 & 10.14 & 10.11\\
    \hline
\end{tabular}
\section{订阅方式和加入编辑部}
编辑部招聘人才,用爱发电,工作轻松,详情可联系QQ:1329527951 客服小祥\\想订阅本消息或获取PDF版(便于查看超链接和往期),可加QQ群:\href{https://qm.qq.com/q/FGX1VYCrGS}{849644979}.
\section{讲座}
\begin{tabular}{|c|c|c|}
    \hline
    往期讲座 & 开展日期 & 刊载日期\\
    \hline\hline
    《聚合物的研发与...》 & 10.24 & 10.3\\
    《电池及电化学能...》 & 11.24 & 10.3\\
    《专利查新与规避...》 & 12.19 & 10.3\\
    《恋爱是门技术活》 & 10.14 & 10.8\\
    《对于人工智能时...》 & 10.16 & 10.9\\
    《中国古代文学中...》 & 10.12 & 10.9\\
    《揭开量化投资的...》 & 10.12 & 10.10\\
    《卡夫卡、现代组...》 & 10.16 & 10.10\\
    《中美博弈及其对...》 & 10.15 & 10.10\\
    《跨代性与跨代平...》 & 10.16 & 10.10\\
    《关于西方社会再...》 & 10.16 & 10.11\\
    《楚国郢都的诗经》 & 10.16 & 10.11\\
      \hline
\end{tabular}\\\\

1.关于西方社会再封建化的哲学评论\\
主讲人:Massimo Carolis(萨莱诺大学哲学教授)\\
主持人:Kangal Kaan(南京大学副教授)\\
时间:10月16日14:00\\
地点:哲学学院(薛光林楼)402室\\

2.南京大学文学院建院110周年系列讲座/南京大学高研院学术前沿讲座第355期\\
题目:时间、地域与共同体:楚国郢都的《诗经》/Time, Place, and Community: The Shijing at the Ying Capital of Chu\\
主讲人:柯马丁 Martin Kern(美国普林斯顿大学讲座教授)\\
主持人:徐兴无(南京大学文学院教授、南京大学高研院院长)\\
时间:2024.10.16(周三)16:00-18:00\\
地点:活水轩(南京大学仙林校区 文学院二楼)\\
注:中文演讲\\
\url{https://mp.weixin.qq.com/s/KYONuCTtzkLaiciIcMrY3g}\\

\section{关于开展国际化科研素养实训课程的选课通知 }
上课时间:2024年10月12日-2025年1月5日,共13周。\\
选课对象:南京大学普通全日制二年级及以上本科在校生,未选修过及选修未通过该课程的学生。选修此类课程的学生需英语听说读写水平较好,一般大学英语四级达到500分、托福80分、雅思6分及以上。\\
流程:教务系统选课(10月11日16:00-10月14日12:00)。登录选课平台(https://xk.nju.edu.cn/)→“2024年秋季学期老生课程补选”,通过“跨专业选课”模块,选择“国际化科研素养实训(计算机科学)”,开课两周内可补退选。\\
注:即选即中\\
具体链接:\url{https://jw.nju.edu.cn/f6/3d/c26263a718397/page.htm}\\

\section{NEC口语角}
地点:鼓楼新教-304\\时间:本周日(10月13号)晚7:00-8:30\\讨论话题:Music,Friendship,Travelling,Major\\报名二维码见附录
\section{心协10月征稿}
投稿内容:记录着生活中引起你情绪波动的时刻的1张摄影照片/图片,并写下背后的故事,300字以内\\截止时间:10月20号19:00\\评奖机制:投稿经筛选后将在心协公众号发出,点赞投票量前十的同学将获得神秘礼品一份;所有过审的投稿者将获得作品印制的定制明信片一张\\投稿二维码见附录 
\section{成员招募 | 性别与亲密关系团体心理辅导}
苏州校区心理工作室在本学期组织开展“性别与亲密关系”团体辅导。本门课程从性本质的四个层面展开,内容涵盖性别刻板印象、女权主义的演化、亲密关系与恋爱、性别安全管理等主题。活动计入“美育”成绩单。\\
授课老师:蒋诗榆(西交利物浦大学心理学讲师/复旦大学临床心理学硕士/苏州大学兼职心询师/瑜伽高级教练)\\
活动地点:苏州校区南雍楼 东105室\\
活动时间:10月18日、10月25日、11月1日、11月7日\\
报名链接及详情见:\url{https://mp.weixin.qq.com/s/HgmU6gYSAhpLMXV0oHC8Gw}\\
\section{商学院运动会}
时间:2024年10月26日\\
一、中场表演节目征集:
    1)招募对象:全体商院本科生(主要面向2023级)\\
    性别要求:女\\
    身高要求:155cm~180cm\\
    本次中场表演形式为啦啦操,会有专人编排舞蹈、组织排练;服装、道具由商学院提供;报名参与者可获得丰富的志愿时长。\\
    2)报名方式:填写报名问卷+加入中场表演群聊\\
    3)报名链接:\url{https://table.nju.edu.cn/dtable/forms/69ddfa20-7d83-40e9-89d2-74484a49274b/}\\
    4)QQ群:726444880\\
二、主持人、播音员召集:\\
召集主持人2名,播音员8名;\\
报名链接:\url{https://table.nju.edu.cn/dtable/forms/bc07fafe-bcc6-4b53-a1a9-99a13844ddc2/}
\section{江苏创青春大赛招募志愿者}
10月21日至24日,江苏青年创青春大赛将在南京市江北新区召开,围绕4个领域举办创新创业赛事和观摩交流等活动,为参赛青年提供展示交流、技能培训、咨询辅导、资本对接等服务。\\
服务时间:10月22日至24日全天(周一至周四)\\
服务地点:江北新区(具体地点另行通知)\\
招募对象:南京大学在籍本科生、研究生\\
招募人数:8-10人\\
参与志愿服务将获得工作餐、服务补贴(80元/天)、志愿服务证书\\
请有意向参与志愿服务的同学在10月14日12点之前填写报名表并提交。\\
报名表链接:\url{https://table.nju.edu.cn/dtable/forms/0a7fd2b5-1a96-44e0-8809-f9653853464c/}\\
提交成功后加入QQ群:760425277\\

\section{南京大学中学班“朋辈导师”活动}
为深化与重点生源学校的务实合作,促进招生与培养联动,助力学生成长成才和未来发展,南京大学在新生中创新设置中学班。今年是南大设立中学班的第二年,中学班数量已增至18所。\\
在“梦想约定站”、“点亮南星”、“南星梦想计划”等南大招协品牌活动中,中学班成员一直是重要的参与群体,发挥着不可或缺的作用。今年,为进一步深化“招培联动”、“双高衔接”工作方法,南大招协计划开展中学班“朋辈导师”活动。\\
活动旨在充分发挥中学班组织优势,凝聚各方力量,赋能人才培养;促进优秀团队的风采展示,并鼓励团队间互相借鉴,继续实现可持续性衔接联动;活动成果将择优在“南大招协”公众号平台展示,为回校宣讲、招生宣传提供原创素材。\\
活动时间:\\
活动拟定于2024年10月正式启动,2025年3月开展中期答辩,2025年5月举行总结答辩与表彰。\\
参加时间:\\
以2023级和2024级中学班成员为参与主体,自愿报名参加。\\
活动形式:\\
各中学班将根据实际情况,积极对接高中目标生源,创新开展线上与线下活动,包括但不限于咨询群答疑、学习经验分享、学业辅导、心理疏导、公益讲座、校园导览等。\\
志愿时长和奖品奖励等细节详见原文:\url{https://mp.weixin.qq.com/s/wPecsgq2h5oYTdi6WHq15w}


\end{multicols}
\
\hrule
\vspace{4mm}
% APPENDIX BEGIN
\centerline{\huge\textbf{附录}}
\begin{figure}[htbp]
    \centering
    \begin{minipage}[b]{0.32\textwidth}
        \centering
        \includegraphics[width=0.5\textwidth]{NEC.jpg}
        \caption{NEC报名}
    \end{minipage}
    \begin{minipage}[b]{0.32\textwidth}
        \centering
        \includegraphics[width=0.5\textwidth]{10月.jpg}
        \caption{心协征稿}
    \end{minipage}
\end{figure}
\end{document}