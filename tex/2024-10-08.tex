% HEAD BEGIN
\documentclass[letterpaper, 12pt]{article}
\usepackage{graphicx}
\usepackage{multicol}
\usepackage{anysize}
\usepackage{fontspec}
\usepackage[fontset=none]{ctex}
\usepackage{tabularx}
\PassOptionsToPackage{hyphens}{url}
\usepackage[breaklinks=true, colorlinks=true]{hyperref}
\expandafter\def\expandafter\UrlBreaks\expandafter{\UrlBreaks\do\a\do\b\do\c\do\d\do\e\do\f\do\g\do\h\do\i\do\j\do\k\do\l\do\m\do\n\do\o\do\p\do\q\do\r\do\s\do\t\do\u\do\v\do\w\do\x\do\y\do\z\do\A\do\B\do\C\do\D\do\E\do\F\do\G\do\H\do\I\do\J\do\K\do\L\do\M\do\N\do\O\do\P\do\Q\do\R\do\S\do\T\do\U\do\V\do\W\do\X\do\Y\do\Z}
% \let\oldurl\url
% \renewcommand{\url}[1]{\begin{sloppypar}\oldurl{#1}\end{sloppypar}}
\setlength\columnsep{30pt}
\marginsize{30pt}{30pt}{10pt}{20pt}
\setmainfont{TeX Gyre Bonum}
\setCJKmainfont[BoldFont=Noto Serif CJK SC Bold, ItalicFont=FandolKai]{Noto Sans CJK SC}
\setlength{\parindent}{0cm}
% \setCJKmonofont{Noto Sans CJK SC}
\begin{document}
\begin{center}
    \Huge\textbf{南哪大专醒前消息}
\end{center}
\vspace{4mm}
\hrule
\renewcommand\tabularxcolumn[1]{m{#1}}
\begin{tabularx}{\textwidth}{>{\hsize.2\hsize}X>{\hsize.6\hsize}X>{\hsize.2\hsize}X}
    \begin{flushleft}
        2024.10.8\, No.82
    \end{flushleft}
    &
    \begin{center}
        \textit{“克明峻德。”}
    \end{center}
    &
    \begin{flushright}
        \textbf{南京市栖霞区}
    \end{flushright}
\end{tabularx}
\vspace{-3.5mm}
\hrule
\vspace{4mm}
% HEAD END
\centerline{\huge\textbf{活动预告}}
\begin{multicols}{2}

\section{Deadline Ongoing}
\begin{tabular}{|c|c|c|}
    \hline
    消息(未见ddl的,不刊) & 截止日期 & 刊载日期\\
    \hline\hline
    仙林校史馆招募讲解员 & 10.30 & 9.12\\
    本科生暑期课程评教 & 10.31 & 9.19\\
    大创训练计划申报 & 11.18 & 9.24\\
    苏州校区音乐会 & 10.19 & 9.25\\
    港澳台生中华文化大赛 & 10.9 & 9.26\\
    心理中心征稿 & 10.10 & 9.28\\
    周末剧场 & 10.10 & 9.28\\
    第十九届大挑 & 10.15 & 9.30\\
    声谷创新基金 & 10.18 & 9.30\\
    午餐读书会 & 10.10 & 9.30\\
    鹰角校招宣讲 & 10.15 & 10.2\\
    大专戏曲知识竞赛 & 10.20 & 10.2\\
    EBSCO数据库检索大赛 & 11.20 & 10.3\\
    Passion街舞公开课 & 10.9 & 10.3\\
    炜华音乐跑 & 12.8 & 10.4\\
    台港澳交流协会招新 & 10.10 & 10.5\\
    马院主题宣讲报名 & 10.25 & 10.5\\
    NJU MAJOR & 10.13 & 10.8\\
    本科毕业生图像补采 & 10.10 & 10.8\\
    新生午餐会报名 & 10.10 & 10.8\\
    世界精神卫生日活动 & 10.10 & 10.8\\
    后革命鲁迅研究征文 & 11.10 & 10.8\\
    街舞社开放活动 & 10.11 & 10.8\\
    心协黑胶唱片活动 & 10.13 & 10.8\\
    黑匣子对谈招募 & 10.11 & 10.8\\
    \hline
\end{tabular}
\section{订阅方式和加入编辑部}
编辑部招聘人才,用爱发电,工作轻松,详情可联系QQ:1329527951 客服小祥\\想订阅本消息或获取PDF版(便于查看超链接),可加QQ群:\href{https://qm.qq.com/q/FGX1VYCrGS}{962626571}.
\section{讲座}
\begin{tabular}{|c|c|c|}
    \hline
    往期讲座 & 开展日期 & 刊载日期\\
    \hline\hline
    《聚合物的研发与...》 & 10.24 & 10.3\\
    《电池及电化学能...》 & 11.24 & 10.3\\
    《专利查新与规避...》 & 12.19 & 10.3\\
    《ChatGPT和生成...》 & 10.9 & 10.3\\
    《从数与形谈起》 & 10.9 & 10.8\\
    物院学术报告会 & 10.10 & 10.8\\
    《恋爱是门技术活》 & 10.14 & 10.8\\
    《宋代佛教书籍史》 & 10.11 & 10.8\\
    《阿赫迈底亚教派...》 & 10.11 & 10.8\\
      \hline
\end{tabular}\\\\
1.本科生论坛——从数与形谈起
题目: 从数与形谈起\\
报告人:丁南庆\\
时间:10月09日(星期三) 16:00-17:30\\
地点:戊己庚四楼北\\
腾讯会议:870-7007-3326\\
摘要: 本报告从“数”与“形”谈起,然后介绍整数的同余、整数分解以及矩阵在密码学中的一些应用,最后介绍范畴的概念。\url{https://mp.weixin.qq.com/s/Luj23LiVxFVEhqufTgRIxQ}

2.物理学院学术报告会(第36期)
题目:Large-scale integrated quantum photonics\\
报告人:王剑威,北京大学\\
时间:2024年10月10日(周四)15:30\\
地点:鼓楼校区唐仲英楼B501\\
第36期学术报告会将通过寇享同步直播,观看方式如下:\\
直播链接:\url{https://www.koushare.com/live/details/37727}\\
于详情中扫描二维码观看直播\url{https://mp.weixin.qq.com/s/Gl-4cnLtOar4Ni1Q0FNy6w}


3.恋爱是门技术活\\
主讲:陈昌凯\\
报名方式与要求说明:\\
时间:10月14日17:00-18:30\\
地点:苏州校区南雍楼西426\\
本课程属于五育项目中的“美”育项目,课程报名人数不超过150人,通过Table链接报名,报满为止。\\
报名链接:\url{https://table.nju.edu.cn/dtable/forms/7064c4a2-82f4-4d0f-bfc3-2d6ba7722f2f/}\\

4.文研讲座349 “大事因缘:佛教与中国文化”系列第六讲\\
题目:轮藏、御书和血经 宋代佛教书籍史初论\\
主讲人:冯国栋,中国社会科学院古代史研究所 副研究员\\
评议人:陈志远,浙江大学\\
主持人:王颂,北京大学哲学系宗教学系 教授\\
时间:2024年10月11日(星期五)19:00\\
地点:北京大学静园二院208会议室\\
将通过文研院视频号、b站和抖音平台,对本场活动进行线上直播,于详情中扫描二维码观看直播\url{https://mp.weixin.qq.com/s/oG2yy1PXlIS-0WZ_sWgUfA}\\

5.学术征文:“后革命时代的鲁迅研究”学术工作坊\\
中国海洋大学文学与新闻传播学院拟于2024年11月底(具体时间待定),举行“后革命时代的鲁迅研究”学术工作坊。工作坊为“政治鲁迅”研究系列会议之一,采用邀请制和征文制结合的方式,为吸引青年学术新锐参与,现拟向全国高校征集在读博士生论文五篇,具体要求如下:\\
征文对象: 在读博士生(包含博士后),不限学校、年级和专业。\\
论文内容: 根据自己的理解,尽量靠拢“后革命时代的鲁迅研究”主题,但也可以提交其他主题的鲁迅研究论文,因为一切皆为后革命时代的症候。\\
投稿需知:须为未曾公开发表的原创学术论文,字数以一万五千字左右为宜,Word稿件体例,具体规范参照近期《中国现代文学研究丛刊》。请在文末附上作者信息,包括姓名、性别、出生年月、所在学校、年级、专业、通信方式(地址、电子邮件和电话)等。\\
投稿时间:2024年11月1日-11月10日,邮件主题请注明:征文+作者+论文题目+学校,请将论文以电子邮件形式发送至邮箱:176690955@qq.com。\\

6.阿赫迈底亚教派在西非:全球非洲本土能动性。\\
题目:阿赫迈底亚教派在西非:全球非洲本土能动性\\
主讲人:邓哲远\\
主持人:王书剑\\
时间:2024.10.11(星期五)15:00-17:00\\
地点:线上/线下结合讲座\\
线下:清华大学科技园C座902教室\\
线上:腾讯会议,报名成功后您将收到会议号及密码等信息\\
讲座语言:中文\\
报名方式:扫码报名,截止日期为10.10中午12:00\\
\url{https://img.picui.cn/free/2024/10/08/67054cd0d3c7e.png}\\



\section{世界精神卫生日活动}
南京大学医院现通知世界精神卫生日活动如下。10 月 10 日中午 12:00 至 13:00 将于鼓楼南园广场与仙林四、五、六食堂前举办世界精神卫生日活动。\newline 活动内容包括漆扇制作、寄语撰写和专家咨询等,或有助于压力纾解与心理健康。\newline 详见 \url{https://mp.weixin.qq.com/s/XhkGPv9e51qt7__0mI2qTQ}
\section{2025届本科毕业生图像秋季补采}
集中补采时间:10月10日(本周四)9:00-17:00\\ 
地点:仙1-201\\
线上补采:无法参加集中采集的同学可于10月8日10:00至10月13日17:00自行采集。为保证照片质量,请同学尽量参加集中采集。\\
采集对象:未参加采集以及参加采集但图像不合格的2025届普通全日制本科毕业生(包括四年制2021级和五年制2020级学生、2023级第二学士学位生)。\\
2020级因无法获取学信网“图像采集码”而未参加过毕业生图像信息采集的延期毕业生。\\
具体链接:\url{https://jw.nju.edu.cn/f4/2a/c26263a717866/page.psp}\\
\section{2024NJU Major}
2024NJU Major报名截止时间:10月8日下午6点(目前已有19支队伍)
\\比赛开始时间:10月13日下午2点
\\赛制:与Major正赛相同,前两轮均为16进8的三败瑞士轮,决定晋级/淘汰的比赛为bo3,其余均为bo1,第三轮为bo3单败淘汰。第一轮在期中前的10月13号和下一个周末,后两轮期中后,时间待定。
\\参赛队员要求:队伍中至少三名本校成员,且外校队员最高段位不得显著高于本校成员(2段左右)。
\\奖品:参赛均有奖金,与各类小奖项奖品。
\\有意向参加可以加入选手群861910073.


\section{高研院“新生午餐会” 第39场}
题目:AI时代传媒业的挑战和机遇\\
谈话人:谭云明 中央财经大学新闻传播系教授 南京大学高研院2024年度访问学者\\
主持人:宗益祥 南京大学新闻传播学院副教授 南京大学高研院第20期驻院学者\\
时间:2024年10月11日(周五)12:20-13:20\\
有意参与的学生以网上抽签的方式获得资格。\\
报名开始时间:10.10 12:30\\
具体链接:\url{https://mp.weixin.qq.com/s/PJmsU-PZZ62lVPRzeM0dg}

\section{passion街舞社}
活动名称:Back to School Vol.1.5\\
“Passionner们在学期初重返校园,汇聚一室,切磋展示,为新学期注入激情与能量。它以battle为主干,穿插showcase等环节,打造好听、好看、好玩的舞蹈狂欢。”\\
时间:2024.10.11 周五 18:30\\
地点:仙林大活111室\\
对象:所有感兴趣者\\
赛制:\\
·海选:各舞种海选三强,可二海 (Locking、Waacking、Popping、Breaking、Hiphop、Jazz六舞种)\\
·组内晋级:裁判加入组成舞种4强-2强  (组内两两battle,其余2人判决)\\
·组间晋级:舞种2强组成12强-6强-3强 (抽签对阵;其余12强选手判罚)\\
·决赛:积分战 (两两battle;其余12强选手判罚)\\
参赛/观赛请加入微信群,见原文:\url{https://mp.weixin.qq.com/s/6uWDG6qczgq4xWc2uhmakQ}\\
转发可领取周边,具体也请参考原文。


\section{心协巧手暖心第1期 | 黑胶唱片DIY活动}
活动主题:黑胶唱片DIY\\
主办单位:南京大学学生心理协会\\
活动时间:2024年10月13日 15:00\\
活动地点:仙林校区三栋南青格庐、鼓楼校区南青格庐\\
活动规模:仙林/鼓楼校区各20人\\
报名问卷和微信群见原文:\url{https://mp.weixin.qq.com/s/bFR2eSM0Tr70xrwdWctXqA}

\section{《青年史》南京场首演 | 对谈招募}
演出时间:\\
10 月 10 日(周四) 19:30\\
10 月 11 日(周五) 19:30\\
演出地点:南京大学仙林校区敬文学生活动中心三楼-黑匣子剧场\\
演出时长:约90分钟\\
票价:扫描二维码免费报名\\
对话时间:10 月 11 日(周五) 14:30—16:00\\
对话地点:同前\\
演出和对谈的报名二维码见原文:\url{https://mp.weixin.qq.com/s/MYt4VD_7P8k1TvMWiFuSeA}

\section{2024南京大学代谢病创新论坛}
2024南京大学代谢病创新论坛现开放观摩名额,欢迎有兴趣的同学填写以下报名链接:\\
\url{https://table.nju.edu.cn/dtable/forms/db91f93f-ec72-4a63-9400-f6c47ab30139/}\\
并加入qq群(830440849),及时关注后续通知。\\
说明:\\
1、鼓楼-仙林有大巴车接送(11号11:00鼓楼-仙林,19:00仙林-鼓楼,12号7:00鼓楼-仙林,19:00仙林-鼓楼,13号7:00鼓楼-仙林,12:30仙林-鼓楼)\\
2、报名成功11号下午开幕式及论坛的,含会议自助午餐和晚餐;12号上午论坛含自助午餐,下午论坛含自助晚餐;13号上午论坛不含餐。\\


\section{男篮院系杯}
小组赛 化生vs工程\\
时间:明日(10月9日)12:30-14:00\\
地点:一组团篮球场
\end{multicols} 


\end{document}