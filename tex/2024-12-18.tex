% HEAD BEGIN
\documentclass[letterpaper, 12pt]{article}
\newsavebox\colbbox
\usepackage{graphicx}
\usepackage{multicol}
\usepackage{anysize}
\usepackage{fontspec}
\usepackage[fontset=none]{ctex}
\usepackage{tabularx}
\usepackage{longtable}
\PassOptionsToPackage{hyphens}{url}
\usepackage[breaklinks=true, colorlinks=true]{hyperref}
\expandafter\def\expandafter\UrlBreaks\expandafter{\UrlBreaks\do\a\do\b\do\c\do\d\do\e\do\f\do\g\do\h\do\i\do\j\do\k\do\l\do\m\do\n\do\o\do\p\do\q\do\r\do\s\do\t\do\u\do\v\do\w\do\x\do\y\do\z\do\A\do\B\do\C\do\D\do\E\do\F\do\G\do\H\do\I\do\J\do\K\do\L\do\M\do\N\do\O\do\P\do\Q\do\R\do\S\do\T\do\U\do\V\do\W\do\X\do\Y\do\Z}
% \let\oldurl\url
% \renewcommand{\url}[1]{\begin{sloppypar}\oldurl{#1}\end{sloppypar}}
\setlength\columnsep{30pt}
\marginsize{30pt}{30pt}{10pt}{20pt}
\setmainfont{TeX Gyre Bonum}
\setCJKmainfont[BoldFont=Noto Serif CJK SC Bold, ItalicFont=FandolKai]{Noto Sans CJK SC}
\setlength{\parindent}{0cm}
% \setCJKmonofont{Noto Sans CJK SC}
\begin{document}
\begin{center}
    \Huge\textbf{南哪大专醒前消息}
\end{center}
\vspace{4mm}
\hrule
\renewcommand\tabularxcolumn[1]{m{#1}}
\begin{tabularx}{\textwidth}{>{\hsize.2\hsize}X>{\hsize.6\hsize}X>{\hsize.2\hsize}X}
    \begin{flushleft}
        2024.12.18\, No.148
    \end{flushleft}
    &
    \begin{center}
        \textit{“秉中持正、求新博闻。”}
    \end{center}
    &
    \begin{flushright}
        \textbf{南京市栖霞区}
    \end{flushright}
\end{tabularx}
\vspace{-3.5mm}
\hrule
\vspace{4mm}
% HEAD END
\centerline{\huge\textbf{活动预告}}
\begin{multicols}{2}
    \section{订阅方式和加入编辑部}  
编辑部招聘人才,用爱发电,工作轻松,详情可联系QQ:1329527951 客服小祥\\想订阅本消息或获取PDF版(便于查看超链接和往期),可加QQ群:\href{https://qm.qq.com/q/VXIW7fgsEe}{849644979}.
\section{Deadline Ongoing}
\setbox\colbbox\vbox{
\makeatletter\col@number\@ne
\begin{longtable}{|c|c|c|}
    \hline
    消息(未见ddl的,不刊) & 截止日期 & 刊载日期\\
    \hline\hline
    安邦征稿 & 1.12 & 11.16\\
    创意物理实验竞赛 & 12.21 & 11.15\\
    仙林通宵自习室 & 1.12 & 11.26\\
    全国大学生家史大赛 & 1.31 & 12.2\\
    金融消费者大赛 & 12.31 & 12.5\\
    花旗杯报名 & 1.3 & 12.6\\
    西安史学论坛征稿 & 3.20 & 12.9\\
    重唱英文评诗会 & 12.21 & 12.10\\
    叶嘉莹纪念征稿 & 12.25 & 12.10\\
    粤语课堂 & 12.22 & 12.11\\
    五院迎新晚会 & 12.21 & 12.12\\
    普通话测试报名 & 12.24 & 12.12\\
    本科评教 & 1.12 & 12.13\\
    排协雪球杯 & 12.28 & 12.13\\
    心协暖冬歌会 & 12.21 & 12.13\\
    12306学生优惠票 & 2.12 & 12.13\\
    校园迷你马拉松报名 & 12.20 & 12.14\\
    南北大联合读书会 & 12.19 & 12.15\\
    新传迎新晚会 & 12.21 & 12.15\\
    外语社团联谊活动 & 12.21 & 12.15\\
    歌魅音乐会 & 12.22 & 12.15\\
    希音编程竞赛 & 12.21 & 12.15\\
    药石杯生化歌赛 & 12.22 & 12.15\\
    交响乐团室内乐 & 12.20 & 12.15\\
    南大新年音乐会 & 12.19 & 12.16\\
    flicker影映 & 12.21 & 12.16\\
    期末考试安排 & 1.12 & 12.17\\
    朝天宫民族团结研学 & 12.20 & 12.17\\
    南大迷你马拉松报名 & 12.20 & 12.17\\
    南大博物馆展览 & 6.16 & 12.17\\
    史院包饺子 & 12.21 & 12.17\\
    地海包饺子 & 12.20 & 12.17\\
    计院包饺子 & 12.21 & 12.17\\
    马院包饺子 & 12.20 & 12.18\\
    软院包饺子 & 12.20 & 12.18\\
    商院包饺子 & 12.21 & 12.18\\
    茶话日和交流会 & 12.21 & 12.18\\
    期末自立营 & 12.23 & 12.18\\
    海岛放映 & 12.21 & 12.18\\
    \hline
\end{longtable}
\unskip
\unpenalty
\unpenalty}\unvbox\colbbox
\end{multicols}
\hrule
\pagebreak
\begin{multicols}{2}

\section{讲座}
\begin{tabularx}{0.5\textwidth}{|X|X|X|}
    \hline
    讲座 & 开展时间 & 刊载时间\\
    \hline\hline
《杨万里对苏轼诗...》 & 12.27 & 12.14\\\hline
《专利查新与规避...》 & 12.19 & 10.3\\\hline
《计算复杂性下界...》 & 12.19 & 12.16\\\hline
《二维半导体中的...》 & 12.19 & 12.16\\\hline
《Designing...》 & 12.19 & 12.16\\\hline
物院学术交流会 & 12.21 & 12.16\\\hline
大城市基层治理的特点与趋势 & 12.20 & 12.17\\\hline
《契诃夫的玫瑰》作者分享会 & 12.19 & 12.18\\\hline
犯罪变化与空间模式 & 12.20 & 12.18\\\hline
社会科学中的理论化工作 & 12.20 & 12.18\\\hline
如果我重写博士论文 & 12.20 & 12.18\\\hline
Active Learning of General Halfspaces & 12.21 & 12.18\\\hline
北太平洋经向模态触发机制的再研究 & 12.19 & 12.18\\\hline
\end{tabularx}

1.《契诃夫的玫瑰》作者分享会\\
时间:12月19日(周四)19:00\\
地点:仙林图书馆一楼校友之家小报告厅\\
主讲人:顾春芳 北京大学艺术学院教授、教育部“长江学者”特聘教授\\
与谈嘉宾:董晓 南京大学文学院院长、教授\\

2.犯罪变化与空间模式\\
主讲人:罗小双 美国阿克伦大学刑事司法系助理教授\\
主持人:柴向南 南京大学社会学院助理教授\\
评议人:刘柳 南京大学新闻传播学院教授\\
讲座时间:12月20日(周五)上午10:00-12:00\\
讲座地点:仙林校区社会学院(河仁楼)118室\\

3.社会科学中的理论化工作\\
主讲人:刘能 北京大学社会学系教授\\
主持人:徐愫 南京大学社会学院副教授、南京大学河仁社会慈善学院副院长\\
时间:12月20日(周五)下午14:00-16:00\\
地点:仙林校区社会学院(河仁楼)118室\\

4.如果我重写博士论文\\
主讲人:姜宇辉 华东师范大学政治与国际关系学院教授\\
主持人:胡翼青 南京大学新闻传播学院教授\\
时间:12月20日(周五)12:00-14:00\\
地点:仙林校区新闻传播学院311\\

5.软件新技术青年学者学术沙龙学术报告\\
Active Learning of General Halfspaces: Label Queries vs Membership Queries\\
时间:2024年12月21日(星期六) 10:30\\
地点:计算机科学技术楼230室\\
马铭辰 博士生 Department of Computer Science, UW-Madison\\
讲座简介见原文\url{https://mp.weixin.qq.com/s/pUO7JleJ-tpEwxjQQAfSQw}\\

6.大气科学学生创新论坛 第一百期\\
报告题目\\
北太平洋经向模态触发机制的再研究\\
时间\\
2024年12月19日(周四) 20:00\\
报告人\\
2020级直博生 王译铭\\
腾讯会议ID\\
230-718-406\\
讲座简介见原文\url{https://mp.weixin.qq.com/s/Xj4V0Plri5yzjfu_a2dcQw}\\

\section{马克思主义学院师生包饺子迎新联欢会}
活动时间:12月20日(周五)16:30-18:30\\
活动地点:教工一食堂(三组团十食堂二楼)\\
活动人员:南京大学马克思主义学院全体师生(消息编辑:西野明日风)\\      

\section{安邦书院|朋导分享会(三)——轻松应对期末}
时间:2024/12/22  16:00\\
地点:新教405\\
朋导介绍:\\
1.翟心怡(化学专业,曾担任化院学生会主席、校会公共权益部部长。我兴趣爱好广泛,平时喜欢健身、摄影和打卡美食。我性格随和,乐观开朗)\\
2.汤崇锹(环境与健康实验班,曾获得郭谢碧蓉优秀奖学金、南京大学优秀学生等荣誉,担任友弈棋社社长,以共同第一作者在《环境化学》期刊发表论文一篇)\\
3.徐佳阳(化学专业,在大一大二担任南京大学学生模拟联合国协会理事作为负责人主持一项国家级大学生创新训练项目;获得国家奖学金、南京大学本科生基础学科专项奖学金特定奖等多项奖学金)\\
报名表和QQ群见\url{https://mp.weixin.qq.com/s/s8jpz8puAro_OpdSRfBgjw}

\section{“茶话日和”线上交流会}
南大日俱再次携手东京大学“茶话日和”社团推出线上交流会活动\\
时间:本周六(12.21)20:00-21:00\\
平台:zoom\\
语言:英语/日语皆可\\
会议号:815 8304 9660\\
密码:669002\\
详情\url{https://mp.weixin.qq.com/s/9-tWwnrYGSLog6xCXGVr7A}\\

\section{期末自习小组和活动组织}
NJU学生爱心联盟 发布\\
时间:2024/12/23 (星期一)—2025/1/5(星期日)\\
地点:鼓楼教室、仙林教室\\
时间安排:每天8点至22点自习;每天9点前为晨读时间,晚上18点到19点为晚读时间\\
招募志愿者监督并记录\\
参与活动及分享各有奖项\\
详情\url{https://mp.weixin.qq.com/s/uX2_MMWpOM5YM6NZdOY8Qg}\\



\section{歌魅音乐会领票}
展台信息:\\
12月19日(周四)12:00-14:00\\
仙林校区五食堂门口\\
12月20日(周五)12:00-14:00\\
鼓楼校区南园喷泉处\\
12月21日(周六)12:00-14:00\\
仙林校区五食堂门口\\
详见\url{https://mp.weixin.qq.com/s/VA5IME8zchyp4cHCmyGL-Q}

\section{海岛放映|小鬼当家}
时间:12月21日(周六)19:00\\
地点:鼓楼校区东大楼310报告厅\\
详见\url{https://mp.weixin.qq.com/s/CJEH-0_5DhPPFz40_7caJg}

\section{暖冬大气树叶拼贴活动}
活动时间:2024年12月20日11:30-14:00\\
活动地点:仙林校区大气科学学院一楼大厅\\
欢迎同学们走出室内,走向校园,搜集校园内的落叶。活动前会在学院一楼放置树叶收集箱,供大家存储提前收集的树叶。\\
参与者将自己收集的独一无二的树叶粘贴到一个巨大的“100”字形海报上,可将自己的愿望、祝福或心情写在树叶上。(如没捡到合适的树叶,也可以到现场领取树叶形状的便利贴)。\\

\section{超图软件职业路径宣讲和专场招聘活动}
主要分为职业路径宣讲及专场招聘会2个部分,招聘会现场还有简历投递及面试环节。\\
时间:12月23日(周一)15:00-17:00\\
地点:昆山楼B537\\
\url{https://mp.weixin.qq.com/s/Ucu1p3YS5gAFtBj5v5HebA}


\section{软件学院师生包饺子活动}
活动时间:12月20日17:30\\
活动地点:鼓楼校区西苑餐厅2楼\\
含有包饺子、预估饺子数量、花式包饺子挑战和抽奖环节。\\
进入原文填表报名\url{https://mp.weixin.qq.com/s/WZPVHn8Tc5eF3_7J78Kouw}\\

\section{商学院“饺满冬至,情满南商”包饺子活动}
时间地点:12.21 星期六 15::30 仙林校区教工一食堂\\
活动内容:包饺子、写对联、蒙眼喂饺子、听歌识曲、节目表演\\
加QQ群\url{1003874890}报名参加活动\\

\section{生科院扭扭棒制作活动}
生科院全体教职工、学生均可报名(因场地原因活动仅限30人,先到先得)\\
活动时间:2024年12月20日15:00-16:30\\
活动地点:生科院A430\\
活动内容\\
1. “手指操”暖场活动\\
2. “柿柿如意”扭扭棒制作\\
进入原文填表报名\url{https://mp.weixin.qq.com/s/tccESG6A4Jm2-Y-OJMJAoQ}\\
\section{2024秋院系杯男排决赛}
外院男排 vs 化学男排\\
时间:12月20日19:00\\
地点:方肇周体育馆副馆\\
\end{multicols} 

\end{document}