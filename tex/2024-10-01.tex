% HEAD BEGIN
\documentclass[letterpaper, 12pt]{article}
\usepackage{graphicx}
\usepackage{multicol}
\usepackage{anysize}
\usepackage{fontspec}
\usepackage[fontset=none]{ctex}
\usepackage{tabularx}
\PassOptionsToPackage{hyphens}{url}
\usepackage[breaklinks=true, colorlinks=true]{hyperref}
\expandafter\def\expandafter\UrlBreaks\expandafter{\UrlBreaks\do\a\do\b\do\c\do\d\do\e\do\f\do\g\do\h\do\i\do\j\do\k\do\l\do\m\do\n\do\o\do\p\do\q\do\r\do\s\do\t\do\u\do\v\do\w\do\x\do\y\do\z\do\A\do\B\do\C\do\D\do\E\do\F\do\G\do\H\do\I\do\J\do\K\do\L\do\M\do\N\do\O\do\P\do\Q\do\R\do\S\do\T\do\U\do\V\do\W\do\X\do\Y\do\Z}
% \let\oldurl\url
% \renewcommand{\url}[1]{\begin{sloppypar}\oldurl{#1}\end{sloppypar}}
\setlength\columnsep{30pt}
\marginsize{30pt}{30pt}{10pt}{20pt}
\setmainfont{TeX Gyre Bonum}
\setCJKmainfont[BoldFont=Noto Serif CJK SC Bold, ItalicFont=FandolKai]{Noto Sans CJK SC}
\setlength{\parindent}{0cm}
% \setCJKmonofont{Noto Sans CJK SC}
\begin{document}
\begin{center}
    \Huge\textbf{南哪大专醒前消息}
\end{center}
\vspace{4mm}
\hrule
\renewcommand\tabularxcolumn[1]{m{#1}}
\begin{tabularx}{\textwidth}{>{\hsize.2\hsize}X>{\hsize.6\hsize}X>{\hsize.2\hsize}X}
    \begin{flushleft}
        2024.10.1\, No.76
    \end{flushleft}
    &
    \begin{center}
        \textit{“忆昔开元全盛日,小邑犹藏万家室。\\稻米流脂粟米白,公私仓廪俱丰实。”}
    \end{center}
    &
    \begin{flushright}
        \textbf{南京市栖霞区}
    \end{flushright}
\end{tabularx}
\vspace{-3.5mm}
\hrule
\vspace{4mm}
% HEAD END
\centerline{\huge\textbf{活动预告}}
\begin{multicols}{2}

\section{Deadline Ongoing}
\begin{tabular}{|c|c|c|}
    \hline
    消息(未见ddl的,不刊) & 截止日期 & 刊载日期\\
    \hline\hline
    仙林校史馆招募讲解员 & 10.30 & 9.12\\
    国优计划报名 & 10.7 & 9.19\\
    本科生暑期课程评教 & 10.31 & 9.19\\
    网易雷火大赛 & 10.7 & 9.22\\
    大创训练计划申报 & 11.18 & 9.24\\
    苏州校区音乐会 & 10.19 & 9.25\\
    外院国庆摄影征集 & 10.7 & 9.25\\
    雨花成长计划课堂报名 & 10.3 & 9.26\\
    港澳台生中华文化大赛 & 10.9 & 9.26\\
    心理中心征稿 & 10.10 & 9.28\\
    周末剧场 & 10.10 & 9.28\\
    历史学院国庆活动 & 10.7 & 9.28\\
    计院国庆桌游会 & 10.5 & 9.29\\
    台湾地区交换项目 & 10.7 & 9.29\\
    第十九届大挑 & 10.15 & 9.30\\
    声谷创新基金 & 10.18 & 9.30\\
    软院国庆桌游会 & 10.7 & 9.30\\
    物院观影会 & 10.3 & 9.30\\
    国际化处全媒体招新 & 10.8 & 9.30\\
    午餐读书会 & 10.10 & 9.30\\
    “周一剧!”第二期 & 10.5 & 9.30\\
    行知书院国庆活动 & 10.2 & 10.1\\
    \hline
\end{tabular}
\section{订阅方式和加入编辑部方式}
编辑部招聘人才,用爱发电,工作轻松,详情可联系QQ:1329527951 客服小祥\\想订阅本消息或获取PDF版(便于查看超链接),可加QQ群:\href{https://qm.qq.com/q/FGX1VYCrGS}{962626571}.
\section{讲座}
\begin{tabular}{|c|c|c|}
    \hline
    往期讲座 & 开展日期 & 刊载日期\\
    \hline\hline
      \hline
\end{tabular}\\\\
\section{Flicker本周放映}
10月5日 18:30 - 21:30 《情人》

费彝民楼 A-410
\section{【计院研究生活动】诗文征集}
以“七十五载庆华诞,十一国庆颂华章”为主题,线上征集创作的三行诗,同时可以附带精彩摄影作品。\\
奖品包括香薰套装、U型枕、小企鹅公仔。\\
投稿征集阶段:10月1日0:00 - 10月6日24:00\\
作品详细要求、提交方式等详见\url{https://mp.weixin.qq.com/s/PHG_0TeOCMc30B0YAoKpEQ}

\section{行知书院国庆系列活动}
活动一

“绘山河锦绣,摄盛世华章”国庆主题征集活动

摄影组(主题二选一)

盛世缩影:围绕祖国国富民强的盛世局面,从自己的感悟和理解出发,抓住生活的碎片与细节,展现今日中国的繁荣昌盛。\\山河巨变:记录家乡的风貌变迁,通过今昔对比,展现当代中国光明恢宏的时代图景,反映出国家在经济发展、城市建设、文化繁荣等方面取得的显著成就。\\
摄影活动报名链接:\url{https://table.nju.edu.cn/dtable/forms/80fc4747-6046-4a6e-8e4d-68168d04b2f5/}

书绘组:选取书法或绘画中的任一形式进行创作,以祖国如今的繁荣昌盛或以祖国七十五年来的沧桑巨变为主题,展现国家发展的光明前景与重任在肩的时代图景。

书绘活动报名链接:\url{https://table.nju.edu.cn/dtable/forms/06ed24cd-6b1b-4832-a450-87a6538d8ec9/}

活动规则详见活动链接\url{https://mp.weixin.qq.com/s/VsgZ5HVNMs_7CUOb9G7haA}\\

活动二:“漆彩华章,扇耀中华”国庆主题手工活动\\
活动时间:2024年10月2日15:00-17:00\\活动地点:新教学楼408\\活动人数:45人
\section{南大后勤场所国庆节服务时间}
伙食:\\
除学生第一食堂、学生第五食堂、鼓楼教工餐厅、教工第一餐厅外,其余食堂国庆节期间正常供餐。\\
学生第一食堂恢复供餐时间为10月7日早餐,学生第五食堂、鼓楼教工餐厅、教工第一餐厅恢复供餐时间为10月8日中餐。\\\\
校内宾馆、超市商店、浴室、水电保障、邮政收发、体育活动场所、大学生活动中心国庆节期间服务时间详见\url{https://mp.weixin.qq.com/s/7wfFMAaTC7Yjhbi6Y1fdRg}
\end{multicols} 
\end{document}