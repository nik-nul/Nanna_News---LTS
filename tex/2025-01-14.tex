% HEAD BEGIN
\documentclass[letterpaper, 12pt]{article}
\newsavebox\colbbox
\usepackage{graphicx}
\usepackage{multicol}
\usepackage{anysize}
\usepackage{fontspec}
\usepackage[fontset=none]{ctex}
\usepackage{tabularx}
\usepackage{longtable}
\PassOptionsToPackage{hyphens}{url}
\usepackage[breaklinks=true, colorlinks=true]{hyperref}
\expandafter\def\expandafter\UrlBreaks\expandafter{\UrlBreaks\do\a\do\b\do\c\do\d\do\e\do\f\do\g\do\h\do\i\do\j\do\k\do\l\do\m\do\n\do\o\do\p\do\q\do\r\do\s\do\t\do\u\do\v\do\w\do\x\do\y\do\z\do\A\do\B\do\C\do\D\do\E\do\F\do\G\do\H\do\I\do\J\do\K\do\L\do\M\do\N\do\O\do\P\do\Q\do\R\do\S\do\T\do\U\do\V\do\W\do\X\do\Y\do\Z}
% \let\oldurl\url
% \renewcommand{\url}[1]{\begin{sloppypar}\oldurl{#1}\end{sloppypar}}
\setlength\columnsep{30pt}
\marginsize{30pt}{30pt}{10pt}{20pt}
\setmainfont{TeX Gyre Bonum}
\setCJKmainfont[BoldFont=Noto Serif CJK SC Bold, ItalicFont=FandolKai]{Noto Sans CJK SC}
\setlength{\parindent}{0cm}
% \setCJKmonofont{Noto Sans CJK SC}
\begin{document}
\begin{center}
    \Huge\textbf{南哪大专醒前消息}
\end{center}
\vspace{4mm}
\hrule
\renewcommand\tabularxcolumn[1]{m{#1}}
\begin{tabularx}{\textwidth}{>{\hsize.2\hsize}X>{\hsize.6\hsize}X>{\hsize.2\hsize}X}
    \begin{flushleft}
        2024.1.15\, No.162
    \end{flushleft}
    &
    \begin{center}
        \textit{“秉中持正、求新博闻。”}
    \end{center}
    &
    \begin{flushright}
        \textbf{南京市栖霞区}
    \end{flushright}
\end{tabularx}
\vspace{-3.5mm}
\hrule
\vspace{4mm}
% HEAD END
\centerline{\huge\textbf{活动预告}}
\begin{multicols}{2}
    \section{订阅方式和加入编辑部}  
编辑部招聘人才,用爱发电,工作轻松,详情可联系QQ:1329527951 客服小祥\\想订阅本消息或获取PDF版(便于查看超链接和往期),可加QQ群:\href{https://qm.qq.com/q/VXIW7fgsEe}{849644979}.
\section{Deadline Ongoing}
\setbox\colbbox\vbox{
\makeatletter\col@number\@ne
\begin{longtable}{|c|c|c|}
    \hline
    消息(未见ddl的,不刊) & 截止日期 & 刊载日期\\
    \hline\hline
    全国大学生家史大赛 & 1.31 & 12.2\\
    西安史学论坛征稿 & 3.20 & 12.9\\
    12306学生优惠票 & 2.12 & 12.13\\
    南大博物馆展览 & 6.16 & 12.17\\
    排超志愿者招募 & 1.16 & 12.25\\
    南星小红书创作 & 2.6 & 12.27\\
    挂职干部选拔 & 2.13 & 12.31\\
    ASC25报名 & 2.21 & 1.6\\
    天文台开放日 & / & 1.6\\
    开甲书院科研作坊 & 2.17 & 1.6\\
    PL读书会 & 2.12 & 1.9\\
    寒假社会实践立项 & 1.13 & 1.9\\
    春季选课 & 1.20 & 1.9\\
    原创剧本联合孵化报名 & 3.20 & 1.10\\
    阅读分享活动征稿 & 3.7 & 1.10\\
    青春杂志征稿 & 1.20 & 1.10\\

    \hline
\end{longtable}
\unskip
\unpenalty
\unpenalty}\unvbox\colbbox
\end{multicols}
\hrule
\pagebreak
\begin{multicols}{2}

\section{讲座}
\begin{tabularx}{0.5\textwidth}{|X|X|X|}
    \hline
    讲座 & 开展时间 & 刊载时间\\
    \hline\hline
湍流噪声仿生学控制的理论建模、机理与优化研究 & 1.16 & 1.15\\\hline
\end{tabularx}
2.物理学院学术报告会(第48期)\\
题   目:湍流噪声仿生学控制的理论建模、机理与优化研究\\
报告人:吕本帅,北京大学\\
时   间:2025年1月16日(周四)14:00\\
地   点:鼓楼校区唐仲英楼B501\\
讲座摘要等见原文\url{https://mp.weixin.qq.com/s/MXrCqPzNf_5EiTGkje1oVg}\\

\section{2024-2025学年“青山同行”线上支教项目招募志愿者}
“青山同行”线上支教是南京大学青年志愿者协会长期举办的支教助学类公益项目。本学年的第二期,南大青协将与云南妥甸中学、浙江黄坦中学合作,为当地学生提供线上一对一辅导。由于地理位置相对闭塞,经济支撑力较差,两所学校目前存在着教学资源相对短缺、学生基础薄弱且部分同学偏科严重、学生课业拓展性不强等问题,学生获取全面素质教育的机会受到限制。

形式:线上一对一教学,每周授课1-2次,每次课程约2小时,可根据学生具体学业压力情况适当调整。课程外,志愿者应提供日常学习方面问题的咨询服务

需求:地理3人,数学29人,物理8人,英语11人,语文7人,科学2人(备注:浙江省中考科学包括物理、化学和生物,建议一对一支教时根据学生情况补习重点细分领域)

群号:914907094

报名问卷:有意向报名的志愿者请填写报名链接,截止时间为1月16日24点。

\url{https://table.nju.edu.cn/dtable/forms/02ddd654-d946-4be9-a9ff-a701bc04459b/}

试讲面试:为保证教学质量,预计于1月18日进行线上志愿者面试。形式为三分钟题目试讲(志愿者自备)+一分半问答。

正式录用:通过面试的志愿者名单将在QQ群内公布。
\section{图书馆寒假开放时间通知}
仙林校区图书馆

开放区域:

图书借阅区(3楼A、B、C区)

报刊阅览区(2楼A、B区)

工具书阅览区(2楼C区)

信息共享大厅(2楼C区)

开放时间:每周一至周五9:00—17:00

注意事项:总服务台合并至报刊阅览区开放,请读者尽可能地集中到2楼报刊阅览区学习。

开放区域:

外文书刊阅览区(4楼B区)

港台与民国文献阅览区(4楼A区)

古籍与地方文献阅览区(4楼C区)

艺术与特藏文献阅览区(5楼C区)

开放时间:每周一、三、五9:00—17:00

注意事项:古籍书库与民国书库藏文献不提供阅览服务。艺术阅览区合并到港台民国文献阅览区。提书服务点在港台民国文献阅览区。\\

苏州校区图书馆

开放区域:图书馆

开放时间:每周一至周五9:00—17:00\\

鼓楼校区图书馆

开放区域:

一楼所有区域

二楼报刊阅览室(206室)

开放时间:每周一至周五9:00—17:00

注意事项:文科借阅室(302室)和理科借阅室(406室)合并至总服务台开放。样本阅览室(421室)每周三开放,开放时间同上。
\end{document}