% HEAD BEGIN
\documentclass[letterpaper, 12pt]{article}
\newsavebox\colbbox
\usepackage{graphicx}
\usepackage{multicol}
\usepackage{anysize}
\usepackage{fontspec}
\usepackage[fontset=none]{ctex}
\usepackage{tabularx}
\usepackage{longtable}
\PassOptionsToPackage{hyphens}{url}
\usepackage[breaklinks=true, colorlinks=true]{hyperref}
\expandafter\def\expandafter\UrlBreaks\expandafter{\UrlBreaks\do\a\do\b\do\c\do\d\do\e\do\f\do\g\do\h\do\i\do\j\do\k\do\l\do\m\do\n\do\o\do\p\do\q\do\r\do\s\do\t\do\u\do\v\do\w\do\x\do\y\do\z\do\A\do\B\do\C\do\D\do\E\do\F\do\G\do\H\do\I\do\J\do\K\do\L\do\M\do\N\do\O\do\P\do\Q\do\R\do\S\do\T\do\U\do\V\do\W\do\X\do\Y\do\Z}
% \let\oldurl\url
% \renewcommand{\url}[1]{\begin{sloppypar}\oldurl{#1}\end{sloppypar}}
\setlength\columnsep{30pt}
\marginsize{30pt}{30pt}{10pt}{20pt}
\setmainfont{TeX Gyre Bonum}
\setCJKmainfont[BoldFont=Noto Serif CJK SC Bold, ItalicFont=FandolKai]{Noto Sans CJK SC}
\setlength{\parindent}{0cm}
% \setCJKmonofont{Noto Sans CJK SC}
\begin{document}
\begin{center}
    \Huge\textbf{南哪大专醒前消息}
\end{center}
\vspace{4mm}
\hrule
\renewcommand\tabularxcolumn[1]{m{#1}}
\begin{tabularx}{\textwidth}{>{\hsize.2\hsize}X>{\hsize.6\hsize}X>{\hsize.2\hsize}X}
    \begin{flushleft}
        2024.11.12\, No.115
    \end{flushleft}
    &
    \begin{center}
        \textit{“秉中持正、求新博闻。”}
    \end{center}
    &
    \begin{flushright}
        \textbf{南京市栖霞区}
    \end{flushright}
\end{tabularx}
\vspace{-3.5mm}
\hrule
\vspace{4mm}
% HEAD END
\centerline{\huge\textbf{活动预告}}
\begin{multicols}{2}
    \section{订阅方式和加入编辑部}  
编辑部招聘人才,用爱发电,工作轻松,详情可联系QQ:1329527951 客服小祥\\想订阅本消息或获取PDF版(便于查看超链接和往期),可加QQ群:\href{https://qm.qq.com/q/VXIW7fgsEe}{849644979}.
\section{Deadline Ongoing}
\setbox\colbbox\vbox{
\makeatletter\col@number\@ne
\begin{longtable}{|c|c|c|}
    \hline
    消息(未见ddl的,不刊) & 截止日期 & 刊载日期\\
    \hline\hline
    紫藤学刊征稿 & 12.15 & 10.22\\
    大创训练计划申报 & 11.18 & 9.24\\
    招生宣传创意征集大赛 & 11.18 & 10.21\\ 
    EBSCO数据库检索大赛 & 11.20 & 10.3\\
    文院征稿 & 11.20 & 10.20\\
    乐跑 & 12.6 & 10.12\\
    国际访学计划申报 & 11.22 & 10.22\\
    南大会征募会设 & 11.15 & 11.1\\
    秉文心理短视频 & 11.25 & 11.3\\
    简历大赛 &11.17 & 11.7\\
    医保补参保 & 11.17 & 11.8\\
    NCQM2024报名 & 11.15 & 11.9\\
    博洽书会 & 11.15 & 11.9\\
    南新读书会 & 11.13 & 11.11\\
    AI爱情主题辩论赛 & 11.16 & 11.11\\
    NCA分享会 & 11.17 & 11.11\\
    走进名企麦当劳 & 11.13 & 11.11\\
    流光《心灵捕手》 & 11.16 & 11.12\\
    \hline
\end{longtable}
\unskip
\unpenalty
\unpenalty}\unvbox\colbbox
\end{multicols}
\hrule
\pagebreak
\begin{multicols}{2}

\section{讲座}
\begin{tabular}{|c|c|c|}
    \hline
    往期讲座 & 开展日期 & 刊载日期\\
    \hline\hline
    《电池及电化学能...》 & 11.24 & 10.3\\
    《专利查新与规避...》 & 12.19 & 10.3\\
    图书馆系列讲座 & 12.3 & 10.20\\
    《教室性别结构对...》 & 11.14 & 11.7\\
    《Learning from AI》 & 11.13 & 11.7\\
    《米洛·劳的真实...》 & 11.13 & 11.11\\
    《Decouple electron...》 & 11.14 & 11.11\\
    《A meta-mathematical...》 & 11.13 & 11.11\\
    《Nakai-Moishezon...》 & 11.13 & 11.11\\
    《我们是天生的赛...》 & 11.13 & 11.11\\
    《都市男人社交与...》 & 11.13 & 11.11\\
    《大数据与传统数...》 & 11.15 & 11.12\\
    《北齐兰陵王的文...》 & 11.13 & 11.12\\
    《自传经典比较研...》 & 11.14 & 11.12\\
    《水下考古漫谈...》 & 11.14 & 11.12\\
    
    \hline
\end{tabular}

1.孙本文社会学论坛第290期
\\题目:大数据与传统数据的四种结合途径
\\主讲人:边燕杰
\\西安交通大学文科资深教授
\\提要:讲座简介国内外社会学家就前三种结合途径的成功范例,着重讲解西安交大团队对于第四种结合途径正在开展的探索
\\时间:11月15日(周五)上午10:00-12:00
\\地点:仙林社会学院河仁楼401室\\
2.假面下的舞者--北齐兰陵王的文化史
\\主讲人:童岭
\\南京大学文学院教授、南京大学文学院副院长
\\提要:以北齐历史的虚实、兰陵王假面乐舞的传播等为例,探讨兰陵王及其背后的文化内涵
\\时间:11月13日(周三)16:15
\\地点:仙林图书馆一楼校友之家小报告厅\\
3,大转型时期的核心问题--自传经典比较研究
\\主讲人:赵白生
\\北京大学外语学院教授、北京大学世界传记研究中心主任
\\提要:将在对比一些比较传记之后,揭示两种不同的自我观所导致的人性的底层逻辑和大转型期的核心问题
\\时间:11月14日(周四)19:00-20:30
\\地点:仙林国际学院C308高研院报告厅\\
4.考古科普讲座
\\题目:探索蓝色文明,水下考古漫谈
\\主讲人:赵东升
\\南京大学历史学院考古文物系博导、南京大学博物馆副馆长
\\讲座简介:讲述什么是考古学、什么是水下考古学
\\时间:11月14日(周四)19:00-20:30
\\地点:仙1-116
\\抽奖福利:三本精美图书,注意当天应到现场领取奖品,否则奖品将会被转赠给现场的幸运同学
\\参与抽奖方式见附录

\section{歌声魅影音乐剧社社员大会}
将播放年度剧《Catch Me If You Can》的官摄\\
时间\\
2024年11月17日19:00-21:30\\
地点\\
仙林校区:仙1-115\\ 
鼓楼校区:费B-201\\
详见\url{https://mp.weixin.qq.com/s/V3GNZp9B09Su4WVgkL6hBg}\\

\section{“挑战杯”相关通知}
各学院须在2024年11月25日(星期一)24∶00之前讲材料打包发送至校团委学术与“双创”部公共邮箱。\\
校赛初赛阶段为2024年12月中旬,校级决赛阶段为2024年12月下旬至2025年1月上旬。\\
校级选拔赛将按一定比例设特等奖、一等奖、二等奖、三等奖,并遴选优秀作品代表我校参加第十九届“挑战杯”全国大学生课外学术科技作品竞赛江苏省选拔赛;另设“优秀组织奖”若干。\\
联系方式,具体要求等见原文\url{https://mp.weixin.qq.com/s/E0Hy8V_1SztIukH_J9pBIQ}\\

\section{南大篮球赛事明日赛程}
研究生院系杯小组赛\\
中美 vs 生科\\
19:30 - 20:30\\
地点:仙林一组团篮球场\\
男篮院系杯小组赛\\
计科 vs 材料\\
12:30 - 14:00\\
地点:一组团篮球场\\

\section{流光影院 | 《心灵捕手》}
时间:\\
11月16日(周六)19:00\\
地点:\\
南京大学仙林校区心理中心\\
南京大学鼓楼校区费B-201\\
报名链接(于11月14日中午12点截止):\url{https://docs.qq.com/form/page/DR0N4Zll5VWRZcExM}\\
活动群及相关礼品等见原文:\url{https://mp.weixin.qq.com/s/9g4NWTGsZgSL7QSK-5m9GA}

\section{五院联合羽毛球赛}
比赛时间:\\
2024年11月23日-24日\\
(具体赛程,另行通知)\\
比赛地点:\\
南京大学仙林校区四组团体育馆\\
一楼羽毛球场\\
比赛对象:\\
计算机学院\\
化学化工学院\\
大气科学学院\&南赫学院\\
教育研究院•陶行知教师教育学院\\
人工智能学院\\
全体学生(包括本科生和研究生)\\
报名方式:\\
1、报名时间:2024年11月12日-2024年11月17日24点
2、报名方式:扫描二维码填写问卷进行报名
\url{https://mp.weixin.qq.com/s/NUbJm7idRI3XGjYARlZ0zQ}

\end{multicols} 

\end{document}