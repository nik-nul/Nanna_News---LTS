% HEAD BEGIN
\documentclass[letterpaper, 12pt]{article}
\newsavebox\colbbox
\usepackage{graphicx}
\usepackage{multicol}
\usepackage{anysize}
\usepackage{fontspec}
\usepackage[fontset=none]{ctex}
\usepackage{tabularx}
\usepackage{longtable}
\PassOptionsToPackage{hyphens}{url}
\usepackage[breaklinks=true, colorlinks=true]{hyperref}
\expandafter\def\expandafter\UrlBreaks\expandafter{\UrlBreaks\do\a\do\b\do\c\do\d\do\e\do\f\do\g\do\h\do\i\do\j\do\k\do\l\do\m\do\n\do\o\do\p\do\q\do\r\do\s\do\t\do\u\do\v\do\w\do\x\do\y\do\z\do\A\do\B\do\C\do\D\do\E\do\F\do\G\do\H\do\I\do\J\do\K\do\L\do\M\do\N\do\O\do\P\do\Q\do\R\do\S\do\T\do\U\do\V\do\W\do\X\do\Y\do\Z}
% \let\oldurl\url
% \renewcommand{\url}[1]{\begin{sloppypar}\oldurl{#1}\end{sloppypar}}
\setlength\columnsep{30pt}
\marginsize{30pt}{30pt}{10pt}{20pt}
\setmainfont{TeX Gyre Bonum}
\setCJKmainfont[BoldFont=Noto Serif CJK SC Bold, ItalicFont=FandolKai]{Source Han Sans SC}
\setlength{\parindent}{0cm}
% \setCJKmonofont{Noto Sans CJK SC}
\begin{document}
\begin{center}
    \Huge\textbf{南哪大专醒前消息}
\end{center}
\vspace{4mm}
\hrule
\renewcommand\tabularxcolumn[1]{m{#1}}
\begin{tabularx}{\textwidth}{>{\hsize.2\hsize}X>{\hsize.6\hsize}X>{\hsize.2\hsize}X}
    \begin{flushleft}
        2025.4.13\, No.219
    \end{flushleft}
    &
    \begin{center}
        \textit{“秉中持正、求新博闻。”}
    \end{center}
    &
    \begin{flushright}
        \textbf{南京市栖霞区}
    \end{flushright}
\end{tabularx}
\vspace{-3.5mm}
\hrule
\vspace{4mm}
% HEAD END
\centerline{\huge\textbf{活动预告}}
\begin{multicols}{2}
\section{订阅方式和加入编辑部}  
编辑部招聘人才,用爱发电,工作轻松,详情可联系QQ:1329527951 客服小千\\想订阅本消息或获取PDF版(便于查看超链接和往期),可加QQ群:\href{https://qm.qq.com/q/4HL41Nt3sQ}{466863272}.
\section{活动清单}
\setbox\colbbox\vbox{
\makeatletter\col@number\@ne
\begin{longtable}{|>{\centering\arraybackslash}m{.3\textwidth}|m{.06\textwidth}|m{.06\textwidth}|}
    \hline
    活动 & 开展时间 & 刊载时间\\
    \hline\hline
    南大版deepseek & / & 2.22\\
    悦读课程群 & / & 2.24\\
    eScience AI科研助手 & / & 3.11\\
    地科博物馆开放安排 & / & 3.22\\ 
    2025年分流和转专业政策通知 & / & 4.7\\
    乐跑 & 5.16 & 3.10\\
    本科生劳育实践 & 7.20 & 2.19\\
    银星杯论文赛 & 4.22 & 2.27\\
    高教社杯 & 4.25 & 3.5\\
    大文大理题目征集 & 期末 & 3.8\\
    5月免费上网 & ? & 3.9\\
    基础学科论坛 & 4.20 & 3.9\\
    外教社杯 & 5.27 & 3.12\\
    江苏创青春赛事 & 4.30 & 3.26\\
    南大数学竞赛 & 4.15 & 3.27\\
    AI素养大赛 & 4.15 & 3.31\\
    浦口音乐跑 & 5.30 & 3.31\\
    红会暑期项目招募 & 4.12 & 4.1\\
    程设大赛 & 4.26 & 4.2\\
    瑞声杯 & 4.20 & 4.4\\
    仙林校区志愿法律咨询 & / & 4.4\\
    青春活力大赛 & 5.17 & 4.7\\
    在校生自愿体检 & 6.20 & 4.8\\
    数智应用大赛 & 4.16 & 4.9\\
    南大购买WPS & / & 4.8\\
    24级程设大赛 & 4.27 & 4.11\\
    南书房支教招新 & 4.15 & 4.11\\
    法治情景剧策划大赛 & 4.23 & 4.11\\
    仙林猫鼠游戏 & 4.19 & 4.12\\
    EL程设大赛 & 4.27 & 4.13\\
    大挑志愿者课3 & 4.16 & 4.13\\
    全国行研大赛 & 4.20 & 4.13\\
    全国ESG大赛 & 4.17 & 4.13\\
    南大网双公开赛报名 & 4.18 & 4.13\\
    
    \hline
\end{longtable}
\unskip
\unpenalty
\unpenalty}\unvbox\colbbox
\end{multicols}
\begin{multicols}{2}
\pagebreak

\section{讲座}
\begin{tabular}{|>{\centering\arraybackslash}m{.3\textwidth}|m{.06\textwidth}|m{.06\textwidth}|}
    \hline
    讲座 & 开展时间 & 刊载时间\\
    \hline\hline
    黑色素瘤防治科普讲座 & 4.13 & 4.8\\\hline
    社交媒体分享实践的语用学研究 & 4.18 & 4.9\\\hline
    基于归纳程序合成的算法自动应用 & 4.15 & 4.10\\\hline
    面向推荐大模型的参数存储系统研究 & 4.15 & 4.10\\\hline
    REUSE IN SWITZERLAND-2 PILOT PROHJECTS BY BAUBÜRO IN SITU & 4.14 & 4.10\\\hline
    特朗普2.0与中美日关系 & 4.16 & 4.10\\\hline
    Deepseek现象中的管理学 & 4.18 & 4.10\\\hline
    智能时代的中国式养老:理论与实践”学术研讨会 & 4.18-20 & 4.10\\\hline
    从感知到疗愈:人脑音乐加工机制 & 4.25 & 4.11\\\hline
    Russian observatories for extragalactic research & 4.14 & 4.11\\\hline
    Accretion-generated rings:coplaner and polar structures & 4.18 & 4.11\\\hline
    Ionizing spotlight of Active Galactic Nucleus & 4.23 & 4.11\\\hline
    用法律知识为职场保驾护航 & 4.16 & 4.12\\\hline
\end{tabular}
%讲座预告写在这。用subsection

\section{2025 EL程序设计大赛}
活动对象:南京大学2024级本科生
\\报名时间:2025年4月11日00:00至2025年4月27日18:00
\\报名方式:填写报名问卷
\\报名细则
\\1. 交互组和算法组相互独立,可以同时参加。小组是参加比赛的基本单位,每组至多4人,1名队长。
\\2. 所有参赛同学必须为大一年级学生(2024级本科生)。
\\3. 组队报名截止后,队伍状态不能更改,即不能新增或减少队员。
\\4. 允许在报名截止前修改组队信息,若需要,请联系工作人员。
\\5. 算法组队伍需在 Online Judge 平台注册账号,并在报名问卷中填写战队账号信息。比赛仅限通过该账号参与,所有积分将统一记录,用于最终评奖。
\\6. 伪造信息,替身代赛等恶劣行为一经发现直接取消成绩。
\\详见:\url{https://mp.weixin.qq.com/s/74PhSV9mNc79KhR-XxXwLg}

\section{南京大学2025年中英集成电路与先进制造本科生科考与科研训练项目招生简章}
1.报名方式
\\请于4月27日17:00之前,填写下方报名链接
\\Table链接报名:https://table.nju.edu.cn/dtable/forms/c4162b40-cce8-44fb-ad0c-431c111dada8/
\\报名表获取:https://box.nju.edu.cn/f/5283bb587fca427e829f/
\\2.选拔方式
\\根据报名材料进行第一轮筛选,通过者于5月14日之前收到邮件通知参加面试选拔。本次科考共计划选拔32名左右成员。
\\3. 咨询QQ群:567405104
\\详见:\url{https://jw.nju.edu.cn/82/3f/c26263a754239/page.htm}

\section{第十九届“挑战杯”竞赛志愿者培训第三期课程发布}
培训时间:4.16(周三)16:00—18:002
\\培训地点:线下 仙Ⅰ-1133
\\培训内容:压力和情绪管理辅导
\\报名方式:扫码填写以下问卷
\\报名注意:填写时间为4.13 14:00-15:00,培训人数限制在40位,按照填写时间先后录取。
\\详见:\url{https://mp.weixin.qq.com/s/7Grc7rWydOXlr0w2PqRs5A}

\section{赛事推介 | 第三届全国大学生预见未来行研大赛}
第三届全国大学生预见未来行研大赛以“携手AI·预见未来”为主题,聚焦“AI+行研”领域,教你如何驾驭AI。
\\报名时间:4月1日 - 4月20日 
\\报名方式:扫描二维码 添加比赛助推官,备注“姓名+专业+行研大赛”报名
\\赛事流程和奖品见原文
\\详见:\url{https://mp.weixin.qq.com/s/FhrK1NF35FKJbR52Ra8n6A}


\section{赛事推介 | 第四届安永x高顿教育ESG大学生创新挑战赛报名}
比赛规则
\\1. 本次比赛面向2022-2024级中国国内及海外在校本科大学生。
\\2. 大赛以4人团队参赛,可跨院校组队,不支持跨城组队,如所在城市未举办初赛,可就近选择其他城市参加。
\\比赛流程
\\本次比赛分为线上初赛、线上复赛、区域决赛、全国总决赛四个阶段,各院校组队参加,每支参赛队伍人数为4人。
\\赛事详情及报名方式见原推https://mp.weixin.qq.com/s/ktUpPPEaAstdpP\_t\_-LXHw
\\详见:\url{https://mp.weixin.qq.com/s/ktUpPPEaAstdpP_t_-LXHw}


\section{2025南京大学网球双打公开赛}
比赛基本信息
\\参赛对象:南京大学全体学生(含非全)
\\                  教职工、毕业校友
\\时间:2025.4.20 上午9点
\\地点:南京大学仙林校区网球场
\\详见:\url{https://mp.weixin.qq.com/s/pSpgeRHbKk86RMTwPhh9Dw}

\section{院级活动}
\begin{tabular}{|>{\centering\arraybackslash}m{.3\textwidth}|m{.06\textwidth}|m{.06\textwidth}|}
\hline
    活动 & 开展时间 & 刊载时间\\
    \hline\hline
    文院剧本创作研讨会 & 9.30 & 3.2\\
    物院征集课程指南 & 6.15 & 3.3\\
    地海征集春日影 & 6.15 & 3.14\\
    社院学术节 & 4.18 & 3.25\\
    五院运动会 & 4.13 & 3.31\\
    五院乒乓球赛 & 4.19 & 3.31\\
    建城影展征集 & 4.16 & 3.31\\
    法院党建征文 & 5.20 & 4.2\\
    地学乒赛 & 4.19 & 4.2\\
    软院征集 & 4.20 & 4.4\\
    地学趣运会 & 4.26 & 4.9\\
    四院音乐节 & 5.11 & 4.7\\
    商院征集 & 5.5 & 4.8\\
    毓秀羽球 & 4.20 & 4.8\\
    大气设计 & 4.18 & 4.8\\
    文院诗歌 & 4.18 & 4.8\\
    化院摄影 & 4.15 & 4.9\\
    毓秀宿舍 & 4.16 & 4.10\\
    社院访企 & 4.16 & 4.11\\
    信管诗会 & 4.14 & 4.12\\
    物院运动打卡 & 5.14 & 4.12\\
    大气留学分享会 & 4.15 & 4.12\\
    地学定向越野 & 4.19 & 4.12\\
    \hline
\end{tabular}


\section{社团活动}
\begin{tabular}{|>{\centering\arraybackslash}m{.3\textwidth}|m{.06\textwidth}|m{.06\textwidth}|}
    \hline
    社团活动 & 开展时间 & 刊载时间\\
    \hline\hline
    天文台开放日 & / & 1.6\\
    重唱诗歌奖征稿 & 4.30 & 3.31\\
    足协体验 & 4.15 & 4.1\\
    轮滑社体验 & 4.17 & 4.1\\
    拳击社体验 & 4.22 & 4.1\\
    轮滑社体验 & 4.22 & 4.1\\
    定向赛 & 4.20 & 4.1\\
    体育舞蹈教学 & 4.25 & 4.1\\
    吉他社歌手招募 & 4.20 & 4.4\\
    吉他社春日音 & 4.26 & 4.4\\
    国学社寄明信片 & 4.14 & 4.4\\
    天健捐衣 & 4.20 & 4.13\\
    \hline
\end{tabular}
%这里是写社团活动的,社团活动就是由社团主办、主要针对社团内部人员的活动。不要把非社团活动写在这里。

\subsection{今日战报\&明日赛程}
软院vs数理
\\45:5
\\地海vs新传
\\10:46
\\今日得分王
\\软院  0号 冯缘  21分
\\明日赛程
\\男篮院系杯小组赛
\\匡院 vs 软院
\\ 14:00 - 15:00
\\地点:一组团篮球场
\\女篮院系杯小组赛
\\材料 vs 软院 
\\10:30 - 11:30
\\地点:一组团篮球场
\\新生杯小组赛
\\数理 vs 地科
\\11: 00 - 12: 00
\\文院 vs 化生
\\12: 30 - 13: 30
\\软院 vs 计科
\\13: 30 - 14: 30
\\健雄 vs 行知
\\14: 30 - 15: 30
\\电子 vs 医学
\\15: 30 -16: 30
\\工试 vs 匡院
\\16: 30 - 17: 30
\\地点: 鼓楼北园篮球场
\\详见:\url{https://mp.weixin.qq.com/s/Y1s-G1q3GhkSABswRAqXbw}


\subsection{活动预热 | “衣”爱启程,暖语励行}
天健社携手咸阳市的两所精神患者医院,发起捐衣活动,诚邀大家捐出旧衣,为患者们送上舒适与幸福。
\\- 01 -
\\活动时间、地点 
\\4月20日(周日)
\\09:00 - 12:00
\\15:00 - 18:00
\\鼓楼校区:南园喷泉广场前
\\(如遇下雨,活动调整至室内进行,具体场地请进咨询QQ群等候通知)
\\- 02 -
\\具体衣物要求
\\1.衣物类别:毛衣裤、外衣裤、短袖、棉衣裤、秋衣秋裤、羽绒服、运动鞋等
\\2.衣物必须提前清洗,干净卫生
\\3.衣物大小需适合1.5米以上的成年人穿着
\\(注:医院强调衣物的实用价值,因此袜子、裙子、短裤、贴身衣裤、破损衣物、除运动鞋外的鞋子等一概不收。接收衣物不怕旧、不怕厚、不怕款式老。志愿者会现场检查衣物是否符合规定,不合规的会直接退还)
\\- 03 -
\\附加活动:书信暖语
\\活动现场,每一位捐赠衣物的同学都可以领取一张明信片,拿起笔,写下南大学子给南京宁海中学和贵州平坝一中的高中学生的鼓励与祝福,遥寄给未曾会面、或将谋面的莘莘学子。
\\(注:参与两项活动的志愿者可以获得1h志愿时长。)
\\详见:\url{https://mp.weixin.qq.com/s/NAAmwT5aEcwCaJSTkWOcBA}

\subsection{篮协明日赛程}
男篮院系杯小组赛
\\物理vs数学 
\\4月14日19:00-20:00
\\地点:一组团篮球场
\\详见:\url{https://mp.weixin.qq.com/s/br_xOqhH-QcGzg48OCA4cA}
\end{multicols}
\end{document}
