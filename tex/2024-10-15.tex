% HEAD BEGIN
\documentclass[letterpaper, 12pt]{article}
\newsavebox\colbbox
\usepackage{graphicx}
\usepackage{multicol}
\usepackage{anysize}
\usepackage{fontspec}
\usepackage[fontset=none]{ctex}
\usepackage{tabularx}
\usepackage{longtable}
\PassOptionsToPackage{hyphens}{url}
\usepackage[breaklinks=true, colorlinks=true]{hyperref}
\expandafter\def\expandafter\UrlBreaks\expandafter{\UrlBreaks\do\a\do\b\do\c\do\d\do\e\do\f\do\g\do\h\do\i\do\j\do\k\do\l\do\m\do\n\do\o\do\p\do\q\do\r\do\s\do\t\do\u\do\v\do\w\do\x\do\y\do\z\do\A\do\B\do\C\do\D\do\E\do\F\do\G\do\H\do\I\do\J\do\K\do\L\do\M\do\N\do\O\do\P\do\Q\do\R\do\S\do\T\do\U\do\V\do\W\do\X\do\Y\do\Z}
% \let\oldurl\url
% \renewcommand{\url}[1]{\begin{sloppypar}\oldurl{#1}\end{sloppypar}}
\setlength\columnsep{30pt}
\marginsize{30pt}{30pt}{10pt}{20pt}
\setmainfont{TeX Gyre Bonum}
\setCJKmainfont[BoldFont=Noto Serif CJK SC Bold, ItalicFont=FandolKai]{Noto Sans CJK SC}
\setlength{\parindent}{0cm}
% \setCJKmonofont{Noto Sans CJK SC}
\begin{document}
\begin{center}
    \Huge\textbf{南哪大专醒前消息}
\end{center}
\vspace{4mm}
\hrule
\renewcommand\tabularxcolumn[1]{m{#1}}
\begin{tabularx}{\textwidth}{>{\hsize.2\hsize}X>{\hsize.6\hsize}X>{\hsize.2\hsize}X}
    \begin{flushleft}
        2024.10.15\, No.89
    \end{flushleft}
    &
    \begin{center}
        \textit{“克明峻德。”}
    \end{center}
    &
    \begin{flushright}
        \textbf{南京市栖霞区}
    \end{flushright}
\end{tabularx}
\vspace{-3.5mm}
\hrule
\vspace{4mm}
% HEAD END
\centerline{\huge\textbf{活动预告}}
\begin{multicols}{2}
    \section{订阅方式和加入编辑部}  
编辑部招聘人才,用爱发电,工作轻松,详情可联系QQ:1329527951 客服小祥\\想订阅本消息或获取PDF版(便于查看超链接和往期),可加QQ群:\href{https://qm.qq.com/q/VXIW7fgsEe}{849644979}.
\section{Deadline Ongoing}
\setbox\colbbox\vbox{
\makeatletter\col@number\@ne
\begin{longtable}{|c|c|c|}
    \hline
    消息(未见ddl的,不刊) & 截止日期 & 刊载日期\\
    \hline\hline
    仙林校史馆招募讲解员 & 10.30 & 9.12\\
    本科生暑期课程评教 & 10.31 & 9.19\\
    大创训练计划申报 & 11.18 & 9.24\\
    苏州校区音乐会 & 10.19 & 9.25\\
    声谷创新基金 & 10.18 & 9.30\\
    大专戏曲知识竞赛 & 10.20 & 10.2\\
    EBSCO数据库检索大赛 & 11.20 & 10.3\\
    炜华音乐跑 & 12.8 & 10.4\\
    马院主题宣讲报名 & 10.25 & 10.5\\
    后革命鲁迅研究征文 & 11.10 & 10.8\\
    “南大新传”编辑部招新 & 10.20 & 10.10\\
    遵义精神宣讲团遴选 & 10.27 & 10.10\\
    心协十月征稿 & 10.20 & 10.11\\
    乐跑 & 12.8 & 10.12\\
    健雄书院院服设计赛 & 10.20 & 10.12\\
    计院迎新晚会征集节目 & 10.25 & 10.12\\
    毓秀素拓友谊赛 & 10.19 & 10.12\\
    新传南新读书会 & 10.16 & 10.13\\
    行知院服设计赛 & 10.21 & 10.15\\
    安邦院服设计赛 & 10.20 & 10.15\\
    CTF竞赛宣讲 & 10.19 & 10.13\\
    行知趣味羽球赛 & 10.19 & 10.13\\
    计院趣味定向赛 & 10.20 & 10.13\\
    林泉钢琴社线上分享 & 10.21 & 10.13\\
    林泉音乐会 & 10.19 & 10.13\\
    普通话考试报名 & 10.28 & 10.14\\
    心协心理博客 & 10.20 & 10.14\\
    有训集体生日会 & 10.19 & 10.14\\
    安邦趣味运动会 & 10.19 & 10.14\\
    羽球书院杯报名 & 10.17 & 10.14\\
    青鸟剧场新戏招募 & 10.27 & 10.14\\
    NUBA招募 & 10.18 & 10.14\\
    新传团学联招新 & 10.18 & 10.15\\
    网易雷火宣讲 & 10.17 & 10.15\\
    法学院师生交流会 & 10.19 & 10.15\\
    理科志愿服务讲堂 & 10.18 & 10.15\\
    商院草地音乐会 & 10.20 & 10.15\\
    \hline
\end{longtable}
\unskip
\unpenalty
\unpenalty}\unvbox\colbbox
\end{multicols}
\hrule
\pagebreak
\begin{multicols}{2}

\section{讲座}
\setbox\colbbox\vbox{
\makeatletter\col@number\@ne
\begin{longtable}{|c|c|c|}
    \hline
    往期讲座 & 开展日期 & 刊载日期\\
    \hline\hline
    《聚合物的研发与...》 & 10.24 & 10.3\\
    《电池及电化学能...》 & 11.24 & 10.3\\
    《专利查新与规避...》 & 12.19 & 10.3\\
    《对于人工智能时...》 & 10.16 & 10.9\\
    《卡夫卡、现代组...》 & 10.16 & 10.10\\
    《跨代性与跨代平...》 & 10.16 & 10.10\\
    《关于西方社会再...》 & 10.16 & 10.11\\
    《楚国郢都的诗经》 & 10.16 & 10.11\\
    《走进ESG暨案例分...》 & 10.17 & 10.12\\
    《作为症状的极端...》 & 10.16 & 10.12\\
    《大学生创新训练...》 & 10.18 & 10.15\\
    《从Riemann-Ze...》 & 10.16 & 10.15\\
    《走进本科生科研...》 & 10.18 & 10.15\\
    \hline  
\end{longtable}
\unskip
\unpenalty
\unpenalty}\unvbox\colbbox
1.满天星学术讲堂\\
主题:大学生创新训练项目指导与经验分享\\
主讲嘉宾:蒋彧(金融与保险学系教授、博士生导师,江苏高校“青蓝工程”中青年学术带头人,国家级大创指导老师),殷瑞(产业经济学系2021级本科生),黄子珊(工商管理系2022级本科生)\\
时间:2024.10.18(周五)18:30-20:30\\
地点:南京仙林校区 逸A-322\\
报名渠道:面向全校同学开放,\url{https://table.nju.edu.cn/dtable/forms/e498a860-3153-49eb-a27c-6fdce4d2fbb4/}\\
注:本次活动可作为项目制课程的一次过程性学习\\
推文链接\url{https://mp.weixin.qq.com/s/-q95DDGmuD1L58HLfaQtuQ}\\

2.数学学院本科生论坛(学生系列第48讲)\\
题目: 从Riemann-Zeta函数谈起\\
报告人:唐晨皓(21级)\\
时间:10月16日(星期三) 16:00-17:30\\
地点:戊己庚四楼北\\
腾讯会议:870-7007-3326\\
摘要: 大量结果表明,L-函数的解析性质蕴含着丰富的算术信息。作为最简单也是最重要的L-函数,Riemann-Zeta函数蕴含着素数分布的规律。我将从Riemann-Zeta函数出发,向大家解释L-函数如何联系解析和算术。最后我将介绍我在虚二次域幅角等分布方面的工作。
\url{https://mp.weixin.qq.com/s/JlWgMn2_HXjmcD7E94BA7A}\\

3.大创宣讲|“走进本科生科研系列”——“创新共领航”讲座\\
主讲嘉宾:吴兴新(生命科学学院教授、博导;指导“探究叶酸受体Ⅱ非依赖于叶酸的功能”与“干眼症小鼠模型构建”两项国家级大创项目,均获得优秀评级)\\
时间:2024.10.18(周五)16:00-18:00\\
地点:南京大学仙林校区 仙2-105\\
注:届时将在线上开放宣讲通道,请关注“创新共领航讲座”QQ群\\
报名方式:扫描推文中二维码,填写报名问卷,并加入QQ群-885936121\\
推文链接:\url{https://mp.weixin.qq.com/s/LKimxV4ZrpJyf7FyPyXSeA}\\



\section{“大美汉字”通识课选课通知}
面向对象:全校本科生开放选课\\
教学方式:线上线下同步教学,请同学们按照自己所在校区选择班级。\\
选课方式:本学期课程为“直选式”,先到先得,开课两周内可进行补退选\\
选课时间:10月15日16:00至11月3日24:00(含开课两周内的补退选)\\
鼓楼校区学生可在“2024年秋季学期新生课程补选”中选课;\\
仙林校区、苏州校区学生可在“2024年秋季学期老生课程补选”中选课;\\
注:每位学生仅限选择所在校区的1个班级,请注意选课校区,不允许跨校区选课,如误选非所在校区班级则选课做无效处理。\\

\section{安邦书院:院服设计大赛}
对象:安邦书院全体24级新生及往届学生\\
时间:即日起至2024年10月20日23:59\\
奖励:第一名:奖金300元+设计成品一件+证书\\
第二名:奖金200元+设计成品一件+证书\\
第三名:奖金100元+设计成品一件+证书\\
最佳人气奖1名:设计成品一件+证书\\
设计要求、报名链接见\url{https://mp.weixin.qq.com/s/SfiTlV_KA0Xina_FdtbTDA}

\section{招募 | 毓琇书院“朋辈薪火”学业讲师团}
对象:大二及以上!!!同学\\
招募组别与科目:\\
1)数学组:微积分I(第一层次)、线性代数(第一层次)、微积分I与线性代数(第二层次)\\
2)物理组:普通物理(力学)、大学物理I\\
电子组:电路分析\\
3)计算机组:C程序设计(一层次)、python\\
4)化学组:大学化学\\
5)英语组:围绕英语听说、读写开展学习活动\\
招募要求:遵纪守法、诚实守信、乐于奉献、作风正派、工作认真、表达和沟通能力强且学习成绩优秀(所辅导课程成绩不低于90分!!!)\\
(注:对于同一课程,大一学年中有一学期课程总评不低于90分即可;对于英语组,要求大一所在层次为一层次或二层次,学年中听说和读写任一课程、任一学期总评不低于90分即可。)\\
奖励:提供与工作时长相匹配的志愿时长或者学科实践劳育时长,并发放工作证,表现优秀的同学还将获得书院颁发的“金牌讲师”称号(还能认识许多有趣的大一学弟学妹们)\\
工作要求、报名链接见\url{https://mp.weixin.qq.com/s/Zjb3FEUiaYslqTuTV7SiwQ}

\section{新传团学联招新}
新闻传播学院团学联面向秉文书院与新闻传播学院全体同学招新,共设办公室、组织部、宣传部、文体发展部、生涯发展部、志愿与权益部以及学术与科创部七个部门。报名截止时间为10月18日18:00。因报名流程比较复杂,故请读者详阅原文:\url{https://mp.weixin.qq.com/s/13SE2j1iedsB5L-2mM2IFg}。

\section{网易雷火游戏策划专场宣讲会}
 时间:10月17日(周四)14:00\\
 地点:十食堂三楼就业中心 307\\
\url{https://www.wjx.top/vm/tpBPrrA.aspx#} 点击链接预约宣讲

\section{行知2024年院服设计大赛}
参赛对象:南京大学行知书院2024级全体本科新生
要求:(1)体现行知书院特色,侧重院服主体图案设计,底色为纯色,避免拼色;可灵活运用行知书院LOGO、口号、吉祥物等书院特色文化元素,院徽、吉祥物素材链接如下:\url{https://box.nju.edu.cn/d/4c80a2bccac44cf1afb0/}\\
(2)需提交完整的院服设计图,设计图至少包含正反两面的视图,可补充院服颜色、版型、领口袖口等细节设计;\\
(3)设计作品需另附100-200字的作品简介,包括但不限于作品创作灵感与元素、蕴含寓意与使用的设计工具等。作品简介最后须附个人信息,包括姓名、学号、联系电话;\\
(4)作品须由个人原创设计,如有侵权嫌疑,将取消参赛资格,并由参赛者承担侵权责任。\\
提交要求:请提交PNG格式的院服设计图,以及100-200字的作品简介Word文档。
设计图和作品简介文档均以“姓名-学号-联系电话-院服设计”命名。\\
提交方式:请在南大云盘提交上传投稿作品及作品简介,提交链接如下:\url{https://box.nju.edu.cn/d/fd0e4fdac92b458a9cb4/}如提交多个作品,请按要求将每件作品分别提交。\\
截止时间:2024年10月21日(周一)15:00

\section{2023级商学院草地音乐会}
活动地点:南京大学仙林校区炜华体育场草地\\
活动时间:2024年10月20日 18:30\\
活动对象:南京大学商学院2023级学生及其他年级同学\\
音乐会简介:共分为四个部分,分别是乐器演奏、个人和团体演唱、乐队表演以及互动大合唱。表演嘉宾来自23级商学院产业经济学等班级,他们将用爱的和弦,倾情献唱自由之乐。\\
注:草地音乐会现场还将发放门票纪念品和发光星星手环\\
节目单请见推文\url{https://mp.weixin.qq.com/s/I2fH9z0hj2NWZDxYdJ0x3A}\\

\section{2024排球新生杯}
比赛时间:小组赛 十月底(待定)\\
淘汰赛(半决赛、决赛) (待定)\\
比赛地点:南京大学鼓楼校区室外排球场(暂定)\\
参赛资格:\\
所有2024级本科⼤⼀新⽣(普通⽣)均可报名参加新⽣杯。\\
(一) 所有具有参赛资格队伍内的队员均具有参赛资格,同⼀名队员只能注册在⼀⽀队伍中\\
(二) ⽐赛前,队⻓必须认真填写记分表中参赛⼈员信息,以备核实(如有极特殊情况,⼈⼿不⾜⽆法参加⽐赛,可以向排协申请并和对⼿说明情况,然后可以找符合要求的未参赛⼈员上场⽐赛)\\
(三) 参赛⼈员须为普通⽣⼊学、且三年内未打过任何体育项⽬⾼⽔平⽐赛的学⽣,⾼⽔平运动员⼀律禁⽌参赛\\
(四) 报名费为20元/人\\
参赛人由排协组织部协调分队\\
(一)队伍分配的原则是按照新生入学的七个书院对应组队\\
上场的成员中女生人数需要大于等于2\\
(二)每支队伍会有一位排协代表担任队伍助理\\
(三)队名由队员自行商议决定\\
赛制等详细说明以及QQ群见原文:\url{https://mp.weixin.qq.com/s/7i_guR_RzpY6JNYL8o6IKA}
报名链接:\url{https://www.wjx.cn/vm/YcJlC5N.aspx}

\section{法学院普惠型DIY讲座 “法韵启智,共筑学梦”——师生分享交流会}
为使大一社科新生更加了解法学专业、同时使大二法学院学生更好地探索法学领域、让对法律感兴趣的同学更深入地理解法学学科,南京大学知行社邀请了来自南京大学法学院的老师及优秀的学长学姐,与各位一同交流并分享法学学习和法律实务的相关事项、经验,为大家答疑解惑。\\
时间:10月19日 14:00-16:00\\
地点:鼓楼校区教室\\
同步线上腾讯会议:286-692-799\\
(具体信息在QQ群内公布) \\
QQ群:666786486
具体信息见原文:\url{https://mp.weixin.qq.com/s/clvVVjgb3GRObtlWZKwoZw}

\section{志愿服务大讲堂(理科专场)}
大讲堂邀请优秀志愿者走进新生书院分享志愿服务经验、服务条例、介绍第二课堂平台、回答志愿服务相关问题,进一步规范志愿服务,宣传志愿服务渠理\\
本次活动面向安邦书院和有训书院。\\
时间:10月18日16:00-18:00\\
线下地点:鼓楼校区南青格庐\\
线上地点:腾讯会议591-9065-0033\\
\end{multicols}
\hrule
\vspace{4mm}
\centerline{\huge\textbf{参考消息}}
\begin{multicols}{2}
\section{美国中学“教育”疯狂的社会达尔文主义实践(一)}
社会达尔文主义是十九世纪兴起于西方的、给人类带来沉重灾难、至今仍有的流毒的重大思想价值体系之一,而其流毒在今日最集中的体现正是美国广大普通中学的“教育”实践。\\

十九世纪,欧洲列强在国内开展工业革命、在全世界各地开展殖民掠夺,对内剥削和镇压劳动人民,对外奴役和屠杀亚非拉群众。近代早期建立在力学体系之上的机械唯物主义政治哲学,如霍布斯社会契约论的思想,不能满足欧洲列强要为自己的种种罪恶行径合理化、正当化的需要。当时,生物学取得重大发展,特别是达尔文出版了《人类的由来》和《物种起源》后,生物学取代力学成为欧洲政治哲学的自然科学基础。\\

达尔文曾阅读马尔萨斯的《人口论》。马尔萨斯认为,为了防止人口过度增长超过物质供应,必须要采取经济措施限制贫穷群众生育,因为下等阶层要承担各种社会问题的主要责任;达尔文对此观点极为赞赏,认为生物进化学说是马尔萨斯的理论在“没有人类智力干预”的领域的应用。虽然生物进化论是纯粹自然科学的理论,但是达尔文并不是纯粹自然科学的作家;同样的,当时欧洲各色思想家也从非自然科学的角度来解读和利用生物进化论。而在工业革命和自然科学最先进的英国,社会上存在一种崇拜个人主义、竞争和强力的“维多利亚气质”。\\

综合以上种种因素,社会达尔文主义在十九世纪应运而生。达尔文主义揭示了人类并不是与自然二元的具有神性的生物;但生物进化论也在当时思想家的演绎下,隐含了人类的自然化和低贱化、从而合理化消灭某些“低劣人群”的观念。社会达尔文主义最根本的三条纲领是宿命决定论、不平均等级制和优胜劣汰选择。\\

宿命决定论主张“个体受制于种族,个人什么都不是,种族和民族代表一切”“个人命运决定于遗传禀赋”。其中又分为以奥斯卡·施密特为代表的,认为自然选择与道德正当二元对立,即使用强力不涉及道德,和以克勒门斯-奥古斯特·罗耶为代表的,认为自然选择与道德正当一元,即有利于生存竞争成功的就是善、惨遭不幸就是错,两派主张;两派主张虽有分歧,总而言之都是恃强凌弱无罪、殖民掠夺有理。\\

美国广大普通中学在开展“教学”实践和行政管理的过程中,往往奉行宿命决定论观念。有人会说,美国普通中学坚持宣传人人都能通过考试改变命运,这和宿命决定论岂不是根本相反的吗?其实,这完全是一种欺骗性的说辞。首先,美国普通中学认为,学生受限于集体,附属于考试,个人什么都不是。并不承认学生除了考试以外,还有其他自由性或可能性。其次,在考试制度中并不存在真正的公平,学生的考试能力与其出身有很大关系:要接受较好的考试训练,就需要家庭高价购买学区房产,其价格绝非贫苦群众能够承受;学生如接受校外考试培训,还需要其家庭额外投入大量财富。其三,在实践中,美国普通中学奉行考试分数就是一切,能够提升分数就是善,不能提升考试分数就是恶,考试分数高的人行为就是对,考试分数低的人行为就是错,的观念;与宿命决定论的内涵如出一辙。\\
\end{multicols} 
\end{document}