% HEAD BEGIN
\documentclass[letterpaper, 12pt]{article}
\newsavebox\colbbox
\usepackage{graphicx}
\usepackage{multicol}
\usepackage{anysize}
\usepackage{fontspec}
\usepackage[fontset=none]{ctex}
\usepackage{tabularx}
\usepackage{longtable}
\PassOptionsToPackage{hyphens}{url}
\usepackage[breaklinks=true, colorlinks=true]{hyperref}
\expandafter\def\expandafter\UrlBreaks\expandafter{\UrlBreaks\do\a\do\b\do\c\do\d\do\e\do\f\do\g\do\h\do\i\do\j\do\k\do\l\do\m\do\n\do\o\do\p\do\q\do\r\do\s\do\t\do\u\do\v\do\w\do\x\do\y\do\z\do\A\do\B\do\C\do\D\do\E\do\F\do\G\do\H\do\I\do\J\do\K\do\L\do\M\do\N\do\O\do\P\do\Q\do\R\do\S\do\T\do\U\do\V\do\W\do\X\do\Y\do\Z}
% \let\oldurl\url
% \renewcommand{\url}[1]{\begin{sloppypar}\oldurl{#1}\end{sloppypar}}
\setlength\columnsep{30pt}
\marginsize{30pt}{30pt}{10pt}{20pt}
\setmainfont{TeX Gyre Bonum}
\setCJKmainfont[BoldFont=Noto Serif CJK SC Bold, ItalicFont=FandolKai]{Source Han Sans SC}
\setlength{\parindent}{0cm}
% \setCJKmonofont{Noto Sans CJK SC}
\begin{document}
\begin{center}
    \Huge\textbf{南哪大专醒前消息}
\end{center}
\vspace{4mm}
\hrule
\renewcommand\tabularxcolumn[1]{m{#1}}
\begin{tabularx}{\textwidth}{>{\hsize.2\hsize}X>{\hsize.6\hsize}X>{\hsize.2\hsize}X}
    \begin{flushleft}
        2025.4.5\, No.211
    \end{flushleft}
    &
    \begin{center}
        \textit{“秉中持正、求新博闻。”}
    \end{center}
    &
    \begin{flushright}
        \textbf{南京市栖霞区}
    \end{flushright}
\end{tabularx}
\vspace{-3.5mm}
\hrule
\vspace{4mm}
% HEAD END
\centerline{\huge\textbf{活动预告}}
\begin{multicols}{2}
\section{订阅方式和加入编辑部}  
编辑部招聘人才,用爱发电,工作轻松,详情可联系QQ:1329527951 客服小千\\想订阅本消息或获取PDF版(便于查看超链接和往期),可加QQ群:\href{https://qm.qq.com/q/4HL41Nt3sQ}{466863272}.
\section{活动清单}
\setbox\colbbox\vbox{
\makeatletter\col@number\@ne
\begin{longtable}{|>{\centering\arraybackslash}m{.3\textwidth}|m{.06\textwidth}|m{.06\textwidth}|}
    \hline
    活动 & 开展时间 & 刊载时间\\
    \hline\hline
    南大版deepseek & / & 2.22\\
    悦读课程群 & / & 2.24\\
    eScience AI科研助手 & / & 3.11\\
    地科博物馆开放安排 & / & 3.22\\ 
    乐跑 & 5.16 & 3.10\\
    本科生劳育实践 & 7.20 & 2.19\\
    银星杯论文赛 & 4.22 & 2.27\\
    高教社杯 & 4.25 & 3.5\\
    南辩院系杯 & 4.12 & 3.6\\
    大文大理题目征集 & 期末 & 3.8\\
    5月免费上网 & ? & 3.9\\
    基础学科论坛 & 4.20 & 3.9\\
    普通话测试 & 4.11 & 3.25\\
    外教社杯 & 5.27 & 3.12\\
    Python比赛 & 4.6 & 3.16\\
    粤歌赛 & 4.12 & 3.24\\
    江苏创青春赛事 & 4.30 & 3.26\\
    悦读测试 & 4.6 & 3.27\\
    南大数学竞赛 & 4.15 & 3.27\\
    AI素养大赛 & 4.15 & 3.31\\
    浦口音乐跑 & 5.30 & 3.31\\
    红会暑期项目招募 & 4.12 & 4.1\\
    程设大赛 & 4.26 & 4.2\\
    主持人大赛报名 & 4.10 & 4.4\\
    春影摄影大赛 & 4.13 & 4.4\\
    奇绩创业宣讲课 & 4.11 & 4.4\\
    瑞声杯 & 4.20 & 4.4\\
    江苏大学生乡村振兴计划 4.7 & 4.4\\
    仙林校区志愿法律咨询 & / & 4.4\\
    天健志愿者招募 & 4.13 & 4.4\\
    外新社征集春日影 & 4.13 & 4.5\\
    \hline
\end{longtable}
\unskip
\unpenalty
\unpenalty}\unvbox\colbbox
\end{multicols}
\begin{multicols}{2}
\pagebreak

\section{讲座}
\begin{tabular}{|>{\centering\arraybackslash}m{.3\textwidth}|m{.06\textwidth}|m{.06\textwidth}|}
    \hline
    讲座 & 开展时间 & 刊载时间\\
    \hline\hline
    秦汉玺印人名考析 & 4.9 & 3.31\\\hline
    先人后事 破局之道 & 4.11 & 3.3\\\hline
    Regularization, Heuristics, and Strategy: A Long Journey Towards Understanding a Few Fundamental yet Fuzzy Concepts in Computing & 4.8 & 4.4\\\hline
    Consumer Awareness, Noisy Certification, and Corporate Social Responsibility under Asymmetric Information & 4.9 & 4.4\\\hline
    Assortment Optimization Under History-Dependent Effects & 4.11 & 4.4\\\hline
    编程语言的设计和实现 & 4.8 & 4.5\\\hline
\end{tabular}
%讲座预告写在这。用subsection
\subsection{直播预告|“诚计划”第146期:冯新宇教授主讲“编程语言的设计和实现”}
主讲人:冯新宇 南京大学教授、华为编程语言首席专家、仓颉编程语言首席架构师
\\时间:4月8日(周二)19:30-21:00
\\直播观看地点请见活动链接
\\详见:\url{https://mp.weixin.qq.com/s/SaYDJYSo-MLD3GcXYezkNg}
%此处写校级活动,请不要把讲座、院级活动和社团活动写在这里orz orz orz
\section{苏州校区校园共享头盔上线}
为保障师生安全出行,南京大学苏州校区推出暖心服务——在校区南门及东门门口设置“自助共享头盔借取点”。如师生匆忙出门忘带头盔,或临时需要应急,在门岗登记即可免费借用。校区将安排专人做好安全头盔的日常保管和卫生消毒工作,师生可放心使用。
\\详见:\url{https://mp.weixin.qq.com/s/K0vFBLdb1ucsFNt5vb45NA}

\section{南雍星光丨2025摄影大赛——金陵寻春 “镜”中春影}
大赛主题:金陵寻春 “镜”中春影
\\参赛对象:南京大学全体师生
\\截止时间:4月13日(周天)24:00
\\参与方式、评选规则和奖品设置请见活动链接
\\详见:\url{https://mp.weixin.qq.com/s/MMKn4JGAIiM8uK7h4bCebg}
\section{院级活动}
\begin{tabular}{|>{\centering\arraybackslash}m{.3\textwidth}|m{.06\textwidth}|m{.06\textwidth}|}
\hline
    活动 & 开展时间 & 刊载时间\\
    \hline\hline
    文院剧本创作研讨会 & 9.30 & 3.2\\
    物院征集课程指南 & 6.15 & 3.3\\
    地海征集春日影 & 6.15 & 3.14\\
    社院学术节 & 4.18 & 3.25\\
    五院运动会 & 4.13 & 3.31\\
    电子南师春日交流 & 4.12 & 3.31\\
    五院乒乓球赛 & 4.19 & 3.31\\
    建城影展征集 & 4.16 & 3.31\\
    法院党建征文 & 5.20 & 4.2\\
    地学乒赛 & 4.19 & 4.2\\
    匡计社商联谊 & 4.13 & 4.2\\
    数院羽球 & 4.12 & 4.4\\
    软院桌游 & 4.6 & 4.4\\
    软院征集 & 4.20 & 4.4\\
    南新读书会 & 4.9 & 4.5\\
    \hline
\end{tabular}
\subsection{会议通知 | 第八届“高校法语专业课程设计与教学方法”研讨会(二号通知)}
会议时间:6月6日-6月8日
\\会议地点:南京大学国际会议中心、南京大学外国语学院
\\主办单位:南京大学外国语学院、外语教学与研究出版社
\\会议日程、报名程序、报名链接请见活动链接
\\详见:\url{https://mp.weixin.qq.com/s/6snI2bv_1r2p26pAG2K2ZA}
\subsection{活动预告 | 企业近距离——荣耀南京研究所}
活动时间:4月10日(周四)下午14:00-16:30
\\参与对象:南京大学计算机学院学生
\\报名方式和活动群聊请见活动链接
\\详见:\url{https://mp.weixin.qq.com/s/TywPbCk7S26N0MPRfKjenw}

\subsection{南新读书会 | 下周预告}
时间:4月9日(周三)19:00
\\地点:新闻传播学院311室
\\1.临床医学的诞生 [法]米歇尔·福柯
\\分享人:林鑫 2024级博士研究生
\\2.神学政治论 [荷]巴鲁赫·斯宾诺莎
\\分享人:赵璇 2024级硕士研究生
\\详见:\url{https://mp.weixin.qq.com/s/Ge5jveM-ACxZf4h-1OE-nQ}

\section{社团活动}
\begin{tabular}{|>{\centering\arraybackslash}m{.3\textwidth}|m{.06\textwidth}|m{.06\textwidth}|}
    \hline
    社团活动 & 开展时间 & 刊载时间\\
    \hline\hline
    天文台开放日 & / & 1.6\\
    重唱诗歌奖征稿 & 4.30 & 3.31\\
    印社讲座 & 4.9 & 4.1\\
    弓箭社体验 & 4.8 & 4.1\\
    飞镖社体验 & 4.8 & 4.1\\
    排协网协体验 & 4.10 & 4.1\\
    杨协体验 & 4.12 & 4.1\\
    足协体验 & 4.15 & 4.1\\
    轮滑社体验 & 4.17 & 4.1\\
    拳击社体验 & 4.22 & 4.1\\
    轮滑社体验 & 4.22 & 4.1\\
    飞盘大赛 & 4.13 & 4.1\\
    五子棋大赛 & 4.13 & 4.1\\
    定向赛 & 4.20 & 4.1\\
    体育舞蹈教学 & 4.25 & 4.1\\
    吉他社歌手招募 & 4.20 & 4.4\\
    吉他社春日音 & 4.26 & 4.4\\
    国学社寄明信片 & 4.14 & 4.4\\
    新火星影映 & 4.6 & 4.5\\
    \hline
\end{tabular}
%这里是写社团活动的,社团活动就是由社团主办、主要针对社团内部人员的活动。不要把非社团活动写在这里。
\subsection{新火星放映 《钢的琴》}
时间:4.6 19:00(本周日)
\\地点:仙林校区? 鼓楼校区费A410
\\简介:钢厂下岗工人陈桂林为了维持生计终日奔波,妻子小菊却想要与他离婚。为了争夺女儿的抚养权,他在身边朋友的帮助用钢铁为女儿打造了一架钢琴。影片借下岗工人陈桂林为女儿造钢琴的荒诞行动,折射计划经济崩塌后工人阶级的身份迷失与精神自救。导演以冷峻的灰蓝色调贯穿全片,手持摄影机穿梭于废弃厂房、烟囱与工人村,构建出极具后工业时代特征的“废墟美学”。该片以粗粝的影像质感与符号化叙事,完成对东北老工业基地转型阵痛的艺术化铭刻,被誉为“一部用钢铁铸造的时代寓言”。
\\欢迎感兴趣的同学前来观影,放映结束后有可选的讲解\&讨论环节。加群了解详细信息:907939564


\subsection{篮协院系杯4月6日(周天)赛程}
男篮院系杯小组赛
\\电子vs文院 17:00 - 18:00
\\软院vs物理 18:00 - 19:00
\\地点:一组团篮球场
\\女篮院系杯小组赛
\\外院vs数理 16:00 - 17:00
\\地点:一组团篮球场
\\详见:\url{https://mp.weixin.qq.com/s/njJYmBJbk-wsHfOMCpWgNg}


\end{multicols}
\end{document}
