
% HEAD BEGIN
\documentclass[letterpaper, 12pt]{article}
\usepackage{graphicx}
\usepackage{multicol}
\usepackage{anysize}
\usepackage{fontspec}
\usepackage[fontset=none]{ctex}
\usepackage{tabularx}
\PassOptionsToPackage{hyphens}{url}
\usepackage[breaklinks=true, colorlinks=true]{hyperref}
\expandafter\def\expandafter\UrlBreaks\expandafter{\UrlBreaks\do\a\do\b\do\c\do\d\do\e\do\f\do\g\do\h\do\i\do\j\do\k\do\l\do\m\do\n\do\o\do\p\do\q\do\r\do\s\do\t\do\u\do\v\do\w\do\x\do\y\do\z\do\A\do\B\do\C\do\D\do\E\do\F\do\G\do\H\do\I\do\J\do\K\do\L\do\M\do\N\do\O\do\P\do\Q\do\R\do\S\do\T\do\U\do\V\do\W\do\X\do\Y\do\Z}
% \let\oldurl\url
% \renewcommand{\url}[1]{\begin{sloppypar}\oldurl{#1}\end{sloppypar}}
\setlength\columnsep{30pt}
\marginsize{30pt}{30pt}{10pt}{20pt}
\setmainfont{TeX Gyre Bonum}
\setCJKmainfont[BoldFont=Noto Serif CJK SC Bold, ItalicFont=FandolKai]{Noto Sans CJK SC}
\setlength{\parindent}{0cm}
% \setCJKmonofont{Noto Sans CJK SC}
\begin{document}
\begin{center}
    \Huge\textbf{南哪大专醒前消息}
\end{center}
\vspace{4mm}
\hrule
\renewcommand\tabularxcolumn[1]{m{#1}}
\begin{tabularx}{\textwidth}{>{\hsize.2\hsize}X>{\hsize.6\hsize}X>{\hsize.2\hsize}X}
    \begin{flushleft}
        2024.9.23\, No.69
    \end{flushleft}
    &
    \begin{center}
        \textit{“克明峻德。”}
    \end{center}
    &
    \begin{flushright}
        \textbf{南京市栖霞区}
    \end{flushright}
\end{tabularx}
\vspace{-3.5mm}
\hrule
\vspace{4mm}
% HEAD END
\centerline{\huge\textbf{活动预告}}
\begin{multicols}{2}
\section{编辑部招聘人才}
编辑部招聘人才,用爱发电,工作轻松,详情可联系QQ:1329527951 客服小祥\\想订阅本消息或获取PDF版(便于查看超链接),可加QQ群:\href{https://qm.qq.com/q/FGX1VYCrGS}{962626571}.
\section{Deadline Ongoing}
\begin{tabular}{|c|c|c|}
    \hline
    消息(未见ddl的,不刊) & 截止日期 & 刊载日期\\
    \hline\hline
    微博新浪大学生诗词大赛 & 9.25 & 9.3\\
    学术金秋,智趣问答 & 9.25 & 9.12\\
    仙林校史馆招募讲解员 & 10.30 & 9.12\\
    管道宣传志愿团队遴选 & 9.27 & 9.12\\
    郑钢教育菁英报名 & 9.24 & 9.14\\
    大气科学学生创新论坛招新 & 9.25 & 9.14\\
    国优计划报名 & 10.7 & 9.19\\
    本科生暑期课程评教 & 10.31 & 9.19\\
    网易雷火大赛 & 10.7 & 9.22\\
    蓝鲸之材产融活动 & 9.26 & 9.22\\
    走近华为报名 & 9.26 & 9.23\\
    
    \hline
\end{tabular}
\section{讲座}
\begin{tabular}{|c|c|c|}
    \hline
    往期讲座 & 开展日期 & 刊载日期\\
    \hline\hline
    《近代中国言论界的...》 & 9.24 & 9.23\\
    《梁启超的“近代”》 & 9.25 & 9.23\\
    《同步在线教学中的...》 & 9.24 & 9.23\\
      \hline
\end{tabular}\\\\



1.《近代中国言论界的第一次论争》

主讲人:李晓东 日本岛根县立大学教授

主持人:李恭忠 南京大学学衡研究院教授

时间:9月24日(周二)16:00-17:30

地点:南京大学仙林校区星云楼310\\





2.《梁启超的“近代”》

主讲人:李晓东 日本岛根县立大学教授

主持人:刘  超 南京大学文学院暨学衡研究院教授

时间:9月25日(周三)16:00-17:30

地点:南京大学仙林校区星云楼310\\

3.《同步在线教学中的大学生学习行为同伴效应》

时间:9.24 13:30-15:30

地点:仙林校区潘琦楼B座103

主讲人:曹宇莲(华东师范大学教育管理系晨晖学者)\\

4. Plato's Parmenides

主讲人:Christoph Horn(波恩大学哲学系讲席教授)

时间:9月26日(周四)10:10-12:00

地点:仙林校区逸C-102
\section{走近华为(南京大学专场)报名启动}
时间:2024年9月26日\\
地点:南京大学(仙林校区)杜厦图书馆 一楼报告厅\\
具体活动议程和报名方法详见链接\url{https://mp.weixin.qq.com/s/dl54p9OgdEezWEG5lIYrKA}
\section{国学社 “影像·华章”献礼国庆创意设计征稿}
参赛内容:\\
(1)“献礼国庆”的剪纸、篆刻、漆扇等作品。\\
(2)“献礼国庆”的正能量照片,能够体现新中国成立以来历史变迁和伟大成就、红色主题摄影等,人、事、物皆可,角度不限。\\
(3)其他优秀传统文化相关创意设计成果。\\
报名方式见\url{https://mp.weixin.qq.com/s/DcO25pqCPMSwbKUVTnAWPw}
\section{9-10月学科竞赛信息速递}
详见:\url{https://mp.weixin.qq.com/s/_q_yzIdkUJzzCp-J4_tcKQ}

内含:

1.全国大学生数学竞赛

2.华为ICT大赛

3.Image Cup微软创新杯

4.2024X-GAME上海智能新能源汽车大数据竞赛

5.第六届全球校园人工智能算法精英大赛

6.第三届全国大学生信息技术与工程创客大赛

7.2024百度搜索文心智能体创新大赛

8.2024年全国大学生计算机系统能力大赛OceanBase数据库创新设计赛

9.2024年全国大学生计算机系统能力大赛PolarDB数据库创新设计赛

10.全球AI攻防挑战赛

注:向南京大学开甲书院公众号编辑团队致敬
\end{multicols} 
\end{document}